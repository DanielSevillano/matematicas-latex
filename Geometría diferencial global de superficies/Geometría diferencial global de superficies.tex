\documentclass{report}
\usepackage[spanish]{babel}
\usepackage{amssymb, amsmath, amsthm, hyperref}

\title{Geometría diferencial global de superficies}
\author{}

\newtheorem{theorem}{Teorema}[chapter]
\newtheorem{corollary}[theorem]{Corolario}
\newtheorem{lemma}[theorem]{Lema}
\newtheorem{proposition}[theorem]{Proposición}

\theoremstyle{remark}
\newtheorem*{remark}{Observación}
\theoremstyle{remark}
\newtheorem*{note}{Nota}

\theoremstyle{definition}
\newtheorem{definition}{Definición}[chapter]
\theoremstyle{definition}
\newtheorem*{properties}{Propiedades}
\theoremstyle{definition}
\newtheorem*{example}{Ejemplo}

\begin{document}
\maketitle
\tableofcontents

\chapter{Introducción}
\section{Superficies regulares}

\begin{definition}
    $S \subset \mathbb{R}^3$ es superficie regular si para todo $p \in S$ existe una aplicación $X : U \to V$, con $U \subset \mathbb{R}^2$ abierto y $V \subset S$ entorno abierto de $p$ en $S$, que verifica:
    \begin{itemize}
        \item $X$ es diferenciable.
        \item $X$ es homeomorfismo.
        \item $dX_q : \mathbb{R}^2 \to \mathbb{R}^3$ es inyectiva para todo $q \in U$.
    \end{itemize}
\end{definition}

\begin{definition}
    \hfill
    \begin{itemize}
        \item $V$ se llama un entorno coordenado de $p$ en $S$.
        \item $X$ es una parametrización de $S$ en $p$ o un sistema local de coordenadas.
        \item $(U, X)$ es una carta en $p$.
        \item $\{ (U_i, X_i), i \in I : \bigcup_i X_i(U_i) = S \}$ es un atlas.
    \end{itemize}
\end{definition}

\begin{remark}
    Las parametrizaciones de una superficie regular no son únicas.
\end{remark}

\begin{proposition}
    Si $f : U \to \mathbb{R}$ es una función diferenciable definida sobre un abierto $U$, entonces el grafo de $f$
    $$G(f) = \{ (x, y, f(x, y)) : (x, y) \in U \}$$
    es una superficie regular.
\end{proposition}

\begin{definition}
    Sea $F : U \subset \mathbb{R}^n \to \mathbb{R}^m$, con $U$ abierto.
    \begin{itemize}
        \item $p \in U$ es un punto crítico de $F$ si $dF_p$ no es sobreyectiva. Entonces $F(p) \in \mathbb{R}^m$ es un valor crítico.
        \item $p$ es un punto regular si no es crítico. Análogamente, $F(p)$ es valor regular si no es crítico.
    \end{itemize}
\end{definition}

\begin{remark}
    Si $f : U \subset \mathbb{R}^3 \to \mathbb{R}$, entonces $df_p$ es sobreyectiva o $df_p = (0,0,0)$.
    Luego $a \in f(U)$ es valor regular de $f$ si y solo si $f_x, f_y, f_z$ no se anulan simultáneamente en ningún punto de $f^{-1}(a)$.
\end{remark}

\begin{proposition}
    Si $f: U \subset \mathbb{R}^2 \to \mathbb{R}$ diferenciable y $a \in f(U)$ es valor regular de $f$, entonces $f^{-1}(a)$ es superficie regular de $\mathbb{R}^3$.
\end{proposition}

\begin{proposition}
    Sea $S \subset \mathbb{R}^3$ una superficie regular.
    Entonces dado $p \in S$, existe $V$ entorno abierto de $p$ en $S$ tal que $V$ es el grafo de una función diferenciable de una de las tres formas siguientes:
    $$z = f(x, y), \quad y = g(x, z), \quad \text{o} \quad x = h(y, z)$$
\end{proposition}

\begin{proposition}
    Sea $S$ una superficie regular y sea $X : U \subset \mathbb{R}^2 \to S$ una aplicación diferenciable, inyectiva y tal que $dX_q$ es inyectiva para todo $q \in U$.
    Entonces $X^{-1} : X(U) \to U$ es continua y, en consecuencia, $X$ es una parametrización de $S$.
\end{proposition}

\begin{remark}
    Es necesario que $S$ sea superficie regular.
\end{remark}

\begin{definition}
    Una superficie parametrizada es una aplicación $X : U \subset \mathbb{R}^2 \to \mathbb{R}^3$ diferenciable.
    Se dice que $X(U) \subset \mathbb{R}^3$ es la traza de $X$.\\
    $X$ es regular si $dX_q : \mathbb{R}^2 \to \mathbb{R}^3$ es inyectiva para todo $q \in U$.
\end{definition}

\begin{proposition}
    Sea $X : U \subset \mathbb{R}^2 \to \mathbb{R}^3$ superficie parametrizada regular y sea $q \in U$.
    Entonces existe $V$ entorno abierto de $q$ en $U$ tal que $X(U)$ es superficie regular.
\end{proposition}

\begin{definition}
    $C \subset \mathbb{R}^3$ es una curva regular si para todo $p \in C$ existe $V$ entorno abierto de $p$ y $\alpha : I \to U \int C$, con $I$ intervalo abierto, tal que:
    \begin{itemize}
        \item $\alpha$ es diferenciable.
        \item $\alpha$ es homeomorfismo.
        \item $d\alpha_t$ = $\alpha'(t)$ es inyectiva para todo $t$.
    \end{itemize}
\end{definition}

\begin{definition}
    Una superficie de revolución es un subconjunto $S \subset \mathbb{R}^3$ obtenida al rotar una curva plana regular $C$ alrededor de un eje contenido en el mismo plano que la curva y que no corte a la curva.
\end{definition}

\section{Cálculo diferencial en superficies regulares}

\begin{definition}
    Sea $f: S \subset \mathbb{R}^3 \to \mathbb{R}$, $X$ parametrización de $S$ y $p \in \mathbb{R}^3$.
    $f$ es diferenciable en $p$ si y solo si $f \circ X$ es diferenciable en $X^{-1}(p)$.
\end{definition}

\begin{remark}
    Esta definición no depende de la parametrización de $S$.
\end{remark}

\begin{definition}
    $f: O \subset S \to \mathbb{R}$ es una función diferenciable en $p \in O$ si para alguna parametrización $X : U \to S$, $p \in X(U)$, se tiene que $f \circ X : U \to \mathbb{R}$ es diferenciable en $X^{-1}(p)$.
\end{definition}

\begin{definition}
    $\varphi : S_1 \to S_2$ es difeomorfismo si es diferenciable, biyectiva y $\varphi^{-1}$ es diferenciable.
\end{definition}

\begin{definition}
    Un vector tangente a $S$ en $p$ es un vector tangente a una curva diferenciable parametrizada que pase por $p$.\\
    Es decir, una curva $\alpha : (-\varepsilon, \varepsilon) \to S$, con $\alpha(0) = p$, $\alpha'(0) \in T_pS$.
\end{definition}

\begin{proposition}
    Sea $X$ una parametrización, $dX_q(\mathbb{R}^2) = T_pS$, con $q = X^{-1}(p)$.
\end{proposition}

\begin{definition}
    Sea $\varphi : O \subset S_1 \to S_2$ una aplicación diferenciable definida en un abierto $O$ de $S_1$ y $p \in O$.
    Consideramos la diferencial de $\varphi$ en $p$
    $$d\varphi_p : T_pS_1 \to T_{\varphi(p)}S_2$$
    Sean $w \in T_pS_1$ y $\alpha : (-\varepsilon, \varepsilon) \to S_1$ curva diferenciable parametrizada con $\alpha(0) = p, \alpha'(0) = w$.
    Entonces $d\varphi_p(w) = (\varphi \circ \alpha)'(0)$.\\
    Se tiene:
    \begin{itemize}
        \item $\varphi \circ \alpha$ es una curva diferenciable parametrizada sobre $S_2$. $(\varphi \circ \alpha)'(0) \in T_{\varphi(p)}S_2$.
        \item $d\varphi_p$ no depende de $\alpha$ y es lineal.
    \end{itemize}
\end{definition}

\begin{definition}
    Sea $S$ superficie regular, $O$ abierto de $S$, $f : O \subset S \to \mathbb{R}$ diferenciable.
    Consideramos la diferencial de $f$ en $p$
    $$df_p : T_pS \to \mathbb{R}$$
    Sean $w \in T_pS$ y $\alpha : (-\varepsilon, \varepsilon) \to S$ curva diferenciable parametrizada con $\alpha(0) = p, \alpha'(0) = w$.
    Entonces $df(w) = (f \circ \alpha)'(0)$.\\
    Verifica:
    \begin{itemize}
        \item Está bien definida y es lineal.
        \item Si $f = F|_S$ con $F : O \subset \mathbb{R}^3 \to \mathbb{R}$, entonces $df_p = dF_p|_{T_pS}$.
    \end{itemize}
\end{definition}

\section{Primera forma fundamental}

\begin{definition}
    Sea $S$ superficie regular.
    Para cada $p \in S$ el producto escalar en $\mathbb{R}^3$ induce una métrica en $T_pS$.
    $$\left\langle w_1, w_2 \right\rangle _p = \left\langle w_1, w_2 \right\rangle, \quad \forall w_1, w_2 \in T_pS$$
    La forma cuadrática asociada se llama primera forma fundamental de $S$ en $p$.
    $$I_p : T_pS \to \mathbb{R}$$
    $$I_p(w) = \left\langle w, w \right\rangle _p = |w|^2 \geq 0$$
\end{definition}

\begin{note}
    Dada una parametrización $X$ de $S$, $\{ X_u, X_v \}_q$ base de $T_pS$, $q = X^{-1}(p)$.
    Llamamos $E$, $F$ y $G$ a los coeficientes de la primera forma fundamental.
    $$E_q = \left\langle X_u(q), X_u(q) \right\rangle$$
    $$F_q = \left\langle X_u(q), X_v(q) \right\rangle$$
    $$G_q = \left\langle X_v(q), X_v(q) \right\rangle$$
    Estas son funciones diferenciables en $X(U)$.
\end{note}

\section{Propiedades de las curvas}

\begin{definition}
    Sea $\alpha : I \to S$ una curva diferenciable parametrizada.
    Se define la longitud de arco como una aplicación $s : I \to \mathbb{R}$, con $t_0 \in I$ fijo, dada por:
    $$s(t) = \int^t_{t_0} |\alpha'(r)|dr = \int^t_{t_0} \sqrt{\left\langle \alpha'(r), \alpha'(r) \right\rangle} dr$$
\end{definition}

\begin{remark}
    Sea $(u_0, v_0) \in U$ fijo.
    $$u \mapsto X(u, v_0), \quad v \mapsto X(u_0, v)$$
    $$s_{v_0}(u) = \int^u_{u_0} |X_u| dr = \int^u_{u_0} \sqrt{E} dr$$
    $v = v_0$ está parametrizada por el arco si y solo si $E(u, v_0) = 1$ para todo $u$.
    $$s_{u_0}(v) = \int^v_{v_0} |X_v| dr = \int^v_{v_0} \sqrt{G} dr$$
    $u = u_0$ está parametrizada por el arco si y solo si $G(u_0, v) = 1$ para todo $v$.\\
    En general, todas las curvas coordenadas de $X$ están parametrizadas por el arco si y solo si $E(u, v) = 1$ y $G(u, v) = 1$, para todo $(u, v) \in U$.
\end{remark}

\begin{definition}
    El ángulo de dos curvas es el menor ángulo que forman las rectas tangentes.
    Sean $\alpha, \beta : I \to S$, con $\alpha(t_0) = \beta(t_0)$.
    $$\cos \theta(t_0) = \frac{\left\langle \alpha'(t_0), \beta'(t_0) \right\rangle}{|\alpha'(t_0)||\beta'(t_0)|}$$
\end{definition}

\begin{remark}
    El ángulo de las curvas coordenadas $u = u_0$ y $v = v_0$ en $X(u_0, v_0)$ es
    $$\cos \theta(u_0, v_0) = \frac{\left\langle X_u, X_v \right\rangle}{|X_u||X_v|} (u_0, v_0) = \frac{F}{\sqrt{EG}} (u_0, v_0)$$
    Las curvas coordenadas de una parametrización $u = cte$, $v = cte$ son ortogonales en todos los puntos de $U$ si y solo si $F(u, v) = 0$ para todo $(u, v) \in U$.\\
    En ese caso se dice que $X$ es una parametrización ortogonal.
\end{remark}

\section{Orientación en superficies}

\begin{properties}
    \hfill
    \begin{itemize}
        \item Dos bases ordenadas de un mismo espacio vectorial representan la misma orientación si el determinante de la matriz de cambio de base es positivo.
              Cada clase de equivalencia es una orientación en $U$.
        \item $\{X_u, X_v\}$ determina una orientación en $S$.
              $\{X_u, X_v, X_u \land X_v\}$ es una base positiva, con $N = \frac{X_u \land X_v}{|X_u \land X_v|}$.
    \end{itemize}
\end{properties}

\begin{definition}
    $S$ es orientable si se puede recubrir con una familia de entornos coordenados tal que en la intersección de dos tales entornos el determinante jacobiano del cambio de coordenadas tiene determinante positivo.
    La elección de tal familia se llama una orientación de $S$ y se dice que $S$ está orientada.
\end{definition}

\begin{definition}
    Sea $V$ abierto de $S$, se llama campo diferenciable de vectores normales unitarios a $N : V \to \mathbb{R}^3$ diferenciable tal que para todo $p \in S$ se tiene que $N(p) \perp T_pS$ y $|N(p)| = 1$.
\end{definition}

\begin{theorem}
    $S$ es orientable si y solo si existe un campo diferenciable $N$ de vectores normales unitarios sobre $S$.
\end{theorem}

\begin{note}
    La elección de dicho campo normal $N$ determina una orientación en $S$.
\end{note}

\begin{proposition}
    Si $S$ es imagen inversa de un valor regular, entonces $S$ es orientable.
\end{proposition}

\section{Segunda forma fundamental}

\begin{definition}
    Sea $S \subset \mathbb{R}^3$ orientable y orientada con orientación $N : S \to \mathbb{R}^3$, con $|N(p)| = 1$ para todo $p \in S$.
    Luego $N(p) \in S^2(1) \equiv S^2 = \{ (x, y, z) \in \mathbb{R}^3 : x^2+y^2+z^2 = 1 \}$.\\
    La aplicación diferenciable $N : S \to S^2$ es la aplicación de Gauss.
\end{definition}

\begin{note}
    Se mira $N$ como aplicación diferenciable entre superficies, no como un campo de vectores.
    El vector normal se toma con origen en el origen de $\mathbb{R}^3$ y el extremo da un punto de $S^2$.
\end{note}

\begin{remark}
    Si se cambia la orientación, también se cambia la aplicación de Gauss.
\end{remark}

\begin{definition}
    La diferencial de la aplicación de Gauss es, para $p \in S$, $dN_p : T_pS \to T_{N(p)}S^2$ lineal.
    Como $N(p) \perp T_pS$ y $N(p) \perp T_{N(p)}S^2$, estos son planos paralelos, así que se pueden identificar y considerar $dN_p : T_pS \to T_pS$ endomorfismo de $T_pS$.
    Este mide la variación en dirección de $N$ sobre las curvas que pasan por $p$ en un entorno de $p$.
    $$dN_p(w) = \frac{d}{dt}|_{t=0} (N \circ \alpha)(t) = (N \circ \alpha)'(0)$$
\end{definition}

\begin{proposition}
    $dN_p$ es autoadjunta respecto a $\left\langle , \right\rangle$, es decir, $\left\langle dN_p(v), w \right\rangle = \left\langle v, dN_p(w) \right\rangle, \forall v, w \in T_pS$.
    Luego se puede asociar la forma bilineal simétrica con la forma cuadrática.
\end{proposition}

\begin{definition}
    Dicha forma cuadrática $\amalg_p : T_pS \to \mathbb{R}$
    $$\amalg_p(w) = -\left\langle dN_p(w), w \right\rangle$$
    es la segunda forma fundamental de $S$ en $p$.
\end{definition}

\begin{remark}
    $S_p = -dN_p$ es el operador de Weingarten en $p$.
\end{remark}

\begin{definition}
    Sea $C$ una curva regular en $S$ que pasa por $p$.
    Se define la curvatura normal de $C$ en $p$ como
    $$k_n(p) = k(p) \left\langle n(p), N(p) \right\rangle$$
\end{definition}

\begin{remark}
    $k_n$ cambia de signo si cambia la orientación en $S$ y no depende de la orientación en $C$.
\end{remark}

\begin{theorem}[Teorema de Meusnier]
    Todas las curvas sobre una superficie regular orientable $S$ que tienen la misma recta tangente en $p \in S$ tienen la misma curvatura normal en $p$.
\end{theorem}

\begin{remark}
    \hfill
    \begin{itemize}
        \item $\amalg_p(w) = k_n(p)$, siendo $w$ unitario.
        \item La curvatura de la sección normal a lo largo de $w$ de $S$ en $p$ es el valor absoluto de la curvatura normal en $p$ de cualquier curva sobre $S$ que pase por $p$ con vector tangente $w$.
              $$k(p) = |\amalg_p(w)|$$
    \end{itemize}
\end{remark}

\begin{theorem}
    Sea $S$ superficie orientable con aplicación de Gauss $N : S \to S^2$.
    Para todo $p \in S$ existe $\{e_1, e_2\}$ base ortonormal de $T_pS$ con $dN_p(e_1) = -k_1e_1, dN_p(e_2) = -k_2e_2$.
    Además, $k_1$ y $k_2$ son el máximo y el mínimo de $\amalg_p$ sobre la circunferencia unidad de $T_pS$, es decir, los valores extremos de las curvaturas normales en $p$.
\end{theorem}

\begin{theorem}[Fórmula de Euler]
    Sea $w \in T_pS$, $|w| = 1$.
    $$\amalg_p(w) = \cos^2(t) k_1 + \sin^2(t) k_2$$
    con $\cos(t) = \left\langle w, e_1 \right\rangle, \sin(t) = \left\langle w, e_2 \right\rangle$, es decir, $t$ es el ángulo que forma $w$ con $e_1$ en la orientación de $T_pS$.
\end{theorem}

\begin{definition}
    $k_1$ y $k_2$ son las curvaturas principales de $S$ en $p$.
    Las direcciones asociadas se llaman direcciones principales en $p$.
\end{definition}

\begin{definition}
    Una curva regular conexa $C$ en $S$ es línea de curvatura de $S$ si para todo $p \in C$ la recta tangente a $C$ en $p$ es una dirección principal en $p$.
\end{definition}

\begin{definition}
    \hfill
    \begin{itemize}
        \item La curvatura de Gauss de $S$ en $p$ se define como $K(p) = det(dN_p)$.
        \item La curvatura media de $S$ en $p$ se define como $H(p) = -\frac{1}{2} tr(dN_p)$.
    \end{itemize}
\end{definition}

\begin{remark}
    En una base ortonormal de direcciones principales
    $$dN_p \equiv
        \begin{pmatrix}
            -k_1 & 0    \\
            0    & -k_2
        \end{pmatrix} \Rightarrow
        K = k_1 k_2, \quad H = \frac{k_1+k_2}{2}$$
\end{remark}

\begin{remark}
    Ante un cambio de orientación, $k_1$, $k_2$ y $H$ cambian de signo, mientras que $K$ no cambia.
\end{remark}

\begin{definition}
    Sea $p \in S$, se puede clasificar según su curvatura de Gauss.
    \begin{itemize}
        \item Si $K(p) > 0$, decimos que $p$ es elíptico.
        \item Si $K(p) < 0$, decimos que $p$ es hiperbólico.
        \item Si $K(p) = 0$ y $dN_p \neq 0$, decimos que $p$ es parabólico.
        \item Si $dN_p \equiv 0$, decimos que $p$ es plano.
    \end{itemize}
\end{definition}

\begin{remark}
    \hfill
    \begin{itemize}
        \item Si $k_1 = k_2$, todas las direcciones son principales.
        \item Si $k_1 \neq k_2$, hay dos direcciones principales y son perpendiculares.
    \end{itemize}
\end{remark}

\begin{definition}
    Un punto $p$ es umbilical si $k_1(p) = k_2(p)$.
    En ese caso, $k_n = k_1 = k_2$ y todas las direcciones de $T_pS$ son principales.
\end{definition}

\begin{remark}
    Los puntos umbilicales solo pueden ser elípticos o planos.
\end{remark}

\begin{theorem}
    Si todos los puntos de una superficie conexa $S$ son umbilicales, entonces $S$ es un abierto de un plano o de una esfera.
\end{theorem}

\begin{definition}
    Una dirección asintótica de $S$ en $p$ es una dirección de $T_pS$ para la cual la curvatura normal es cero.
\end{definition}

\begin{definition}
    Una curva o línea asintótica es una curva regular conexa $C$ en $S$ tal que, para todo $p \in C$, la recta tangente a $C$ en $p$ es una dirección asintótica.
\end{definition}

\begin{note}
    $C$ es una curva asintótica si y solo si $k_n(p) = k(p) \left\langle n, N \right\rangle = 0$.
\end{note}

\begin{remark}
    \hfill
    \begin{itemize}
        \item En un punto elíptico no hay direcciones asintóticas.\\
              $k_1$ y $k_2$ tienen el mismo signo, así que $k_2 < k_n < k_1$.
        \item En un punto plano todas las direcciones son asintóticas.
        \item En un punto hiperbólico hay dos direcciones asintóticas.\\
              $k_1 \cos^2(t) + k_2 \sin^2(t) = 0$ tiene dos soluciones.
        \item En un punto parabólico hay una dirección asintótica.
              $k_1 \cos^2(t) = 0$ tiene una solución.
    \end{itemize}
\end{remark}

\begin{proposition}
    Sea $S$ orientada, con $N = \frac{X_u \land X_v}{|X_u \land X_v|}$. En la base $\{ X_u, X_v \}$
    $$\amalg_p =
        \begin{pmatrix}
            e & f \\
            f & g
        \end{pmatrix},$$
    $$e = \left\langle N, X_{uu} \right\rangle = \frac{det(X_u, X_v, X_{uu})}{\sqrt{EG-F^2}}$$
    $$g = \left\langle N, X_{vv} \right\rangle = \frac{det(X_u, X_v, X_{uv})}{\sqrt{EG-F^2}}$$
    $$f = \left\langle N, X_{uv} \right\rangle = \frac{det(X_u, X_v, X_{vv})}{\sqrt{EG-F^2}}$$
\end{proposition}

\begin{proposition}
    $$dN_p = -I_p^{-1} \amalg_p$$
\end{proposition}

\begin{proposition}
    $$K = \frac{eg-f^2}{EG-F^2}$$
    $$H = \frac{1}{2} \frac{eG - 2fF + gE}{EG-F^2}$$
    $K$ y $H$ son diferenciables.
\end{proposition}

\begin{remark}
    Si una parametrización de una superficie regular es tal que $F = f = 0$, entonces las matrices de $I_p$ y $\amalg_p$ son diagonales y $dN_p$ es diagonal en $\{X_u, X_v\}$.
    $$dN_p = -
        \begin{pmatrix}
            E & 0 \\
            0 & G
        \end{pmatrix}^{-1}
        \begin{pmatrix}
            e & 0 \\
            0 & g
        \end{pmatrix} = -
        \begin{pmatrix}
            \frac{e}{E} & 0           \\
            0           & \frac{g}{G}
        \end{pmatrix}$$
    Así que las curvaturas principales son $\frac{e}{E}$ y $\frac{g}{G}$.
\end{remark}

\section{Geometría intrínseca}

\begin{definition}
    $\varphi : S \to \bar{S}$ es isometría si:
    \begin{itemize}
        \item $\varphi$ es difeomorfismo.
        \item $d\varphi_p : T_pS \to T_{\varphi(p)}\bar{S}$ es isometría lineal para todo $p \in S$.
    \end{itemize}
    Se dice que $S$ y $\bar{S}$ son isométricas.
\end{definition}

\begin{remark}
    Como $\varphi$ es difeomorfismo, entonces $d\varphi_p$ es isomorfismo.
    Luego $d\varphi_p$ es isometría lineal si y solo si conserva el producto escalar.\\
    Esto es, si para cada $w_1, w_2 \in T_pS$
    $$\left\langle w_1, w_2 \right\rangle _p = \left\langle d\varphi_p(w_1), d\varphi_p(w_2) \right\rangle _{\varphi(p)}$$
\end{remark}

\begin{definition}
    Sea $V$ entorno abierto de $p$ en $S$.
    $\varphi : V \to \bar{S}$ es isometría local si existe $\bar{V}$ entorno abierto de $\varphi(p)$ en $\bar{S}$ tal que $\varphi : V \to \bar{V}$ es isometría.\\
    $S$ es localmente isométrica a $\bar{S}$ si para todo $p \in S$ existe isometría local a $\bar{S}$.
    Si además $\bar{S}$ es localmente isométrica a $S$, se dice que $S$ y $\bar{S}$ son localmente isométricas.
\end{definition}

\begin{remark}
    Ser localmente isométricas no implica que exista $\varphi : S \to \bar{S}$ isometría local, porque puede que no sea la misma en todos los puntos.
\end{remark}

\begin{proposition}
    Sean $X: U \subset \mathbb{R}^2 \to S, \bar{X}: U \subset \mathbb{R}^2 \to \bar{S}$ parametrizaciones tales que $E = \bar{E}, F = \bar{F}, G = \bar{G}$ en $U$.
    Entonces $\varphi = \bar{X} \circ X^{-1}: X(U) \to \bar{X}(U)$ es isometría, es decir, $\varphi: X(U) \to \bar{S}$ es isometría local.
\end{proposition}

\begin{proposition}
    Sea $\varphi: S \to \bar{S}$ isometría, $X: U \to S$ parametrización.
    Entonces $\bar{X} = \varphi \circ X: U \to \bar{S}$ es una parametrización de $\bar{S}$ en $\varphi(p)$ con $E = \bar{E}, F = \bar{F}, G = \bar{G}$ en $U$.
\end{proposition}

\begin{note}
    La inversa y la composición de isometrías son isometrías.
\end{note}

\begin{proposition}
    Sea $\varphi: S \to \bar{S}$ difeomorfismo.
    \begin{enumerate}
        \item $\varphi$ es isometría si y solo si $\varphi$ conserva la longitud de arco de las curvas parametrizadas en $S$.
        \item Si $\varphi$ es isometría entonces $\varphi$ conserva ángulos y áreas.
    \end{enumerate}
\end{proposition}

\begin{definition}
    $\varphi: S \to \bar{S}$ es aplicación conforme si
    \begin{itemize}
        \item $\varphi$ es difeomorfismo.
        \item Para todo $p \in S$, $w_1, w_2 \in T_pS$
              $$\left\langle d\varphi_p(w_1), d\varphi_p(w_2) \right\rangle _{\varphi(p)} = \lambda^2(p) \left\langle w_1, w_2 \right\rangle _p$$
              donde $\lambda^2$ es una función diferenciable no nula sobre $S$.
    \end{itemize}
    Se dice que $S$ y $\bar{S}$ son conformes.
\end{definition}

\begin{definition}
    $\varphi: V \to \bar{S}$, con $V$ entorno abierto de $p$ en $S$, es aplicación conforme local en $p$ si existe un entorno abierto $\bar{V}$ de $\varphi(p)$ en $\bar{S}$ tal que $\varphi: V \to \bar{V}$ es aplicación conforme.
    Si para todo $p \in S$ existe una aplicación conforme local en $p$, entonces $S$ es localmente conforme a $\bar{S}$.
\end{definition}

\begin{remark}
    Una isometría es una aplicación conforme con $\lambda(p) = 1, \forall p$.
\end{remark}

\begin{note}
    Una aplicación conforme no conserva longitudes de curvas, mientras que conserva los ángulos.
\end{note}

\begin{proposition}
    Sean $X: U \subset \mathbb{R}^2 \to S, \bar{X}: U \subset \mathbb{R}^2 \to \bar{S}$ parametrizaciones tales que $\bar{E} = \lambda^2 E, \bar{F} = \lambda^2 F, \bar{G} = \lambda^2 G$ en $U$, con $\lambda^2$ diferenciable y $\lambda^2 \neq 0$ en $U$.
    Entonces $\varphi = \bar{X} \circ X^{-1}: X(U) \to \bar{S}$ es una aplicación conforme local.
\end{proposition}

\begin{corollary}
    Dos superficies cualesquiera son localmente conformes.
\end{corollary}

\begin{proposition}
    Sea $\varphi: S \to \bar{S}$ aplicación conforme, $X: U \to S$ parametrización.
    Entonces $\bar{X} = \varphi \circ X: U \to \bar{S}$ es una parametrización de $\bar{S}$ en $\varphi(p)$ con $\bar{E} = \lambda^2 E, \bar{F} = \lambda^2 F, \bar{G} = \lambda^2 G$ en $U$.
\end{proposition}

\section{El teorema de Gauss}

\begin{proposition}
    Sea $S$ una superficie regular orientable y orientada, con orientación $N: S \to S^2$ y sea $X: U \subset \mathbb{R}^2 \to S$ una parametrización de $S$ compatible con la orientación $N$.
    Es decir, $N = \frac{X_u \land X_v}{|X_u \land X_v|}$ en $U$.
    A cada punto de $X(U)$ se le puede asignar un triedro $\{X_u, X_v, N\}$, que es una base de $\mathbb{R}^3$.
    Derivando estos vectores con respecto de $u$ y $v$ obtenemos las siguientes ecuaciones:
    $$\left\{
        \begin{array}{lcl}
            X_{uu} & = & \Gamma^1_{11}X_u + \Gamma^2_{11}X_v + eN         \\
            X_{uv} & = & \Gamma^1_{12}X_u + \Gamma^2_{12}X_v + fN = X{vu} \\
            X_{vv} & = & \Gamma^1_{22}X_u + \Gamma^2_{22}X_v + gN         \\
            N_u    & = & a_{11}X_u + a_{21}X_v                            \\
            N_v    & = & a_{12}X_u + a_{22}X_v
        \end{array}
        \right.$$
    Estas se conocen como ecuaciones de Gauss-Weingarten.
    Los $\Gamma^k_{ij}$ se denominan los símbolos de Christoffel de $S$ en la parametrización $X$.
\end{proposition}

\begin{proposition}
    Consideramos las relaciones
    $$\left\{
        \begin{array}{lcl}
            (X_{uu})_v - (X_{uv})_u & = & 0 \\
            (X_{vv})_u - (X_{vu})_v & = & 0 \\
            N_{uv} - N_{vu}         & = & 0
        \end{array}
        \right.$$
    Estas son las condiciones que hacen que el sistema de ecuaciones en derivadas parciales que definen las ecuaciones de Gauss-Weingarten sea integrable dada una condición inicial.\\
    Sustituyendo las ecuaciones de Gauss-Weingarten en estas relaciones y resolviendo llegamos a la fórmula de Gauss si $E \neq 0$:
    $$\Gamma^1_{11} \Gamma^2_{12} + \Gamma^2_{11} \Gamma^2_{22} + (\Gamma^2_{11})_v - \Gamma^1_{12} \Gamma^2_{11} - \Gamma^2_{12} \Gamma^2_{12} - (\Gamma^2_{12})_u = EK$$
\end{proposition}

\begin{remark}
    Como $E \neq 0$, la fórmula de Gauss permite obtener la curvatura de Gauss a partir de los símbolos de Christoffel, es decir, que $K$ es intrínseco.
\end{remark}

\begin{note}
    Existen otras dos versiones de la fórmula de Gauss cuando $F \neq 0$ y $G \neq 0$, respectivamente.
\end{note}

\begin{theorem}[Teorema Egregium de Gauss]
    La curvatura de Gauss de una superficie regular es invariante frente a isometrías locales.
\end{theorem}

\begin{remark}
    El recíproco del teorema Egregium no es cierto en general.
    Sin embargo, si dos superficies tienen la misma curvatura de Gauss constante, entonces dos entornos cualesquiera suficientemente pequeños de esas superficies son isométricos.
\end{remark}

\begin{note}
    Siempre es posible calcular los símbolos de Christoffel en términos de los coeficientes de la primera forma fundamental y de sus derivadas.
\end{note}

\begin{remark}
    Todos los conceptos geométricos y propiedades que se expresan en términos de los símbolos de Christoffel son invariantes por isometrías, puesto que solo dependen de la primera forma fundamental.
\end{remark}

\begin{proposition}
    Las ecuaciones de Mainardi-Codazzi son
    $$\left\{
        \begin{array}{lcl}
            e_v - f_u & = & e\Gamma^1_{12} + f(\Gamma^2_{12} - \Gamma^1_{11}) - g\Gamma^2_{11} \\
            f_v - g_u & = & e\Gamma^1_{22} + f(\Gamma^2_{22} - \Gamma^1_{12}) - g\Gamma^2_{12}
        \end{array}
        \right.$$
\end{proposition}

\begin{note}
    Las fórmulas de Gauss y las ecuaciones de Mainardi-Codazzi se conocen como ecuaciones de compatibilidad de la teoría de superficies.
\end{note}

\begin{theorem}[Teorema de Bonnet]
    Sean $E, F, G, e, f, g$ funciones diferenciables definidas en un conjunto abierto $V \subset \mathbb{R}^2$ con $E > 0$, $G > 0$, y que verifican la fórmula de Gauss y las ecuaciones de Mainardi-Codazzi y $EG-F^2 > 0$.
    Entonces para cada $q \in V$ existe un entorno abierto $U$ de $q$ en $V$ y un difeomorfismo
    $$X: U \to X(U) \subset \mathbb{R}^3$$
    tal que la superficie regular $X(U)$ tiene a $E, F, G$ y a $e, f, g$ como coeficientes de la primera y la segunda formas fundamentales respectivamente.\\
    Además, si $U$ es conexo y si
    $$\bar{X}: U \to \bar{X}(U) \subset \mathbb{R}^3$$
    es otro difeomorfismo que satisface las mismas condiciones, entonces existe un movimiento rígido directo de $\mathbb{R}^3$, $\psi: \mathbb{R}^3 \to \mathbb{R}^3$, tal que $\bar{X} = \psi \circ X$.
\end{theorem}

\section{Parametrizaciones especiales}

\begin{definition}[Parametrización ortogonal]
    Se dice que una parametrización de una superficie regular $S$, $X: U \subset \mathbb{R}^2 \to S$, $(u, v) \in U$, es ortogonal si la función $\left\langle X_u, X_v \right\rangle$ es idénticamente cero en $U$, es decir, si las curvas coordenadas de $X$ se cortan ortogonalmente en cualquier punto de $X(U)$.
\end{definition}

\begin{remark}
    En una parametrización ortogonal $F = 0$ en $X(U)$, es decir, la matriz de $I_p$ en la base $\{X_u, X_v\}_q$ de $T_pS$ es diagonal.
    $$I_p \equiv
        \begin{pmatrix}
            E & 0 \\
            0 & G
        \end{pmatrix}$$
    Esto permite obtener la siguiente expresión de la curvatura de Gauss.
    $$K = -\frac{1}{2\sqrt{EG}} \left( \left( \frac{E_v}{\sqrt{EG}}\right)_v + \left( \frac{G_u}{\sqrt{EG}}\right)_u \right) $$
\end{remark}

\begin{theorem}
    Dado un punto $p$ cualquiera de una superficie regular $S$ existe una parametrización ortogonal de $S$ en $p$.
\end{theorem}

\begin{definition}[Parametrización por líneas de curvatura]
    Una parametrización de una superficie regular $S$, $X: U \to S$, es una parametrización por líneas de curvatura si $F = f = 0$, es decir, si las curvas coordenadas son las líneas de curvatura.
\end{definition}

\begin{remark}
    En este caso, respecto a $\{X_u, X_v\}$, las matrices de $I_p$ y $\amalg_p$ son diagonales.
    $$I_p \equiv
        \begin{pmatrix}
            E & 0 \\
            0 & F
        \end{pmatrix}, \quad
        \amalg_p \equiv
        \begin{pmatrix}
            e & 0 \\
            0 & g
        \end{pmatrix}$$
    Luego, calculando $dN_p$, obtenemos que $\frac{e}{E}$ y $\frac{g}{G}$ son las curvaturas principales.
\end{remark}

\begin{proposition}
    Las curvas coordenadas de una parametrización en un entorno sin puntos umbilicales son las líneas de curvatura si y solo si $f = F = 0$ en todos los puntos del entorno.
\end{proposition}

\begin{corollary}
    Las curvas coordenadas de una parametrización en un entorno de un punto no umbilical son las líneas de curvatura si y solo si $F = f = 0$ en el entorno.
\end{corollary}

\begin{theorem}
    Sea $p$ un punto no umbilical de una superficie regular $S$.
    Entonces existe una parametrización por líneas de curvatura en un entorno de $p$.
\end{theorem}

\chapter{Desplazamiento paralelo}
\section{Derivada covariante}

\begin{definition}
    Una curva parametrizada $\alpha : [0, l] \to S$ es la restricción a $[0, l]$ de una aplicación diferenciable de $(-\varepsilon, l+\varepsilon)$, $\varepsilon>0$, en $S$.
    Si $\alpha(0) = p$ y $\alpha(l) = q$ se dice que $\alpha$ une $p$ con $q$.
\end{definition}

\begin{definition}
    Un campo vectorial tangente a $S$ a lo largo de una curva parametrizada $\alpha : [0, l] \to S$ es una correspondencia $w$ que asigna a cada $t \in [0, l]$ un vector $w(t) \in T_{\alpha(t)}S$.
    El campo vectorial $w$ se dice diferenciable en $t_0 \in [0, l]$ si para alguna parametrización $X(u, v)$ en $\alpha(t_0)$ se tiene
    $$w(\alpha(t)) \equiv w(t) = a(t)X_u + b(t)X_v$$
    con $a$ y $b$ funciones diferenciables en $t_0$.\\
    $w$ es diferenciable en $[0, l]$ si es diferenciable para todo $t \in I$.
\end{definition}

\begin{example}
    El campo $\alpha'(t)$ de vectores tangentes a $\alpha$ es un campo vectorial diferenciable a lo largo de $\alpha$.
\end{example}

\begin{definition}
    Sea $w$ un campo vectorial diferenciable a lo largo de $\alpha : [0, l] \to S$.
    Se denomina derivada covariante de $w$ en $t$ para $t \in [0, l]$ y se representa $\frac{Dw}{dt}(t)$ a la proyección ortogonal del vector $\frac{dw}{dt}(t)$ sobre $T_{\alpha(t)}S$.
\end{definition}

\begin{remark}
    Si dos superficies $S$ y $\bar{S}$ son tangentes a lo largo de una curva parametrizada $\alpha$, como los planos tangentes coinciden la derivada covariante de cualquier campo $w$ a lo largo de $\alpha$ es la misma para $S$ y $\bar{S}$.
\end{remark}

\begin{definition}
    Se dice que un campo vectorial $w$ a lo largo de una curva parametrizada es paralelo si la derivada covariante es cero en todos los puntos.
    $$\frac{Dw}{dt} = 0, \quad \forall t \in [0, l]$$
    Equivalentemente, un campo es paralelo si y solo si $\frac{dw}{dt}$ es normal a $S$ en $\alpha(t)$ para todo $t$.
\end{definition}

\begin{proposition}
    Sean $w_1$ y $w_2$ dos campos vectoriales paralelos a lo largo de $\alpha : [0, l] \to S$.
    Entonces $\left\langle w_1(t), w_2(t) \right\rangle$ es constante.
    En particular, $|w_1(t)|$ y $|w_2(t)|$ son constantes y el ángulo entre $w_1(t)$ y $w_2(t)$ es constante.
\end{proposition}

\begin{note}
    Un campo vectorial $w$ a lo largo de $\alpha$ es paralelo si y solo si $\frac{dw}{dt}$ es proporcional al normal a $S$, es decir, $\frac{dw}{dt} = \lambda(t) N(\alpha(t))$.
\end{note}

\begin{proposition}
    Sean $\alpha : [0, l] \to S$ una curva parametrizada, $w$ un campo de vectores diferenciable a lo largo de $\alpha$, $t_0 \in [0, l]$ y $X(u, v)$ una parametrización de $S$ en $\alpha(t_0)$.
    Existe $\varepsilon>0$ tal que $\alpha(t_0-\varepsilon, t_0+\varepsilon) \subset X(U)$ por ser $X(U)$ abierto y $\alpha$ continua. Supongamos
    $$\alpha(t) = X(u(t), v(t))$$ y $$w(t) \equiv w(\alpha(t)) = a(t)X_u + b(t)X_v, \quad a, b \text{ diferenciables}$$
    Utilizando las ecuaciones de Gauss se tiene que la parte tangente de $\frac{dw}{dt}$ es
    \begin{align*}
        \frac{Dw}{dt} & = (a' + au'\Gamma^1_{11} + (av' + bu')\Gamma^1_{12} + bv'\Gamma^1_{22}) X_u + \\
                      & + (b' + au'\Gamma^2_{11} + (av' + bu')\Gamma^2_{12} + bv'\Gamma^2_{22}) X_v
    \end{align*}
\end{proposition}

\begin{proposition}
    Sea $\alpha : [0, l] \to S$ una curva parametrizada sobre $S$ y sea $w_0 \in T_{\alpha(t_0)}S, t_0 \in [0, l]$.
    Entonces existe un único campo vectorial $w(t)$ paralelo a lo largo de $\alpha$ con $w(t_0) = w_0$.
\end{proposition}

\section{Desplazamiento paralelo}

\begin{definition}
    Sea $\alpha : [0, l] \to S$ una curva parametrizada y sea $w_0 \in T_{\alpha(t_0)}S$, $t_0 \in [0, l]$.
    Sea $w$ el único campo vectorial paralelo a lo largo de $\alpha$ con $w(t_0) = w_0$.
    El vector $w(t_1) \in T_{\alpha(t_1)}S$, $t_1 \in [0, l]$, es el desplazamiento paralelo de $w_0$ a lo largo de $\alpha$ en el punto $\alpha(t_1)$.
\end{definition}

\begin{properties}
    \hfill
    \begin{enumerate}
        \item Si $\alpha : [0, l] \to S$ es regular, entonces el desplazamiento paralelo no depende de la parametrización de $\alpha([0, l])$.
        \item Dados $p, q \in S$ y una curva parametrizada $\alpha: [0, l] \to S$ con $\alpha(0) = p$ y $\alpha(l) = q$, se puede definir
              $$P_\alpha : T_pS \to T_qS$$
              como la aplicación que asigna a cada $w_0 \in T_pS$ el transporte paralelo de $w_0$ a lo largo de $\alpha$ en $q = \alpha(l)$.
              Está bien definido por la unicidad de $w$ y es una isometría.
        \item Si dos superficies $S$ y $\bar{S}$ son tangentes a lo largo de una curva parametrizada $\alpha$, como la derivada covariante a lo largo de $\alpha$ es igual para ambas superficies, entonces dado $w_0 \in T_{\alpha(t_0)}S$ el campo vectorial paralelo $w(t)$ a lo largo de $\alpha$ tal que $w(t_0) = w_0$ es paralelo en las dos superficies.
              Por tanto, por unicidad del desplazamiento paralelo, el desplazamiento paralelo de $w_0$ a lo largo de $\alpha$ es el mismo con respecto a $S$ y a $\bar{S}$.
    \end{enumerate}
\end{properties}

\begin{definition}
    Una aplicación $\alpha : [0, l] \to S$ es una curva parametrizada regular a trozos si
    \begin{enumerate}
        \item $\alpha$ es continua.
        \item Existe un subdivisión de $[0, l]$
              $$0 = t_0 < t_1 < \dots < t_k < t_{k+1} = l$$
              tal que la restricción de $\alpha$ a $[t_i, t_{i+1}]$ es una curva parametrizada regular.
              Llamaremos a cada $\alpha |_{[t_i, t_{i+1}]}$ un arco regular de $\alpha$.
    \end{enumerate}
\end{definition}

\begin{note}
    Si $s_0 \in [t_i, t_{i+1}]$, se puede definir su desplazamiento paralelo a lo largo del arco regular $\alpha |_{[t_i, t_{i+1}]}$ de la forma habitual y se toma $w(t_{i+1})$ como valor inicial para el desplazamiento paralelo en $[t_{i+1}, t_{i+2}]$, y así sucesivamente hasta definirlo sobre $\alpha$.
\end{note}

\section{Geodésicas}

\begin{definition}
    Se dice que una curva parametrizada no constante $\gamma : [0, l] \to S$ es geodésica parametrizada si su campo de vectores tangente $\gamma'(t)$ es paralelo, es decir,
    $$\frac{D\gamma'}{dt}(t) = 0, \quad \forall t \in [0, l]$$
\end{definition}

\begin{remark}
    Puesto que estamos con curvas parametrizadas no regulares, una geodésica parametrizada puede presentar autointersecciones.
    Sin embargo, su campo tangente nunca se anula y, en consecuencia, la parametrización es regular.
\end{remark}

\begin{definition}
    Una curva regular conexa $C$ en $S$ es una geodésica si, para cada $p \in C$, la parametrización $\alpha(s)$ de un entorno coordenado de $p$ en $C$ por la longitud de arco $s$ es una geodésica parametrizada, es decir, $\alpha'(s)$ es un campo paralelo a lo largo de $\alpha$.
\end{definition}

\begin{remark}
    \hfill
    \begin{enumerate}
        \item Cada recta contenida en una superficie es una geodésica por ser $\alpha''(s) = 0$.
        \item Si no es una recta, se pueden caracterizar las geodésicas.\\
              Una curva regular conexa $C$ en $S$ con curvatura $K$ no nula es una geodésica si y solo si el normal a $C$ en cada $p \in C$ y el normal a $S$ en $p$ son proporcionales.
    \end{enumerate}
\end{remark}

\begin{proposition}[Ecuaciones diferenciales de las geodésicas]
    Sea $\gamma : [0, l] \to S$ una curva parametrizada de $S$ y sea $X(u, v)$ una parametrización de $S$ en un entorno abierto $X(U)$ de $\gamma(t_0), t_0 \in [0, l]$.
    Como $X(U)$ es abierto y $\gamma$ es continua, existe un intervalo abierto $J \in [0, l]$ con $t_0 \in J$ tal que $\gamma(J) \subset X(U)$ y $$\gamma(t) = X(u(t), v(t)), \quad t \in J$$
    $\gamma$ es una geodésica si y solo si se satisface el sistema de ecuaciones diferenciales
    $$\left\{ \begin{array}{lcl}
            u'' + \Gamma^1_{11}(u')^2 + 2\Gamma^1_{12}u'v' + \Gamma^1_{22}(v')^2 = 0 \\
            v'' + \Gamma^2_{11}(u')^2 + 2\Gamma^2_{12}u'v' + \Gamma^2_{22}(v')^2 = 0
        \end{array}
        \right.$$
    para cada intervalo $J \subset [0, l]$ tal que $\gamma(J)$ esté contenido en un entorno coordenado.
\end{proposition}

\begin{proposition}
    Sea $S$ superficie regular.
    Dado $p \in S$ y $w_0 \in T_pS$, $w_0 \neq 0$, existe $\varepsilon > 0$ y una única geodésica parametrizada $\gamma : (-\varepsilon, \varepsilon) \to S$ tal que $\gamma(0) = p$ y $\gamma'(0) = w_0$.
\end{proposition}

\begin{remark}
    Para cada punto y para cada dirección tangente hay una geodésica.
    Consideramos $w_0 \neq 0$ porque para $w_0 = 0$ la solución de la ecuación es una curva constante, que no es geodésica por definición.
    Si la solución que obtenemos es $(u(t), v(t))$, la geodésica es $X(u(t), v(t))$.
    Esta no depende de la parametrización. Si para otra parametrización se obtuviese otra geodésica, al escribirla en $X$ tendría que verificarse el sistema, luego por unicidad es la misma.
\end{remark}

\begin{remark}
    El módulo de $w$ es el de $w_0$ en todos los puntos, es decir, el módulo de la condición inicial se mantiene a lo largo de la geodésica porque las ecuaciones son las del transporte paralelo, que mantiene el módulo.
\end{remark}

\begin{remark}
    Una geodésica se puede autointersecar pero, por unicidad de la geodésica que pasa por un punto en una dirección, dos geodésicas no pueden ser tangentes en un punto.
\end{remark}

\begin{corollary}
    Las isometrías locales llevan geodésicas en geodésicas.
\end{corollary}

\section{Curvatura geodésica}

\begin{definition}
    Se define el valor algebraico de la derivada covariante de un campo diferenciable $w$ de vectores unitarios en $t \in [0, l]$ como
    $$\left[ \frac{Dw}{dt} \right] = \left\langle \frac{Dw}{dt}, N \land w \right\rangle = \lambda(t), \quad t \in [0, l]$$
    Su signo depende de la orientación de $S$.
\end{definition}

\begin{definition}
    Sea $C$ una curva regular orientada contenida en una superficie orientada $S$ y sea $\alpha(s)$ una parametrización de $C$ por la longitud de arco $s$ en un entorno de $p$ en $C$.
    La curvatura geodésica de $C$ en $p$ es el valor algebraico de la derivada covariante de $\alpha'(s)$ en $p$.
    $$k_g(s) = \left[ \frac{D\alpha'}{ds}(s) \right] = \left\langle \frac{D\alpha'}{ds}(s), N(s) \land \alpha'(s) \right\rangle = \left\langle \frac{d\alpha'}{ds}(s), N(s) \land \alpha'(s) \right\rangle$$
\end{definition}

\begin{remark}
    \hfill
    \begin{enumerate}
        \item El signo de $k_g(s)$ depende de la orientación de $C$ y de la orientación de $S$.
        \item Las geodésicas se caracterizan por tener curvatura geodésica cero.
              $$\frac{D\alpha'}{ds} = k_g(N \land \alpha') = 0 \Leftrightarrow k_g = 0$$
        \item Para una curva $\alpha : [0, l] \to S$ parametrizada por el arco $s$ tenemos definida la curvatura normal
              $$k_n = k \left\langle n, N \right\rangle = \left\langle a'', N \right\rangle$$
              y la curvatura geodésica
              $$k_g = \left\langle \alpha'', N \land \alpha' \right\rangle$$
              Considerando en cada $\alpha(s)$ la base ortonormal $\left\{ \alpha', N \land \alpha', N \right\}$
              \begin{align*}
                   & \left\langle \alpha', \alpha' \right\rangle = 1 \Rightarrow \left\langle \alpha'', \alpha' \right\rangle = 0 \Rightarrow                      \\
                   & \Rightarrow \alpha''(s) = \left\langle \alpha'', N \land \alpha' \right\rangle (N \land \alpha') + \left\langle \alpha'', N \right\rangle N = \\
                   & = k_g(s) (N \land \alpha'(s)) + k_n(s)N
              \end{align*}
              Por otra parte, $\alpha''(s) = k(s)n(s)$.
              Puesto que $n$, $N$ y $N \land \alpha'$ son unitarios, igualando módulo al cuadrado en estas dos expresiones
              $$k^2 = k_g^2 + k_n^2$$
        \item Cuando dos superficies son tangentes a lo largo de una curva regular $C$, el valor absoluto de la curvatura geodésica de $C$ es el mismo respecto a cualquiera de las dos superficies.
        \item En el plano $k_n = 0$ en cualquier dirección, así que $k_g^2(s) = k^2(s)$.
    \end{enumerate}
\end{remark}

\begin{proposition}
    Sea $X(u, v)$ una parametrización ortogonal $(F=0)$ de un entorno de una superficie orientada $S$ compatible con la orientación y sea $w(t)$ un campo diferenciable de vectores unitarios a lo largo de la curva $X(u(t), v(t))$. Entonces
    $$\left[ \frac{Dw}{dt} \right] = \frac{1}{2\sqrt{EG}} \left( G_u \frac{dv}{dt} - E_v \frac{du}{dt} \right) + \frac{d\varphi}{dt}$$
    donde $\varphi(t)$ es el ángulo de $\frac{X_u}{\sqrt{E}}$ a $w$ en la orientación de $S$.
\end{proposition}

\begin{theorem}[Teorema de Liouville]
    Sea $\alpha : [0, l] \to S$ una curva parametrizada por la longitud de arco $s$ en una superficie orientada $S$.
    Sea $X(u, v)$ una parametrización ortogonal de $S$ en $\alpha(s_0)$ compatible con la orientación de $S$ y sea $\varphi(s)$ el ángulo que forma $X_u$ con $\alpha'(s)$ en la orientación de $S$. Entonces
    $$k_g = {(k_g)}_1\cos\varphi + {(k_g)}_2\sin\varphi + \frac{d\varphi}{ds}$$
    donde ${(k_g)}_1$ y ${(k_g)}_2$ son las curvaturas geodésicas de las curvas coordenadas $v = cte$ y $u = cte$, respectivamente.
    $${(k_g)}_1 = -\frac{E_v}{2E\sqrt{G}}, \quad {(k_g)}_2 = \frac{G_u}{2G\sqrt{E}}$$
\end{theorem}

\begin{corollary}
    Si las curvas coordenadas de una parametrización ortogonal de una superficie $S$ son geodésicas, entonces las geodésicas del entorno son las curvas que forman un ángulo constante con las curvas coordenadas.
\end{corollary}

\chapter{Aplicación exponencial}

\section{Aplicación exponencial}

\begin{definition}
    Dada una superficie regular $S$, un punto $p \in S$ y un vector no nulo $w \in T_pS$ existe una única geodésica parametrizada
    $$\gamma : (-\varepsilon, \varepsilon) \to S$$
    con $\gamma(0) = p$ y $\gamma'(0) = w$.
    Representaremos esta geodésica como $\gamma(t,p,w)$, o $\gamma(t,w)$ si hemos fijado $p$, para indicar la dependencia respecto a $w$.
\end{definition}

\begin{lemma}[Lema de homogeneidad de las geodésicas]
    Si la geodésica $\gamma(t,w)$ está definida en $(-\varepsilon, \varepsilon)$, entonces la geodésica $\gamma(t, \lambda w)$, $\lambda \neq 0$, está definida en $(\frac{\varepsilon}{|\lambda|}, \frac{\varepsilon}{|\lambda|})$ y $\gamma(t, \lambda w) = \gamma(\lambda t, w)$.
\end{lemma}

\begin{remark}
    No son curvas parametrizadas iguales si $\lambda \neq 1$, aunque sí es igual su traza.
    En ambos casos, al ser geodésicas, la traza se recorre con velocidad constante.
\end{remark}

\begin{definition}
    Sea $w \in T_pS \setminus \{0\}$ tal que $\gamma(|w|, \frac{w}{|w|}) = \gamma(1, w)$ está definido.
    Entonces la exponencial en $p$ de $w$ está dada por
    $$exp_p(w) = \gamma(1, w), \quad exp_p(0) = p$$
\end{definition}

\begin{proposition}
    Dado $p \in S$ existe $\varepsilon>0$ tal que $exp_p$ está definida y es diferenciable en un disco abierto de radio $\varepsilon$ centrado en el origen de $T_pS$
    $$B_\varepsilon = \{ w \in T_pS : |w| < \varepsilon \}$$
\end{proposition}

\begin{proposition}
    La aplicación $exp_p : B_\varepsilon \subset T_pS \to S$ es un difeomorfismo en un entorno abierto $U \subset B_\varepsilon$ del origen de $T_pS$.

    \begin{proof}
        Vamos a probar que la diferencial de $exp_p$ en $0 \in T_pS$ es no singular y a aplicar el teorema de la función inversa.
        $$d(exp_p)_0: T_0(T_pS) \to T_pS$$
        Podemos identificar $T_0(T_pS) \equiv T_pS$, puesto que un vector tangente en 0 a $T_pS$ es un vector de $T_pS$.
        Veamos que $d(exp_p)_0 = Id_{T_pS}$.
        La curva $\alpha(t) = tw, w \in T_pS$, verifica $\alpha(0) = 0 \in T_pS, \alpha'(0) = w$.
        Por tanto,
        \begin{align*}
            d(exp_p)_0(w) & = \frac{d}{dt} (exp_p \circ \alpha)(t)|_{t=0} = \frac{d}{dt} (exp_p(tw))|_{t=0} =          \\
                          & = \frac{d}{dt} \gamma(1, tw)|_{t=0} = \frac{d}{dt} \gamma(t, w)|_{t=0} = \gamma'(0, w) = w
        \end{align*}
        Luego $d(exp_p)_0 = Id$, así que la diferencial de $exp_p$ es no singular.
        Por el teorema de la función inversa, existe un entorno abierto de 0, $U \subset B_\varepsilon \subset T_pS$, tal que
        $$exp_p|_U: U \subset T_pS \to exp_p(U) \subset S$$
        es un difeomorfismo.
    \end{proof}
\end{proposition}



\begin{definition}
    Se dice que $V \subset S$ es un entorno normal de $p$ si $V = exp_p(U)$ para un entorno abierto $U$ de $0 \in T_pS$ tal que $exp_p : U \to V$ es un difeomorfismo.
\end{definition}

\section{Sistemas de coordenadas}

\begin{definition}
    Las coordenadas normales se obtienen al elegir en el plano $T_pS$, $p \in S$, dos vectores ortogonales unitarios, es decir, una base ortonormal $\{e_1, e_2\} \subset T_pS$.\\
    Como $exp_p : U \to V \subset S$ es un difeomorfismo y $U$ es abierto, $exp_p$ es una parametrización en $p$.
    Si $q \in V = exp_p(U)$, entonces existe un único $w \in U$ tal que $q = exp_p(w)$.
    Si $w = ue_1 + ve_2 \in U$, se dice que $q$ tiene coordenadas normales $(u, v)$.
    Para cada base ortonormal de $T_pS$ se tienen unas coordenadas normales en $V$.
    $$X(u, v) = exp_p(ue_1 + ve_2) = q$$
\end{definition}

\begin{properties}
    \hfill
    \begin{enumerate}
        \item En un entorno normal $V$ centrado en $p$ las geodésicas que pasan por $p$ son imagen por $exp_p$ de rectas vectoriales en $U$, con $V = exp_p(U)$. Se llaman geodésicas radiales.
        \item Como todas las geodésicas que pasan por $p$ son geodésicas radiales, entonces para todo $q \in V$ la geodésica que une $p$ con $q$ contenida en $V$ es única.
        \item Los coeficientes de la primera forma fundamental en $p$ de $X(u, v) = exp_p(ue_1 + ve_2)$ son $$E(p) = G(p) = 1, \quad F(p) = 0$$
              Esto es solo cierto en $p$. De serlo en todo punto de $V$, $exp_p$ sería una isometría y la superficie sería localmente isométrica a un plano.
    \end{enumerate}
\end{properties}

\begin{definition}
    Sea $(\rho, \theta)$ un sistema de coordenadas polares en el plano $T_pS$, donde $\rho>0$ es el radio y $\theta \in (0, 2\pi)$ es el ángulo con respecto a una semirrecta cerrada $l$ con extremo en $0 \in T_pS$.
    Supongamos $exp_p(l) = L$. Entonces $$exp_p : U-l \to V-L$$ sigue siendo un difeomorfismo y se puede parametrizar $V-L$ con las coordenadas $(\rho, \theta)$, que se llaman coordenadas geodésicas polares.
    No están definidas en $p$ porque no lo están en $0 \in T_pS$.
    Si $\{e_1, e_2\}$ es una base ortonormal de $T_pS$ tal que $l = \{ \tilde{\rho}e_1 \in T_pS : \tilde{\rho} \geq 0 \}$, se tiene $$X(\rho, \theta) = exp_p(\rho \cos(\theta)e_1 + \rho \sin(\theta)e_2)$$
\end{definition}

\begin{proposition}
    Sea $X: U-l \to V-L$ una parametrización de $S$ por coordenadas geodésicas polares $(\rho, \theta)$, $\rho>0$, $\theta \in (0, 2\pi)$.
    Entonces los coeficientes de la primera forma fundamental verifican
    \begin{align*}
         & E(\rho, \theta) = 1,                          & F(\rho, \theta) = 0                          \\
         & \lim\limits_{\rho \to 0} G(\rho, \theta) = 0, & \lim\limits_{\rho \to 0} (\sqrt{G})_\rho = 1
    \end{align*}
\end{proposition}

\begin{remark}
    $E$, $F$ y $G$ no están definidas en $p$.
\end{remark}

\section{Teorema de Minding}

\begin{theorem}[Teorema de Minding]
    Dos superficies cualesquiera con la misma curvatura de Gauss constante son localmente isométricas.\\
    Concretamente, si $S_1$ y $S_2$ son dos superficies regulares con la misma curvatura de Gauss $K$ constante, dados $p_1 \in S_1$, $p_2 \in S_2$ y bases ortonormales $\{e_1,e_2\}$ de $T_{p_1}S_1$, $\{f_1,f_2\}$ de $T_{p_2}S_2$ existen entornos abiertos $V_1$ de $p_1$, $V_2$ de $p_2$ y una isometría $\psi : V_1 \to V_2$ tal que $d\psi_{p_1}(e_1) = f_1$, $d\psi_{p_1}(e_2) = f_2$.

    \begin{proof}
        Sean $V_1 = exp_{p_1}(U_1)$ y $V_2 = exp_{p_2}(U_2)$ entornos normales de $p_1$ y $p_2$ respectivamente en $S_1$ y $S_2$.
        Sea $\varphi: T_{p_1}S_1 \to T_{p_2}S_2$ la isometría lineal definida por $\varphi(e_1) = f_1, \varphi(e_2) = f_2$.
        Restringiendo si fuera necesario, podemos suponer que $\varphi(U_1) = U_2$.
        Definimos: $$\psi: V_1 \to V_2, \quad \psi = exp_{p_2} \circ \varphi \circ exp_{p_1}^{-1}$$
        $\varphi$ es isomorfismo lineal, luego es un difeomorfismo.
        Como estamos en entornos normales, $exp_{p_1}$ y $exp_{p_2}$ son difeomorfismos, así que $\psi$ es difeomorfismo por composición.\\
        Veamos que $\psi$ es isometría.
        Consideramos en $U_1$ coordenadas polares $(\rho, \theta)$ con eje $l = \{ ue_1 : u \geq 0 \}$ para la orientación $\{ e_1, e_2 \}$.
        Si $L_1 = exp_{p_1}(l)$, en $V_1 - L_1$ tenemos un sistema de coordenadas geodésicas polares centrado en $p_1$.
        Entonces $(\rho, \theta)$ son coordenadas geodésicas polares en $\psi(V_1 - L_1)$ puesto que
        \begin{align*}
            \psi(exp_{p_1}(\rho \cos(\theta) e_1 + \rho \sin(\theta) e_2)) & = exp_{p_2} (\varphi(\rho \cos(\theta) e_1 + \rho \sin(\theta) e_2)) = \\
                                                                           & = exp_{p_2}(\rho \cos(\theta) f_1 + \rho \sin(\theta) f_2)
        \end{align*}
        Así que $\psi(V_1 - L_1)$ es un entorno coordenado polar centrado en $p_2$.
        Como $S_1$ y $S_2$ tienen la misma curvatura de Gauss constante y los coeficientes de la primera forma fundamental son iguales en los puntos correspondientes de $V_1 - L_1$ y $\psi(V_1 - L_1)$, entonces $\psi|_{V_1 - L_1}$ es isometría.\\
        Además, $\psi(p_1) = exp_{p_2}\varphi(0) = exp_{p_2}(0) = p_2$ y $\psi(V_1) = V_2$, así que por continuidad $\psi$ es isometría de $V_1$ en $V_2$.
        Además, $$d\psi_{p_1} = (dexp_{p_2})_{\psi(0)} \circ (d\varphi)_0 \circ (dexp_{p_1})_0^{-1}$$
        Como $(dexp_p)_0 = Id$ y $d\varphi = \varphi$ por ser $\varphi$ lineal, entonces $d\psi_{p_1} = \varphi$ luego $d\psi_{p_1}(e_i) = \varphi(e_i) = f_i, i = 1, 2$.
    \end{proof}
\end{theorem}

\begin{corollary}
    Sean $S_1$ y $S_2$ superficies regulares con curvatura de Gauss $K_1$ y $K_2$ respectivamente.
    Si $K_1$ y $K_2$ son constantes, entonces son equivalentes:
    \begin{enumerate}
        \item $S_1$ y $S_2$ son localmente isométricas.
        \item $K_1 = K_2$.
    \end{enumerate}
    Así, una superficie con curvatura de Gauss constante $K$ es localmente isométrica a una de las siguientes superficies:
    \begin{itemize}
        \item Si $K>0$, a una esfera de radio $r$ tal que $K = \frac{1}{r^2}$.
        \item Si $K=0$, a un plano.
        \item Si $K<0$, a una pseudoesfera de pseudorradio $r$ tal que $K = -\frac{1}{r^2}$.
    \end{itemize}
\end{corollary}

\begin{note}
    La pseudoesfera es la superficie de revolución obtenida cuando la tractriz gira alrededor del eje $z$.
\end{note}

\section{Propiedades variacionales de las geodésicas}

\begin{definition}
    Para cada punto $p \in S$ de una superficie regular existe un $\delta>0$ tal que la aplicación $$exp_p : B_\delta(0) \subset T_pS \to S$$ es un difeomorfismo sobre su imagen.
    A esta imagen, que representaremos por $B_\delta(p)$, se le llama bola geodésica de centro $p$ y radio $\delta$.
    Es el interior de un círculo geodésico.
\end{definition}

\begin{proposition}
    Sea $B_\delta(p)$ una bola geodésica de radio $\delta$ con centro en un punto $p$ de una superficie $S$.
    Si $\alpha : [0, l] \to B_\delta(p) \subset S$ es una curva parametrizada regular con $\alpha(0) = p$, entonces
    $$L^l_0(\alpha) \geq L^l_0(\gamma)$$
    donde $\gamma$ es la única geodésica radial parametrizada por el arco definida en $[0, l]$ que une $p$ con $\alpha(l)$.
    Si se da la igualdad, entonces las trazas de $\alpha$ y de $\gamma$ coinciden.
\end{proposition}

\begin{corollary}
    Para cada punto $p$ de una superficie regular $S$ existe un entorno abierto $W$ de $p$ en $S$ tal que si $\gamma : [0, l] \to W$ es una geodésica parametrizada con $\gamma(0) = p$ y $\gamma(t_1) = q$, $t_1 \in [0, l]$, entonces para cualquier curva parametrizada regular $\alpha : [0, t_1] \to S$ uniendo $p$ con $q$ se tiene
    $$L^{t_1}_0(\alpha) \geq L^{t_1}_0(\gamma)$$
    Además, $L^{t_1}_0(\alpha) = L^{t_1}_0(\gamma)$ si y solo si $\alpha$ y $\gamma$ tienen la misma traza.
\end{corollary}

\begin{proposition}
    Sea $\alpha : I \to S$ una curva parametrizada regular cuyo parámetro es proporcional a la longitud de arco.
    Si la longitud de arco entre cualquier par de puntos $t_0, t_1 \in I$ es menor o igual que la longitud de arco de cualquier curva parametrizada que una $\alpha(t_0)$ con $\alpha(t_1)$, entonces $\alpha$ es una geodésica parametrizada.
\end{proposition}

\begin{proposition}
    Dado $p \in S$, existe un entorno abierto $W$ de $p$ en $S$ y un número $\delta>0$ tal que si $q \in W$, entonces $exp_q$ es un difeomorfismo en $B_\delta(0) \subset T_qS$ y $W \subset exp_q(B_\delta(0))$, esto es, $W$ es entorno normal de todos los puntos.\\
    $W$ se llama un entorno totalmente normal con $\delta>0$ asociado.
\end{proposition}

\begin{corollary}
    Dados dos puntos $q_1, q_2 \in W$ existe una única geodésica minimizante que es de longitud menor que $\delta$ y que une $q_1$ y $q_2$.
\end{corollary}

\begin{definition}
    Se dice que una geodésica parametrizada $\gamma$ que une los puntos $p, q \in S$ es minimal o minimizante si su longitud es menor o igual que la de cualquier curva parametrizada diferenciable a trozos que une $p$ y $q$.
\end{definition}

\chapter{Geometría diferencial global}
\section{Superficies congruentes}

\begin{definition}
    Dos superficies $S$ y $\bar{S}$ se dicen congruentes si existe un movimiento rígido de $\mathbb{R}^3$, $F : \mathbb{R}^3 \to \mathbb{R}^3$, tal que $F(S) = \bar{S}$.
    En particular, $F|_S : S \to \bar{S}$ es una isometría, luego dos superficies congruentes son isométricas.
\end{definition}

\begin{definition}
    Se dice que una superficie es rígida si cualquier superficie isométrica a ella es congruente con ella. Es decir, que salvo movimiento rígido la única superficie isométrica a $S$ es $S$.
\end{definition}

\section{Hessiano de una función}

\begin{definition}
    Sea $S$ una superficie y $f : S \to \mathbb{R}$ una función diferenciable.
    Si $p \in S$ es un punto crítico de $f$, se define el hessiano de $f$ en $p$ de la siguiente forma.\\
    Dado $w \in T_pS$, si $\alpha : (-\varepsilon, \varepsilon) \to S$ es una curva diferenciable con $\alpha(0) = p$ y $\alpha'(0) = w$, entonces
    $$Hess_pf(w) = \frac{d^2(f \circ \alpha)}{dt^2}|_{t=0} = (f \circ \alpha)''(0)$$
    Así, $Hess_pf : T_pS \to \mathbb{R}$.
\end{definition}

\begin{proposition}
    \hfill
    \begin{enumerate}
        \item El hessiano está bien definido, es decir, no depende de la curva elegida.
        \item $Hess_pf$ es una forma cuadrática sobre $T_pS$.
    \end{enumerate}
\end{proposition}

\begin{proposition}
    Sea $f : S \to \mathbb{R}$, con $p$ crítico para $f$.
    \begin{enumerate}
        \item \begin{itemize}
                  \item Si $Hess_pf$ es definida negativa, entonces $p$ es máximo local estricto de $f$.
                  \item Si $Hess_pf$ es definida positiva, entonces $p$ es mínimo local estricto de $f$.
              \end{itemize}
        \item Si $p$ es extremo local de $f$, entonces $Hess_pf$ es semidefinida.
              \begin{itemize}
                  \item Si $p$ es máximo local, entonces $Hess_pf$ es semidefinida negativa.
                  \item Si $p$ es mínimo local, entonces $Hess_pf$ es semidefinida positiva.
              \end{itemize}
    \end{enumerate}
\end{proposition}

\begin{proposition}
    Sea $S$ superficie regular orientable, $p \in S$.
    El punto $p$ es elíptico si y solo si $p$ es máximo local estricto de una función distancia al cuadrado desde un $p_0 \in \mathbb{R}^3$, con $p \neq p_0$.
\end{proposition}

\begin{corollary}
    Toda superficie compacta orientable tiene un punto elíptico.
\end{corollary}

\begin{corollary}
    No hay superficies minimales compactas.
\end{corollary}

\begin{lemma}[Lema de Hilbert]
    Sea $S$ superficie regular y $p \in S$. Supongamos
    \begin{itemize}
        \item $p$ es un máximo local para la función $K_1$.
        \item $p$ es un mínimo local para la función $K_2$.
        \item $K_1(p) > K_2(p)$.
    \end{itemize}
    Entonces $K(p) \leq 0$.
\end{lemma}

\begin{note}
    Como $K_1 \neq K_2$, necesariamente $K_2 \leq 0$. Así que $p$ es hiperbólico o parabólico.
\end{note}

\begin{corollary}
    Sea $S$ una superficie regular y $p \in S$. Si $p$ verifica
    \begin{itemize}
        \item $p$ es elíptico, esto es, $K(p) > 0$.
        \item $p$ es máximo local para la función $K_1$.
        \item $p$ es mínimo local para la función $K_2$.
        \item $K_1 \geq K_2$.
    \end{itemize}
    Entonces $p$ es un punto umbilical de $S$.
\end{corollary}

\begin{note}
    Este corolario es equivalente al lema de Hilbert.
\end{note}

\section{Rigidez de la esfera}

\begin{theorem}
    Sea $S$ una superficie regular conexa y compacta con curvatura de Gauss $K = cte$.
    Entonces $S$ es una esfera.

    \begin{proof}
        Como $S$ es compacta, $S$ tiene al menos un punto elíptico y por tanto $K$ es una constante positiva.
        Por otra parte, $k_1$ y $k_2$ son funciones continuas sobre $S$ y $S$ es compacta, así que deben alcanzar máximo y mínimo.
        Sea $p \in S$ el punto donde $k_1(p)$ es máximo.
        Como $k_1 k_2 = K = cte > 0$, entonces $k_2$ alcanza el mínimo en $p$.
        Por el lema de Hilbert, $p$ es un punto umbilical.
        Veamos que todos los puntos de $S$ son umbilicales.
        Sea $q \in S$ arbitrario, $$k_1(p) \geq k_1(q) \geq k_2(q) \geq k_2(p) = k_1(p)$$
        Por tanto, $k_1 = k_2$ para todo $q \in S$, así que todos los puntos de $S$ son umbilicales.
        Como $S$ es conexa y $K>0$, por el teorema de clasificación de superficies umbilicales tenemos que $S$ es un abierto de una esfera de radio $r$, con $K = \frac{1}{r^2}$.
        Como además $S$ es compacta, entonces $S$ es cerrada en $S^2$.
        Por tanto, $S = S^2$.
    \end{proof}
\end{theorem}

\begin{theorem}[Teorema de rigidez de la esfera]
    Si $\varphi : S^2(r) \to S$ es una isometría de una esfera $S^2(r) \subset \mathbb{R}^3$ de radio $r$ sobre una superficie regular $S = \varphi(S^2(r)) \subset \mathbb{R}^3$, entonces $S$ es una esfera de radio $r$ y $\varphi$ es la restricción a $S^2(r)$ de un movimiento rígido de $\mathbb{R}^3$.
\end{theorem}

\section{Teorema de Hopf-Rinow}

\begin{definition}
    Una superficie regular $S$ es completa si, para cada punto $p \in S$, cualquier geodésica parametrizada $\gamma : [0, \varepsilon) \to S$ con $\gamma(0) = p$ se puede extender a una geodésica parametrizada definida en toda la recta real.
    Esto es, existe $\bar{\gamma} : \mathbb{R} \to S$ geodésica parametrizada con $\bar{\gamma}|_{[0, \varepsilon)} = \gamma$.
\end{definition}

\begin{remark}
    La completitud es equivalente a que la aplicación exponencial esté definida en todo $T_pS$.
\end{remark}

\begin{definition}
    Una aplicación continua $\alpha : [a, b] \to S$ es una curva parametrizada diferenciable a trozos si existe una partición de $[a, b]$
    $$a = t_0 < t_1 < \dots < t_{k+1} = b$$
    tal que $\alpha$ es diferenciable en $(t_i, t_{i+1})$.
    La longitud de $\alpha$ se define por $$l(\alpha) = \sum^k_{i=0} \int^{t_{i+1}}_{t_i} |\alpha'(t)|dt$$
\end{definition}

\begin{proposition}
    Sea $S$ una superficie regular conexa.
    Dados $p, q \in S$, existe una curva parametrizada diferenciable a trozos que une $p$ con $q$.
\end{proposition}

\begin{definition}
    Se define la distancia intrínseca de $p$ a $q$, con $p, q \in S$, como $$d(p, q) = \inf \{ l(\alpha) : \alpha \in D^\infty(p,q) \}$$ donde $D^\infty(p, q)$ es el conjunto de curvas parametrizadas diferenciables a trozos que unen $p$ con $q$.
\end{definition}

\begin{proposition}
    Sea $d_0$ la métrica inducida en $S \subset \mathbb{R}^3$, $d_0(p, q) = ||p-q||$.\\
    $d$ y $d_0$ son distancias equivalentes, esto es, inducen la misma topología en $S$.
\end{proposition}

\begin{proposition}
    Una superficie regular $S$ es completa si y solo si $(S, d)$ es espacio métrico completo, esto es, cada sucesión de Cauchy en $(S, d)$ converge a un punto de $S$.
\end{proposition}

\begin{definition}
    Se dice que una geodésica $\gamma$ que une los puntos $p, q \in S$ es minimal si su longitud es menor o igual que la de cualquier curva parametrizada diferenciable a trozos que una $p$ y $q$.
\end{definition}

\begin{theorem}[Teorema de Hopf-Rinow]
    Sea $S$ una superficie completa.
    Dados dos puntos $p, q \in S$, existe una geodésica minimal que une $p$ y $q$.
\end{theorem}

\section{Teorema de Gauss-Bonnet}

\begin{definition}
    Sea $\alpha: [0, l] \to S$, con $S$ superficie regular, una aplicación continua.
    Decimos que $\alpha$ es una curva parametrizada, regular a trozos, cerrada y simple si
    \begin{itemize}
        \item Existe una partición de $[0, l]$
              $$0 = t_0 < t_1 < \dots < t_k < t_{k+1} = l$$
              tal que $\alpha$ es diferenciable y regular en cada $[t_i, t_{i+1}]$.
        \item $\alpha(0) = \alpha(l)$
        \item Si $\bar{t} \neq \hat{t}, \bar{t}, \hat{t} \in [0, l)$, entonces $\alpha(\bar{t}) \neq \alpha(\hat{t})$.
    \end{itemize}
\end{definition}

\begin{definition}
    Sea $\alpha$ una curva parametrizada, regular a trozos, cerrada y simple.
    \begin{itemize}
        \item Los puntos $\alpha(t_i)$ se llaman los vértices de $\alpha$.
        \item Las trazas de $\alpha([t_i, t_{i+1}])$ se llaman los arcos regulares de $\alpha$.
        \item La traza de $\alpha$ se llama una curva cerrada regular a trozos.
    \end{itemize}
\end{definition}

\begin{definition}
    Por la condición de regularidad, por cada vértice existen el límite por la izquierda y por la derecha.
    Supongamos que $S$ está orientada y sea $|\theta_i|$, con $0 < |\theta_i| \leq \pi$ la menor determinación del ángulo de $\alpha'(t_i - 0)$ a $\alpha'(t_i + 0)$.
    \begin{itemize}
        \item Si $|\theta_i| \neq \pi$, damos a $\theta_i$ el signo del determinante $det(\alpha'(t_i-0), \alpha'(t_i+0), N)$.
              Observamos que el signo viene determinado por la orientación de $S$.
        \item Si $|\theta_i| = \pi$, por la regularidad se tiene que existe $\varepsilon'>0$ tal que $det(\alpha'(t_i-\varepsilon), \alpha'(t_i+\varepsilon), N)$ no cambio de signo para todo $0 < \varepsilon < \varepsilon'$.
              Damos a $\theta_i$ este signo.
    \end{itemize}
    El ángulo con signo $\theta_i$ es el ángulo externo en el vértice $\alpha(t_i)$.
\end{definition}

\begin{definition}
    Sea $S$ una superficie orientada, se dice que una región $R$ es una región simple si $R$ es homeomorfa a un disco y la frontera de $R$ es la traza de una curva parametrizada regular a trozos, cerrada y simple $\alpha: I \to S$.
\end{definition}

\begin{definition}
    Sea $X: U \subset \mathbb{R}^2 \to S$ una parametrización de $S$ compatible con su orientación y sea $R \subset X(u)$ una región acotada de $S$.
    La integral $$\int\int_{X^{-1}(R)} f(u, v) \sqrt{EG-F^2} dudv$$ donde $f$ es una función diferenciable sobre $S$ no depende de $X$.
    Se denomina la integral de $f$ sobre la región $R$ y se nota $\int\int_R f d\sigma$.
    En particular, $\int\int_R d\sigma = A(R) = \text{ área de } R$.
\end{definition}

\begin{theorem}[Teorema de Gauss-Bonnet]
    Sea $S$ una superficie orientada y sea $X: U \to S$ una parametrización de $S$ compatible con la orientación de $S$ y tal que $U$ sea homeomorfo a un disco abierto.
    Sea $R \subset X(U)$ una región simple de $S$ y sea $\alpha: I \to S$ tal que $Fr(R) = \alpha(I)$.
    Supongamos que $\alpha$ está orientada positivamente y parametrizada por la longitud de arco $s$.
    Sean $\alpha(s_0), \dots \alpha(s_k)$ y $\theta_0, \dots, \theta_k$ los vértices y los ángulos externos de $\alpha$ respectivamente.
    Entonces
    $$\sum_{i=0}^k \int_{s_i}^{s_{i+1}} k_g(s) ds + \int\int_R K d\sigma + \sum_{i=0}^k \theta_i = 2\pi$$
    donde $k_g(s)$ es la curvatura geodésica de los arcos regulares de $\alpha$ y $K$ es la curvatura de Gauss de $S$.
\end{theorem}

\end{document}