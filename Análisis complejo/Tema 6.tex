\chapter{Funciones enteras. Crecimiento y distribución de los ceros}
\begin{theorem}[Teorema de Liouville]
    Si $f$ es una función entera y acotada, entonces $f$ es constante.
\end{theorem}

\begin{theorem}[Generalización del teorema de Liouville]
    Sea $f$ entera tal que existen $\alpha > 0$, $c > 0$ y $R_0 > 0$ con $|f(z)| \leq c|z|^\alpha$ para todo $|z| \geq R_0$.
    Entonces $f$ es un polinomio de grado $E(\alpha)$ y por tanto tiene $E(\alpha)$ ceros.
\end{theorem}

\begin{proof}
    Sea $f(z) = \sum_{n=0}^\infty a_nz^n$.
    Sea $R > 0$,
    $$a_n = \frac{f^{(n)}(0)}{n!} = \frac{1}{2\pi i} \int_{|z|=R} \frac{f(z)}{z^{n+1}}dz \leq \max_{|z|=R} |f(z)|\frac{1}{R^n} \leq c\frac{|z|^\alpha}{R} = cR^{\alpha-n} \xrightarrow[R \to \infty, n > E(\alpha)]{} 0$$
    Por tanto, $a_n = 0$ para todo $n > E(\alpha)$.
\end{proof}

\begin{remark}
    \hfill
    \begin{enumerate}
        \item Parece que si el crecimiento está controlado, el número de ceros también lo está.
        \item Al revés esto no ocurre.
              Por ejemplo, con la exponencial.
    \end{enumerate}
\end{remark}

\section{La fórmula de Jensen}
La fórmula de Jensen permite controlar el número de ceros de una función holomorfa sabiendo restricciones sobre su crecimiento.

\begin{lemma}
    Sea $s > 0$.
    Entonces
    $$\frac{1}{2\pi} \int_{-\pi}^\pi \log|1+se^{i\theta}|d\theta = \log^+(s)$$
    donde
    $$\log^+(s) = \begin{cases}
            \log(s) & \text{si } s \geq 1  \\
            0       & \text{si } 0 < s < 1
        \end{cases}$$
\end{lemma}

\begin{proof}
    Sabemos que $\Log(1-z)$ es holomorfa en $\mathbb{D}$.
    \begin{itemize}
        \item Si $s < 1$, entonces por el teorema del valor medio
              $$\frac{1}{2\pi} \int_{-\pi}^\pi \Log|1-se^{i\theta}|d\theta = 0$$

        \item Si $s > 1$,
              \begin{align*}
                   & \frac{1}{2\pi} \int_{-\pi}^\pi \log|1-se^{i\theta}|d\theta = \frac{1}{2\pi} \int_{-\pi}^\pi \log\left|se^{i\theta}\left(\frac{1}{s}e^{-i\theta}-1\right)\right|d\theta =                                                                   \\
                   & = \frac{1}{2\pi} \int_{-\pi}^\pi \log(s)d\theta + \frac{1}{2\pi} \int_{-\pi}^\pi \log\left|1-\frac{1}{s}e^{-i\theta}\right|d\theta = \log(s) - \frac{1}{2\pi} \int_{-\pi}^\pi \log\left|1-\frac{1}{s}e^{i\varphi}\right|d\varphi = \log(s)
              \end{align*}

        \item Sea $s = 1$.
              $$|1-e^{i\theta}|^2 = (1-e^{i\theta})(1-e^{-i\theta}) = 2-2\cos(\theta)$$
              Sabemos que
              $$\cos(\theta) = \cos\left(2\frac{\theta}{2}\right) = \cos^2\left(\frac{\theta}{2}\right) - \sin^2\left(\frac{\theta}{2}\right) = 1-2\sin^2\left(\frac{\theta}{2}\right)$$
              Luego $|1-e^{i\theta}|^2 = 4\sin^2\left(\frac{\theta}{2}\right)$.
              \begin{align*}
                   & \frac{1}{2\pi} \int_{-\pi}^\pi \log|1-e^{i\theta}|d\theta = \frac{1}{2\pi} \int_{-\pi}^\pi \log\left(2\left|\sin\left(\frac{\theta}{2}\right)\right|\right)d\theta = \log(2) + \frac{1}{2\pi} \int_{-\pi}^\pi \log\left|\sin\left(\frac{\theta}{2}\right)\right|d\theta = \\
                   & = \log(2) + \frac{1}{\pi} \int_0^{\pi} \log\left(\sin\left(\frac{\theta}{2}\right)\right)d\theta = \log(2) + \frac{2}{\pi} \int_{0}^{\pi/2} \log(\sin(\varphi))d\varphi
              \end{align*}
              Ahora bien,
              \begin{align*}
                   & \int_0^{\pi/2} \log(\sin(\theta))d\theta = \int_0^{\pi/2} \log\left(2\sin\left(\frac{\theta}{2}\right)\cos\left(\frac{\theta}{2}\right)\right)d\theta =                                        \\
                   & = \frac{\pi}{2}\log(2) + \int_0^{\pi/2} \log\left(\sin\left(\frac{\theta}{2}\right)\right)d\theta + \int_0^{\pi/2} \log\left(\cos\left(\frac{\theta}{2}\right)\right)d\theta =                 \\
                   & = \frac{\pi}{2}\log(2) + \int_0^{\pi/2} \log\left(\sin\left(\frac{\theta}{2}\right)\right)d\theta + \int_0^{\pi/2} \log\left(\sin\left(\frac{\pi}{2} - \frac{\theta}{2}\right)\right)d\theta = \\
                   & = \frac{\pi}{2}\log(2) + 2\int_0^{\pi/2} \log(\sin(\varphi))d\varphi - 2\int_{\pi/2}^{\pi/4} \log(\sin(\varphi))d\varphi = \frac{\pi}{2}\log(2) + 2\int_0^{\pi/2} \log(\sin(\varphi))d\varphi
              \end{align*}
              Así que
              $$\int_0^{\pi/2} \log(\sin(\theta))d\theta = -\frac{\pi}{2}\log(2)$$
              Por tanto,
              $$\frac{1}{2\pi} \int_{-\pi}^\pi \log|1-e^{i\theta}|d\theta = 0$$
    \end{itemize}
\end{proof}

\begin{theorem}[Fórmula de Jensen]
    Sea $f$ una función holomorfa en $D(0, R)$ con $f(0) \neq 0$ y tal que tiene ceros $\{a_n\}$ de modo que
    $$|a_1| \leq |a_2| \leq |a_3| \leq \dots$$
    Sea $\rho \in (0, R)$ y $n(\rho, f) = \#\{a_n : |a_n| \leq \rho\}$.
    Entonces
    $$\frac{1}{2\pi} \int_{-\pi}^\pi \log|f(\rho e^{i\theta})|d\theta = \log|f(0)| + \sum_{k=1}^{n(\rho, f)} \log\left(\frac{\rho}{|a_k|}\right)$$
\end{theorem}

\begin{proof}
    Si $\rho = 0$,
    $$\frac{1}{2\pi} \int_{-\pi}^\pi \log|f(0)|d\theta = \log|f(0)|$$
    Sea ahora $\rho \in (0, R)$.
    Supongamos que $f$ no tiene ceros en $\overline{D(0, \rho)}$.
    Entonces existe $\rho'$ con $0 < \rho < \rho' < R$ tal que $f$ es holomorfa en $D(0, \rho')$ y $f$ no tiene ceros en $D(0, \rho')$.
    Así, como $\Log(f(z))$ es holomorfa en $D(0, \rho')$, tenemos que $\log|f(z)|$ es armónica en $D(0, \rho')$.
    Por el teorema del valor medio,
    $$\frac{1}{2\pi} \int_{-\pi}^\pi \log|f(\rho e^{i\theta})|d\theta = \log|f(0)|$$
    En general, consideramos
    $$F(z) = \frac{f(z)}{\prod_{k=1}^{n(\rho, f)} (z-a_k)}$$
    Es holomorfa en $D(0, R)$ y no se anula en $\overline{D(0, \rho)}$.
    Por el caso anterior,
    \begin{align*}
         & \frac{1}{2\pi} \int_{-\pi}^\pi \log\left(\frac{|f(\rho e^{i\theta})|}{\prod_{k=1}^{n(\rho, f)} |\rho e^{i\theta} - a_k|}\right)d\theta = \frac{1}{2\pi} \int_{-\pi}^\pi \log|f(\rho e^{i\theta})|d\theta - \frac{1}{2\pi} \int_{-\pi}^\pi \log\left(\prod_{k=1}^{n(\rho, f)} |\rho e^{i\theta} - a_k|\right)d\theta = \\
         & = \log|f(0)| + \sum_{k=1}^{n(\rho, f)} \log\left(\frac{1}{|a_k|}\right)
    \end{align*}
    Sin embargo, para dar este paso es necesario que las integrales converjan por separado.
    Sea $a_k = \rho_ke^{i\theta_k}$.
    \begin{align*}
         & \frac{1}{2\pi} \int_{-\pi}^\pi \log|\rho e^{i\theta} - \rho_ke^{i\theta_k}|d\theta = \frac{1}{2\pi} \int_{-\pi}^\pi \log\left|\rho e^{i\theta}\left(1-\frac{\rho_k}{\rho}e^{i(\theta_k-\theta)}\right)\right|d\theta = \\
         & = \frac{1}{2\pi} \int_{-\pi}^\pi \log|\rho e^{i\theta}|d\theta + \frac{1}{2\pi} \int_{-\pi}^\pi \log\left|1-\frac{\rho_k}{\rho}e^{i(\theta_k-\theta)}\right|d\theta =                                                  \\
         & = \log(\rho) - \frac{1}{2\pi} \int_{-\pi}^\pi \log\left|1-\frac{\rho_k}{\rho}e^{i\varphi}\right|d\varphi = \log(\rho)
    \end{align*}
\end{proof}

\begin{definition}
    Se llama función contadora de Nevanlinna a la cantidad
    $$N(\rho, f) = \sum_{\{n : |a_n| \leq \rho\}} \log\left(\frac{\rho}{|a_n|}\right)$$
\end{definition}

\begin{remark}
    \hfill
    \begin{enumerate}
        \item $$N(\rho, f) = \sum_{\{n : |a_n| \leq \rho\}} \log\left(\frac{\rho}{|a_n|}\right) = \sum_{k=1}^{n(\rho, f)} \frac{\rho}{|a_k|} = \sum_{k=1}^\infty \log^+\left(\frac{\rho}{|a_k|}\right)$$
        \item $$N(\rho, f) = \int_0^\rho \frac{n(t, f)}{t}dt$$
              \begin{proof}
                  Sean $f$ y $\rho$.
                  Por simplicidad, llamamos $n(\rho) = n(\rho, f)$.
                  Existen $a_1, \dots, a_{n(\rho)}$ ceros de $f$ de modo que
                  $$|a_1| \leq |a_2| \leq \dots \leq |a_{n(\rho)}|$$
                  Van a existir $p$ radios $\rho_1 < \dots < \rho_p$ tal que cada cero $a_k$ va a tener asociado un único $\rho_j$ de la forma $|a_k| = \rho_j$.
                  Entenderemos que $n(\rho_0) = 0$.
                  \begin{align*}
                       & N(\rho, f) = \sum_{\{n : |a_n| \leq \rho\}} \log\left(\frac{\rho}{|a_n|}\right) = \sum_{j=1}^p \sum_{\{n : |a_n| \leq \rho_j\}} \log\left(\frac{\rho}{|a_n|}\right) = \sum_{j=1}^p \log\left(\frac{\rho}{\rho_j}\right) \sum_{\{n : |a_n| \leq \rho_j\}} 1 =      \\
                       & = \sum_{j=1}^p \log\left(\frac{\rho}{\rho_j}\right)\left(n(\rho_j)-n(\rho_{j-1})\right) =                                                                                                                                                                         \\
                       & = \log\left(\frac{\rho}{\rho_1}\right)(n(\rho_1)-n(\rho_0)) + \log\left(\frac{\rho}{\rho_2}\right)(n(\rho_2)-n(\rho_1)) + \dots + \log\left(\frac{\rho}{\rho_p}\right)(n(\rho_p)-n(\rho_{p-1})) =                                                                 \\
                       & = n(\rho_1)\left(\log\left(\frac{\rho}{\rho_1}\right) - \log\left(\frac{\rho}{\rho_2}\right)\right) + n(\rho_2)\left(\log\left(\frac{\rho}{\rho_2}\right) - \log\left(\frac{\rho}{\rho_3}\right)\right) + \dots + n(\rho_p)\log\left(\frac{\rho}{\rho_p}\right) = \\
                       & = n(\rho_1)\log\left(\frac{\rho_2}{\rho_1}\right) + n(\rho_2)\log\left(\frac{\rho_3}{\rho_2}\right) + \dots + n(\rho_{p-1})\log\left(\frac{\rho_p}{\rho_{p-1}}\right) + n(\rho_p)\log\left(\frac{\rho}{\rho_p}\right) =                                           \\
                       & = \int_{\rho_1}^{\rho_2} \frac{n(t)}{t}dt + \int_{\rho_2}^{\rho^3} \frac{n(t)}{t}dt + \dots + \int_{\rho_{p-1}}^{\rho_p} \frac{n(t)}{dt}dt + \int_{\rho_p}^\rho \frac{n(t)}{t}dt = \int_{\rho_1}^\rho \frac{n(t)}{t}dt = \int_0^\rho \frac{n(t)}{t}dt
                  \end{align*}
              \end{proof}
    \end{enumerate}
\end{remark}

Existen diversas generalizaciones de la fórmula de Jensen.

\begin{theorem}
    Sea $f(z) = c_Nz^N + c_{N+1}z^{N+1} + \dots$ una función holomorfa en $D(0, R)$ y sea $\rho \in [0, R)$ tal que $\{a_n\}_{n=1}^\infty$ son los ceros de $f$ en $D(0, R)$ y además están ordenados respetando la multiplicidad.
    Entonces:
    $$\frac{1}{2\pi} \int_{-\pi}^\pi \log|f(\rho e^{i\theta})|d\theta = \log|c_N\rho^N| + \sum_{k=1}^{n(\rho, f)} \log\left(\frac{\rho}{|a_k|}\right)$$
\end{theorem}

\begin{proof}
    Consideramos
    $$F(z) = \frac{f(z)}{c_Nz^N}$$
    $F$ es holomorfa en $D(0, R)$, $F(0) = 1$ y los ceros de $F$ son los mismos que los de $f$.

    Aplicando la fórmula de Jensen,
    \begin{align*}
         & \frac{1}{2\pi} \int_{-\pi}^\pi \log|f(\rho e^{i\theta})d\theta = \log|F(0)| + \sum_{k=1}^{n(\rho, f)} \log\left(\frac{\rho}{|a_k|}\right) \Leftrightarrow                                                             \\
         & \Leftrightarrow \frac{1}{2\pi} \int_{-\pi}^\pi \log|f(\rho e^{i\theta})|d\theta - \frac{1}{2\pi} \int_{-\pi}^\pi \log|c_N\rho^N|d\theta = \sum_{k=1}^{n(\rho, f)} \log\left(\frac{\rho}{|a_k|}\right) \Leftrightarrow \\
         & \Leftrightarrow \frac{1}{2\pi} \int_{-\pi}^\pi \log|f(\rho e^{i\theta})|d\theta = \log|c_N\rho^N| + \sum_{k=1}^{n(\rho, f)} \log\left(\frac{\rho}{|a_k|}\right)
    \end{align*}
\end{proof}

Vamos a ver cómo aplicar la fórmula de Jensen para ver que si el crecimiento de $f$ está controlado entonces sus ceros también lo están.

\begin{definition}
    Llamamos módulo máximo de $f$ a
    $$M_\infty(\rho, f) = \sup_{|z|=\rho} |f(z)|$$
\end{definition}

\begin{remark}
    Si $f$ es holomorfa,
    $$M_\infty(\rho, f) = \max_{|z|=\rho} |f(z)| = \max_{|z|\leq\rho} |f(z)|$$
\end{remark}

\begin{theorem}
    Sea $f$ holomorfa en $D(0, R)$ tal que $f(0) \neq 0$ y $0 \leq s \leq \rho < R$.
    Entonces:
    $$n(s, f)\log\left(\frac{\rho}{s}\right) \leq \log(M_\infty(\rho, f)) - \log|f(0)|$$
\end{theorem}

\begin{proof}
    \begin{align*}
        n(s, f)\log\left(\frac{\rho}{s}\right) & = n(s, f)\int_s^\rho \frac{1}{t}dt \leq \int_s^\rho \frac{n(t, f)}{t}dt \leq \int_0^\rho \frac{n(t, f)}{t}dt =           \\
                                               & = \frac{1}{2\pi} \int_{-\pi}^\pi \log|f(\rho e^{i\theta})|d\theta - \log|f(0)| \leq \log(M_\infty(\rho, f)) - \log|f(0)|
    \end{align*}
\end{proof}

\begin{example}
    Sea $f$ entera con $|f(0)| = 1$ y sea $\rho = es$.
    Entonces:
    $$n(s, f) \leq \log(M_\infty(es, f))$$
\end{example}

\begin{remark}
    Si $f(z) = c_Nz^N + \dots$, el teorema anterior se puede escribir de la forma:
    $$n(s, f)\log\left(\frac{\rho}{s}\right) \leq \log(M_\infty(\rho, f)) - \log|c_N\rho^N|$$
\end{remark}

\section{La fórmula de Poisson-Jensen}
\begin{theorem}
    Sea $0 < R \leq \infty$ y sea $f$ holomorfa en $D(0, R)$.
    Sean $s \in (0, R)$ y $z \in D(0, s)$, entonces:
    $$\frac{1}{2\pi} \int_{-\pi}^\pi \log|f(se^{i\theta})\frac{s^2-|z|^2}{|s-ze^{-i\theta}|^2}d\theta = \log|f(z)| + \sum_{\{|a_n| \leq s, a_n \text{ cero de } f\}} \log^+\left|\frac{s^2-\bar{a}_nz}{s(z-a_n)}\right|$$
\end{theorem}