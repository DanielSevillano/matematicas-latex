\documentclass{report}
\usepackage[hmargin = 3cm, vmargin = 2.5cm]{geometry}
\usepackage[spanish]{babel}
\usepackage{amssymb, amsmath, amsthm, parskip, hyperref}

\title{Análisis complejo}
\author{Matemáticas en LaTeX}

\newtheorem{theorem}{Teorema}[chapter]
\newtheorem{corollary}[theorem]{Corolario}
\newtheorem{lemma}[theorem]{Lema}
\newtheorem{proposition}[theorem]{Proposición}

\theoremstyle{remark}
\newtheorem*{remark}{Observación}
\theoremstyle{remark}
\newtheorem*{note}{Nota}
\theoremstyle{remark}
\newtheorem*{notation}{Notación}

\theoremstyle{definition}
\newtheorem{definition}{Definición}[chapter]
\theoremstyle{definition}
\newtheorem*{properties}{Propiedades}
\theoremstyle{definition}
\newtheorem*{example}{Ejemplo}
\theoremstyle{definition}
\newtheorem*{exercise}{Ejercicio}

\newcommand{\Res}{\mathrm{Res}}
\newcommand{\Arg}{\mathrm{Arg}}
\newcommand{\Log}{\mathrm{Log}}
\newcommand{\dist}{\mathrm{dist}}
\newcommand{\Aut}{\mathrm{Aut}}
\newcommand{\sop}{\mathrm{sop}}
\renewcommand{\Re}{\mathrm{Re}}
\renewcommand{\Im}{\mathrm{Im}}

\begin{document}
\maketitle
\tableofcontents

\chapter{Modelo de regresión lineal simple}
\section{Introducción}

La regresión lineal es un modelo matemático que nos permite establecer la relación de dependencia entre una variable dependiente $Y$ y una variable independiente $X$.

Nos interesan las relaciones de la forma $y = f(x) + u$, donde $u$ es una variable aleatoria a la que llamamos perturbación.
En el caso de la regresión lineal simple, el modelo será de la forma $$y = \beta_0 + \beta_1x + u$$
con $\beta_0$ y $\beta_1$ parámetros.
Llamamos intercepto a $\beta_0$ y pendiente a $\beta_1$.

\section{Modelo e hipótesis}
Sea $X$ una variable aleatoria cuantitativa, $Y$ una variable aleatoria continua y $(x_1, y_1), (x_2, y_2), \dots (x_n, y_n)$ un conjunto de datos.
Entonces el modelo de regresión lineal simple es
$$y_i = \beta_0 + \beta_1x_i + u_i, \quad i = 1, \dots n$$

\subsection*{Hipótesis del modelo}
\begin{enumerate}
    \item $E(u_i) = 0, \quad \forall i = 1, \dots n$.
    \item $V(u_i) = \sigma^2, \quad \forall i = 1, \dots n$ (homocedasticidad)
    \item $u_i \sim N(0, \sigma^2), \quad \forall i = 1, \dots n$ (normalidad)
    \item $E(u_i u_j) = 0, \quad \forall i \neq j$ (independencia)
\end{enumerate}

\begin{note}
    En realidad, la cuarta hipótesis es de incorrelación ($Cov(u_i, u_j) = 0$).
    $$Cov(u_i, u_j) = E(u_i u_j) - E(u_i)E(u_j) = E(u_i u_j)$$
    Sin embargo, bajo normalidad la incorrelación y la independencia son equivalentes.
\end{note}

Podemos escribir las mismas hipótesis en términos de $y_i, \quad \forall i = 1, \dots n$.
\begin{enumerate}
    \item $E(y_i | x_i) = E(\beta_0 + \beta_1x_i + u_i) = \beta_0 + \beta_1x_i, \quad \forall i = 1, \dots n$ (linealidad)
    \item $V(y_i | x_i) = V(\beta_0 + \beta_1x_i + u_i) = \sigma^2, \quad \forall i = 1, \dots n$ (homocedasticidad)
    \item $y_i | x_i \sim N(\beta_0 + \beta_1x_i, \sigma^2), \quad \forall i = 1, \dots n$ (normalidad)
    \item $Cov(y_i, y_j) = 0, \quad \forall i \neq j$ (independencia)
\end{enumerate}

Podemos dar un significado real a $\beta_0$ y $\beta_1$:
\begin{itemize}
    \item $\beta_0$ es el valor medio de la variable $Y$ cuando $x_i$ toma el valor 0.
          $$E(y_i | x_i = 0) = \beta_0, \quad i = 1, \dots n$$
    \item $\beta_1$ es la variación media que experimenta la variable $Y$ cuando $X_i$ aumenta en una unidad.
          $$E(y_i | x_i+1) - E(y_i | x_i) = \beta_1, \quad i = 1, \dots n$$
\end{itemize}

\section{Estimación de los parámetros}
Queremos estimar $\beta_0$, $\beta_1$ y $\sigma^2$.
Con los estimadores $\hat{\beta_0}$ y $\hat{\beta_1}$ podemos estimar $$\hat{E}(y_i | x_i) = \hat{\beta_0} + \hat{\beta_1}x_i, \quad i = 1, \dots n$$

\subsection*{Método de máxima verosimilitud}
$y_i | x_i \sim N(\beta_0 + \beta_1x_i, \sigma^2), \quad i = 1, \dots, n$, así que podemos encontrar estimadores de máxima verosimilitud para los parámetros y para $\sigma^2$.

Usando el método de máxima verosimilitud llegamos las ecuaciones normales de la regresión:
$$\begin{cases}
        \dfrac{\partial \log{L}}{\partial \beta_0} = \frac{1}{\sigma^2} \sum_{i=1}^n (y_i - \beta_0 - \beta_1x_i) = 0 \\
        \dfrac{\partial \log(L)}{\partial \beta_1} = \frac{1}{\sigma^2} \sum_{i=1}^n x_i(y_i - \beta_0 - \beta_1x_i) = 0
    \end{cases}$$

\begin{notation}
    $\hat{y_i} = \hat{E}(y_i | x_i) = \hat{\beta_0} + \hat{\beta_1}x_i$
\end{notation}

Si definimos el error o residuo como $e_i = y_i - \hat{y_i} = y_i - \hat{\beta_0} - \hat{\beta_1}x_i$, podemos escribir las ecuaciones normales de regresión de la siguiente forma:
$$\begin{cases}
        \sum_{i=1}^n e_i = 0 \\
        \sum_{i=1}^n x_ie_i = 0
    \end{cases}$$

Resolviendo este sistema, obtenemos los estimadores:
\begin{align*}
    \hat{\beta_1}  & = \frac{s_{XY}}{s_X^2}                                                                                 \\
    \hat{\beta_0}  & = \bar{y} - \frac{s_{XY}}{s_X^2}\bar{x}                                                                \\
    \hat{\sigma^2} & = \frac{1}{n} \sum_{i=1}^n (y_i - \hat{\beta_0} - \hat{\beta_1}x_i)^2 = \frac{1}{n} \sum_{i=1}^n e_i^2
\end{align*}

La ecuación de la recta resultante es:
$$\hat{y_i} = \bar{y} + \frac{s_{XY}}{s_X^2}(x_i - \bar{x})$$

\subsection*{Estimación por mínimos cuadrados}
Queremos minimizar la suma de los cuadrados de los errores $\sum_{i=1}^n e_i^2$, donde $e_i = y_i - \hat{y_i}$.
Para ello minimizamos la función $M(\beta_0, \beta_1) = \sum_{i=1}^n (y_i - \beta_0 - \beta_1x_i)^2$.
$$\begin{cases}
        \frac{\partial M}{\partial \beta_0}(\beta_0, \beta_1) = -2\sum(y_i - \beta_0 - \beta_1x_i) = 0 \\
        \frac{\partial M}{\partial \beta_1}(\beta_0, \beta_1) = -2\sum x_i(y_i - \beta_0 - \beta_1x_i) = 0
    \end{cases}$$

Simplificando obtenemos las ecuaciones normales de la regresión, como antes.
Así que los estimadores de $\beta_0$ y $\beta_1$ por máxima verosimilitud coinciden con los estimadores por mínimos cuadrados.

\subsection*{Estimación de la varianza}
Partiendo del estimador $\bar{\sigma^2} = \frac{1}{n} \sum_{i=1}^n e_i^2$ obtenido previamente, podemos llegar a una expresión equivalente:
$$\hat{\sigma^2} = s_Y^2 - \frac{s_{XY}^2}{s_X^2}$$

Veamos si este estimador es insesgado calculando su esperanza.
$$E(\hat{\sigma^2}) = E(\frac{\sum_{i=1}^n e_i^2}{n}) = \frac{1}{n}E(\sum_{i=1}^n e_i^2) = \frac{1}{n}\sigma^2(n-2)$$

\begin{note}
    $\frac{1}{\sigma^2} \sum_{i=1}^n e_i^2 \sim \chi^2_{n-2}, \quad E(\frac{1}{\sigma^2} \sum_{i=1}^n e_i^2) = n-2$
\end{note}

Observamos que este estimador no es insesgado. Consideramos entonces:
$$s_R^2 = \frac{1}{n-2}\sum_{i=1}^n e_i^2$$
Este sí es un estimador insesgado de $\sigma^2$ y le llamamos varianza residual.
Tenemos la relación $s_R^2 = \frac{n}{n-2} \hat{\sigma^2}$.

\section{Propiedades de los estimadores}
Podemos escribir $\hat{\beta_1}$ de la forma:
$$\hat{\beta_1} = \sum_{i=1}^n w_iy_i, \quad w_i = \frac{x_i - \bar{x}}{ns_X^2}$$
Por las hipótesis del modelo, $y_i$ son normales e independientes, luego $\hat{\beta_1} \sim N$.
Podemos calcular:

\begin{itemize}
    \item $E(\hat{\beta_1}) = \beta_1$ (estimador insesgado)
    \item $V(\hat{\beta_1}) = \frac{\sigma^2}{ns_X^2}$
\end{itemize}
Por tanto, $\hat{\beta_1} \sim N(\beta_1, \frac{\sigma^2}{ns_X^2})$.

De forma análoga, podemos escribir:
$$\hat{\beta_0} = \sum_{i=1}^n (\frac{1}{n} - \bar{x}w_i)$$
Como las $y_i$ son normales e independientes, $\hat{\beta_0} \sim N$.
Calculamos:

\begin{itemize}
    \item $E(\hat{\beta_0}) = \beta_0$ (estimador insesgado)
    \item $V(\hat{\beta_0}) = \frac{\sigma^2}{n} (1 + \frac{\bar{x}^2}{s_X^2})$
\end{itemize}
Por tanto, $\hat{\beta_0} \sim N(\beta_0, \frac{\sigma^2}{n} (1 + \frac{\bar{x}^2}{s_X^2}))$.

En cuanto a $s_R^2$, sabemos que $\frac{1}{\sigma^2} \sum_{i_1}^n e_i^2 \sim \chi^2_{n-2}$.
Obtenemos que:
\begin{itemize}
    \item $E(s_R^2) = \sigma^2$
    \item $V(s_R^2) = \frac{2}{n-2} (\sigma^2)^2$
\end{itemize}

\section{Intervalos de confianza para los parámetros}
\subsection*{Intervalos de confianza para $\beta_1$}
\subsubsection*{Caso 1: $\sigma^2$ conocida}
Sabemos que $\hat{\beta_1} \sim N(\beta_1, \frac{\sigma^2}{ns_X^2})$. Entonces:
$$\frac{\hat{\beta_1} - \beta_1}{\frac{\sigma}{\sqrt{ns_X^2}}} \sim N(0, 1)$$
Por tanto, el intervalo de confianza para $\beta_1$ a nivel de significación $\alpha$ es:
$$IC_{1-\alpha}(\beta_1) = \left( \hat{\beta_1} - z_{1-\frac{\alpha}{2}} \frac{\sigma}{\sqrt{ns_X^2}}, \hat{\beta_1} + z_{1-\frac{\alpha}{2}} \frac{\sigma}{\sqrt{ns_X^2}} \right)$$
donde $z_{1-\frac{\alpha}{2}}$ es el percentil de orden $(1 - \frac{\alpha}{2}) 100\%$ de una variable aleatoria $Z \sim N(0, 1)$.

\subsubsection*{Caso 2: $\sigma^2$ desconocida}
$$\begin{cases}
        \frac{\hat{\beta_1} - \beta_1}{\frac{\sigma}{\sqrt{ns_X^2}}} \sim N(0, 1) \\
        \frac{\sum_{i=1}^n e_i^2}{\sigma^2} = \frac{(n-2)s_R^2}{\sigma^2} \sim \chi^2_{n-2}
    \end{cases} \Rightarrow
    \frac{\frac{\hat{\beta_1} - \beta_1}{\frac{\sigma}{\sqrt{ns_X^2}}}}{\sqrt{\frac{(n-2)s_R^2}{\sigma^2} \frac{1}{n-2}}} = \frac{\hat{\beta_1} - \beta_1}{\frac{s_R}{\sqrt{ns_X^2}}} \sim t_{n-2}$$
Luego el intervalo de confianza para $\beta_1$ a nivel de significación $\alpha$ es:
$$IC_{1-\alpha}(\beta_1) = \left( \hat{\beta_1} - t_{n-2, 1-\frac{\alpha}{2}} \frac{s_R}{\sqrt{ns_X^2}}, \hat{\beta_1} + t_{n-2, 1-\frac{\alpha}{2}} \frac{s_R}{\sqrt{ns_X^2}} \right)$$
donde $s_R = +\sqrt{s_R^2}$ y $t_{n-2, 1-\frac{\alpha}{2}}$ es el percentil de orden $(1 - \frac{\alpha}{2}) 100\%$ de una variable aleatoria $T \sim t_{n-2}$.

\subsection*{Intervalos de confianza para $\beta_0$}
\subsubsection*{Caso 1: $\sigma^2$ conocida}
Sabemos que $\hat{\beta_0} \sim N\left( \beta_0, \frac{\sigma^2}{n} \left( 1 + \frac{\bar{x}}{s_X^2} \right) \right)$.
Entonces, el intervalo de confianza para $\beta_0$ a nivel de significación $\alpha$ es:
$$IC_{1-\alpha}(\beta_0) = \left( \hat{\beta_0} - z_{1-\frac{\alpha}{2}} \frac{\sigma}{\sqrt{n}} \sqrt{1 + \frac{\bar{x}^2}{s_X^2}}, \hat{\beta_0} + z_{1-\frac{\alpha}{2}} \frac{\sigma}{\sqrt{n}} \sqrt{1 + \frac{\bar{x}^2}{s_X^2}} \right)$$
donde $z_{1-\frac{\alpha}{2}}$ es el percentil de orden $(1 - \frac{\alpha}{2}) 100\%$ de una variable aleatoria $Z \sim N(0, 1)$.

\subsubsection*{Caso 2: $\sigma^2$ desconocida}
Razonando de forma análoga al caso de $\beta_1$, tenemos que:
$$\frac{\hat{\beta_0} - \beta_0}{\frac{s_R}{n} \sqrt{1 + \frac{\bar{x}^2}{s_X^2}}}$$
Por tanto, el intervalo de confianza para $\beta_0$ a nivel de significación $\alpha$ es:
$$IC_{1-\alpha}(\beta_0) = \left( \hat{\beta_0} - t_{n-2, 1-\frac{\alpha}{2}} \frac{s_R}{\sqrt{n}} \sqrt{1 + \frac{\bar{x}^2}{s_X^2}}, \hat{\beta_0} + t_{n-2, 1-\frac{\alpha}{2}} \frac{s_R}{\sqrt{n}} \sqrt{1 + \frac{\bar{x}^2}{s_X^2}} \right)$$
donde $t_{n-2, 1-\frac{\alpha}{2}}$ es el percentil de orden $(1 - \frac{\alpha}{2}) 100\%$ de una variable aleatoria $T \sim t_{n-2}$.

\subsection*{Intervalos de confianza para $\sigma^2$}
Sabemos que $\frac{\sum_{i=1}^n e_i^2}{\sigma^2} = \frac{(n-2)s_R^2}{\sigma^2} \sim \chi^2_{n-2}$.
Queremos que $P(a < \sigma^2 < b) = 1-\alpha$.
\begin{align*}
    P(a < \sigma^2 < b) & = P\left( \frac{1}{b} < \frac{1}{\sigma^2} < \frac{1}{a} \right) =                          \\
                        & = P\left( \frac{(n-2)s_R^2}{b} < \frac{(n-2)s_R^2}{\sigma^2} < \frac{(n-2)s_R^2}{a} \right)
\end{align*}
Luego:
$$\begin{cases}
        \frac{(n-2)s_R^2}{b} = \chi^2_{n-2, \frac{\alpha}{2}} \Rightarrow b = \frac{(n-2)s_R^2}{\chi^2_{n-2, \frac{\alpha}{2}}} \\
        \frac{(n-2)s_R^2}{a} = \chi^2_{n-2, 1 - \frac{\alpha}{2}} \Rightarrow a = \frac{(n-2)s_R^2}{\chi^2_{n-2, 1 - \frac{\alpha}{2}}}
    \end{cases}$$
Por tanto, el intervalo de confianza para $\sigma^2$ a nivel de significación $\alpha$ tiene por extremos $a$ y $b$, es decir:
$$IC_{1-\alpha}(\sigma^2) = (a, b)$$

\section{Contraste de la regresión}
Consideramos el siguiente contraste de hipótesis:
$$\begin{cases}
        H_0 : \beta_1 = 0 \Leftrightarrow E(y|x) = \beta_0 \\
        H_1 : \beta_1 \neq 0 \Leftrightarrow E(y|x) = \beta_0 + \beta_1x
    \end{cases}$$
Fijamos el nivel de significación $\alpha$.
Podemos resolverlo de cuatro formas distintas.

\subsection*{Intervalos de confianza}
Sea $IC_{1-\alpha}(\beta_1)$ el intervalo de confianza para $\beta_1$ a nivel de significación $\alpha$.
Entonces:
\begin{itemize}
    \item Aceptamos $H_0$ a nivel de significación $\alpha$ si $0 \in IC_{1-\alpha}(\beta_1)$.
    \item Rechazamos $H_0$ a nivel de significación $\alpha$ en caso contrario.
\end{itemize}

\subsection*{Estadístico $T$}
Sabemos que $\frac{\hat{\beta_1} - \beta_1}{\frac{s_R}{\sqrt{ns_X^2}}} \sim t_{n-2}$.
Entonces $T = \frac{\hat{\beta_1}}{\frac{s_R}{\sqrt{ns_X^2}}} \sim t_{n-2}$ si $H_0$ es cierto.

Tomamos un $t_{exp}$.
\begin{itemize}
    \item Si $t_{exp} \in (-t_{n-2, 1-\frac{\alpha}{2}}, t_{n-2, 1-\frac{\alpha}{2}})$, o equivalentemente $|t_{exp}| \leq t_{n-2, 1-\frac{\alpha}{2}}$, aceptamos $H_0$ a nivel de significación $\alpha$.
    \item En caso contrario, rechazamos $H_0$ a nivel de significación $\alpha$.
\end{itemize}

\subsection*{Valor $p$}
Sea $p$ el valor $p$ o $p$-valor de la distribución. Entonces:
\begin{itemize}
    \item Si $p \geq \alpha$, aceptamos $H_0$ a nivel de significación $\alpha$.
    \item En caso contrario, rechazamos $H_0$ a nivel de significación $\alpha$.
\end{itemize}

\subsection*{Tabla ANOVA}
Partimos de que podemos escribir:
$$\sum_{i=1}^n (y_i - \bar{y})^2 = \sum_{i=1}^n (y_i - \hat{y_i})^2 + \sum_{i=1}^n (\hat{y_i} - \bar{y})^2$$
Definimos:
\begin{itemize}
    \item Variabilidad total:
          $$VT = \sum_{i=1}^n (y_i - \bar{y})^2 = ns_Y^2$$
    \item Variabilidad no explicada:
          $$VNE = \sum_{i=1}^n (y_i - \hat{y_i})^2 = \sum_{i=1}^n e_i^2 = (n-2)s_R^2 = n\hat{\sigma^2} = n(s_Y^2 - \hat{\beta_1}^2s_X^2)$$
    \item Variabilidad explicada:
          $$VE = \sum_{i=1}^n (\hat{y_i} - \bar{y})^2 = VT - VNE = n\hat{\beta_1}^2 s_X^2$$
\end{itemize}
Observamos que:
$$\frac{VNE}{\sigma^2} = \frac{\sum_{i=1}^n e_i^2}{\sigma^2} \sim \chi^2_{n-2}$$
Además, como $\frac{\hat{\beta_1} - \beta_1}{\frac{\sigma}{\sqrt{ns_X^2}}} \sim N(0,1)$, entonces:
$$\frac{(\hat{\beta_1} - \beta_1)^2}{\frac{\sigma^2}{ns_X^2}} \sim \chi^2_1$$
Luego $\frac{\hat{\beta_1}^2}{\frac{\sigma^2}{ns_X^2}} = \frac{n\hat{\beta_1}^2s_X^2}{\sigma^2} = \frac{VE}{\sigma^2} \sim \chi^2_1$ si $H_0$ es cierta.

Consideramos ahora $F = \frac{\frac{VE}{\sigma^2} / 1}{\frac{VNE}{\sigma^2} / (n-2)} = \frac{VE}{s_R^2}$.
Observamos que $F \sim F_{1, n-2}$ si $H_0$ es cierta.

Tomamos un $F_{exp}$.
\begin{itemize}
    \item Aceptamos $H_0$ a nivel de significación $\alpha$ si $F_{exp} \leq F_{1, n-2, 1-\alpha}$.
    \item En caso contrario, rechazamos $H_0$ a nivel de significación $\alpha$.
\end{itemize}

La tabla ANOVA es de la forma:
\begin{center}
    \begin{tabular}{| c | c | c | c |}
        \hline
        Fuentes & Suma de cuadrados                   & Grados de libertad & Cocientes   \\
        \hline
        $VE$    & $\sum_{i=1}^n(\hat{y_i}-\bar{y})^2$ & 1                  & $VE/1$      \\
        $VNE$   & $\sum_{i=1}^n(y_i - \hat{y_i})^2$   & $n-2$              & $VNE/(n-2)$ \\
        $VT$    & $\sum_{i=1}^n(y_i - \bar{y})^2$     & $n-1$              &             \\
        \hline
    \end{tabular}
\end{center}
También se incluyen columnas para $F_{exp}$ y $p$-valor.

\begin{remark}
    Existe la siguiente relación entre $t_{exp}$ y $F_exp$:
    $$t_{exp}^2 = F_{exp}$$
\end{remark}

\section{Evaluación del ajuste}
Existen dos coeficientes para evaluar el ajuste del modelo: el coeficiente de correlación lineal y el coeficiente de determinación.

\subsection*{Coeficiente de correlación lineal}
El coeficiente de correlación lineal se define como:
$$r = \frac{s_{XY}}{s_X s_Y}, \quad -1 \leq r \leq 1$$
\begin{itemize}
    \item Si $r = 1$, se tiene dependencia lineal exacta positiva.
    \item Si $r = -1$, se tiene dependencia lineal exacta negativa.
    \item Si $r = 0$, las variables están incorreladas linealmente.
\end{itemize}
Se dice que el ajuste es bueno si $|r|$ es cercano a 1.
Si por el contrario $r$ se aproxima a 0, entonces las variables no tienen relación lineal.

\subsection*{Coeficiente de determinación}
El coeficiente de determinación se define como:
$$R^2 = \frac{VE}{VT}, \quad 0 \leq R^2 \leq 1$$
\begin{itemize}
    \item Si $R^2 = 1$ entonces $VE = VT$ luego $VNE = \sum_{i=1}^n (y_i - \hat{y_i})^2 = \sum_{i=1}^n e_i^2 = 0$.
          Por tanto, $e_i = 0$ para todo $i = 1, \dots n$, así que el ajuste lineal es exacto.
    \item Si $R^2 = 0$ entonces $VE = 0$, luego $VT = VNE$.
          Así que el ajuste lineal es pésimo.
\end{itemize}

\begin{theorem}
    El coeficiente de determinación coincide con el coeficiente de correlación lineal al cuadrado.
    Es decir, $$r^2 = R^2$$

    \begin{proof}
        $$R^2 = \frac{VE}{VT} = \frac{n\hat{\beta_1}^2s_X^2}{ns_Y^2} = \frac{\left(\frac{s_{XY}}{s_X^2}\right)^2 s_X^2}{s_Y^2} = \frac{s_{XY}^2}{s_X^2 s_Y^2} = r^2$$
    \end{proof}
\end{theorem}

\section{Predicción}

\subsection*{Estimación de las medias condicionadas}
Llamamos $m_0 = E(y | x=x_0) = \beta_0 + \beta_1x_0$.
Observamos que $m_0$ es un parámetro que podemos estimar de la forma:
$$\hat{m_0} = \hat{E}(y | x=x_0) = \hat{\beta_0} + \hat{\beta_1}x_0$$

\begin{theorem}
    $$\hat{m_0} \sim N\left( m_0, \frac{\sigma^2}{n} \left(1+\frac{(x_0-\bar{x})^2}{s_X^2}\right) \right)$$
\end{theorem}

\subsubsection*{Intervalos de confianza para $m_0$}
Podemos calcular los intervalos de confianza para $m_0$ con nivel de confianza de $100(1-\alpha)\%$.

Si $\sigma^2$ es conocida,
$$\left( \hat{m_0} - z_{1-\frac{\alpha}{2}} \frac{\sigma}{\sqrt{n}} \sqrt{1 + \frac{(x_0-\bar{x})^2}{s_X^2}}, \hat{m_0} + z_{1-\frac{\alpha}{2}} \frac{\sigma}{\sqrt{n}} \sqrt{1 + \frac{(x_0-\bar{x})^2}{s_X^2}} \right)$$

Si $\sigma^2$ es desconocida,
$$\left( \hat{m_0} - t_{n-2, 1-\frac{\alpha}{2}} \frac{s_R}{\sqrt{n}} \sqrt{1 + \frac{(x_0-\bar{x})^2}{s_X^2}}, \hat{m_0} + t_{n-2, 1-\frac{\alpha}{2}} \frac{s_R}{\sqrt{n}} \sqrt{1 + \frac{(x_0-\bar{x})^2}{s_X^2}} \right)$$

\subsection*{Predicción de una observación futura}
Dado un conjunto de datos $(x_1, y_1), \dots, (x_n, y_n)$ y dado $x_0$ queremos predecir:
$$y_0 = \beta_0 + \beta_1 x_0 + u_0$$
donde $u_0$ es independiente a $u_1, \dots, u_n$ con $u_0 \sim N(0, \sigma^2)$.
Observamos que $y_0$ es una variable aleatoria, con estimador $\hat{y_0} = \hat{\beta_0} + \hat{\beta_1}x_0$.
La estimación puntual es:
$$\hat{y_0} = \hat{\beta_0} + \hat{\beta_1}x_0 = \hat{m_0}$$

Consideramos el error:
$$e_0 = y_0 - \hat{y_0} = \beta_0 + \beta_1x_0 + u_0 - (\hat{\beta_0} + \hat{\beta_1}x_0)$$
que también es una variable aleatoria.

\begin{theorem}
    $$e_0 \sim \left( 0, \sigma^2 \left(1+\frac{1}{n}+\frac{(x_0-\bar{x})^2}{ns_X^2} \right) \right)$$
\end{theorem}

\subsubsection*{Intervalos de pronóstico para $y_0$}
Podemos calcular los intervalos de pronóstico $IP_{1-\alpha}(y_0)$ para $y_0$ con contenido probabilístico $1-\alpha$.

Si $\sigma^2$ es conocida,
$$\left( \hat{y_0} - z_{1-\frac{\alpha}{2}} \sigma \sqrt{1 + \frac{1}{n} + \frac{(x_0-\bar{x})^2}{ns_X^2}}, \hat{y_0} + z_{1-\frac{\alpha}{2}} \sigma \sqrt{1 + \frac{1}{n} + \frac{(x_0-\bar{x})^2}{ns_X^2}} \right)$$

Si $\sigma^2$ es desconocida,
$$\left( \hat{y_0} - t_{n-2, 1-\frac{\alpha}{2}} s_R \sqrt{1 + \frac{1}{n} + \frac{(x_0-\bar{x})^2}{ns_X^2}}, \hat{y_0} + t_{n-2, 1-\frac{\alpha}{2}} s_R \sqrt{1 + \frac{1}{n} + \frac{(x_0-\bar{x})^2}{ns_X^2}} \right)$$

\section{Análisis de residuos y observaciones atípicas e influyentes}
\subsection*{Residuos}
El residuo de un dato es la diferencia entre su valor y la predicción mediante el modelo.
$$e_i = y_i - \hat{y_i}, \quad \forall i = 1, \dots n$$
El análisis de los residuos puede darnos información sobre el ajuste del modelo.

\subsection*{Observaciones atípicas}
Una observación atípica es un valor que es numéricamente distinto al resto de los datos.
Visualmente, es un dato que se sale del patrón.
Las observaciones atípicas pueden ser indicativas de errores de observación o errores en el modelo.
Un error de observación se debe a datos que pertenecen a una población diferente del resto de muestras, mientras que un error en el modelo puede ser debido a que la muestra depende una variable desconocida que no se han tenido en cuenta.

\subsection*{Observaciones influyentes}
Una observación influyente $(x_A, y_A)$ es una observación atípica cuya exclusión produce un cambio drástico en la recta de regresión.
Puede ser causada por un error de observación o por un modelo incorrecto.
Algunas posibles causas de que el modelo sea incorrecto son:
\begin{itemize}
    \item La relación entre $x$ e $y$ no es lineal cerca de $x_A$.
    \item La varianza aumenta mucho con $x$.
    \item Una variable desconocida ha tomado un valor distinto en $x_A$.
\end{itemize}

\subsection*{Puntos palanca}
Los puntos palanca son observaciones con un valor alto de $p_i$.
Estos tienen la capacidad de alterar en gran medida la recta de regresión.

\section{Transformaciones}
Cuando el diagrama de dispersión entre las dos variables o el de los residuos presenta indicios de incumplimiento de alguna hipótesis básica, entonces hay que abandonar el modelo inicial por uno menos simple o bien aplicar alguna transformación a los datos.
\chapter{Familias normales}
\section{Familias normales}

\begin{theorem}[Teorema de convergencia de Weierstrass]
    Sea $D$ abierto en $\mathbb{C}$ y sean $\{f_n\}_{n=1}^\infty$ una sucesión de funciones holomorfas en $D$ y $f: D \to \mathbb{C}$.
    Si $f_n \xrightarrow[n \to \infty]{} f$ uniformemente en cada subconjunto compacto de $D$, entonces $f$ es holomorfa en $D$ y $f_n' \xrightarrow[n \to \infty]{} f'$ uniformemente en cada subconjunto compacto.
    Para todo $k \in \mathbb{N}$, $f^{(k)}_n \xrightarrow[n \to \infty]{} f^{(k)}$ uniformemente en cada compacto.
\end{theorem}

\begin{definition}
    Sea $D$ un abierto en $\mathbb{C}$ y sea $\mathcal{F}$ una familia de funciones holomorfas en $D$.
    Diremos que $\mathcal{F}$ es finitamente normal si para cada sucesión $\{f_n\}_{n=1}^\infty$ en $\mathcal{F}$ existe una subsucesión $\{f_{n_k}\}_{k=1}^\infty$ de $\{f_n\}$ que converge uniformemente en cada subconjunto compacto de $D$.
\end{definition}

\begin{remark}
    El límite $f$ de tal subsucesión es una función holomorfa en $D$, pero no tiene por qué pertenecer a $\mathcal{F}$.
\end{remark}

\begin{definition}
    Sea $D$ un abierto en $\mathbb{C}$ y sea $\mathcal{F}$ una familia de funciones holomorfas en $D$.
    Diremos que $\mathcal{F}$ es compacta si para cada sucesión $\{f_n\}_{n=1}^\infty$ en $\mathcal{F}$ existe una subsucesión $\{f_{n_k}\}_{k=1}^\infty$ de $\{f_n\}$ que converge uniformemente en cada subconjunto compacto de $D$ a una función que pertenece a $\mathcal{F}$.
\end{definition}

En el conjunto $Hol(D)$ de las funciones holomorfas en $D$, con $D$ abierto en $\mathbb{C}$, se puede definir una distancia $d$ tal que $(Hol(D), d)$ es un espacio métrico completo, y en el que:
$$f_n \xrightarrow{d} f \Leftrightarrow f_n \to f \text{ uniformemente en cada subconjunto compacto de } D$$
Si $\mathcal{F} \subset Hol(D)$, $\mathcal{F}$ es finitamente normal si y solo si $\mathcal{F}$ es relativamente compacto.
Los compactos coinciden con la definición de familia compacta dada.

\section{El teorema de Montel}
\begin{lemma}
    Sea $D$ un abierto en $\mathbb{C}$ y $\mathcal{F}$ una familia de funciones holomorfas en $D$.
    Entonces son equivalentes:
    \begin{enumerate}
        \item $\mathcal{F}$ está uniformemente acotada en cada subconjunto compacto de $D$.
        \item Para cada $a \in D$ existe $r_a > 0$ con $D(a, r_a) \subset D$ y $f$ está uniformemente acotada en $D(a, r_a)$.
    \end{enumerate}
\end{lemma}

\begin{lemma}
    Sea $D$ abierto en $\mathbb{C}$ y sean $f_n: D \to \mathbb{C}$ para $n = 1, 2, \dots$ y $f: D \to \mathbb{C}$.
    Entonces son equivalentes:
    \begin{enumerate}
        \item $f_n \to f$ uniformemente en cada subconjunto compacto de $D$.
        \item Para cada $a \in D$ existe $r_a > 0$ con $D(a, r_a) \subset D$ tal que $f_n \to f$ uniformemente en $D(a, r_a)$.
    \end{enumerate}
\end{lemma}

\begin{lemma}
    Sean $C_1, C_2 \in \mathbb{C}$, con $C_1 \cap C_2 = \emptyset$ y $C_1, C_2 \neq \emptyset$.
    Si $C_1$ es compacto y $C_2$ es cerrado, entonces:
    $$dist(C_1, C_2) = \inf\{|z_1-z_2| : z_1 \in C_1, z_2 \in C_2\} > 0$$
\end{lemma}

\begin{remark}
    Si $C_1$ no es compacto no es cierto en general.
\end{remark}

\begin{lemma}
    Sea $A \subset \mathbb{C}$, $A \neq \emptyset$ y sea
    $$F: \mathbb{C} \to \mathbb{R}, \; F(z) = dist(z, A) = \inf \{|z-a| : a \in A\}$$
    Entonces $F$ es continua y $F(z) = 0$ para todo $z \in A$.
    Si además $A$ es cerrado, entonces $F(z) = \min \{|z-a| : a \in A\}$ para todo $z \in \mathbb{C}$.
\end{lemma}

\begin{lemma}
    Sea $A \subset \mathbb{C}$, $A \neq \emptyset$ y sea $\varepsilon > 0$.
    Consideramos los conjuntos:
    \begin{align*}
        B & = \{z \in \mathbb{C} : dist(z, A) < \varepsilon\}    \\
        C & = \{z \in \mathbb{C} : dist(z, A) \leq \varepsilon\}
    \end{align*}
    Entonces $B$ es abierto y $C$ es cerrado, con $A \subset B \subset C$.
    Si además $A$ es acotado, entonces $B$ es acotado y $C$ es compacto.
\end{lemma}

\begin{proposition}
    Sea $D$ un abierto en $\mathbb{C}$ y $\mathcal{F}$ una familia de funciones holomorfas en $D$.
    Supongamos que $\mathcal{F}$ está uniformemente acotada en $D$.
    Sea $K$ un subconjunto compacto de $D$.
    Entonces existe $A > 0$ tal que:
    $$|f(z_2)-f(z_1)| \leq A|z_2-z_1|, \quad \forall z_1, z_2 \in K, \; \forall f \in \mathcal{F}$$
\end{proposition}

\begin{proof}
    Sea $M > 0$ tal que $|f(z)| \leq M$ para todo $z \in D$ y para toda $f \in \mathcal{F}$.
    Sean $K \subset D$, $K$ compacto.
    Sea $d > 0$ con $d < dist(K, \mathbb{C} \setminus D)$.
    Si $D = \mathbb{C}$, tomamos $d > 0$ cualquiera.
    Sea $z_0 \in K$.
    Entonces $D(z_0, d) \subset D$.
    De hecho, podemos tomar $\varepsilon > 0$ tal que $D(z_0, d+\varepsilon) \subset D$.
    Dada $f \in \mathcal{F}$, por la fórmula de Cauchy,
    $$f'(z) = \frac{1}{2\pi i} \int_{|\xi-z_0|=d} \frac{f(\xi)}{(\xi-z)^2}d\xi \quad \text{si } z \in D\left(z_0, \frac{d}{2}\right)$$
    Entonces:
    $$|f'(z)| \leq \frac{1}{2\pi}2\pi \max_{|\xi-z_0|=d} \frac{|f(\xi)|}{|\xi-z|^2}$$
    Podemos acotar:
    $$|\xi-z| = |(\xi-z_0) + (z_0-z)| \geq |\xi-z_0| - |z_0-z| \geq d - \frac{d}{2} = \frac{d}{2}$$
    Así que $|\xi-z|^2 \geq \frac{d^2}{4} > 0$.
    Luego:
    $$|f'(z)| \leq d\frac{M}{d^2/4} = \frac{4M}{d}$$
    Hemos probado que si $z_0 \in K$, $f \in \mathcal{F}$ y $z \in D\left(z_0, \frac{d}{2}\right) \subset D$, entonces $|f'(z)| \leq \frac{4M}{d}$.

    Ahora, sean $z_1, z_2 \in K$ y $f \in \mathcal{F}$.
    Supongamos que $|z_1-z_2| < \frac{d}{2}$.
    Si $\xi \in [z_1, z_2]$, entonces $z_2 \in D\left(z_1, \frac{d}{2}\right) \subset D$ y $|\xi-z_1| \leq |z_1-z_2| < \frac{d}{2}$, $\xi \in D\left(z_1, \frac{d}{2}\right)$.
    Entonces $\xi \in D$ y $|f'(\xi)| \leq \frac{4M}{d}$.
    Por tanto:
    $$|f(z_2)-f(z_1)| = \left|\int_{[z_1, z_2]} f'(\xi)d\xi\right| \leq |z_2-z_1| \max_{\xi \in [z_1, z_2]} |f'(\xi)| \leq |z_2-z_1| \frac{4M}{d}$$
    Entonces, si $z_1, z_2 \in K$, $|z_1-z_2| < \frac{d}{2}$ y $f \in \mathcal{F}$, se tiene que:
    $$|f(z_2)-f(z_1)| \leq A|z_2-z_1|$$

    Ahora, si $z_1, z_2 \in K$, $|z_2-z_1| \geq \frac{d}{2}$ y $f \in \mathcal{F}$, tenemos:
    $$|f(z_2)-f(z_1)| \leq |f(z_2)| + |f(z_1)| \leq 2M = 2M \frac{d}{2}\frac{2}{d} \leq \frac{4M}{d}|z_2-z_1| = A|z_2-z_1|$$
\end{proof}

\begin{theorem}[Teorema de Arzelá-Ascoli]
    Sean $(X_1, d_1)$ y $(X_2, d_2)$ dos espacios métricos, siendo $(X_1, d_1)$ separable y $(X_2, d_2)$ completo.
    Sea $\mathcal{F}$ una familia de aplicaciones continuas de $X_1$ en $X_2$ que verifica:
    \begin{enumerate}
        \item $\mathcal{F}$ es puntualmente equicontinua.
              Es decir, dado $x \in X_1$ se verifica que, para todo $\varepsilon > 0$, existe $\delta > 0$ tal que si $y \in X_1$ con $d_1(x, y) < \delta$, entonces $d_2(f(x), f(y)) < \varepsilon$ para toda $f \in \mathcal{F}$.
        \item Para todo $x \in X_1$, el conjunto $\{f(x) : f \in \mathcal{F}\}$ es relativamente compacto.
    \end{enumerate}
    Entonces, si $\{f_n\}_{n=1}^\infty$ es una sucesión en $\mathcal{F}$, existe una subsucesión $\{f_{n_k}\}_{k=1}^\infty$ de $\{f_n\}$ que converge uniformemente en cada subconjunto compacto de $X_1$.
\end{theorem}

\begin{theorem}[Teorema de Montel]
    Sea $D$ un abierto en $\mathbb{C}$ y sea $\mathcal{F}$ una familia de funciones holomorfas en $D$.
    Entonces son equivalentes:
    \begin{enumerate}
        \item $\mathcal{F}$ es finitamente normal.
        \item $\mathcal{F}$ está uniformemente acotada en cada subconjunto compacto de $D$.
              Es decir, para cada $K \subset D$, $K$ compacto, existe $M_k > 0$ tal que $|f(z)| \leq M_k$ para toda $f \in \mathcal{F}$ y para todo $z \in K$.
    \end{enumerate}
\end{theorem}

% Demostración (1 + 2)

\begin{remark}
    \hfill
    \begin{enumerate}
        \item Sea $D$ un abierto en $\mathbb{C}$.
              Si $\mathcal{F}$ es una familia finitamente normal de funciones holomorfas en $D$, entonces la familia $\mathcal{F}' = \{f' : f \in \mathcal{F}\}$ es finitamente normal.
              En general, si $k \in \mathbb{N}$, la familia $\mathcal{F}^{(k)} = \{f^{(k)} : f \in \mathcal{F}\}$ es finitamente normal.

              Sea $\{g_n\}_{n=1}^\infty$ en $\mathcal{F}'$.
              Entonces $g_n = f_n'$, $f_n \in \mathcal{F}$.
              Existe $\{f_{n_k}\}_{k=1}^\infty$ subsucesión de $\{f_n\}$ que converge uniformemente en cada subconjunto compacto de $D$ a una función $f$ holomorfa en $D$.
              Entonces $g_{n_k} = f_{n_k}' \to f'$ uniformemente en cada subconjunto compacto de $D$.

        \item Sea $D$ abierto en $\mathcal{C}$ y sea $\mathcal{G}$ familia finitamente normal de funciones holomorfas en $D$ con $\mathcal{F} \subset \mathcal{G}$.
              Entonces $\mathcal{F}$ es finitamente normal.
    \end{enumerate}
\end{remark}
\chapter{El teorema de Riemann de la aplicación conforme}

\section{Preliminares}
Recordemos algunos conceptos y resultados.

\begin{definition}
    Sea $D$ un dominio en $\mathbb{C}$ y sea $f$ una función holomorfa en $D$.
    \begin{itemize}
        \item $g$ es una rama de $\sqrt{f}$ en $D$ si $g: D \to \mathbb{C}$ es una función continua tal que $g(z)^2 = f(z)$ para todo $z \in D$.
        \item $g$ es una rama de $\log(f)$ en $D$ si $g: D \to \mathbb{C}$ es una función continua tal que $e^{g(z)} = f(z)$ para todo $z \in D$.
    \end{itemize}
\end{definition}

\begin{proposition}
    Sean $D$ un dominio en $\mathbb{C}$ y $f$ una función holomorfa y nunca nula en $D$.
    \begin{enumerate}
        \item Si $g$ es una rama de $\sqrt{f}$ en $D$, entonces $g$ es holomorfa en $D$ y $g'(z) = \frac{f'(z)}{2g(z)}$ para todo $z \in D$.
        \item Si $g$ es un rama de $\log(f)$ en $D$, entonces $g$ es holomorfa en $D$ y $g'(z) = \frac{f'(z)}{f(z)}$ para todo $z \in D$.
        \item Existe una rama de $\log(f)$ en $D$ si y solo si $\frac{f'}{f}$ tiene primitiva en $D$.
    \end{enumerate}
\end{proposition}

\begin{proposition}
    Sean $D$ un dominio en $\mathbb{C}$ y $f: D \to \mathbb{C}$ una función continua en $D$.
    Entonces $f$ tiene primitiva en $D$ si y solo si $\int_\gamma f(z)dz = 0$ para todo camino cerrado $\gamma$ en $D$.
\end{proposition}

\begin{definition}
    Si $D$ es un dominio en $\mathbb{C}$ y $\Gamma$ es un ciclo en $D$, se dice que $\Gamma$ es homólogo a cero módulo $D$, y se denota $\Gamma \sim 0 (mod D)$, si $n(\Gamma, a) = 0$ para todo $a \in \mathbb{C} \setminus D$.
\end{definition}

\begin{theorem}[Versión general del teorema de Cauchy]
    Sea $D$ un dominio en $\mathbb{C}$ y sea $\Gamma$ un ciclo en $D$.
    Las dos siguientes condiciones son equivalentes:
    \begin{enumerate}
        \item $\Gamma \sim 0 (mod D)$.
        \item $\int_\Gamma f(z)dz = 0$ para toda función $f$ holomorfa en $D$.
    \end{enumerate}
\end{theorem}

\section{Dominios simplemente conexos}
\begin{definition}
    Si $D$ es un dominio en $\mathbb{C}$, se dice que $D$ es simplemente conexo si $\mathbb{C}^\ast \setminus D$ es conexo.
\end{definition}

Hay una serie de caracterizaciones para los dominios simplemente conexos, que se pueden deducir de los resultados anteriores.

\begin{theorem}
    Sea $D$ un dominio en $\mathbb{C}$.
    Las siguientes condiciones son equivalentes:
    \begin{enumerate}
        \item $D$ es simplemente conexo.
        \item Todo ciclo en $D$ es homólogo a cero módulo $D$.
        \item Todo camino cerrado en $D$ es homólogo a cero módulo $D$.
        \item $\int_\Gamma f(z)dz = 0$ para toda $f$ holomorfa en $D$ y para todo ciclo $\Gamma$ en $D$.
        \item $\int_\gamma f(z)dz = 0$ para toda $f$ holomorfa en $D$ y para todo camino cerrado $\gamma$ en $D$.
        \item Toda función holomorfa en $D$ tiene primitiva.
        \item Para toda función $f$ holomorfa y nunca nula en $D$, existe una rama de $\log(f)$ en $D$.
        \item Para toda función $f$ holomorfa y nunca nula en $D$, existe una rama de $\sqrt{f}$ en $D$.
    \end{enumerate}
\end{theorem}

Recordamos que:
\begin{itemize}
    \item Dos dominios $D_1$ y $D_2$ en $\mathbb{C}^\ast$ son conformemente equivalentes si existe una aplicación conforme $f$ de $D_1$ sobre $D_2$.
    \item En el conjunto de los dominios en $\mathbb{C}^\ast$, el ser conformemente equivalentes es una relación de equivalencia.
    \item Si $D_1$ y $D_2$ son dos dominios en $\mathbb{C}^\ast$ que son conformemente equivalentes, entonces $D_1$ es simplemente conexo si y solo si $D_2$ es simplemente conexo.
    \item $\mathbb{C}^\ast$, $\mathbb{C}$ y el disco unidad $\mathbb{D} = \{z \in \mathbb{C} : |z| < 1\}$ son tres dominios simplemente conexos en $\mathbb{C}^\ast$, que no son conformemente equivalentes.
    \item El único dominio en $\mathbb{C}^\ast$ conformemente equivalente a $\mathbb{C}^\ast$ es $\mathbb{C}^\ast$.
\end{itemize}

Vamos a ver que, además de $\mathbb{C}\ast$, $\mathbb{C}$ y $\mathbb{D}$, no hay más dominios simplemente conexos en $\mathbb{C}^\ast$ módulo la relación de equivalencia.
Es decir, si $D$ es un dominio simplemente conexo en $\mathbb{C}^\ast$, entonces $D$ es conformemente equivalente a uno de los tres: $\mathbb{C}^\ast$, $\mathbb{C}$ o $\mathbb{D}$.
Por tanto se tendrá que, si $D$ es un dominio simplemente conexo en $\mathbb{C}$, con $D \neq \mathbb{C}$, entonces $D$ es conformemente equivalente a $\mathbb{D}$.

\begin{definition}
    Sea $D$ un dominio en $\mathbb{C}^\ast$.
    Llamamos automorfismos de $D$ a aquellas aplicaciones conformes de $D$ sobre $D$.
    El conjunto de todos los automorfismos de $D$ se denota $Aut(D)$, y tiene estructura de grupo con la composición.
\end{definition}

Tenemos que
\begin{align*}
    Aut(\mathbb{C}^\ast) & = \mathcal{M}                                                                                                                                                       \\
    Aut(\mathbb{C})      & = \{f_{\alpha, \beta} : f_{\alpha, \beta}(z) = \alpha z + \beta, \alpha, \beta \in \mathbb{C}, \alpha \neq 0\} = \{T \in \mathcal{M} : T(\mathbb{C}) = \mathbb{C}\} \\
    Aut(\mathbb{D})      & = \{\lambda T_a : \lambda \in \mathbb{C}, |\lambda| = 1, a \in \mathbb{D}\} = \{T \in \mathcal{M} : T(\mathbb{D}) = \mathbb{D}\}
\end{align*}

\begin{example}
    Veamos algunos ejemplos de dominios en $\mathbb{C}$ para los que podemos encontrar una aplicación conforme del dominio sobre $\mathbb{D}$.
    \begin{enumerate}
        \item Un disco abierto, $D(a, R)$, $a \in \mathbb{C}, R > 0$.
              \begin{align*}
                  \mathbb{D} & \to D(a, R)    \\
                  z          & \mapsto a + rz
              \end{align*}
        \item Un semiplano.
              \begin{align*}
                  \mathbb{D} & \to \mathbb{H} = \{z \in \mathbb{C} : \Re(z) > 0\} \\
                  z          & \mapsto P(z)
              \end{align*}
              donde $P(z) = \frac{1+z}{1-z}$, es una aplicación conforme.

              Componiendo con una rotación y una traslación, vemos que $\mathbb{D}$ es conformemente equivalente a cualquier semiplano.
              \begin{align*}
                  \mathbb{D} & \to \mathbb{C}              \\
                  z          & \mapsto a + e^{i\theta}P(z)
              \end{align*}
              con $a \in \mathbb{C}$ y $\theta \in \mathbb{R}$.
        \item El exterior de un disco, $\{z \in \mathbb{C} : |z-a| > R\} \cup \{\infty\}$, $a \in \mathbb{C}, R > 0$.
              \begin{align*}
                  D(a, R) & \to \{z \in \mathbb{C} : |z-a| > R\} \cup \{\infty\} \\
                  z       & \mapsto \frac{1}{z}
              \end{align*}
        \item El plano menos una semirrecta, $\mathbb{C} \setminus \{a + re^{i\theta}, r \geq 0\}$, $a \in \mathbb{C}, \theta \in \mathbb{R}$.
              \begin{align*}
                  \mathbb{H} & \to \mathbb{C} \setminus (-\infty, 0] \\
                  z          & \mapsto z^2
              \end{align*}
              es una aplicación conforme.
              Así que
              \begin{align*}
                  \mathbb{H} & \to \mathbb{C} \setminus (-\infty, 0]           \\
                  z          & \mapsto P(z)^2 = \left(\frac{1+z}{1-z}\right)^2
              \end{align*}
              es una aplicación conforme.

              Componiendo con una rotación y una traslación, vemos que $\mathbb{D}$ es conformemente equivalente al plano menos una semirrecta cualquiera.
        \item La función exponencial no es inyectiva.
              $$z = x + iy_0 \mapsto e^z = e^{x + iy_0} = e^x(\cos(y_0) + i\sin(y_0))$$
              Es inyectiva en cualquier banda horizontal abierta de amplitud menor o igual que $2\pi$.
              Por ejemplo,
              $$\exp: \left\{z \in \mathbb{C} : |\Im(z)| < \frac{\pi}{2}\right\} \to \mathbb{H}$$
              es una aplicación conforme.
              Como $\mathbb{H}$ es conformemente equivalente a $\mathbb{D}$, tenemos que esta banda es conformemente equivalente a $\mathbb{D}$.

              Componiendo con el producto por un número real, una rotación y una traslación, vemos que $\mathbb{D}$ es conformemente equivalente a cualquier banda.
        \item Sectores.
              $$\left\{z \in \mathbb{C} : |\Im(z)| < \frac{\alpha}{2}\right\} \xrightarrow{\exp} S$$
              donde $S$ es el sector de vértice 0 y amplitud $\alpha$, es una aplicación conforme.
        \item $\mathbb{D}^+ = \{z \in \mathbb{D} : \Im(z) > 0\}$.
              $$\mathbb{D}^+ \xrightarrow{P} \{z \in \mathbb{C} : \Re(z) > 0, \Im(z) > 0\}$$
              es una aplicación conforme. El dominio $\{z \in \mathbb{C} : \Re(z) > 0, \Im(z) > 0\}$ es un sector, así que es conformemente equivalente a $\mathbb{D}$.
    \end{enumerate}
\end{example}

\section{El teorema de Riemann de la aplicación conforme}
\begin{theorem}[Teorema de Riemann de la aplicación conforme]
    Sea $D$ un dominio simplemente conexo en $\mathbb{C}$ con $D \neq \mathbb{C}$ y sea $z_0 \in D$.
    Entonces existe una única aplicación conforme $f$ de $D$ sobre $\mathbb{D}$ tal que $f(z_0) = 0$ y $f'(z_0) > 0$.
\end{theorem}

\begin{remark}
    \hfill
    \begin{enumerate}
        \item Existen infinitas aplicaciones conformes de $D$ sobre $\mathbb{D}$.
              Basta cambiar el punto $z_0$ o componer con una rotación.
        \item Para la demostración, las condiciones
              \begin{enumerate}
                  \item $D$ simplemente conexo.
                  \item $D \neq \mathbb{C}$.
              \end{enumerate}
              solo las vamos a utilizar para deducir que:
              \begin{itemize}
                  \item $\mathbb{C} \setminus D$ tiene más de un punto.
                  \item Si $h$ es holomorfa y nunca nula en $D$, existe una rama de $\sqrt{h}$ en $D$.
              \end{itemize}
    \end{enumerate}
\end{remark}

\begin{theorem}
    Sea $D$ un dominio en $\mathbb{C}$ tal que:
    \begin{enumerate}
        \item $\mathbb{C} \setminus D$ tiene más de un punto.
        \item Para toda función $h$ holomorfa y nunca nula en $D$, existe una rama de $\sqrt{h}$ en $D$.
    \end{enumerate}
    Sea $z_0 \in D$.
    Entonces existe una única aplicación conforme $f$ de $D$ sobre $\mathbb{D}$ tal que $f(z_0) = 0$ y $f'(z_0) > 0$.
\end{theorem}


\begin{proof}
    Sea $\mathcal{F} = \{f : f \text{ es holomorfa e inyectiva en } D, f(D) \subset \mathbb{D}, f(z_0) = 0\}$.
    \begin{enumerate}
        \item Veamos que $\mathcal{F} \neq \emptyset$.
              Por (1), existen $a, b \in \mathbb{C} \setminus D$ con $a \neq b$.
              Sea $\varphi(z) = \frac{z-a}{z-b}$, $z \in D$.
              $$\begin{vmatrix}
                      1 & -a \\
                      1 & -b
                  \end{vmatrix} = -b + a \neq 0 \Rightarrow \varphi \in \mathcal{M}$$
              $\varphi$ es holomorfa e inyectiva en $D$ y $\varphi$ es nunca nula en $D$, porque $a, b \notin D$.
              Por (2), existe $\psi$ rama de $\sqrt{\varphi}$ en $D$.
              $\psi$ es holomorfa e inyectiva en $D$ y $\psi$ es nunca nula.

              Además, se tiene que si $w \in \psi(D)$, entonces $-w \notin \psi(D)$.
              Veámoslo.
              Supongamos que $w \in \psi(D)$ y $-w \in \psi(D)$.
              Entonces:
              \begin{align*}
                  w  & = \psi(z_1), \quad z_1 \in D \\
                  -w & = \psi(z_2), \quad z_2 \in D
              \end{align*}
              $$\psi(z_1)^2 = w^2 = (-w)^2 = \psi(z_2)^2 \Leftrightarrow \varphi(z_1) = \varphi(z_2) \Leftrightarrow z_1 = z_2 \Leftrightarrow w = -w \Leftrightarrow w = 0 \in \psi(D)$$
              Sin embargo, $\psi$ es nunca nula en $D$.

              Tomamos $w_0 \in \psi(D)$.
              Como $\psi(D)$ es abierto, existe $r > 0$ tal que $\overline{D}(0, r) \subset \psi(D)$.
              Entonces, si $z \in D$ se tiene que $\psi(z) \in \psi(D)$ y por tanto $-\psi(z) \notin \psi(D)$, de manera que $-\psi(z) \notin \overline{D}(w_0, r)$.
              Es decir,
              $$|-\psi(z)-w_0| > r \Leftrightarrow |\psi(z)+w_0| > r > 0 \Leftrightarrow \frac{r}{|\psi(z)+w_0} < 1$$
              Sea $h(z) = \frac{r}{\psi(z)+w_0}$, $z \in D$.
              $h$ es holomorfa e inyectiva en $D$ y $|h(z)| < 1$ para todo $z \in D$, luego $h(D) \subset \mathbb{D}$.
              Consideramos la transformación de Möbius $S_{h(z_0)}(z) = \frac{z-h(z_0)}{1-\overline{h(z_0)}z}$.
              Sabemos que $S_{h(z_0)}(\mathbb{D}) = \mathbb{D}$ y $S_{h(z_0)}(h(z_0)) = 0$.
              Por tanto, $f = S_{h(z_0)} \circ h \in \mathcal{F}$.

        \item $\mathcal{F}$ está uniformemente acotada en $D$.
              Por el teorema de Montel, $\mathcal{F}$ es finitamente normal.

        \item Sea $M = \sup_{f \in \mathcal{F}} |f'(z_0)|$, $0 \leq M \leq \infty$.
              Si $f \in \mathcal{F}$, $f$ es holomorfa e inyectiva en $D$, por lo que $f'(z_0) \neq 0$.
              Entonces $M \neq 0$.
              Como $\mathcal{F}' = \{f' : f \in \mathcal{F}\}$ es finitamente normal, entonces está uniformemente acotada en $\{z_0\}$.
              Entonces $\{f'(z_0) : f \in \mathcal{F}\}$ está acotado, así que $M = \sup_{f \in \mathcal{F}} |f'(z_0)| < \infty$.
              Por tanto, $0 < M < \infty$.

              Tomamos una sucesión $\{f_n\}_{n=1}^\infty$ en $\mathcal{F}$ tal que $\lim\limits_{n \to \infty} |f_n'(z_0)| = M$.
              Como $\mathcal{F}$ es finitamente normal, existe $\{f_{n_k}\}_{k=1}^\infty$ subsucesión de $\{f_n\}$ que converge a una función $F$ uniformemente en cada subconjunto comapcto de $D$.
              Entonces $F$ es holomorfa en $D$ y cada $f_{n_k}$ es holomorfa e inyectiva en $D$.
              Por el segundo teorema de Hurwitz, $F$ es inyectiva o constante.
              Como $f_{n_k}' \to F'$ uniformemente en cada subconjunto compacto de $D$, se tiene que $f_{n_k}'(z_0) \to F'(z_0)$, así que $|f_{n_k}'(z_0)| \to |F'(z_0)| = M > 0$.
              Luego $F$ no es constante.
              Entonces $F$ es inyectiva.
              Además, $F(z_0)= \lim\limits_{k \to \infty} f_{n_k}(z_0) = 0$ porque $f_{n_z}(z_0) = 0$.
              Si $z \in D$, $F(z) = \lim\limits_{k \to \infty} f_{n_k}(z)$, así que $|F(z)| = \lim\limits_{k \to \infty} |f_{n_z}(z)| \leq 1$ porque $|f_{n_z}(z)| < 1$.
              Pero si $|F(z)| = 1$ para algún $z \in D$, por el principio del máximo $F$ sería constante, lo cual es imposible.
              Por tanto, $|F(z)| < 1$ para todo $z \in D$, luego $F(D) \subset \mathbb{D}$.
              Entonces $F \in \mathcal{F}$ y $|F'(z_0)| = M$.

        \item Veamos que $F(D) = \mathbb{D}$.
              Supongamos por reducción al absurdo que existe $\alpha \in \mathbb{D} \setminus F(D)$.
              Consideramos $S_\alpha(z) = \frac{z-\alpha}{1-\bar{\alpha}z}$ transformación de Möbius con $S_\alpha(\mathbb{D}) = \mathbb{D}$ y $S_\alpha(\alpha) = 0$.
              Sea $h = S_\alpha \circ F$.
              $h$ es holomorfa e inyectiva en $D$.
              Además, $h$ es nunca nula en $D$ y $h(D) \subset \mathbb{D}$.
              Por (2), existe $g$ una rama de $\sqrt{h}$ en $D$, es decir, $g^2 = h$ en $D$.
              $g$ es holomorfa, inyectiva y nunca nula en $D$, con $g(D) \subset \mathbb{D}$.
              Sea $G = S_{g(z_0)} \circ g$.
              $G$ es holomorfa e inyectiva en $D$, con $G(D) \subset \mathbb{D}$ y $G(z_0) = 0$.
              Por tanto, $G \in \mathcal{F}$.

              Calculemos $|G'(z_0)|$.
              $$G'(z_0) = g'(z_0)S_{g(z_0)}'(g(z_0))$$
              En primer lugar, hallamos la derivada de $S_a$.
              $$S_a'(z) = \frac{1-\bar{a}z + (z-a)\bar{a}}{(1-\bar{a}z)^2} = \frac{1-|a|^2}{(1-\bar{a}z)^2}$$
              Observamos que $S_a'(a) = \frac{1}{1-|a|^2}$ y $S_a'(0) = 1-|a|^2$.
              Así que:
              $$G'(z_0) = g'(z_0)\frac{1}{1-|g(z_0)|^2} \Rightarrow |G'(z_0)| = \frac{|g'(z_0)|}{1-|g(z_0)|^2}$$
              Como $g^2 = h$ en $D$, también tenemos que $2gg' = h$ en $D$.
              Luego $|g(z_0)|^2 = |h(z_0)|$ y $2|g(z_0)||g'(z_0)| = |h'(z_0)|$.
              Entonces:
              $$|G'(z_0)| = \frac{|h'(z_0)|}{2|g(z_0)|}\frac{1}{1-|g(z_0)|^2} = \frac{|h'(z_0)|}{2\sqrt{|h(z)|}}\frac{1}{1-|h(z_0)|}$$
              Calculamos también:
              \begin{align*}
                  h'(z_0) & = F'(z_0)S_\alpha'(F(z_0)) = F'(z_0)S_\alpha'(0) = F'(z_0)(1-|\alpha|^2)   \\
                  h(z_0)  & = S_\alpha(F(z_0)) = S_\alpha(0) = -\alpha \Rightarrow |h(z_0)| = |\alpha|
              \end{align*}
              Por tanto:
              $$|G'(z_0)| = \frac{|F'(z_0)|(1-|\alpha|^2)}{2\sqrt{|\alpha|}(1-|\alpha|)} = M\frac{1-|\alpha|^2}{2\sqrt{|\alpha|}(1-|\alpha|)} = M\frac{1+|\alpha|}{2\sqrt{|\alpha|}}$$

              Veamos que $\frac{1+|\alpha|}{2\sqrt{|\alpha|}} > 1$.
              \begin{align*}
                   & \frac{1+|\alpha|}{2\sqrt{|\alpha|}} > 1 \Leftrightarrow 1+|\alpha| > 2\sqrt{|\alpha|} \Leftrightarrow 1+\alpha - 2\sqrt{|\alpha|} > 0 \Leftrightarrow (1-\sqrt{|\alpha|})^2 > 0 \Leftrightarrow 1-\sqrt{|\alpha|} \neq 0 \Leftrightarrow \\
                   & \Leftrightarrow \sqrt{|\alpha|} \neq 1 \Leftrightarrow |\alpha| \neq 1
              \end{align*}
              Como $\alpha \in \mathbb{D}$, la desigualdad se cumple.
              Por tanto, $|G'(z_0)| > M$, con $G \in \mathcal{F}$ y $M = \sup_{f \in \mathcal{F}} |f'(z_0)|$, luego llegamos a contradicción.
              Entonces, $F(D) = \mathbb{D}$.

        \item Tenemos $F \in \mathcal{F}$, $|F'(z_0)| = M$ y $F(D) = \mathbb{D}$.
              $F$ es una aplicación conforme de $D$ sobre $\mathbb{D}$ con $F(z_0) = 0$.
              Falta que $F'(z_0) > 0$.

              Queremos encontrar $\lambda \in \mathbb{C}$ con $|\lambda| = 1$ tal que $f = \lambda F$ verifique que $f'(z_0) > 0$.
              $$f'(z_0) = \lambda F'(z_0) > 0 \Leftrightarrow f'(z_0) = |\lambda||F'(z_0)| = |F'(z_0)| = M \Rightarrow \lambda = \frac{M}{F'(z_0)}$$
              Sea $\lambda = \frac{M}{F'(z_0)} \in \mathbb{C}$, con $|\lambda| = \frac{M}{|F'(z_0)|} = 1$, $F'(z_0) \neq 0$.
              Sea $f = \lambda F$.
              $f$ es holomorfa e inyectiva en $D$, con $f(D) = \mathbb{D}$, $f(z_0) = 0$ y $f'(z_0) = \lambda F'(z_0) = \frac{M}{F'(z_0)}F'(z_0) = M > 0$.

        \item Veamos que esta aplicación conforme es única.
              Supongamos $f_1, f_2: D \to \mathbb{D}$ aplicación conforme, con $f_j(z_0) = 0$, $f_j'(z_0) > 0$.
              Sea $g = f_1 \circ f_2^{-1}$.
              $$g: \mathbb{D} \xrightarrow{f_2^{-1}} D \xrightarrow{f_1} \mathbb{D}$$
              $g$ es una aplicación conforme de $\mathbb{D}$ sobre $\mathbb{D}$, así que es de la forma
              $$g(z) = \lambda T_a(z) = \lambda \frac{z+a}{1+\bar{a}z}, \quad \lambda \in \mathbb{C}, \; |\lambda| = 1, \; a \in \mathbb{D}$$
              Como $g(0) = 0$,
              $$g(0) = \lambda a = 0 \Rightarrow a = 0 \Rightarrow g(z) = \lambda z$$
              Como $g \circ f_2 = f_1$ en $D$,
              $$f_2'(z_0)g'(f_2(z_0)) = f_1'(z_0) \Leftrightarrow f_2'(z_0)g'(0) = f_1'(z_0) \Leftrightarrow g'(0) = \frac{f_1'(z_0)}{f_2'(z_0)} > 0$$
              Como $g'(0) = \lambda > 0$ y $|\lambda| = 1$, entonces $\lambda = 1$.
              Por tanto, $g(z) = z \Leftrightarrow f_1 = f_2$.
    \end{enumerate}
\end{proof}

\subsection*{Otros enunciados equivalentes}
\begin{theorem}[Teorema de Riemann]
    Sea $D$ un dominio simplemente conexo en $\mathbb{C}$ con $D \neq \mathbb{C}$ y sea $z_0 \in D$.
    Entonces existe una única aplicación conforme de $\mathbb{D}$ sobre $D$ tal que $f(0) = z_0$ y $f'(0) > 0$.
\end{theorem}


\begin{theorem}[Teorema de Riemann: enunciado equivalente]
    Sea $D$ un dominio simplemente conexo en $\mathbb{C}$ con $D \neq \mathbb{C}$ y sea $z_0 \in D$.
    Entonces existe un único $R > 0$ tal que existe una aplicación conforme $f$ de $D$ sobre $D(0, R)$ con $f(z_0) = 0$ y $f'(z_0) = 1$.
    Además, esta $f$ es única.

    A este número $R$ se le denomina radio conforme interior a $D$ en $z_0$ y se denota $r(D, z_0)$.
\end{theorem}

\begin{proof}
    Este enunciado es equivalente al teorema de Riemann.
    Si $D$ es un dominio simplemente conexo en $\mathbb{C}$ con $D \neq \mathbb{C}$ y $z_0 \in D$.
    \begin{enumerate}
        \item Si $f: D \to \mathbb{D}$ aplicación conforme, $f(z_0) = 0$ y $f'(z_0) > 0$, entonces si $R = \frac{1}{f'(z_0)} > 0$ se tiene que $g = Rf: D \to D(0, R)$ es una aplicación conforme con $g(z_0) = 0$ y $g'(z_0) = 1$.
        \item Si $R > 0$, $g: D \to D(0, R)$ es una aplicación conforme con $g(z_0) = 0$ y $g'(z_0) = 1$, entonces $f = \frac{1}{R}g: D \to \mathbb{D}$ es una aplicación conforme con $f(z_0) = 0$ y $f'(z_0) = \frac{1}{R}g'(z_0) = \frac{1}{R} > 0$.
    \end{enumerate}
\end{proof}

\begin{remark}
    Hemos visto que cualquier dominio $D$ simplemente conexo en $\mathbb{C}$ con $D \neq \mathbb{C}$ es conformemente equivalente a $\mathbb{D}$.
    Entonces, si $D_1$ y $D_2$ son dominios simplemente conexos en $\mathbb{C}$ con $D_1, D_2 \neq \mathbb{C}$, $D_1$ y $D_2$ son conformemente equivalentes.
\end{remark}

En $\mathbb{C}^\ast$ tenemos el siguiente resultado.

\begin{theorem}
    Sea $D$ un dominio simplemente conexo en $\mathbb{C}^\ast$ tal que $\mathbb{C}^\ast \setminus D$ tiene más de un punto.
    Entonces $D$ es conformemente equivalente a $\mathbb{D}$.
\end{theorem}

\begin{proof}
    \hfill
    \begin{itemize}
        \item Si $D \subset \mathbb{C}$, entonces $D$ es un dominio simplemente conexo en $\mathbb{C}$ con $D \neq \mathbb{C}$, ya que $\mathbb{C}^\ast \setminus D$ tiene más de un punto.
              Entonces $D$ es conformemente equivalente a $\mathbb{D}$.
        \item Si $\infty \in D$, entonces tomamos $a, b \in \mathbb{C}^\ast \setminus D$ con $a \neq b$.
              Entonces $a, b \in \mathbb{C} \setminus D$.
              Sea $T(z) = \frac{1}{z-a}$, $z \in \mathbb{C} \setminus \{a\}$, con $T(a) = \infty$ y $T(\infty) = 0$.
              $T: \mathbb{C}^\ast \to \mathbb{C}\ast$ es una aplicación conforme.
              Entonces $D \xrightarrow{T} T(D) = D'$ es una aplicación conforme y $D'$ es un dominio simplemente conexo en $\mathbb{C}^\ast$.

              Como $a, b \notin D$, $T(a) = \infty \notin D'$, así que $D'$ es un dominio en $\mathbb{C}$.
              Además, $T(b) \notin D'$ con $T(b) \in \mathbb{C}$, luego $D \neq \mathbb{C}$.
              $D'$ es un dominio simplemente conexo en $\mathbb{C}$, con $D' \neq \mathbb{C}$.
              Por tanto, $D'$ es conformemente equivalente a $\mathbb{D}$.
              Como $D'$ es conformemente equivalente a $D$, entonces $D$ es conformemente equivalente a $\mathbb{D}$.
    \end{itemize}
\end{proof}
\chapter{Inferencia bayesiana}
\section{Teorema de Bayes}
\begin{theorem}[Teorema de Bayes]
    Sea $(\Omega, \mathcal{A}, P)$ un espacio de probabilidad.
    Sea $\{A_1, \dots, A_n\} \subset \mathcal{A}$ una partición de $\Omega$ y sea $B \in \mathcal{A}$ tal que $P(B) > 0$ y del que se conocen $P(B|A_i)$, para $i = 1, \dots, n$.
    Entonces
    $$P(A_i|B) = \frac{P(B|A_i)P(A_i)}{\sum_{j=1}^n P(B|A_j)P(A_j)}, \quad \forall i = 1, \dots, n$$
    donde:
    \begin{itemize}
        \item $P(A_j)$, $j = 1, \dots, n$, se llaman probabilidades a priori.
        \item $P(B|A_j)$, $j = 1, \dots, n$, se llaman verosimilitudes.
        \item $P(A_j|B)$, $j= 1, \dots, n$, se llaman probabilidades a posteriori.
    \end{itemize}

    Esta se conoce como la fórmula de Bayes.
\end{theorem}

\begin{remark}
    Las probabilidades a posteriori son proporcionales al producto de verosimilitudes y probabilidades a priori.
    $$P(A_i|B) \propto P(B|A_i)P(A_i)$$
\end{remark}

\begin{example}
    Una caja contiene dos monedas: una moneda legal $M_1$ y otra con una cara en cada lado $M_2$.

    En primer lugar, se selecciona una de las dos monedas al azar, se lanza y sale cara.
    Veamos cuál es la probabilidad de que la moneda lanzada sea la legal.

    Para ello definimos los sucesos:
    \begin{itemize}
        \item $C_i$: en el lanzamiento $i$ sale cara.
        \item $F_i$: en el lanzamiento $i$ sale cruz.
    \end{itemize}

    Usamos el teorema de Bayes:
    \begin{center}
        \begin{tabular}{| c | c | c |}
            \hline
            Probabilidad a priori  & Verosimilitudes            & Probabilidad a posteriori  \\
            \hline
            $P(M_1) = \frac{1}{2}$ & $P(C_1|M_1) = \frac{1}{2}$ & $P(M_1|C_1) = \frac{1}{3}$ \\
            $P(M_2) = \frac{1}{2}$ & $P(C_1|M_2) = 1$           & $P(M_2|C_1) = \frac{2}{3}$ \\
            \hline
        \end{tabular}
    \end{center}

    Lanzamos de nuevo la moneda elegida y se obtiene otra cara.
    Veamos cuál es la probabilidad de que la moneda lanzada sea la legal.

    Podemos usar el carácter secuencial del teorema de Bayes y usar los resultados anteriores.

    \begin{center}
        \begin{tabular}{| c | c | c |}
            \hline
            Probabilidad a priori      & Verosimilitudes            & Probabilidad a posteriori           \\
            \hline
            $P(M_1|C_1) = \frac{1}{3}$ & $P(C_2|M_1) = \frac{1}{2}$ & $P(M_1|C_1 \cap C_2) = \frac{1}{5}$ \\
            $P(M_2|C_1) = \frac{2}{3}$ & $P(C_2|M_2) = 1$           & $P(M_2|C_1 \cap C_2) = \frac{4}{5}$ \\
            \hline
        \end{tabular}
    \end{center}
\end{example}

\section{Teorema de Bayes generalizado}
\begin{theorem}[Teorema de Bayes generalizado]
    Sean $\vec{x} = (x_1, \dots, x_n)$ una muestra y $\theta$ una variable aleatoria.
    Sea $f_\theta$ la distribución a priori y $f(\vec{x}|\theta)$ la función de verosimilitud.
    Entonces:
    $$f(\theta|\vec{x}) = \frac{f(\vec{x}|\theta)f_\theta(\theta)}{f(\vec{x})}$$
    donde:
    $$f(\vec{x}) = \begin{cases}
            \sum_{i=1}^n f(\vec{x}|\theta_i)f_\theta(\theta_i)    & \text{si es discreta} \\
            \int_\Theta f(\vec{x}, \theta)f_\theta(\theta)d\theta & \text{si es continua}
        \end{cases}$$
\end{theorem}

\begin{remark}
    Para los clásicos, $\theta$ es un parámetro fijo y desconocido.
    En cambio, para los bayesianos $\theta$ es una variable aleatoria.
\end{remark}

\begin{example}
    Supongamos que tenemos una moneda y queremos estimar la probabilidad $p$ de obtener cara.
    Supongamos que nuestras creencias a priori sobre $p$ se pueden describir por una distribución uniforme en $(0, 1)$.
    Realizamos el experimento de tirar la moneda 12 veces y obtenemos 9 caras y 3 cruces.

    Definimos la variable aleatoria:
    $$X = \begin{cases}
            1 & \text{si sale cara } (C) \\
            0 & \text{si sale cruz } (F)
        \end{cases}, \quad X|p \sim Ber(p)$$
    Queremos estimar $P(C) = P(X = 1) = p$ a partir de nuestra muestra $\vec{x} = (x_1, \dots, x_{12})$, com $\sum_{i=1}^{12} x_i = 9$.

    Como $p \sim U(0, 1)$, su distribución a priori es $f(p) = 1$ si $p \in (0, 1)$.
    Calculamos la función de verosimilitud:
    $$L(\vec{x}, p) = \prod_{i=1}^{12} f(x_i|p) = \prod_{i=1}^{12} p^{x_i}(1-p)^{1-x_i} = p^9(1-p)^3$$
    Podemos hallar la distribución a posteriori:
    $$f(p|\vec{x}) \propto p^9(1-p)^3 \Rightarrow p|\vec{x} \sim Be(10, 4)$$
\end{example}

\begin{note}
    Si $X \sim Be(p, q)$ beta, entonces:
    $$f_X(x) \propto x^{p-1}(1-x)^{q-1}$$
\end{note}

\section{Familias de distribución conjugadas}
Las familias de distribución conjugadas son aquellas en las que las distribuciones a priori y a posteriori son de la misma familia.

\subsection*{Muestras de la distribución Bernoulli}
Sean $x_i|\theta \sim Ber(\theta)$ y $\theta \sim Be(p, q)$.
Su distribución a priori es:
$$f_\theta(\theta) \propto \theta^{p-1}(1-\theta)^{q-1}, \quad \theta \in (0, 1)$$
Dada una muestra $\vec{x}$, calculamos la función de verosimilitud:
$$L(\vec{x}, \theta) = \prod_{i=1}^n f(x_i|\theta) \propto \theta^{\sum_{i=1}^n x_i}(1-\theta)^{n-\sum_{i=1}^n x_i}$$
Luego la distribución a posteriori es:
\begin{align*}
    f(\theta|\vec{x}) & \propto f_\theta(\theta)L(\vec{x}, \theta) \propto \theta^{p+\sum_{i=1}^n x_i-1}(1-\theta)^{n+q-\sum_{i=1}^n x_i+1} \\
                      & \Rightarrow \theta|\vec{x} \sim Be\left(p+\sum_{i=1}^n x_i, n+q-\sum_{i=1}^n x_i\right)
\end{align*}
Por tanto, la beta es una familia conjugada respecto de muestras de la Bernoulli.

\subsection*{Muestras de la distribución de Poisson}
Sean $x_i|\lambda \sim Po(\lambda)$ y $\lambda \sim Ga(a, p)$.
Su distribución a priori es:
$$f_\lambda(\lambda) = \frac{a^p}{\Gamma(p)}e^{-a\lambda}\lambda^{p-1}, \quad \lambda, a, p > 0$$
Dada una muestra $\vec{x}$, calculamos la función de verosimilitud:
$$L(\vec{x}, \lambda) = \prod_{i=1}^n f(x_i|\lambda) \propto \prod_{i=1}^n e^{-\lambda}\lambda^{x_i} = e^{-n\lambda}\lambda^{\sum_{i=1}^n x_i}$$
Luego la distribución a posteriori es:
\begin{align*}
    f(\lambda|\vec{x}) & \propto f_\lambda(\lambda)L(\vec{x}, \lambda) \propto e^{-(a+n)\lambda}\lambda^{\sum_{i=1}^n x_i+p-1} \\
                       & \Rightarrow \lambda|\vec{x} \sim Ga(a+n, p+\sum_{i=1}^n x_i)
\end{align*}
Por tanto, la gamma es una familia conjugada respecto de muestras de la Poisson.

\subsection*{Muestras de la distribución normal}
\begin{lemma}
    $$A(z-a)^2 + B(z-b)^2 = (A+B)\left(z - \frac{Aa+Bb}{A+B}\right)^2 + \frac{AB}{A+B}(a-b)^2$$
\end{lemma}

\subsubsection*{Media desconocida y precisión conocida}
Sea $x_i|\mu \sim N(\mu, p)$ con media $\mu$ desconocida y precisión $p$ conocida y sea $\mu \sim N(m_0, p_0)$.
Su distribución a priori es:
$$f_\mu(\mu) = \frac{\sqrt{p_0}}{\sqrt{2\pi}} e^{\frac{p_0}{2}(\mu-m_0)^2} \propto e^{-\frac{p_0}{2}(\mu-m_0)^2}$$
Dada una muestra $\vec{x}$, calculamos la función de verosimilitud:
$$L(\vec{x}, \mu) = \prod_{i=1}^n f(x_i|\mu) \propto \prod_{i=1}^n e^{-\frac{p}{2}(x_i-\mu)^2} = e^{-\frac{p}{2}\sum_{i=1}^n (x_i-\mu)^2}$$

\begin{note}
    \begin{align*}
        \sum_{i=1}^n (x_i-\mu)^2 & = \sum_{i=1}^n (x_i-\bar{x}+\bar{x}-\mu)^2 =                                                   \\
                                 & = \sum_{i=1}^n (x_i-\bar{x})^2 + n(\bar{x}-\mu)^2 + 2(\bar{x}-\mu)\sum_{i=1}^n (x_i-\bar{x}) = \\
                                 & = (n-1)s^2 + n(\bar{x}-\mu)^2
    \end{align*}
\end{note}

Así que:
$$L(\vec{x}, \mu) = e^{-\frac{p}{2}((n-1)s^2+n(\bar{x}-\mu)^2)} \propto e^{-\frac{np}{2}(\bar{x}-\mu)^2}$$
Luego la distribución a posteriori es:
$$f(\mu|\vec{x}) \propto f_\mu(\mu)L(\vec{x}, \mu) \propto e^{-\frac{p}{2}(\mu-m_0)^2}e^{-\frac{np}{2}(\bar{x}-\mu)^2} = e^{-\frac{1}{2}(p_0(\mu-m_0)^2+np(\bar{x}-\mu)^2)}$$
Usando el lema previo, queda:
\begin{align*}
    f(\mu, \vec{x}) & \propto e^{-\frac{1}{2}(a(\mu-b)^2+c)} \propto e^{-\frac{a}{2}(\mu-b)^2} \\
                    & \Rightarrow \mu|\vec{x} \sim N(b, pr = a)
\end{align*}
donde
$$a = p_0+np, \quad b = \frac{p_0m_0+np\bar{x}}{p_0+np}$$
Por tanto, la normal es una familia conjugada respecto de muestras de la normal con media desconocida y precisión conocida.

\subsubsection*{Media conocida y precisión desconocida}
Sea $x_i|\tau \sim N(\mu, \tau)$ y sea $\tau \sim Ga(a_0, p_0)$.
Su distribución a priori es:
$$f_\tau(\tau) = \frac{a_0^p}{\Gamma(p_0)}e^{-a_0\tau}\tau^{p_0-1}, \quad \tau, a_0, p_0 > 0$$
Dada una muestra $\vec{x}$, calculamos la función de verosimilitud:
$$L(\vec{x}, \tau) = \prod_{i=1}^n f(x_i|\tau) \propto \prod_{i=1}^n \sqrt{\tau}e^{-\frac{\tau}{2}(x_i-\mu)^2} = \tau^{\frac{n}{2}}e^{-\frac{\tau}{2}\sum_{i=1}^n(x_i-\mu)^2}$$
Luego la distribución a posteriori es:
\begin{align*}
    f(\tau|\vec{x}) & \propto f_\tau(\tau)L(\vec{x}, \tau) \propto \tau^{\frac{n}{2}+p_0-1}e^{-\tau\left(a_0+\frac{1}{2}\sum_{i=1}^n(x_i-\mu)^2\right)} \\
                    & \Rightarrow \tau|\vec{x} \sim Ga(a_n, p_n)
\end{align*}
donde
$$a_n = a_0 + \frac{1}{2}\sum_{i=1}^n(x_i-\mu)^2, \quad p_n = \frac{n}{2}+p_0$$
Por tanto, la gamma es una familia conjugada respecto de muestras de la normal con media conocida y precisión desconocida.

\begin{definition}
    Sea $X \sim Ga(a, p)$, consideramos $Y = \frac{1}{X}$.
    Entonces $Y \sim GaI(a, p)$ gamma invertida.
    Su función de densidad es:
    $$f_Y(y) = \frac{a^p}{\Gamma(p)} e^{-\frac{a}{y}} y^{-(p+1)}, \quad y > 0$$
\end{definition}

\subsubsection*{Media conocida y varianza desconocida}
Sea $x_i|\sigma^2 \sim N(\mu, \sigma^2)$ con varianza $\sigma^2$ desconocida y sea $\sigma^2 \sim GaI(a_0, p_0)$.
Su distribución a priori es:
$$f_{\sigma^2}(\sigma^2) = \frac{a_0^{p_0}}{\Gamma(p_0)}e^{-\frac{a_0}{\sigma^2}}(\sigma^2)^{-(p_0+1)}, \quad a_0, p_0 > 0$$
Dada una muestra $\vec{x}$, calculamos la función de verosimilitud:
$$L(\vec{x}, \sigma^2) = \prod_{i=1}^n f(x_i|\sigma^2) \propto \prod_{i=1}^n \frac{1}{\sqrt{\sigma^2}}e^{-\frac{1}{2\sigma^2}(x_i-\mu)^2} = (\sigma^2)^{-\frac{n}{2}}e^{-\frac{1}{2\sigma^2}\sum_{i=1}^n(x_i-\mu)^2}$$
Luego la distribución a posteriori es:
\begin{align*}
    f(\sigma^2|\vec{x}) & \propto f_{\sigma^2}(\sigma^2)L(\vec{x}, \sigma^2) \propto (\sigma^2)^{-\left(p_0+\frac{n}{2}+1\right)}e^{-\frac{1}{\sigma^2}}\left(a_0+\frac{1}{2}\sum_{i=1}^n(x_i-\mu)^2\right) \\
                        & \Rightarrow \sigma^2|\vec{x} \sim GaI(a_n, p_n)
\end{align*}
donde
$$a_n = a_0 + \frac{1}{2}\sum_{i=1}^n(x_i-\mu)^2, \quad p_n = p_0 + \frac{n}{2}$$
Por tanto, la gamma invertida es una familia conjugada respecto de muestras de la normal con media conocida y varianza desconocida.

\begin{definition}
    Decimos que $(\mu, \tau) \sin NGa(m_0, \tau_0, a_0, p_0)$ normal gamma, con $m_0 \in \mathbb{R}, \tau_0, a_0, p_0 > 0$, si:
    $$\mu|\tau \sim N(m_0, pr = \tau\tau_0) \text{ y } \tau \sim Ga(a_0, p_0), \quad \mu \in \mathbb{R}, \tau > 0$$
    Su función de densidad es:
    $$f(\mu, \tau) = \frac{\sqrt{\tau_0}}{\sqrt{2\pi}} \frac{a_0^{p_0}}{\Gamma(p_0)} \tau^{p_0 - \frac{1}{2}} e^{-\tau \left(a_0 + \frac{\tau_0}{2}(\mu-m_0)^2\right)}$$
\end{definition}

\begin{definition}
    Si $T \sim t_n$ y $X = \mu + \frac{1}{\sqrt{p}}T$, entonces $X \sim t(\mu, p, n)$, donde $\mu$ es la media y $p$ es el parámetro de escala.
    Su función de densidad es:
    $$f_X(x) = \frac{\Gamma\left(\frac{n+1}{2}\right)\sqrt{p}}{\Gamma\left(\frac{1}{2}\right)\Gamma\left(\frac{n}{2}\right)\sqrt{n}} \left(1 + \frac{p}{n}(x-\mu)^2\right)^{-\frac{n+1}{2}}$$
    Verifica que:
    $$E(X) = \mu, \quad V(X) = \frac{1}{p} \frac{n}{n-2}$$
\end{definition}

\begin{remark}
    La distribución $t_1$ se llama distribución de Cauchy.
    Además:
    $$t_n \xrightarrow[n \to \infty]{} N(0, 1)$$
\end{remark}

\begin{theorem}
    Si $(\mu, \tau) \sim NGa(m_0, \tau_0, a_0, p_0)$, entonces:
    $$\mu \sim t\left(m_0, \frac{p_0\tau_0}{a_0}, 2p_0\right)$$
\end{theorem}

\begin{corollary}[Génesis bayesiana de la $t$ de Student]
    $$\begin{cases}
            \mu|\tau \sim N(0, pr = \tau) \\
            \tau \sim Ga\left(\frac{n}{2}, \frac{n}{2}\right)
        \end{cases} \Rightarrow \mu \sim t(0, 1, n) \equiv t_n$$
\end{corollary}

\subsubsection*{Media y precisión desconocidas}
Sean $x_i|\mu, \tau \sim N(\mu, \tau)$ y $(\mu, \tau) \sim NGa(m_0, \tau_0, a_0, p_0)$.
Se puede comprobar que la normal gamma es una familia conjugada respecto de muestras de la normal con media y precisión desconocidas.
% Completar

\section{Distribuciones a priori no informativas}
\begin{definition}
    La información de Fisher para $\theta$ se define como:
    $$J(\theta) = -E\left(\frac{\partial^2 \log(f(x|\theta))}{\partial\theta^2}\right)$$
\end{definition}

\begin{proposition}[Regla de Jeffreys]
    $$f_\theta(\theta) \propto \sqrt{J(\theta)}$$
\end{proposition}

\begin{remark}
    La reglas de Jeffreys no da densidades en general.
    A aquellas que no son densidades se les llama densidades impropias.
\end{remark}
\chapter{Introducción a la inferencia no paramétrica}
\section{Introducción}
Hasta ahora los tests de hipótesis han sido utilizados para contrastar la veracidad de hipótesis acerca de los parámetros de la distribución de una población.
Sin embargo, en muchas ocasiones, es necesario emitir un juicio estadístico sobre la distribución poblacional en su conjunto.
Los problemas de este tipo que se plantean de manera habitual son los siguientes:
\begin{itemize}
    \item \textbf{Contrastes de bondad de ajuste.}
          Decidir, a la vista de una muestra aleatoria de una población, si puede admitirse que la distribución poblacional coincide con una cierta distribución dada o pertenece a un determinado tipo de distribuciones.
    \item \textbf{Contrastes de homogeneidad.}
          Analizar si varias muestras aleatorias provienen de poblaciones con la misma distribución teórica, de forma que puedan ser utilizadas conjuntamente para inferencias posteriores acerca de esta; o, por el contrario, son muestras de poblaciones con distinta distribución, que no pueden agruparse como información homogénea acerca de una única distribución.
    \item \textbf{Contrastes de independencia.}
          Estudiar, en el caso en que se observen dos o más características de los elementos de la población, si las características observadas pueden considerarse independientes, y se puede proceder a su análisis por separado; o, por el contrario, existe relación estadística entre ellas.
\end{itemize}

\section{Contrastes de bondad de ajuste}
\subsection*{Primer caso}
Consideremos una muestra aleatoria $(X_1, \dots, X_n)$ de una variable aleatoria $X$ con distribución desconocida.
Para decidir si es razonable admitir que la distribución de $X$ viene dada por un determinado modelo de probabilidad $P$, resolvemos el siguiente contraste de hipótesis:
$$\begin{cases}
        H_0: \text{ El modelo de probabilidad de } X \text{ es } P \\
        H_1: \text{ El modelo de probabilidad de } X \text{ no es } P
    \end{cases}$$

\begin{note}
    El modelo $P$ debe estar completamente especificado.
\end{note}

Para contrastar $H_0$ frente a $H_1$ hacemos una partición arbitraria del espacio muestral de la población en $k$ clases $A_1, \dots, A_k$.
Después, para cada $A_i$ consideramos las siguientes frecuencias absolutas:
\begin{itemize}
    \item $O_i$: frecuencia observada en $A_i$, número de elementos de la muestra que están en la clase $A_i$.
    \item $e_i$: frecuencia esperada de la clase $A_i$ si la hipótesis $H_0$ es cierta, $nP(A_i)$.
\end{itemize}

El estadístico que utilizaremos es:
$$\sum_{i=1}^k \frac{(O_i-e_i)^2}{e_i}$$
que tiene aproximadamente cuando $n$ es grande una distribución $\chi^2_{k-1}$ si $H_0$ es cierta.

Para que la aproximación sea razonablemente buena, además de tener una muestra suficientemente grande, es necesario que el valor esperado de cada clase sea suficientemente grande.
Si la muestra procede de $P$, es de esperar que haya valores parecidos para $O_i$ y $e_i$ y, por tanto, este estadístico debería tomar valores próximos a cero.

Rechazaremos $H_0$ a nivel de significación $\alpha$ si:
$$\sum_{i=1}^k \frac{(O_i-e_i)^2}{e_i} > \chi^2_{k-1, 1-\alpha}$$
En caso contrario, aceptaremos $H_0$ a nivel de significación $\alpha$.

\begin{example}
    Después de lanzar un dado 300 veces, se obtienen las siguientes frecuencias:
    \begin{center}
        \begin{tabular}{ | l | c c c c c c | }
            \hline
            Resultado  & 1  & 2  & 3  & 4  & 5  & 6  \\
            \hline
            Frecuencia & 43 & 49 & 56 & 45 & 66 & 41 \\
            \hline
        \end{tabular}
    \end{center}

    Veamos si se puede afirmar que el dado es regular a nivel de significación $\alpha = 0.05$.

    Disponemos de una muestra aleatoria de $n = 300$ lanzamientos de un dado.
    Para decidir si el dado es regular o no, llevamos a cabo un contraste de bondad de ajuste a nivel de significación $\alpha$:
    $$\begin{cases}
            H_0: \text{ El dado es regular: } P(1) = \dots = P(6) = \frac{1}{6} \\
            H_1: \text{ El dado es irregular}
        \end{cases}$$

    La tabla de frecuencias observadas y esperadas es:
    \begin{center}
        \begin{tabular}{| c | c c c c c c |}
            \hline
            $A_i$ & 1  & 2  & 3  & 4  & 5  & 6  \\
            \hline
            $O_i$ & 43 & 49 & 56 & 45 & 66 & 41 \\
            $e_i$ & 50 & 50 & 50 & 50 & 50 & 50 \\
            \hline
        \end{tabular}
    \end{center}
    donde las frecuencias esperadas bajo $H_0$ han sido calculadas como:
    $$e_i = nP(A_i) = 300 \frac{1}{6} = 50$$

    Usaremos el estadístico de contraste:
    $$\sum_{i=1}^k \frac{(O_i-e_i)^2}{e_i}$$
    que tiene aproximadamente una distribución $\chi^2_{k-1}$ si $H_0$ es cierta.

    Rechazaremos $H_0$ a nivel de significación $\alpha$ si:
    $$\sum_{i=1}^k \frac{(O_i-e_i)^2}{e_i} > \chi^2_{k-1, 1-\alpha}$$
    En nuestro caso:
    $$\sum_{i=1}^6 \frac{(O_i-e_i)^2}{e_i} = 8.96, \quad \chi^2_{5, 0.95} = 11.07$$

    Por tanto, aceptamos $H_0$ a nivel de significación $\alpha = 0.05$, es decir, aceptamos que el dado es regular a nivel de significación $\alpha = 0.05$.
\end{example}

\begin{example}
    Nos dicen que un programa de ordenador genera observaciones de una distribución $N(0, 1)$.
    Como no estamos seguros de ello, obtenemos una muestra aleatoria de 450 observaciones mediante dicho programa, obteniendo los siguientes resultados:
    \begin{itemize}
        \item 30 observaciones menores que -2.
        \item 80 observaciones entre -2 y -1.
        \item 140 observaciones entre -1 y 0.
        \item 110 observaciones entre 0 y 1.
        \item 60 observaciones entre 1 y 2.
        \item 30 observaciones mayores que 2.
    \end{itemize}
    Veamos si se puede aceptar que el programa funciona correctamente a nivel de significación $\alpha = 0.01$.

    Disponemos de una muestra aleatoria de $n = 450$ observaciones generadas por el programa.
    Los posibles resultados de estas observaciones se agrupan en seis clases:
    \begin{align*}
        A_1 & = (-\infty, -2) & A_2 & = (-2, -1)    \\
        A_3 & = (-1, 0)       & A_4 & = (0, 1)      \\
        A_5 & = (1, 2)        & A_6 & = (2, \infty)
    \end{align*}
    Para decidir si el programa funciona correctamente o no, llevamos a cabo un contraste de bondad de ajuste a nivel de significación $\alpha$:
    $$\begin{cases}
            H_0: \text{ El programa funciona correctamente: provienen de } N(0, 1) \\
            H_1: \text{ El programa no funciona correctamente}
        \end{cases}$$

    La tabla de frecuencias observadas y esperadas es:
    \begin{center}
        \begin{tabular}{| c | c c c c c c |}
            \hline
            $A_i$    & $(-\infty, -2)$ & $(-2, -1)$ & $(-1, 0)$ & $(0, 1)$ & $(1, 2)$ & $(2, \infty)$ \\
            \hline
            $O_i$    & 30              & 80         & 140       & 110      & 60       & 30            \\
            $P(A_i)$ & 0.0228          & 0.1359     & 0.3413    & 0.3413   & 0.1359   & 0.0228        \\
            $e_i$    & 10.26           & 61.155     & 153.585   & 154.585  & 61.155   & 10.26         \\
            \hline
        \end{tabular}
    \end{center}
    donde las frecuencias esperadas bajo $H_0$ han sido calculadas de la forma:
    $$e_i = nP(A_i) = 450P(A_i)$$
    y los valores $P(A_i)$ se han calculado a partir de la tabla de la función de distribución de la normal estándar.

    Usaremos el estadístico de contraste:
    $$\sum_{i=1}^k \frac{(O_i-e_i)^2}{e_i}$$
    que tiene aproximadamente una distribución $\chi^2_{k-1}$ si $H_0$ es cierta.

    Rechazaremos $H_0$ a nivel de significación $\alpha$ si:
    $$\sum_{i=1}^k \frac{(O_i-e_i)^2}{e_i} > \chi^2_{k-1, 1-\alpha}$$
    En nuestro caso:
    $$\sum_{i=1}^6 \frac{(O_i-e_i)^2}{e_i} = 95.358, \quad \chi^2_{5, 0.99} = 15.086$$

    Por tanto, rechazamos $H_0$ a nivel de significación $\alpha = 0.01$, es decir, podemos afirmar que el programa no funciona correctamente a nivel de significación $\alpha = 0.01$.
\end{example}

\subsection*{Segundo caso}
El contraste de bondad de ajuste se puede plantear también en una situación más general.
Consideremos una muestra aleatoria $(X_1, \dots, X_n)$ de una variable aleatoria $X$ con distribución desconocida.
Para decidir si es razonable admitir que la distribución de $X$ viene dada por algún modelo de probabilidad de una cierta familia $P_\theta$, con $\theta = (\theta_1, \dots, \theta_r)$, resolveremos el siguiente contraste de hipótesis:
$$\begin{cases}
        H_0: \text{ El modelo de probabilidad } X \text{ es de la familia } \{P_\theta: \theta \in \Theta\} \\
        H_1: \text{ El modelo de probabilidad } X \text{ no es de la familia } \{P_\theta: \theta \in \Theta\}
    \end{cases}$$

Para contrastar $H_0$ frente a $H_1$ hacemos nuevamente una partición arbitraria del espacio muestral de la población en $k$ clases $A_1, \dots, A_k$.
Después, para cada $A_i$ consideramos las siguientes frecuencias absolutas:
\begin{itemize}
    \item $O_i$: frecuencia observada en $A_i$, número de elementos de la muestra que están en la clase $A_i$.
    \item $e_i$: frecuencia esperada de la clase $A_i$ si la hipótesis $H_0$ es cierta, $nP_\theta(A_i) \approx nP_{\hat{\theta}}(A_i)$, donde $\hat{\theta}$ es el estimador de máxima verosimilitud de $\theta$.
\end{itemize}

El estadístico que utilizaremos es:
$$\sum_{i=1}^k \frac{(O_i-e_i)^2}{e_i}$$
que tiene aproximadamente cuando $n$ es grande una distribución $\chi^2_{k-r-1}$ si $H_0$ es cierta.

Rechazaremos $H_0$ a nivel de significación $\alpha$ si:
$$\sum_{i=1}^k \frac{(O_i-e_i)^2}{e_i} > \chi^2_{k-r-1, 1-\alpha}$$
En caso contrario, aceptaremos $H_0$ a nivel de significación $\alpha$.

\begin{example}
    En el transcurso de dos horas, el número de llamadas por minuto recibidas en una centralita fue el siguiente:
    \begin{center}
        \begin{tabular}{| l | c c c c c c c |}
            \hline
            Número de llamadas por minuto & 0 & 1  & 2  & 3  & 4  & 5  & 6 \\
            \hline
            Frecuencia                    & 6 & 18 & 32 & 35 & 17 & 10 & 2 \\
            \hline
        \end{tabular}
    \end{center}
    Veamos si se puede aceptar que el número de llamadas por minuto sigue una distribución de Poisson.

    Disponemos de una muestra de $n = 120$ minutos, en los cuales registramos el número de llamadas que se han producido.
    El número de llamadas por minutos lo clasificamos en las siguientes clases:
    $$\{0\}, \quad \{1\}, \quad \{2\}, \quad \{3\}, \quad \{4\}, \quad \{\geq 5\}$$
    Las dos últimas clases las hemos agrupado para evitar frecuencias demasiado bajas.

    Sea $X$ el número de llamadas por minuto, realizamos el siguiente contraste:
    $$\begin{cases}
            H_0: X \sim \text{ Poisson}  \\
            H_1: X \nsim \text{ Poisson} \\
        \end{cases}$$

    La tabla de frecuencias observadas y esperadas es:
    \begin{center}
        \begin{tabular}{| c | c c c c c c |}
            \hline
            $A_i$    & $\{0\}$ & $\{1\}$ & $\{2\}$ & $\{3\}$ & $\{4\}$ & $\{\geq 5\}$ \\
            \hline
            $O_i$    & 6       & 18      & 32      & 35      & 17      & 12           \\
            $P(A_i)$ & 0.0743  & 0.19318 & 0.2510  & 0.2176  & 0.1414  & 0.1226       \\
            $e_i$    & 8.92    & 23.17   & 30.12   & 26.11   & 16.97   & 14.71        \\
            \hline
        \end{tabular}
    \end{center}
    donde las frecuencias observadas bajo $H_0$ han sido calculadas como:
    $$e_i = nP(A_i) = 120P(A_i)$$
    y los valores de $P(A_i)$ se han calculado a partir de una distribución de Poisson de parámetro $\hat{\lambda} = \bar{x} = 2.6$.

    Usaremos el estadístico de contraste:
    $$\sum_{i=1}^k \frac{(O_i-e_i)^2}{e_i}$$
    que tiene aproximadamente una distribución $\chi^2_{k-r-1}$ si $H_0$ es cierta.

    Rechazaremos $H_0$ a nivel de significación $\alpha$ si:
    $$\sum_{i=1}^k \frac{(O_i-e_i)^2}{e_i} > \chi^2_{k-r-1, 1-\alpha}$$
    En nuestro caso:
    $$\sum_{i=1}^6 \frac{(O_i-e_i)^2}{e_i} = 5.75, \quad \chi^2_{4, 0.95} = 9.488$$

    Por tanto, aceptamos $H_0$ a nivel de significación $\alpha = 0.05$, es decir, podemos aceptar que el número de llamadas por minuto sigue una distribución de Poisson a nivel de significación $\alpha = 0.05$.
\end{example}

\section{Contrastes de homogeneidad}
Supongamos que disponemos de $p$ muestras aleatorias independientes tomadas de $p$ poblaciones sobre una característica común $X$ a todas ellas:
\begin{align*}
    (X_{11}, \dots, X_{1n_1}) \\
    \dots                     \\
    (X_{p1}, \dots, X_{pn_p})
\end{align*}
con $n_1 + \dots + n_p = n$.

Queremos ver si, a la vista de las muestras obtenidas, es razonable admitir que todas las poblaciones tienen una distribución común, es decir, si son poblaciones homogéneas.
Por tanto, tenemos el contraste:
$$\begin{cases}
        H_0: \text{ Las } p \text{ poblaciones tienen una distribución común} \\
        H_1: \text{ Las } p \text{ poblaciones no tienen una distribución común}
    \end{cases}$$

Para contrastar $H_0$ frente a $H_1$ hacemos nuevamente una partición arbitraria del espacio muestral común a las $p$ poblaciones en $k$ clases $A_1, \dots, A_k$.

Después, definimos para la clase $A_i$ y para la muestra de la población $j$-ésima:
\begin{itemize}
    \item $O_{ij}$: frecuencia observada en la clase $A_i$ con la muestra $j$-ésima.
    \item $e_{ij}$: frecuencia esperada en la clase $A_i$ con la muestra $j$-ésima si todas las poblaciones tienen la distribución común $P$, $n_jP(A_i)$.
\end{itemize}

El estadístico utilizado es:
$$\sum_{j=1}^p \sum_{i=1}^k \frac{(O_{ij}-e_{ij})^2}{e_{ij}}$$
que tiene aproximadamente cuando $n$ es grande una distribución $\chi^2_{(k-1)(p-1)}$ si $H_0$ es cierta.

Rechazaremos $H_0$ a nivel de significación $\alpha$ si:
$$\sum_{j=1}^p \sum_{i=1}^k \frac{(O_{ij}-e_{ij})^2}{e_{ij}} > \chi^2_{(k-1)(p-1), 1-\alpha}$$
En caso contrario, aceptaremos $H_0$ a nivel de significación $\alpha$.

\begin{example}
    Una fábrica de automóviles quiere averiguar si la preferencia de modelo tiene relación con el sexo de los clientes.
    Se toman dos muestras aleatorias de 1000 hombres y 1000 mujeres, observándose las siguientes preferencias:
    \begin{center}
        \begin{tabular}{| c | c c |}
            \hline
            Modelo & Hombre & Mujer \\
            \hline
            $A$    & 340    & 350   \\
            $B$    & 400    & 270   \\
            $C$    & 260    & 380   \\
            \hline
        \end{tabular}
    \end{center}
    Veamos si son homogéneas las preferencias entre hombres y mujeres a nivel de significación $\alpha = 0.01$.

    Disponemos de una muestra aleatoria de 1000 mujeres y otra de 1000 hombres, clasificadas según sus preferencias por los modelos $A$, $B$ y $C$.
    El número total de datos es $n = 2000$.

    Para decidir si las preferencias en las dos poblaciones son homogéneas, planteamos el siguiente contraste de homogeneidad:
    $$\begin{cases}
            H_0: \text{ Las preferencias son homogéneas} \\
            H_1: \text{ Las preferencias no son homogéneas}
        \end{cases}$$

    La tabla de frecuencias observadas es:
    \begin{center}
        \begin{tabular}{| c | c c | c |}
            \hline
            $O_{ij}$ & Hombres & Mujeres &      \\
            \hline
            $A$      & 340     & 350     & 690  \\
            $B$      & 400     & 270     & 670  \\
            $C$      & 260     & 380     & 640  \\
            \hline
                     & 1000    & 1000    & 2000 \\
            \hline
        \end{tabular}
    \end{center}

    Las frecuencias esperadas se calculan de la forma:
    $$e_{ij} = \frac{(\sum_{i=1}^k O_{ij})(\sum_{j=1}^p O_{ij})}{n}$$
    obteniéndose la siguiente tabla de frecuencias esperadas:
    \begin{center}
        \begin{tabular}{| c | c c |}
            \hline
            $e_{ij}$ & Hombres & Mujeres \\
            \hline
            $A$      & 345     & 345     \\
            $B$      & 335     & 335     \\
            $C$      & 320     & 320     \\
            \hline
        \end{tabular}
    \end{center}

    Usaremos el estadístico de contraste:
    $$\sum_{j=1}^p \sum_{i=1}^k \frac{(O_{ij}-e_{ij})^2}{e_{ij}}$$
    que tiene aproximadamente una distribución $\chi^2_{(k-1)(p-1)}$ si $H_0$ es cierta.

    Rechazaremos $H_0$ a nivel de significación $\alpha$ si:
    $$\sum_{j=1}^p \sum_{i=1}^k \frac{(O_{ij}-e_{ij})^2}{e_{ij}} > \chi^2_{(k-1)(p-1), 1-\alpha}$$
    En nuestro caso:
    $$\sum_{j=1}^3 \sum_{i=1}^3 \frac{(O_{ij}-e_{ij})^2}{e_{ij}} = 47.87, \quad \chi^2_{2, 0.99} = 9.210$$

    Por tanto, rechazamos $H_0$ a nivel de significación $\alpha = 0.01$, es decir, podemos concluir que no son homogéneas las preferencias entre hombres y mujeres a nivel de significación $\alpha = 0.01$.
\end{example}

% Problema

\begin{exercise}
    Para probar la efectividad de una vacuna contra cierta enfermedad, se realizó un experimento observando el comportamiento de 100 personas vacunadas y 100 personas sin vacunar.
    Los resultados fueron los siguientes:
    \begin{center}
        \begin{tabular}{| c | c c |}
            \hline
                         & Enfermaron & No enfermaron \\
            \hline
            Vacunados    & 4          & 96            \\
            No vacunados & 6          & 94            \\
            \hline
        \end{tabular}
    \end{center}
    Queremos contrastar la efectividad de la vacuna con los datos obtenidos.

    Para ello consideramos el contraste:
    $$\begin{cases}
            H_0: \text{Las poblaciones son homogéneas, luego la vacuna no es efectiva} \\
            H_1: \text{La vacuna es efectiva}
        \end{cases}$$

    La tabla de frecuencias observadas es:
    \begin{center}
        \begin{tabular}{| c | c c | c |}
            \hline
            $O_{ij}$     & Enfermaron & No enfermaron &     \\
            \hline
            Vacunados    & 4          & 96            & 100 \\
            No vacunados & 6          & 94            & 100 \\
            \hline
                         & 10         & 190           & 200 \\
            \hline
        \end{tabular}
    \end{center}

    Las frecuencias esperadas se calculan de la forma:
    $$e_{ij} = \frac{(\sum_{i=1}^k O_{ij})(\sum_{j=1}^p O_{ij})}{n}$$
    obteniéndose la siguiente tabla de frecuencias esperadas:
    \begin{center}
        \begin{tabular}{| c | c c |}
            \hline
            $e_{ij}$     & Enfermaron & No enfermaron \\
            \hline
            Vacunados    & 5          & 95            \\
            No vacunados & 5          & 95            \\
            \hline
        \end{tabular}
    \end{center}

    Usaremos el estadístico de contraste:
    $$\sum_{j=1}^p \sum_{i=1}^k \frac{(O_{ij}-e_{ij})^2}{e_{ij}}$$
    que tiene aproximadamente una distribución $\chi^2_{(k-1)(p-1)}$ si $H_0$ es cierta.

    Rechazaremos $H_0$ a nivel de significación $\alpha$ si:
    $$\sum_{j=1}^p \sum_{i=1}^k \frac{(O_{ij}-e_{ij})^2}{e_{ij}} > \chi^2_{(k-1)(p-1), 1-\alpha}$$
    En nuestro caso:
    $$\sum_{j=1}^2 \sum_{i=1}^2 \frac{(O_{ij}-e_{ij})^2}{e_{ij}} = 0.4211, \quad \chi^2_{1, 0.95} = 3.8415$$

    Por tanto, aceptamos $H_0$ a nivel de significación $\alpha = 0.05$, es decir, podemos concluir que la vacuna no es eficaz a nivel de significación $\alpha = 0.05$.
\end{exercise}

% Problema

\section{Contrastes de independencia}
Supongamos que queremos estudiar si dos características $X$ e $Y$ de una población están relacionadas o no.
Para hacer este estudio, obtenemos una muestra aleatoria de $n$ pares de valores de estas características:
$$((X_1, Y_1), \dots, (X_n, Y_n))$$
Queremos ver si, a la vista de la muestra, tiene sentido admitir que $X$ e $Y$ son independientes.
Por tanto, tenemos el contraste:
$$\begin{cases}
        H_0: X \text{ e } Y \text{ son independientes} \\
        H_1: X \text{ e } Y \text{ no son independientes}
    \end{cases}$$

Tomamos una partición arbitraria del espacio muestral en $kp$ clases:
$$A_1 \times B_1, \dots, A_k \times B_p$$
Estas $kp$ clases corresponden a tomar las clases $A_1, \dots, A_k$ para la característica $X$ y las clases $B_1, \dots, B_p$ para la característica $Y$.

Llamamos:
\begin{itemize}
    \item $O_{ij}$: frecuencia observada en la clase $A_i \times B_j$.
    \item $e_{ij}$: frecuencia esperada en la clase $A_i \times B_j$ si la hipótesis nula es cierta, $nP(A_i)P(B_j)$.
\end{itemize}

El estadístico utilizado es:
$$\sum_{j=1}^p \sum_{i=1}^k \frac{(O_{ij}-e_{ij})^2}{e_{ij}}$$
que tiene aproximadamente cuando $n$ es grande una distribución $\chi^2_{(k-1)(p-1)}$ si $H_0$ es cierta.

\begin{remark}
    Este estadístico coincide con el que utilizábamos en el contraste de homogeneidad.
\end{remark}

Rechazaremos $H_0$ a nivel de significación $\alpha$ si:
$$\sum_{j=1}^p \sum_{i=1}^k \frac{(O_{ij}-e_{ij})^2}{e_{ij}} > \chi^2_{(k-1)(p-1), 1-\alpha}$$
En caso contrario, aceptaremos $H_0$ a nivel de significación $\alpha$.

\begin{example}
    Se desea evaluar la efectividad de una nueva vacuna antigripal. Para ello, se decide suministrar dicha vacuna, de manera voluntaria y gratuita, a una pequeña comunidad.
    La vacuna se administra en dos dosis, separadas por un período de dos semanas, de forma que alguna de las personas han recibido una sola dosis, otras han recibido las dos y otras personas no han recibido ninguna.
    La siguiente tabla muestra los resultados que se registraron durante la siguiente primavera en 1000 habitantes de la comunidad elegidos al azar:
    \begin{center}
        \begin{tabular}{| c | c c c |}
            \hline
                     & No vacunados & Una dosis & Dos dosis \\
            \hline
            Gripe    & 24           & 9         & 13        \\
            No gripe & 289          & 100       & 565       \\
            \hline
        \end{tabular}
    \end{center}
    Veamos si proporcionan estos datos suficiente evidencia estadística para indicar una dependencia entre la clasificación respecto a la vacuna y la protección frente a la gripe a nivel de significación $\alpha = 0.05$.

    Disponemos de una muestra aleatoria de $n = 1000$ habitantes clasificados según dos características: número de dosis recibidas y protección frente a la gripe.

    Para decidir si existe dependencia entre estas dos características, planteamos un contraste de independencia:
    $$\begin{cases}
            H_0: \text{ Las dos características son independientes} \\
            H_1: \text{ Existe dependencia entre las dos características}
        \end{cases}$$

    La tabla de frecuencias observadas es:
    \begin{center}
        \begin{tabular}{| c | c c c | c |}
            \hline
            $O_{ij}$ & No vacunados & Una dosis & Dos dosis &      \\
            \hline
            Gripe    & 24           & 9         & 13        & 46   \\
            No gripe & 289          & 100       & 565       & 954  \\
            \hline
                     & 313          & 109       & 578       & 1000 \\
            \hline
        \end{tabular}
    \end{center}

    Las frecuencias esperadas se calculan de la forma:
    $$e_{ij} = \frac{(\sum_{i=1}^k O_{ij})(\sum_{j=1}^p O_{ij})}{n}$$
    obteniéndose la siguiente tabla de frecuencias esperadas:
    \begin{center}
        \begin{tabular}{| c | c c c |}
            \hline
            $e_{ij}$ & No vacunados & Una dosis & Dos dosis \\
            \hline
            Gripe    & 14.4         & 5         & 26.6      \\
            No gripe & 298.6        & 104       & 551.4     \\
            \hline
        \end{tabular}
    \end{center}

    Usaremos el estadístico de contraste:
    $$\sum_{j=1}^p \sum_{i=1}^k \frac{(O_{ij}-e_{ij})^2}{e_{ij}}$$
    que tiene aproximadamente una distribución $\chi^2_{(k-1)(p-1)}$ si $H_0$ es cierta.

    Rechazaremos $H_0$ a nivel de significación $\alpha$ si:
    $$\sum_{j=1}^p \sum_{i=1}^k \frac{(O_{ij}-e_{ij})^2}{e_{ij}} > \chi^2_{(k-1)(p-1), 1-\alpha}$$
    En nuestro caso:
    $$\sum_{j=1}^3 \sum_{i=1}^2 \frac{(O_{ij}-e_{ij})^2}{e_{ij}} = 17.35, \quad \chi^2_{2, 0.95} = 5.991$$

    Por tanto, rechazamos $H_0$ a nivel de significación $\alpha = 0.05$, es decir, podemos concluir que existe dependencia entre el número de dosis recibidas y la protección frente a la gripe a nivel de significación $\alpha = 0.05$.
\end{example}

% Problema
\chapter{Funciones enteras. Crecimiento y distribución de los ceros}
\begin{theorem}[Teorema de Liouville]
    Si $f$ es una función entera y acotada, entonces $f$ es constante.
\end{theorem}

\begin{theorem}[Generalización del teorema de Liouville]
    Sea $f$ entera tal que existen $\alpha > 0$, $c > 0$ y $R_0 > 0$ con $|f(z)| \leq c|z|^\alpha$ para todo $|z| \geq R_0$.
    Entonces $f$ es un polinomio de grado $E(\alpha)$ y por tanto tiene $E(\alpha)$ ceros.
\end{theorem}

\begin{proof}
    Sea $f(z) = \sum_{n=0}^\infty a_nz^n$.
    Sea $R > 0$,
    $$a_n = \frac{f^{(n)}(0)}{n!} = \frac{1}{2\pi i} \int_{|z|=R} \frac{f(z)}{z^{n+1}}dz \leq \max_{|z|=R} |f(z)|\frac{1}{R^n} \leq c\frac{|z|^\alpha}{R} = cR^{\alpha-n} \xrightarrow[R \to \infty, n > E(\alpha)]{} 0$$
    Por tanto, $a_n = 0$ para todo $n > E(\alpha)$.
\end{proof}

\begin{remark}
    \hfill
    \begin{enumerate}
        \item Parece que si el crecimiento está controlado, el número de ceros también lo está.
        \item Al revés esto no ocurre.
              Por ejemplo, con la exponencial.
    \end{enumerate}
\end{remark}

\section{Fórmula de Jensen}
La fórmula de Jensen permite controlar el número de ceros de una función holomorfa sabiendo restricciones sobre su crecimiento.

\begin{lemma}
    Sea $s > 0$.
    Entonces
    $$\frac{1}{2\pi} \int_{-\pi}^\pi \log|1+se^{i\theta}|d\theta = \log^+(s)$$
    donde
    $$\log^+(s) = \begin{cases}
            \log(s) & \text{si } s \geq 1  \\
            0       & \text{si } 0 < s < 1
        \end{cases}$$
\end{lemma}

% Demostración

\begin{theorem}[Fórmula de Jensen]
    Sea $f$ una función holomorfa en $D(0, R)$ con $f(0) \neq 0$ y tal que tiene ceros $\{a_n\}$ de modo que
    $$|a_1| \leq |a_2| \leq |a_3| \leq \dots$$
    Sea $\rho \in (0, R)$ y $n(\rho, f) = \#\{a_n : |a_n| \leq \rho\}$.
    Entonces
    $$\frac{1}{2\pi} \int_{-\pi}^\pi \log|f(\rho e^{i\theta})|d\theta = \log|f(0)| + \sum_{k=1}^{n(\rho, f)} \log\left(\frac{\rho}{|a_k|}\right)$$
\end{theorem}

% Demostración

\begin{definition}
    Se llama función contadora de Nevanlinna a la cantidad
    $$N(\rho, f) = \sum_{\{n : |a_n| \leq \rho\}} \log\left(\frac{\rho}{|a_n|}\right)$$
\end{definition}

\begin{remark}
    \hfill
    \begin{enumerate}
        \item $$N(\rho, f) = \sum_{\{n : |a_n| \leq \rho\}} \log\left(\frac{\rho}{|a_n|}\right) = \sum_{k=1}^{n(\rho, f)} \frac{\rho}{|a_k|} = \sum_{k=1}^\infty \log^+\left(\frac{\rho}{|a_k|}\right)$$
        \item $$N(\rho, f) = \int_0^\rho \frac{n(t, f)}{t}dt$$
    \end{enumerate}
\end{remark}
\chapter{Resolución de sistemas y ecuaciones diferenciales lineales con coeficientes constantes}
\section{Exponencial de una matriz cuadrada}
\begin{definition}
    Dada $A \in \mathcal{M}_n(\mathbb{R})$, se llama matriz exponencial de $A$ y se escribe $e^A$ a la matriz de $\mathcal{M}_n(\mathbb{R})$ definida por:
    $$e^A = \sum_{k=0}^\infty \frac{1}{k!}A^k$$
\end{definition}

\begin{proposition}[Propiedades de la exponencial]
    \hfill
    \begin{enumerate}
        \item Si $\Theta \in \mathcal{M}_n(\mathbb{R})$ es la matriz nula, entonces $e^\Theta = I_n$.
        \item Si $I \in \mathcal{M}_n(\mathbb{R})$ es la matriz identidad, $e^I = eI$.
        \item Si $A, B \in \mathcal{M}_n(\mathbb{R})$ y $A$ y $B$ conmutan, entonces $Ae^B = e^BA$
        \item Si $A, B \in \mathcal{M}_n(\mathbb{R})$ y $A$ y $B$ conmutan, entonces $e^{A+B} = e^Ae^B$
        \item Si $A \in \mathcal{M}_n(\mathbb{R})$, entonces $e^A$ es invertible y su inversa es $(e^A)^{-1} = e^{-A}$.
    \end{enumerate}
\end{proposition}

\section{Determinación de una matriz fundamental}
\begin{theorem}
    Supongamos que $I \subset \mathbb{R}$ es intervalo no degenerado, que $B \in \mathcal{C}^1(I, \mathcal{M}_n(\mathbb{R}))$ y que $B$ y $B'$ conmutan para todo $t \in I$.
    Entonces, fijado $t_0 \in I$, la aplicación:
    \begin{align*}
        \Phi_{t_0}: I & \to \mathcal{M}_n(\mathbb{R})             \\
        t_0           & \mapsto \Phi_{t_0}(t) = e^{B(t) - B(t_0)}
    \end{align*}
    es la matriz fundamental canónica de $x' = B'(t)x$ en $t_0$.
\end{theorem}

\begin{corollary}
    Si $A \in \mathcal{M}_n(\mathbb{R})$ y $b \in \mathcal{C}(I, \mathbb{R}^n)$ entonces, para cada $t_0 \in I$ y cada $x^0 \in \mathbb{R}^n$, el problema de Cauchy
    $$(P): \begin{cases}
            x' = Ax + b \\
            x(t_0) = x^0
        \end{cases}$$
    tiene solución única en $I$.

    Además, esta solución viene dada por:
    $$\varphi(t) = e^{(t-t_0)A}x^0 + \int_{t_0}^t e^{(t-s)A}b(s)ds, \quad t \in I$$
\end{corollary}

\section{Cálculo de la exponencial de una matriz}
\begin{enumerate}
    \item Si $A = diag(\lambda_1, \dots, \lambda_n)$, entonces:
          $$e^{tA} = diag(e^{t\lambda_1}, \dots, e^{t\lambda_n})$$
    \item Si $A$ es semejante a una matriz diagonal $D$, es decir, existe $P$ invertible tal que $A = PDP^{-1}$, entonces:
          $$e^{tA} = Pe^{tD}P^{-1}$$
    \item Si $A$ es diagonal por bloques, es decir, $A = diag(A_1, \dots, A_l)$ siendo cada $A_j \in \mathcal{M}_{r_j}(\mathbb{R})$ con $r_1 + \dots + r_l = n$, entonces:
          $$e^{tA} = diag(e^{tA_1}, \dots, e^{tA_l})$$
\end{enumerate}

\end{document}