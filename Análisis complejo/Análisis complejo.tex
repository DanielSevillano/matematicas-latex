\documentclass{report}
\usepackage[hmargin = 3cm, vmargin = 2.5cm]{geometry}
\usepackage[spanish]{babel}
\usepackage{amssymb, amsmath, amsthm, hyperref, parskip}

\title{Análisis complejo}
\author{}

\newtheorem{theorem}{Teorema}[chapter]
\newtheorem{corollary}[theorem]{Corolario}
\newtheorem{lemma}[theorem]{Lema}
\newtheorem{proposition}[theorem]{Proposición}

\theoremstyle{remark}
\newtheorem*{remark}{Observación}
\theoremstyle{remark}
\newtheorem*{note}{Nota}
\theoremstyle{remark}
\newtheorem*{notation}{Notación}

\theoremstyle{definition}
\newtheorem{definition}{Definición}[chapter]
\theoremstyle{definition}
\newtheorem*{properties}{Propiedades}
\theoremstyle{definition}
\newtheorem*{example}{Ejemplo}
\theoremstyle{definition}
\newtheorem*{exercise}{Ejercicio}

\newcommand{\Arg}{\mathrm{Arg}}
\newcommand{\Log}{\mathrm{Log}}
\newcommand{\dist}{\mathrm{dist}}
\renewcommand{\Re}{\mathrm{Re}}
\renewcommand{\Im}{\mathrm{Im}}

\begin{document}
\maketitle
\tableofcontents

\chapter*{Preliminares}
\addcontentsline{toc}{chapter}{Preliminares}
\begin{definition}
    Si $a \in \mathbb{C}$ y $0 \leq R_1 < R_2 \leq \infty$, se define la corona de centro $a$ y radios $R_1$ y $R_2$ como:
    $$A(a, R_1, R_2) = \{z \in \mathbb{C} : R_1 < |z-a| < R_2\}$$
\end{definition}

\begin{theorem}
    Si $a \in \mathbb{C}$, $0 \leq R_1 < R_2 \leq \infty$ y $f$ es holomorfa en $A(a, R_1, R_2)$, entonces existe una única sucesión $\{a_n\}_{-\infty}^\infty$ en $\mathbb{C}$ tal que:
    \begin{itemize}
        \item $\sum_{-\infty}^\infty a_n(z-a)^n$ converge para todo $z \in A(a, R_1, R_2)$.
        \item $f(z) = \sum_{-\infty}^\infty a_n(z-a)^n$ para todo $z \in A(a, R_1, R_2)$.
    \end{itemize}

    Para cada $n \in \mathbb{Z}$,
    $$a_n = \frac{1}{2\pi i} \int_\gamma \frac{f(z)}{(z-a)^{n+1}}dz$$
    siendo $\gamma$ cualquiera camino que esté en $A(a, R_1, R_2)$ con $n(\gamma, a) = 1$

    Además, la serie $\sum_{-\infty}^\infty a_n(z-a)^n$ converge absoluta y uniformemente a cada subconjunto compacto de $A(a, R_1, R_2)$.

    A esta serie se le llama desarrollo de Laurent de $f$ en $A(a, R_1, R_2)$.
\end{theorem}

\begin{definition}
    $f$ tiene una singularidad aislada en $a \in \mathbb{C}$ si existe $R > 0$ tal que $f$ está definida y es holomorfa en $D(a, R) \setminus \{a\} = A(a, 0, R)$.
\end{definition}

Podemos considerar el desarrollo de Laurent de $f$ en $D(a, R) \setminus \{a\}$.
Existe una única sucesión en $\mathbb{C}$, $\{a_n\}_{-\infty}^\infty$, tal que:
$$f(z) = \sum_{-\infty}^\infty a_n(z-a)^n, \quad z \in D(a, R) \setminus \{a\}$$

Como la sucesión $\{a_n\}_{-\infty}^\infty$ no depende de $R$, a este desarrollo se le puede llamar desarrollo de Laurent de $f$ en $a$ o en un entorno perforado de $a$.

\begin{proposition}
    Sea $f$ una función con una singularidad aislada en $a \in \mathbb{C}$ y sea $\sum_{-\infty}^\infty a_n(z-a)^n$ el desarrollo de Laurent de $f$ en $a$.
    Entonces:
    \begin{enumerate}
        \item $a$ es una singularidad evitable de $f$ $\Leftrightarrow$ $a_n = 0$ si $n < 0$ $\Leftrightarrow$ $\{n < 0 : a_n \neq 0\} = \emptyset$.
        \item $a$ es un polo de orden $N$ de $f$ $\Leftrightarrow$ $a_{-N} \neq 0$ y $a_n = 0$ si $n < -N$.
              Luego $a$ es un polo de $f$ $\Leftrightarrow$ $\{n < 0 : a_n \neq 0\}$ es finito y no vacío.
        \item $a$ es una singularidad esencial de $f$ $\Leftrightarrow$ $\{n < 0 : a_n \neq 0\}$ es infinito.
    \end{enumerate}
\end{proposition}

\begin{definition}
    $f$ tiene una singularidad aislada en $\infty$ si existe $R > 0$ tal que $f$ es holomorfa en $\{z \in \mathbb{C} : |z| > R\}$.
    \begin{enumerate}
        \item Es una singularidad evitable de $f$ si $\lim\limits_{z \to \infty} f(z)$ existe en $\mathbb{C}$.
        \item Es un polo de $f$ si $\lim\limits_{z \to \infty} f(z) = \infty$.
        \item Es una singularidad esencial en otro caso.
    \end{enumerate}
\end{definition}

Si $f$ tiene una singularidad aislada en $\infty$, entonces $f$ es holomorfa en $\{z \in \mathbb{C} : |z| > R\}$ para un cierto $R > 0$.
Entonces la función $g(z) = f\left(\frac{1}{z}\right)$ es holomorfa en $D\left(0, \frac{1}{R}\right) \setminus \{0\}$, por lo que tiene una singularidad aislada en 0.

Entonces:
\begin{enumerate}
    \item $f$ tiene una singularidad evitable en $\infty$ $\Leftrightarrow$ $g$ tiene una singularidad evitable en 0.
    \item $f$ tiene un polo en $\infty$ $\Leftrightarrow$ $g$ tiene un polo en 0.
    \item $f$ tiene una singularidad esencial en $\infty$ $\Leftrightarrow$ $g$ tiene una singularidad esencial en 0.
\end{enumerate}

\begin{proposition}
    Sea $f$ una función con una singularidad aislada en $\infty$.
    Entonces:
    \begin{enumerate}
        \item $\infty$ es una singularidad evitable de $f$ $\Leftrightarrow$ $f$ está acotada en un entorno perforado de $\infty$.
              Es decir, si existe $R > 0$ tal que $f$ es holomorfa y está acotada en $\{z \in \mathbb{C} : |z| > R\}$.
        \item $\infty$ es un polo de $f$ $\Leftrightarrow$ existe $N \in \mathbb{N}$ tal que $\lim\limits_{z \to \infty} \frac{f(z)}{z^N}$ existe en $\mathbb{C}$ y es distinto de 0.
              En este caso, $N$ es único y se denomina el orden de $\infty$ como polo de $f$.
        \item $\infty$ es una singularidad esencial de $f$ $\Leftrightarrow$ $f(\{z \in \mathbb{C} : |z| > R\})$ es denso en $\mathbb{C}$ para todo $R > 0$ suficientemente grande.
    \end{enumerate}
\end{proposition}

\begin{remark}
    En (2), el orden de $\infty$ como polo de $f$ coincide con el orden de 0 como polo de $f\left(\frac{1}{z}\right)$.
\end{remark}

Si $f$ tiene una singularidad aislada en $\infty$, entonces existe $R > 0$ tal que $f$ es holomorfa en $\{z \in \mathbb{C} : |z| > R\} = A(0, R, \infty)$.
Podemos considerar el desarrollo de Laurent de $f$ en $A(0, R, \infty)$: existe una única sucesión $\{a_n\}_{-\infty}^\infty$ en $\mathbb{C}$ tal que:
$$f(z) = \sum_{-\infty}^\infty a_nz^n, \quad \text{para todo } z \in \mathbb{C} \text{ con } |z| > R$$

Como no depende de $R$, se le puede llamar desarrollo de Laurent de $f$ en $\infty$.

\begin{proposition}
    Sea $f$ una función con una singularidad aislada en $\infty$ y sea $\sum_{-\infty}^\infty a_nz^n$ el desarrollo de Laurent de $f$ en $\infty$.
    Entonces:
    \begin{enumerate}
        \item $\infty$ es una singularidad evitable de $f$ $\Leftrightarrow$ $a_n = 0$ si $n > 0$.
        \item $\infty$ es un polo de $f$ de orden $N$ $\Leftrightarrow$ $a_N \neq 0$ y $a_n = 0$ si $n > N$.
        \item $\infty$ es una singularidad esencial de $f$ $\Leftrightarrow$ $\{n > 0 : a_n \neq 0\}$ es infinito.
    \end{enumerate}
\end{proposition}

\begin{definition}
    Si $f$ tiene una singularidad aislada en $a \in \mathbb{C}$ y $\sum_{-\infty}^\infty a_n(z-a)^n$ es el desarrollo de Laurent de $f$ en $a$, se define $Res(f, a) = a_{-1}$.
\end{definition}

\begin{proposition}
    Sea $a \in \mathbb{C}$ y $f$ una función con una singularidad aislada en $a$.
    Sea $R > 0$ tal que $f$ es holomorfa en $D(a, R) \setminus \{a\}$.
    Entonces, para todo $r \in (0, R)$, se tiene que:
    $$Res(f, a) = \frac{1}{2\pi i} \int_{|z-a| = r} f(z)dz$$
\end{proposition}

\begin{proposition}
    Sea $f$ una función con una singularidad aislada en $\infty$.
    Sea $R > 0$ tal que $f$ es holomorfa en $\{z \in \mathbb{C} : |z| > R\}$.
    Se define:
    $$Res(f, \infty) = \frac{-1}{2\pi i} \int_{|z| = r} f(z)dz, \quad \text{siendo } r > R$$
\end{proposition}

\begin{proposition}
    Si $f$ tiene una singularidad aislada en $\infty$ y $\sum_{-\infty}^\infty a_nz^n$ es el desarrollo de Laurent de $f$ en $\infty$, entonces $Res(f, \infty) = -a_{-1}$.
\end{proposition}

\begin{theorem}[Teorema de los residuos]
    Sea $D$ un dominio en $\mathbb{C}$ y sea $f$ holomorfa en $D$ salvo por singularidades aisladas, es decir, existe $A \subset D$, $A$ sin puntos de acumulación en $D$, tal que $f$ es holomorfa en $D \setminus A$.
    Sea $\gamma$ un camino cerrado en $D \setminus A$, con $n(\gamma, z) = 0$ para todo $z \in \mathbb{C} \setminus D$.
    Entonces:
    $$\frac{1}{2\pi i} \int_\gamma f(z)dz = \sum_{a \in A} Res(f, a)n(\gamma, a)$$
\end{theorem}

\begin{theorem}[Teorema de la función inversa]
    Sea $D$ un dominio en $\mathbb{C}$ y sea $f$ holomorfa en $D$, con $a \in D$ tal que $f'(a) \neq 0$.
    Entonces existen $U, V$ abiertos en $\mathbb{C}$ con $a \in U \subset D$, $f(a) \in V$, tales que:
    \begin{enumerate}
        \item $f$ es inyectiva en $U$.
        \item $f(U) = V$.
        \item $f'(z) \neq 0$ para todo $z \in U$.
        \item $f^{-1}: V \to U$ es holomorfa y además:
              $$(f^{-1})'(f(z)) = \frac{1}{f'(z)}, \quad \forall z \in U$$
    \end{enumerate}
\end{theorem}

\begin{theorem}
    Sea $D$ un dominio en $\mathbb{C}$ y sean $f$ holomorfa en $D$ no constante y $a \in D$.
    Sea $n$ el orden de $a$ como cero de $f-f(a)$, es decir, el primer natural para el que $f^{(n)}(a) \neq 0$.
    Entonces $f$ es localmente una aplicación $n \to 1$ cerca de $a$.
    Es decir, existe $\alpha > 0$ con $D(a, \alpha) \subset D$ tal que para todo $0 < \varepsilon < \alpha$ existe $\delta > 0$ tal que cada punto $w \in D(f(a), \delta) \setminus \{f(a)\}$ es la imagen de exactamente $n$ puntos distintos $z_1, z_2, \dots z_n \in D(a, \varepsilon) \setminus \{a\}$.
    En particular, $f(D(a, \varepsilon)) \supset D(f(a), \delta)$.
\end{theorem}

\begin{definition}
    Sea $D$ abierto en $\mathbb{C}$ y sea $f$ holomorfa en $D$ salvo por polos.
    Si $a \in D$ es un polo de $f$, se tiene que $\lim\limits_\{z \to a\} f(z) = \infty$.
    Definimos $f(a) = \infty$.
    Entonces $f: D \to \mathbb{C}^\ast$ y es continua.
    Se dice que $f$ es meromorfa en $D$.
\end{definition}

\begin{theorem}
    Sea $D$ un dominio en $\mathbb{C}$ y sea $f$ meromorfa en $D$, con $a \in D$ un polo de orden $n$ de $f$.
    Entonces $f$ es localmente una aplicación $n \to 1$ cerca de $a$.
    Es decir, existe $\alpha > 0$ tal que $D(a, \alpha) \subset D$, $f$ es holomorfa en $D(a, \alpha) \setminus \{a\}$ y se verifica que para todo $0 < \varepsilon < \alpha$ existe $R > 0$ tal que cada punto $w \in \mathbb{C}$ con $|w| > R$ es la imagen de exactamente $n$ puntos distintos $z_1, z_2, \dots z_n \in D(a, \varepsilon) \setminus \{a\}$.
    En particular, $f(D(a, \varepsilon) \setminus \{a\}) \supset \{w \in \mathbb{C} : |w| > R\}$.
\end{theorem}


\begin{theorem}
    Sea $f$ una función con un polo de orden $n$ en $\infty$.
    Entonces $f$ es localmente una aplicación $n \to 1$ cerca de $\infty$.
    Es decir, existe $R_0 > 0$ tal que $f$ es holomorfa en $\{z \in \mathbb{C} : |z| > R_0\}$ y se verifica que para todo $R > R_0$ existe $R' > 0$ tal que cada punto $w \in \mathbb{C}$ con $|w| > R'$ es la imagen de exactamente $n$ puntos distintos $z_1, \dots, z_n$ de $\{z \in \mathbb{C} : |z| > R\}$.
    En particular, $f(\{z \in \mathbb{C} : |z| > R\}) \supset \{w \in \mathbb{C} : |w| > R\}$.
\end{theorem}

\begin{theorem}[Teorema de la aplicación abierta]
    Sea $D$ un dominio en $\mathbb{C}$ y sea $f: D \to \mathbb{C}$ holomorfa y no constante.
    Entonces $f$ es una aplicación abierta.
    En particular, $f(D)$ es un dominio.
\end{theorem}

\begin{lemma}
    Sea $D$ un dominio en $\mathbb{C}$ y sea $f$ holomorfa en $D$.
    \begin{itemize}
        \item Sea $a \in D$.
              Entonces $f'(a) \neq 0$ si y solo si $f$ es inyectiva en un entorno de $a$.
        \item Si $f$ es inyectiva en $D$, entonces $f'(z) \neq 0$ para todo $z \in D$.
    \end{itemize}
\end{lemma}

\section*{Aplicaciones conformes}
\begin{definition}
    Sea $D$ un dominio en $\mathbb{C}$ y sea $f: D \to \mathbb{C}$ holomorfa e inyectiva.
    Sea $D' = f(D)$.
    Entonces:
    \begin{itemize}
        \item $D'$ es un dominio en $D$.
        \item $f: D \to D'$ es biyectiva.
        \item $f^{-1}: D' \to D$ es holomorfa.
    \end{itemize}

    En ese caso decimos que $f$ es una aplicación conforme de $D$ sobre $D'$.
\end{definition}

\begin{remark}
    \hfill
    \begin{enumerate}
        \item Si $f$ es una aplicación conforme de $D$ sobre $D'$, entonces $f^{-1}$ es una aplicación conforme de $D'$ sobre $D$.
        \item Si $D_1$, $D_2$ y $D_3$ son dominios en $\mathbb{C}$ con $f$ aplicación conforme de $D_1$ sobre $D_2$ y $g$ aplicación conforme de $D_2$ sobre $D_3$, entonces $g \circ f$ es una aplicación cnforme de $D_1$ sobre $D_3$.
    \end{enumerate}
\end{remark}

\begin{definition}
    Si $D_1$ y $D_2$ son dominios en $\mathbb{C}$, se dice que $D_1$ y $D_2$ son conformemente equivalentes si existe una aplicación conforme $f$ de $D_1$ sobre $D_2$.

    En el conjunto de los dominios en $\mathbb{C}$, se tiene la relación de equivalencia "ser conformemente equivalentes".
\end{definition}

\begin{definition}
    Sea $D$ un dominio en $\mathbb{C}$.
    $D$ es simplemente conexo si $\mathbb{C}^\ast \setminus D$ es conexo.
    Equivalentemente, $D$ es simplemente conexo si todo camino cerrado $\gamma$ en $D$ es homólogo a cero módulo $D$, es decir, $n(\gamma, z) = 0$ para todo $z \in \mathbb{C} \setminus D$.
\end{definition}

\begin{theorem}
    Sean $D_1$ y $D_2$ dos dominios en $\mathbb{C}$ que son conformemente equivalentes.
    Entonces $D_1$ es simplemente conexo si y solo si $D_2$ es simplemente conexo.
\end{theorem}

\begin{definition}
    Si $z_1, z_2 \in \mathbb{C} \setminus \{0\}$, el ángulo formado por $z_1$ y $z_2$ se define como:
    $$\theta(z_1, z_2) = \arg \frac{z_2}{z_1} \in (-\pi, \pi]$$
\end{definition}

\begin{remark}
    Si $z_1, z_2 \in \mathbb{C} \setminus \{0\}$ y $\lambda_1, \lambda_2 > 0$, entonces $\theta(\lambda_1z_1, \lambda_2z_2) = \theta(z_1, z_2)$.
\end{remark}

\begin{definition}
    Sea $\gamma$ un camino con origen en un punto $a \in \mathbb{C}$.
    Se dice que $\gamma$ es regular en $a$ si existe una parametrización $\mathcal{C}^1$ a trozos de $\gamma$, $\gamma: [0, 1] \to \mathbb{C}$, tal que $\gamma'(0) \neq 0$.
\end{definition}

\begin{definition}
    Sean $\gamma_1$ y $\gamma_2$ dos caminos con origen $a \in \mathbb{C}$ que son regulares en $a$.
    El ángulo que forman $\gamma_1$ y $\gamma_2$ en $a$, $\theta_a(\gamma_1, \gamma_2)$, se define como sigue.

    Sean $\gamma_1, \gamma_2: [0, 1] \to \mathbb{C}$ parametrizaciones $\mathcal{C}^1$ a trozos de $\gamma_1$ y $\gamma_2$ respectivamente tales que $\gamma_1'(0), \gamma_2'(0) \neq 0$.
    Entonces $\theta_a(\gamma_1, \gamma_2) = \theta(\gamma_1'(0), \gamma_2'(0))$.
\end{definition}

\begin{definition}
    Si $\gamma$ es una curva en $\mathbb{C}$ y $f: sop(\gamma) \to \mathbb{C}$ es continua, se define la curva imagen de $\gamma$ por $f$ como la curva $\Gamma$ que tiene por parametrización $f \circ \gamma$, siendo $\gamma$ una parametrización de $\gamma$.
\end{definition}

\begin{definition}
    Sea $D$ un dominio en $\mathbb{C}$ y sean $f$ holomorfa en $D$ y $a \in D$.
    Diremos que $f$ preserva ángulos en $a$ o que $f$ es conforme en $a$ si se verifica lo siguiente.

    Si $\gamma_1$ y $\gamma_2$ son caminos con origen $a$, regulares en $a$, entonces las curvas imagen de $\Gamma_1$ y $\Gamma_2$ por $f$ de $\gamma_1$ y $\gamma_2$ respectivamente son caminos con oriden $f(a)$, que son regulares en $f(a)$ y se tiene que:
    $$\theta_{f(a)}(\Gamma_1, \Gamma_2) = \theta_a(\gamma_1, \gamma_2)$$
\end{definition}

\begin{theorem}
    Sea $D$ un dominio en $\mathbb{C}$ y sean $f$ holomorfa en $D$ y $a \in D$.
    Si $f'(a) \neq 0$, entonces $f$ es conforme en $a$.
\end{theorem}

\begin{proof}
    Sean $\gamma_1$ y $\gamma_2$ caminos en $D$, con origen en $a$ y regulares en $a$.
    Sean $\gamma_1, \gamma_2: [0, 1] \to \mathbb{C}$ parametrizaciones de $\gamma_1$ y $\gamma_2$ respectivamente, ambas $\mathcal{C}^1$ a trozos con $\gamma_1'(0), \gamma_2'(0) \neq 0$.
    Consideramos las curvas imagen de $\gamma_1$ y $\gamma_2$ por $f$:
    \begin{align*}
        \Gamma_1 & = f \circ \gamma_1: [0, 1] \to \mathbb{C} \\
        \Gamma_2 & = f \circ \gamma_2: [0, 1] \to \mathbb{C}
    \end{align*}
    $\Gamma_1$ y $\Gamma_2$ son $\mathcal{C}^1$ a trozos.
    Además, $\Gamma_1$ y $\Gamma_2$ son caminos con origen $f(a)$, porque:
    $$\Gamma_1(0) = f(\gamma_1(0)) = f(a) = f(\gamma_2(0)) = \Gamma_2(0)$$
    Observamos que $\Gamma_1$ y $\Gamma_2$ son regulares en $a$:
    \begin{align*}
        \Gamma_1'(0) & = f'(\gamma_1(0))\gamma_1'(0) = f'(a)\gamma_1(0) \neq 0 \\
        \Gamma_2'(0) & = f'(\gamma_2(0))\gamma_2'(0) = f'(a)\gamma_2(0) \neq 0
    \end{align*}
    Por tanto:
    $$\theta_{f(a)}(\Gamma_1, \Gamma_2) = \theta(\Gamma_1'(0), \Gamma_2'(0)) = \arg \frac{\Gamma_2'(0)}{\Gamma_1'(0)} = \theta(\gamma_1'(0), \gamma_2'(0)) = \theta_a(\gamma_1, \gamma_2)$$
\end{proof}

\begin{example}[Contraejemplo]
    Sean $D = \mathbb{C}$, $f(z) = z^2$ y $a = 0$.
    Observamos que $f'(a) = 0$.
    Sea $\gamma_1$ el segmento $[0, 1]$ y $\gamma_2$ el segmento $[0, i]$.
    Es claro que $\theta_0(\gamma_1, \gamma_2) = \frac{\pi}{2}$.
    Si consideramos las curvas imagen de $\gamma_1$ y $\gamma_2$ por $f$, $\Gamma_1$ y $\Gamma_2$, podemos ver que $\Gamma_1$ es el segmento $[0, 1]$ y $\Gamma_2$ el segmento $[0, -1]$, que tienen $\theta_0(\Gamma_1, \Gamma_2) = \pi \neq \frac{\pi}{2}$.
\end{example}

De hecho, se tiene la equivalencia.
Sea $D$ un dominio en $\mathbb{C}$ y sean $f$ holomorfa en $D$ y $a \in D$.
Entonces $f'(a) \neq 0$ $\Leftrightarrow$ $f$ es conforme en $a$.
\chapter{Introducción}
\section{Superficies regulares}

\begin{definition}
    $S \subset \mathbb{R}^3$ es superficie regular si para todo $p \in S$ existe una aplicación $X : U \to V$, con $U \subset \mathbb{R}^2$ abierto y $V \subset S$ entorno abierto de $p$ en $S$, que verifica:
    \begin{itemize}
        \item $X$ es diferenciable.
        \item $X$ es homeomorfismo.
        \item $dX_q : \mathbb{R}^2 \to \mathbb{R}^3$ es inyectiva para todo $q \in U$.
    \end{itemize}
\end{definition}

\begin{definition}
    \hfill
    \begin{itemize}
        \item $V$ se llama un entorno coordenado de $p$ en $S$.
        \item $X$ es una parametrización de $S$ en $p$ o un sistema local de coordenadas.
        \item $(U, X)$ es una carta en $p$.
        \item $\{ (U_i, X_i), i \in I : \bigcup_i X_i(U_i) = S \}$ es un atlas.
    \end{itemize}
\end{definition}

\begin{remark}
    Las parametrizaciones de una superficie regular no son únicas.
\end{remark}

\begin{proposition}
    Si $f : U \to \mathbb{R}$ es una función diferenciable definida sobre un abierto $U$, entonces el grafo de $f$
    $$G(f) = \{ (x, y, f(x, y)) : (x, y) \in U \}$$
    es una superficie regular.
\end{proposition}

\begin{definition}
    Sea $F : U \subset \mathbb{R}^n \to \mathbb{R}^m$, con $U$ abierto.
    \begin{itemize}
        \item $p \in U$ es un punto crítico de $F$ si $dF_p$ no es sobreyectiva. Entonces $F(p) \in \mathbb{R}^m$ es un valor crítico.
        \item $p$ es un punto regular si no es crítico. Análogamente, $F(p)$ es valor regular si no es crítico.
    \end{itemize}
\end{definition}

\begin{remark}
    Si $f : U \subset \mathbb{R}^3 \to \mathbb{R}$, entonces $df_p$ es sobreyectiva o $df_p = (0,0,0)$.
    Luego $a \in f(U)$ es valor regular de $f$ si y solo si $f_x, f_y, f_z$ no se anulan simultáneamente en ningún punto de $f^{-1}(a)$.
\end{remark}

\begin{proposition}
    Si $f: U \subset \mathbb{R}^2 \to \mathbb{R}$ diferenciable y $a \in f(U)$ es valor regular de $f$, entonces $f^{-1}(a)$ es superficie regular de $\mathbb{R}^3$.
\end{proposition}

\begin{proposition}
    Sea $S \subset \mathbb{R}^3$ una superficie regular.
    Entonces dado $p \in S$, existe $V$ entorno abierto de $p$ en $S$ tal que $V$ es el grafo de una función diferenciable de una de las tres formas siguientes:
    $$z = f(x, y), \quad y = g(x, z), \quad \text{o} \quad x = h(y, z)$$
\end{proposition}

\begin{proposition}
    Sea $S$ una superficie regular y sea $X : U \subset \mathbb{R}^2 \to S$ una aplicación diferenciable, inyectiva y tal que $dX_q$ es inyectiva para todo $q \in U$.
    Entonces $X^{-1} : X(U) \to U$ es continua y, en consecuencia, $X$ es una parametrización de $S$.
\end{proposition}

\begin{remark}
    Es necesario que $S$ sea superficie regular.
\end{remark}

\begin{definition}
    Una superficie parametrizada es una aplicación $X : U \subset \mathbb{R}^2 \to \mathbb{R}^3$ diferenciable.
    Se dice que $X(U) \subset \mathbb{R}^3$ es la traza de $X$.\\
    $X$ es regular si $dX_q : \mathbb{R}^2 \to \mathbb{R}^3$ es inyectiva para todo $q \in U$.
\end{definition}

\begin{proposition}
    Sea $X : U \subset \mathbb{R}^2 \to \mathbb{R}^3$ superficie parametrizada regular y sea $q \in U$.
    Entonces existe $V$ entorno abierto de $q$ en $U$ tal que $X(U)$ es superficie regular.
\end{proposition}

\begin{definition}
    $C \subset \mathbb{R}^3$ es una curva regular si para todo $p \in C$ existe $V$ entorno abierto de $p$ y $\alpha : I \to U \int C$, con $I$ intervalo abierto, tal que:
    \begin{itemize}
        \item $\alpha$ es diferenciable.
        \item $\alpha$ es homeomorfismo.
        \item $d\alpha_t$ = $\alpha'(t)$ es inyectiva para todo $t$.
    \end{itemize}
\end{definition}

\begin{definition}
    Una superficie de revolución es un subconjunto $S \subset \mathbb{R}^3$ obtenida al rotar una curva plana regular $C$ alrededor de un eje contenido en el mismo plano que la curva y que no corte a la curva.
\end{definition}

\section{Cálculo diferencial en superficies regulares}

\begin{definition}
    Sea $f: S \subset \mathbb{R}^3 \to \mathbb{R}$, $X$ parametrización de $S$ y $p \in \mathbb{R}^3$.
    $f$ es diferenciable en $p$ si y solo si $f \circ X$ es diferenciable en $X^{-1}(p)$.
\end{definition}

\begin{remark}
    Esta definición no depende de la parametrización de $S$.
\end{remark}

\begin{definition}
    $f: O \subset S \to \mathbb{R}$ es una función diferenciable en $p \in O$ si para alguna parametrización $X : U \to S$, $p \in X(U)$, se tiene que $f \circ X : U \to \mathbb{R}$ es diferenciable en $X^{-1}(p)$.
\end{definition}

\begin{definition}
    $\varphi : S_1 \to S_2$ es difeomorfismo si es diferenciable, biyectiva y $\varphi^{-1}$ es diferenciable.
\end{definition}

\begin{definition}
    Un vector tangente a $S$ en $p$ es un vector tangente a una curva diferenciable parametrizada que pase por $p$.\\
    Es decir, una curva $\alpha : (-\varepsilon, \varepsilon) \to S$, con $\alpha(0) = p$, $\alpha'(0) \in T_pS$.
\end{definition}

\begin{proposition}
    Sea $X$ una parametrización, $dX_q(\mathbb{R}^2) = T_pS$, con $q = X^{-1}(p)$.
\end{proposition}

\begin{definition}
    Sea $\varphi : O \subset S_1 \to S_2$ una aplicación diferenciable definida en un abierto $O$ de $S_1$ y $p \in O$.
    Consideramos la diferencial de $\varphi$ en $p$
    $$d\varphi_p : T_pS_1 \to T_{\varphi(p)}S_2$$
    Sean $w \in T_pS_1$ y $\alpha : (-\varepsilon, \varepsilon) \to S_1$ curva diferenciable parametrizada con $\alpha(0) = p, \alpha'(0) = w$.
    Entonces $d\varphi_p(w) = (\varphi \circ \alpha)'(0)$.\\
    Se tiene:
    \begin{itemize}
        \item $\varphi \circ \alpha$ es una curva diferenciable parametrizada sobre $S_2$. $(\varphi \circ \alpha)'(0) \in T_{\varphi(p)}S_2$.
        \item $d\varphi_p$ no depende de $\alpha$ y es lineal.
    \end{itemize}
\end{definition}

\begin{definition}
    Sea $S$ superficie regular, $O$ abierto de $S$, $f : O \subset S \to \mathbb{R}$ diferenciable.
    Consideramos la diferencial de $f$ en $p$
    $$df_p : T_pS \to \mathbb{R}$$
    Sean $w \in T_pS$ y $\alpha : (-\varepsilon, \varepsilon) \to S$ curva diferenciable parametrizada con $\alpha(0) = p, \alpha'(0) = w$.
    Entonces $df(w) = (f \circ \alpha)'(0)$.\\
    Verifica:
    \begin{itemize}
        \item Está bien definida y es lineal.
        \item Si $f = F|_S$ con $F : O \subset \mathbb{R}^3 \to \mathbb{R}$, entonces $df_p = dF_p|_{T_pS}$.
    \end{itemize}
\end{definition}

\section{Primera forma fundamental}

\begin{definition}
    Sea $S$ superficie regular.
    Para cada $p \in S$ el producto escalar en $\mathbb{R}^3$ induce una métrica en $T_pS$.
    $$\left\langle w_1, w_2 \right\rangle _p = \left\langle w_1, w_2 \right\rangle, \quad \forall w_1, w_2 \in T_pS$$
    La forma cuadrática asociada se llama primera forma fundamental de $S$ en $p$.
    $$I_p : T_pS \to \mathbb{R}$$
    $$I_p(w) = \left\langle w, w \right\rangle _p = |w|^2 \geq 0$$
\end{definition}

\begin{note}
    Dada una parametrización $X$ de $S$, $\{ X_u, X_v \}_q$ base de $T_pS$, $q = X^{-1}(p)$.
    Llamamos $E$, $F$ y $G$ a los coeficientes de la primera forma fundamental.
    $$E_q = \left\langle X_u(q), X_u(q) \right\rangle$$
    $$F_q = \left\langle X_u(q), X_v(q) \right\rangle$$
    $$G_q = \left\langle X_v(q), X_v(q) \right\rangle$$
    Estas son funciones diferenciables en $X(U)$.
\end{note}

\section{Propiedades de las curvas}

\begin{definition}
    Sea $\alpha : I \to S$ una curva diferenciable parametrizada.
    Se define la longitud de arco como una aplicación $s : I \to \mathbb{R}$, con $t_0 \in I$ fijo, dada por:
    $$s(t) = \int^t_{t_0} |\alpha'(r)|dr = \int^t_{t_0} \sqrt{\left\langle \alpha'(r), \alpha'(r) \right\rangle} dr$$
\end{definition}

\begin{remark}
    Sea $(u_0, v_0) \in U$ fijo.
    $$u \mapsto X(u, v_0), \quad v \mapsto X(u_0, v)$$
    $$s_{v_0}(u) = \int^u_{u_0} |X_u| dr = \int^u_{u_0} \sqrt{E} dr$$
    $v = v_0$ está parametrizada por el arco si y solo si $E(u, v_0) = 1$ para todo $u$.
    $$s_{u_0}(v) = \int^v_{v_0} |X_v| dr = \int^v_{v_0} \sqrt{G} dr$$
    $u = u_0$ está parametrizada por el arco si y solo si $G(u_0, v) = 1$ para todo $v$.\\
    En general, todas las curvas coordenadas de $X$ están parametrizadas por el arco si y solo si $E(u, v) = 1$ y $G(u, v) = 1$, para todo $(u, v) \in U$.
\end{remark}

\begin{definition}
    El ángulo de dos curvas es el menor ángulo que forman las rectas tangentes.
    Sean $\alpha, \beta : I \to S$, con $\alpha(t_0) = \beta(t_0)$.
    $$\cos \theta(t_0) = \frac{\left\langle \alpha'(t_0), \beta'(t_0) \right\rangle}{|\alpha'(t_0)||\beta'(t_0)|}$$
\end{definition}

\begin{remark}
    El ángulo de las curvas coordenadas $u = u_0$ y $v = v_0$ en $X(u_0, v_0)$ es
    $$\cos \theta(u_0, v_0) = \frac{\left\langle X_u, X_v \right\rangle}{|X_u||X_v|} (u_0, v_0) = \frac{F}{\sqrt{EG}} (u_0, v_0)$$
    Las curvas coordenadas de una parametrización $u = cte$, $v = cte$ son ortogonales en todos los puntos de $U$ si y solo si $F(u, v) = 0$ para todo $(u, v) \in U$.\\
    En ese caso se dice que $X$ es una parametrización ortogonal.
\end{remark}

\section{Orientación en superficies}

\begin{properties}
    \hfill
    \begin{itemize}
        \item Dos bases ordenadas de un mismo espacio vectorial representan la misma orientación si el determinante de la matriz de cambio de base es positivo.
              Cada clase de equivalencia es una orientación en $U$.
        \item $\{X_u, X_v\}$ determina una orientación en $S$.
              $\{X_u, X_v, X_u \land X_v\}$ es una base positiva, con $N = \frac{X_u \land X_v}{|X_u \land X_v|}$.
    \end{itemize}
\end{properties}

\begin{definition}
    $S$ es orientable si se puede recubrir con una familia de entornos coordenados tal que en la intersección de dos tales entornos el determinante jacobiano del cambio de coordenadas tiene determinante positivo.
    La elección de tal familia se llama una orientación de $S$ y se dice que $S$ está orientada.
\end{definition}

\begin{definition}
    Sea $V$ abierto de $S$, se llama campo diferenciable de vectores normales unitarios a $N : V \to \mathbb{R}^3$ diferenciable tal que para todo $p \in S$ se tiene que $N(p) \perp T_pS$ y $|N(p)| = 1$.
\end{definition}

\begin{theorem}
    $S$ es orientable si y solo si existe un campo diferenciable $N$ de vectores normales unitarios sobre $S$.
\end{theorem}

\begin{note}
    La elección de dicho campo normal $N$ determina una orientación en $S$.
\end{note}

\begin{proposition}
    Si $S$ es imagen inversa de un valor regular, entonces $S$ es orientable.
\end{proposition}

\section{Segunda forma fundamental}

\begin{definition}
    Sea $S \subset \mathbb{R}^3$ orientable y orientada con orientación $N : S \to \mathbb{R}^3$, con $|N(p)| = 1$ para todo $p \in S$.
    Luego $N(p) \in S^2(1) \equiv S^2 = \{ (x, y, z) \in \mathbb{R}^3 : x^2+y^2+z^2 = 1 \}$.\\
    La aplicación diferenciable $N : S \to S^2$ es la aplicación de Gauss.
\end{definition}

\begin{note}
    Se mira $N$ como aplicación diferenciable entre superficies, no como un campo de vectores.
    El vector normal se toma con origen en el origen de $\mathbb{R}^3$ y el extremo da un punto de $S^2$.
\end{note}

\begin{remark}
    Si se cambia la orientación, también se cambia la aplicación de Gauss.
\end{remark}

\begin{definition}
    La diferencial de la aplicación de Gauss es, para $p \in S$, $dN_p : T_pS \to T_{N(p)}S^2$ lineal.
    Como $N(p) \perp T_pS$ y $N(p) \perp T_{N(p)}S^2$, estos son planos paralelos, así que se pueden identificar y considerar $dN_p : T_pS \to T_pS$ endomorfismo de $T_pS$.
    Este mide la variación en dirección de $N$ sobre las curvas que pasan por $p$ en un entorno de $p$.
    $$dN_p(w) = \frac{d}{dt}|_{t=0} (N \circ \alpha)(t) = (N \circ \alpha)'(0)$$
\end{definition}

\begin{proposition}
    $dN_p$ es autoadjunta respecto a $\left\langle , \right\rangle$, es decir, $\left\langle dN_p(v), w \right\rangle = \left\langle v, dN_p(w) \right\rangle, \forall v, w \in T_pS$.
    Luego se puede asociar la forma bilineal simétrica con la forma cuadrática.
\end{proposition}

\begin{definition}
    Dicha forma cuadrática $\amalg_p : T_pS \to \mathbb{R}$
    $$\amalg_p(w) = -\left\langle dN_p(w), w \right\rangle$$
    es la segunda forma fundamental de $S$ en $p$.
\end{definition}

\begin{remark}
    $S_p = -dN_p$ es el operador de Weingarten en $p$.
\end{remark}

\begin{definition}
    Sea $C$ una curva regular en $S$ que pasa por $p$.
    Se define la curvatura normal de $C$ en $p$ como
    $$k_n(p) = k(p) \left\langle n(p), N(p) \right\rangle$$
\end{definition}

\begin{remark}
    $k_n$ cambia de signo si cambia la orientación en $S$ y no depende de la orientación en $C$.
\end{remark}

\begin{theorem}[Teorema de Meusnier]
    Todas las curvas sobre una superficie regular orientable $S$ que tienen la misma recta tangente en $p \in S$ tienen la misma curvatura normal en $p$.
\end{theorem}

\begin{remark}
    \hfill
    \begin{itemize}
        \item $\amalg_p(w) = k_n(p)$, siendo $w$ unitario.
        \item La curvatura de la sección normal a lo largo de $w$ de $S$ en $p$ es el valor absoluto de la curvatura normal en $p$ de cualquier curva sobre $S$ que pase por $p$ con vector tangente $w$.
              $$k(p) = |\amalg_p(w)|$$
    \end{itemize}
\end{remark}

\begin{theorem}
    Sea $S$ superficie orientable con aplicación de Gauss $N : S \to S^2$.
    Para todo $p \in S$ existe $\{e_1, e_2\}$ base ortonormal de $T_pS$ con $dN_p(e_1) = -k_1e_1, dN_p(e_2) = -k_2e_2$.
    Además, $k_1$ y $k_2$ son el máximo y el mínimo de $\amalg_p$ sobre la circunferencia unidad de $T_pS$, es decir, los valores extremos de las curvaturas normales en $p$.
\end{theorem}

\begin{theorem}[Fórmula de Euler]
    Sea $w \in T_pS$, $|w| = 1$.
    $$\amalg_p(w) = \cos^2(t) k_1 + \sin^2(t) k_2$$
    con $\cos(t) = \left\langle w, e_1 \right\rangle, \sin(t) = \left\langle w, e_2 \right\rangle$, es decir, $t$ es el ángulo que forma $w$ con $e_1$ en la orientación de $T_pS$.
\end{theorem}

\begin{definition}
    $k_1$ y $k_2$ son las curvaturas principales de $S$ en $p$.
    Las direcciones asociadas se llaman direcciones principales en $p$.
\end{definition}

\begin{definition}
    Una curva regular conexa $C$ en $S$ es línea de curvatura de $S$ si para todo $p \in C$ la recta tangente a $C$ en $p$ es una dirección principal en $p$.
\end{definition}

\begin{definition}
    \hfill
    \begin{itemize}
        \item La curvatura de Gauss de $S$ en $p$ se define como $K(p) = det(dN_p)$.
        \item La curvatura media de $S$ en $p$ se define como $H(p) = -\frac{1}{2} tr(dN_p)$.
    \end{itemize}
\end{definition}

\begin{remark}
    En una base ortonormal de direcciones principales
    $$dN_p \equiv
        \begin{pmatrix}
            -k_1 & 0    \\
            0    & -k_2
        \end{pmatrix} \Rightarrow
        K = k_1 k_2, \quad H = \frac{k_1+k_2}{2}$$
\end{remark}

\begin{remark}
    Ante un cambio de orientación, $k_1$, $k_2$ y $H$ cambian de signo, mientras que $K$ no cambia.
\end{remark}

\begin{definition}
    Sea $p \in S$, se puede clasificar según su curvatura de Gauss.
    \begin{itemize}
        \item Si $K(p) > 0$, decimos que $p$ es elíptico.
        \item Si $K(p) < 0$, decimos que $p$ es hiperbólico.
        \item Si $K(p) = 0$ y $dN_p \neq 0$, decimos que $p$ es parabólico.
        \item Si $dN_p \equiv 0$, decimos que $p$ es plano.
    \end{itemize}
\end{definition}

\begin{remark}
    \hfill
    \begin{itemize}
        \item Si $k_1 = k_2$, todas las direcciones son principales.
        \item Si $k_1 \neq k_2$, hay dos direcciones principales y son perpendiculares.
    \end{itemize}
\end{remark}

\begin{definition}
    Un punto $p$ es umbilical si $k_1(p) = k_2(p)$.
    En ese caso, $k_n = k_1 = k_2$ y todas las direcciones de $T_pS$ son principales.
\end{definition}

\begin{remark}
    Los puntos umbilicales solo pueden ser elípticos o planos.
\end{remark}

\begin{theorem}
    Si todos los puntos de una superficie conexa $S$ son umbilicales, entonces $S$ es un abierto de un plano o de una esfera.
\end{theorem}

\begin{definition}
    Una dirección asintótica de $S$ en $p$ es una dirección de $T_pS$ para la cual la curvatura normal es cero.
\end{definition}

\begin{definition}
    Una curva o línea asintótica es una curva regular conexa $C$ en $S$ tal que, para todo $p \in C$, la recta tangente a $C$ en $p$ es una dirección asintótica.
\end{definition}

\begin{note}
    $C$ es una curva asintótica si y solo si $k_n(p) = k(p) \left\langle n, N \right\rangle = 0$.
\end{note}

\begin{remark}
    \hfill
    \begin{itemize}
        \item En un punto elíptico no hay direcciones asintóticas.\\
              $k_1$ y $k_2$ tienen el mismo signo, así que $k_2 < k_n < k_1$.
        \item En un punto plano todas las direcciones son asintóticas.
        \item En un punto hiperbólico hay dos direcciones asintóticas.\\
              $k_1 \cos^2(t) + k_2 \sin^2(t) = 0$ tiene dos soluciones.
        \item En un punto parabólico hay una dirección asintótica.
              $k_1 \cos^2(t) = 0$ tiene una solución.
    \end{itemize}
\end{remark}

\begin{proposition}
    Sea $S$ orientada, con $N = \frac{X_u \land X_v}{|X_u \land X_v|}$. En la base $\{ X_u, X_v \}$
    $$\amalg_p =
        \begin{pmatrix}
            e & f \\
            f & g
        \end{pmatrix},$$
    $$e = \left\langle N, X_{uu} \right\rangle = \frac{det(X_u, X_v, X_{uu})}{\sqrt{EG-F^2}}$$
    $$g = \left\langle N, X_{vv} \right\rangle = \frac{det(X_u, X_v, X_{uv})}{\sqrt{EG-F^2}}$$
    $$f = \left\langle N, X_{uv} \right\rangle = \frac{det(X_u, X_v, X_{vv})}{\sqrt{EG-F^2}}$$
\end{proposition}

\begin{proposition}
    $$dN_p = -I_p^{-1} \amalg_p$$
\end{proposition}

\begin{proposition}
    $$K = \frac{eg-f^2}{EG-F^2}$$
    $$H = \frac{1}{2} \frac{eG - 2fF + gE}{EG-F^2}$$
    $K$ y $H$ son diferenciables.
\end{proposition}

\begin{remark}
    Si una parametrización de una superficie regular es tal que $F = f = 0$, entonces las matrices de $I_p$ y $\amalg_p$ son diagonales y $dN_p$ es diagonal en $\{X_u, X_v\}$.
    $$dN_p = -
        \begin{pmatrix}
            E & 0 \\
            0 & G
        \end{pmatrix}^{-1}
        \begin{pmatrix}
            e & 0 \\
            0 & g
        \end{pmatrix} = -
        \begin{pmatrix}
            \frac{e}{E} & 0           \\
            0           & \frac{g}{G}
        \end{pmatrix}$$
    Así que las curvaturas principales son $\frac{e}{E}$ y $\frac{g}{G}$.
\end{remark}

\section{Geometría intrínseca}

\begin{definition}
    $\varphi : S \to \bar{S}$ es isometría si:
    \begin{itemize}
        \item $\varphi$ es difeomorfismo.
        \item $d\varphi_p : T_pS \to T_{\varphi(p)}\bar{S}$ es isometría lineal para todo $p \in S$.
    \end{itemize}
    Se dice que $S$ y $\bar{S}$ son isométricas.
\end{definition}

\begin{remark}
    Como $\varphi$ es difeomorfismo, entonces $d\varphi_p$ es isomorfismo.
    Luego $d\varphi_p$ es isometría lineal si y solo si conserva el producto escalar.\\
    Esto es, si para cada $w_1, w_2 \in T_pS$
    $$\left\langle w_1, w_2 \right\rangle _p = \left\langle d\varphi_p(w_1), d\varphi_p(w_2) \right\rangle _{\varphi(p)}$$
\end{remark}

\begin{definition}
    Sea $V$ entorno abierto de $p$ en $S$.
    $\varphi : V \to \bar{S}$ es isometría local si existe $\bar{V}$ entorno abierto de $\varphi(p)$ en $\bar{S}$ tal que $\varphi : V \to \bar{V}$ es isometría.\\
    $S$ es localmente isométrica a $\bar{S}$ si para todo $p \in S$ existe isometría local a $\bar{S}$.
    Si además $\bar{S}$ es localmente isométrica a $S$, se dice que $S$ y $\bar{S}$ son localmente isométricas.
\end{definition}

\begin{remark}
    Ser localmente isométricas no implica que exista $\varphi : S \to \bar{S}$ isometría local, porque puede que no sea la misma en todos los puntos.
\end{remark}

\begin{proposition}
    Sean $X: U \subset \mathbb{R}^2 \to S, \bar{X}: U \subset \mathbb{R}^2 \to \bar{S}$ parametrizaciones tales que $E = \bar{E}, F = \bar{F}, G = \bar{G}$ en $U$.
    Entonces $\varphi = \bar{X} \circ X^{-1}: X(U) \to \bar{X}(U)$ es isometría, es decir, $\varphi: X(U) \to \bar{S}$ es isometría local.
\end{proposition}

\begin{proposition}
    Sea $\varphi: S \to \bar{S}$ isometría, $X: U \to S$ parametrización.
    Entonces $\bar{X} = \varphi \circ X: U \to \bar{S}$ es una parametrización de $\bar{S}$ en $\varphi(p)$ con $E = \bar{E}, F = \bar{F}, G = \bar{G}$ en $U$.
\end{proposition}

\begin{note}
    La inversa y la composición de isometrías son isometrías.
\end{note}

\begin{proposition}
    Sea $\varphi: S \to \bar{S}$ difeomorfismo.
    \begin{enumerate}
        \item $\varphi$ es isometría si y solo si $\varphi$ conserva la longitud de arco de las curvas parametrizadas en $S$.
        \item Si $\varphi$ es isometría entonces $\varphi$ conserva ángulos y áreas.
    \end{enumerate}
\end{proposition}

\begin{definition}
    $\varphi: S \to \bar{S}$ es aplicación conforme si
    \begin{itemize}
        \item $\varphi$ es difeomorfismo.
        \item Para todo $p \in S$, $w_1, w_2 \in T_pS$
              $$\left\langle d\varphi_p(w_1), d\varphi_p(w_2) \right\rangle _{\varphi(p)} = \lambda^2(p) \left\langle w_1, w_2 \right\rangle _p$$
              donde $\lambda^2$ es una función diferenciable no nula sobre $S$.
    \end{itemize}
    Se dice que $S$ y $\bar{S}$ son conformes.
\end{definition}

\begin{definition}
    $\varphi: V \to \bar{S}$, con $V$ entorno abierto de $p$ en $S$, es aplicación conforme local en $p$ si existe un entorno abierto $\bar{V}$ de $\varphi(p)$ en $\bar{S}$ tal que $\varphi: V \to \bar{V}$ es aplicación conforme.
    Si para todo $p \in S$ existe una aplicación conforme local en $p$, entonces $S$ es localmente conforme a $\bar{S}$.
\end{definition}

\begin{remark}
    Una isometría es una aplicación conforme con $\lambda(p) = 1, \forall p$.
\end{remark}

\begin{note}
    Una aplicación conforme no conserva longitudes de curvas, mientras que conserva los ángulos.
\end{note}

\begin{proposition}
    Sean $X: U \subset \mathbb{R}^2 \to S, \bar{X}: U \subset \mathbb{R}^2 \to \bar{S}$ parametrizaciones tales que $\bar{E} = \lambda^2 E, \bar{F} = \lambda^2 F, \bar{G} = \lambda^2 G$ en $U$, con $\lambda^2$ diferenciable y $\lambda^2 \neq 0$ en $U$.
    Entonces $\varphi = \bar{X} \circ X^{-1}: X(U) \to \bar{S}$ es una aplicación conforme local.
\end{proposition}

\begin{corollary}
    Dos superficies cualesquiera son localmente conformes.
\end{corollary}

\begin{proposition}
    Sea $\varphi: S \to \bar{S}$ aplicación conforme, $X: U \to S$ parametrización.
    Entonces $\bar{X} = \varphi \circ X: U \to \bar{S}$ es una parametrización de $\bar{S}$ en $\varphi(p)$ con $\bar{E} = \lambda^2 E, \bar{F} = \lambda^2 F, \bar{G} = \lambda^2 G$ en $U$.
\end{proposition}

\section{El teorema de Gauss}

\begin{proposition}
    Sea $S$ una superficie regular orientable y orientada, con orientación $N: S \to S^2$ y sea $X: U \subset \mathbb{R}^2 \to S$ una parametrización de $S$ compatible con la orientación $N$.
    Es decir, $N = \frac{X_u \land X_v}{|X_u \land X_v|}$ en $U$.
    A cada punto de $X(U)$ se le puede asignar un triedro $\{X_u, X_v, N\}$, que es una base de $\mathbb{R}^3$.
    Derivando estos vectores con respecto de $u$ y $v$ obtenemos las siguientes ecuaciones:
    $$\left\{
        \begin{array}{lcl}
            X_{uu} & = & \Gamma^1_{11}X_u + \Gamma^2_{11}X_v + eN         \\
            X_{uv} & = & \Gamma^1_{12}X_u + \Gamma^2_{12}X_v + fN = X{vu} \\
            X_{vv} & = & \Gamma^1_{22}X_u + \Gamma^2_{22}X_v + gN         \\
            N_u    & = & a_{11}X_u + a_{21}X_v                            \\
            N_v    & = & a_{12}X_u + a_{22}X_v
        \end{array}
        \right.$$
    Estas se conocen como ecuaciones de Gauss-Weingarten.
    Los $\Gamma^k_{ij}$ se denominan los símbolos de Christoffel de $S$ en la parametrización $X$.
\end{proposition}

\begin{proposition}
    Consideramos las relaciones
    $$\left\{
        \begin{array}{lcl}
            (X_{uu})_v - (X_{uv})_u & = & 0 \\
            (X_{vv})_u - (X_{vu})_v & = & 0 \\
            N_{uv} - N_{vu}         & = & 0
        \end{array}
        \right.$$
    Estas son las condiciones que hacen que el sistema de ecuaciones en derivadas parciales que definen las ecuaciones de Gauss-Weingarten sea integrable dada una condición inicial.\\
    Sustituyendo las ecuaciones de Gauss-Weingarten en estas relaciones y resolviendo llegamos a la fórmula de Gauss si $E \neq 0$:
    $$\Gamma^1_{11} \Gamma^2_{12} + \Gamma^2_{11} \Gamma^2_{22} + (\Gamma^2_{11})_v - \Gamma^1_{12} \Gamma^2_{11} - \Gamma^2_{12} \Gamma^2_{12} - (\Gamma^2_{12})_u = EK$$
\end{proposition}

\begin{remark}
    Como $E \neq 0$, la fórmula de Gauss permite obtener la curvatura de Gauss a partir de los símbolos de Christoffel, es decir, que $K$ es intrínseco.
\end{remark}

\begin{note}
    Existen otras dos versiones de la fórmula de Gauss cuando $F \neq 0$ y $G \neq 0$, respectivamente.
\end{note}

\begin{theorem}[Teorema Egregium de Gauss]
    La curvatura de Gauss de una superficie regular es invariante frente a isometrías locales.
\end{theorem}

\begin{remark}
    El recíproco del teorema Egregium no es cierto en general.
    Sin embargo, si dos superficies tienen la misma curvatura de Gauss constante, entonces dos entornos cualesquiera suficientemente pequeños de esas superficies son isométricos.
\end{remark}

\begin{note}
    Siempre es posible calcular los símbolos de Christoffel en términos de los coeficientes de la primera forma fundamental y de sus derivadas.
\end{note}

\begin{remark}
    Todos los conceptos geométricos y propiedades que se expresan en términos de los símbolos de Christoffel son invariantes por isometrías, puesto que solo dependen de la primera forma fundamental.
\end{remark}

\begin{proposition}
    Las ecuaciones de Mainardi-Codazzi son
    $$\left\{
        \begin{array}{lcl}
            e_v - f_u & = & e\Gamma^1_{12} + f(\Gamma^2_{12} - \Gamma^1_{11}) - g\Gamma^2_{11} \\
            f_v - g_u & = & e\Gamma^1_{22} + f(\Gamma^2_{22} - \Gamma^1_{12}) - g\Gamma^2_{12}
        \end{array}
        \right.$$
\end{proposition}

\begin{note}
    Las fórmulas de Gauss y las ecuaciones de Mainardi-Codazzi se conocen como ecuaciones de compatibilidad de la teoría de superficies.
\end{note}

\begin{theorem}[Teorema de Bonnet]
    Sean $E, F, G, e, f, g$ funciones diferenciables definidas en un conjunto abierto $V \subset \mathbb{R}^2$ con $E > 0$, $G > 0$, y que verifican la fórmula de Gauss y las ecuaciones de Mainardi-Codazzi y $EG-F^2 > 0$.
    Entonces para cada $q \in V$ existe un entorno abierto $U$ de $q$ en $V$ y un difeomorfismo
    $$X: U \to X(U) \subset \mathbb{R}^3$$
    tal que la superficie regular $X(U)$ tiene a $E, F, G$ y a $e, f, g$ como coeficientes de la primera y la segunda formas fundamentales respectivamente.\\
    Además, si $U$ es conexo y si
    $$\bar{X}: U \to \bar{X}(U) \subset \mathbb{R}^3$$
    es otro difeomorfismo que satisface las mismas condiciones, entonces existe un movimiento rígido directo de $\mathbb{R}^3$, $\psi: \mathbb{R}^3 \to \mathbb{R}^3$, tal que $\bar{X} = \psi \circ X$.
\end{theorem}

\section{Parametrizaciones especiales}

\begin{definition}[Parametrización ortogonal]
    Se dice que una parametrización de una superficie regular $S$, $X: U \subset \mathbb{R}^2 \to S$, $(u, v) \in U$, es ortogonal si la función $\left\langle X_u, X_v \right\rangle$ es idénticamente cero en $U$, es decir, si las curvas coordenadas de $X$ se cortan ortogonalmente en cualquier punto de $X(U)$.
\end{definition}

\begin{remark}
    En una parametrización ortogonal $F = 0$ en $X(U)$, es decir, la matriz de $I_p$ en la base $\{X_u, X_v\}_q$ de $T_pS$ es diagonal.
    $$I_p \equiv
        \begin{pmatrix}
            E & 0 \\
            0 & G
        \end{pmatrix}$$
    Esto permite obtener la siguiente expresión de la curvatura de Gauss.
    $$K = -\frac{1}{2\sqrt{EG}} \left( \left( \frac{E_v}{\sqrt{EG}}\right)_v + \left( \frac{G_u}{\sqrt{EG}}\right)_u \right) $$
\end{remark}

\begin{theorem}
    Dado un punto $p$ cualquiera de una superficie regular $S$ existe una parametrización ortogonal de $S$ en $p$.
\end{theorem}

\begin{definition}[Parametrización por líneas de curvatura]
    Una parametrización de una superficie regular $S$, $X: U \to S$, es una parametrización por líneas de curvatura si $F = f = 0$, es decir, si las curvas coordenadas son las líneas de curvatura.
\end{definition}

\begin{remark}
    En este caso, respecto a $\{X_u, X_v\}$, las matrices de $I_p$ y $\amalg_p$ son diagonales.
    $$I_p \equiv
        \begin{pmatrix}
            E & 0 \\
            0 & F
        \end{pmatrix}, \quad
        \amalg_p \equiv
        \begin{pmatrix}
            e & 0 \\
            0 & g
        \end{pmatrix}$$
    Luego, calculando $dN_p$, obtenemos que $\frac{e}{E}$ y $\frac{g}{G}$ son las curvaturas principales.
\end{remark}

\begin{proposition}
    Las curvas coordenadas de una parametrización en un entorno sin puntos umbilicales son las líneas de curvatura si y solo si $f = F = 0$ en todos los puntos del entorno.
\end{proposition}

\begin{corollary}
    Las curvas coordenadas de una parametrización en un entorno de un punto no umbilical son las líneas de curvatura si y solo si $F = f = 0$ en el entorno.
\end{corollary}

\begin{theorem}
    Sea $p$ un punto no umbilical de una superficie regular $S$.
    Entonces existe una parametrización por líneas de curvatura en un entorno de $p$.
\end{theorem}
\chapter{Familias normales}

\section{Familias normales}
\begin{theorem}[Teorema de convergencia de Weierstrass]
    Sea $D$ abierto en $\mathbb{C}$ y sean $\{f_n\}_{n=1}^\infty$ una sucesión de funciones holomorfas en $D$ y $f: D \to \mathbb{C}$.
    Si $f_n \xrightarrow[n \to \infty]{} f$ uniformemente en cada subconjunto compacto de $D$, entonces $f$ es holomorfa en $D$ y $f_n' \xrightarrow[n \to \infty]{} f'$ uniformemente en cada subconjunto compacto.
    Para todo $k \in \mathbb{N}$, $f^{(k)}_n \xrightarrow[n \to \infty]{} f^{(k)}$ uniformemente en cada compacto.
\end{theorem}

\begin{definition}
    Sea $D$ un abierto en $\mathbb{C}$ y sea $\mathcal{F}$ una familia de funciones holomorfas en $D$.
    Diremos que $\mathcal{F}$ es finitamente normal si para cada sucesión $\{f_n\}_{n=1}^\infty$ en $\mathcal{F}$ existe una subsucesión $\{f_{n_k}\}_{k=1}^\infty$ de $\{f_n\}$ que converge uniformemente en cada subconjunto compacto de $D$.
\end{definition}

\begin{remark}
    El límite $f$ de tal subsucesión es una función holomorfa en $D$, pero no tiene por qué pertenecer a $\mathcal{F}$.
\end{remark}

\begin{definition}
    Sea $D$ un abierto en $\mathbb{C}$ y sea $\mathcal{F}$ una familia de funciones holomorfas en $D$.
    Diremos que $\mathcal{F}$ es compacta si para cada sucesión $\{f_n\}_{n=1}^\infty$ en $\mathcal{F}$ existe una subsucesión $\{f_{n_k}\}_{k=1}^\infty$ de $\{f_n\}$ que converge uniformemente en cada subconjunto compacto de $D$ a una función que pertenece a $\mathcal{F}$.
\end{definition}

En el conjunto $Hol(D)$ de las funciones holomorfas en $D$, con $D$ abierto en $\mathbb{C}$, se puede definir una distancia $d$ tal que $(Hol(D), d)$ es un espacio métrico completo, y en el que:
$$f_n \xrightarrow{d} f \Leftrightarrow f_n \to f \text{ uniformemente en cada subconjunto compacto de } D$$
Si $\mathcal{F} \subset Hol(D)$, $\mathcal{F}$ es finitamente normal si y solo si $\mathcal{F}$ es relativamente compacto.
Los compactos coinciden con la definición de familia compacta dada.

\section{El teorema de Montel}
\begin{lemma}
    Sea $D$ un abierto en $\mathbb{C}$ y $\mathcal{F}$ una familia de funciones holomorfas en $D$.
    Entonces son equivalentes:
    \begin{enumerate}
        \item $\mathcal{F}$ está uniformemente acotada en cada subconjunto compacto de $D$.
        \item Para cada $a \in D$ existe $r_a > 0$ con $D(a, r_a) \subset D$ y $f$ está uniformemente acotada en $D(a, r_a)$.
    \end{enumerate}
\end{lemma}

\begin{lemma}
    Sea $D$ abierto en $\mathbb{C}$ y sean $f_n: D \to \mathbb{C}$ para $n = 1, 2, \dots$ y $f: D \to \mathbb{C}$.
    Entonces son equivalentes:
    \begin{enumerate}
        \item $f_n \to f$ uniformemente en cada subconjunto compacto de $D$.
        \item Para cada $a \in D$ existe $r_a > 0$ con $D(a, r_a) \subset D$ tal que $f_n \to f$ uniformemente en $D(a, r_a)$.
    \end{enumerate}
\end{lemma}

\begin{lemma}
    Sean $C_1, C_2 \in \mathbb{C}$, con $C_1 \cap C_2 = \emptyset$ y $C_1, C_2 \neq \emptyset$.
    Si $C_1$ es compacto y $C_2$ es cerrado, entonces:
    $$dist(C_1, C_2) = \inf\{|z_1-z_2| : z_1 \in C_1, z_2 \in C_2\} > 0$$
\end{lemma}

\begin{remark}
    Si $C_1$ no es compacto no es cierto en general.
\end{remark}

\begin{lemma}
    Sea $A \subset \mathbb{C}$, $A \neq \emptyset$ y sea
    $$F: \mathbb{C} \to \mathbb{R}, \; F(z) = dist(z, A) = \inf \{|z-a| : a \in A\}$$
    Entonces $F$ es continua y $F(z) = 0$ para todo $z \in A$.
    Si además $A$ es cerrado, entonces $F(z) = \min \{|z-a| : a \in A\}$ para todo $z \in \mathbb{C}$.
\end{lemma}

\begin{lemma}
    Sea $A \subset \mathbb{C}$, $A \neq \emptyset$ y sea $\varepsilon > 0$.
    Consideramos los conjuntos:
    \begin{align*}
        B & = \{z \in \mathbb{C} : dist(z, A) < \varepsilon\}    \\
        C & = \{z \in \mathbb{C} : dist(z, A) \leq \varepsilon\}
    \end{align*}
    Entonces $B$ es abierto y $C$ es cerrado, con $A \subset B \subset C$.
    Si además $A$ es acotado, entonces $B$ es acotado y $C$ es compacto.
\end{lemma}

\begin{proposition}
    Sea $D$ un abierto en $\mathbb{C}$ y $\mathcal{F}$ una familia de funciones holomorfas en $D$.
    Supongamos que $\mathcal{F}$ está uniformemente acotada en $D$.
    Sea $K$ un subconjunto compacto de $D$.
    Entonces existe $A > 0$ tal que:
    $$|f(z_2)-f(z_1)| \leq A|z_2-z_1|, \quad \forall z_1, z_2 \in K, \; \forall f \in \mathcal{F}$$
\end{proposition}

\begin{proof}
    Sea $M > 0$ tal que $|f(z)| \leq M$ para todo $z \in D$ y para toda $f \in \mathcal{F}$.
    Sean $K \subset D$, $K$ compacto.
    Sea $d > 0$ con $d < dist(K, \mathbb{C} \setminus D)$.
    Si $D = \mathbb{C}$, tomamos $d > 0$ cualquiera.
    Sea $z_0 \in K$.
    Entonces $D(z_0, d) \subset D$.
    De hecho, podemos tomar $\varepsilon > 0$ tal que $D(z_0, d+\varepsilon) \subset D$.
    Dada $f \in \mathcal{F}$, por la fórmula de Cauchy,
    $$f'(z) = \frac{1}{2\pi i} \int_{|\xi-z_0|=d} \frac{f(\xi)}{(\xi-z)^2}d\xi \quad \text{si } z \in D\left(z_0, \frac{d}{2}\right)$$
    Entonces:
    $$|f'(z)| \leq \frac{1}{2\pi}2\pi \max_{|\xi-z_0|=d} \frac{|f(\xi)|}{|\xi-z|^2}$$
    Podemos acotar:
    $$|\xi-z| = |(\xi-z_0) + (z_0-z)| \geq |\xi-z_0| - |z_0-z| \geq d - \frac{d}{2} = \frac{d}{2}$$
    Así que $|\xi-z|^2 \geq \frac{d^2}{4} > 0$.
    Luego:
    $$|f'(z)| \leq d\frac{M}{d^2/4} = \frac{4M}{d}$$
    Hemos probado que si $z_0 \in K$, $f \in \mathcal{F}$ y $z \in D\left(z_0, \frac{d}{2}\right) \subset D$, entonces $|f'(z)| \leq \frac{4M}{d}$.

    Ahora, sean $z_1, z_2 \in K$ y $f \in \mathcal{F}$.
    Supongamos que $|z_1-z_2| < \frac{d}{2}$.
    Si $\xi \in [z_1, z_2]$, entonces $z_2 \in D\left(z_1, \frac{d}{2}\right) \subset D$ y $|\xi-z_1| \leq |z_1-z_2| < \frac{d}{2}$, $\xi \in D\left(z_1, \frac{d}{2}\right)$.
    Entonces $\xi \in D$ y $|f'(\xi)| \leq \frac{4M}{d}$.
    Por tanto:
    $$|f(z_2)-f(z_1)| = \left|\int_{[z_1, z_2]} f'(\xi)d\xi\right| \leq |z_2-z_1| \max_{\xi \in [z_1, z_2]} |f'(\xi)| \leq |z_2-z_1| \frac{4M}{d}$$
    Entonces, si $z_1, z_2 \in K$, $|z_1-z_2| < \frac{d}{2}$ y $f \in \mathcal{F}$, se tiene que:
    $$|f(z_2)-f(z_1)| \leq A|z_2-z_1|$$

    Ahora, si $z_1, z_2 \in K$, $|z_2-z_1| \geq \frac{d}{2}$ y $f \in \mathcal{F}$, tenemos:
    $$|f(z_2)-f(z_1)| \leq |f(z_2)| + |f(z_1)| \leq 2M = 2M \frac{d}{2}\frac{2}{d} \leq \frac{4M}{d}|z_2-z_1| = A|z_2-z_1|$$
\end{proof}

\begin{theorem}[Teorema de Arzelá-Ascoli]
    Sean $(X_1, d_1)$ y $(X_2, d_2)$ dos espacios métricos, siendo $(X_1, d_1)$ separable y $(X_2, d_2)$ completo.
    Sea $\mathcal{F}$ una familia de aplicaciones continuas de $X_1$ en $X_2$ que verifica:
    \begin{enumerate}
        \item $\mathcal{F}$ es puntualmente equicontinua.
              Es decir, dado $x \in X_1$ se verifica que, para todo $\varepsilon > 0$, existe $\delta > 0$ tal que si $y \in X_1$ con $d_1(x, y) < \delta$, entonces $d_2(f(x), f(y)) < \varepsilon$ para toda $f \in \mathcal{F}$.
        \item Para todo $x \in X_1$, el conjunto $\{f(x) : f \in \mathcal{F}\}$ es relativamente compacto.
    \end{enumerate}
    Entonces, si $\{f_n\}_{n=1}^\infty$ es una sucesión en $\mathcal{F}$, existe una subsucesión $\{f_{n_k}\}_{k=1}^\infty$ de $\{f_n\}$ que converge uniformemente en cada subconjunto compacto de $X_1$.
\end{theorem}

\begin{theorem}[Teorema de Montel]
    Sea $D$ un abierto en $\mathbb{C}$ y sea $\mathcal{F}$ una familia de funciones holomorfas en $D$.
    Entonces son equivalentes:
    \begin{enumerate}
        \item $\mathcal{F}$ es finitamente normal.
        \item $\mathcal{F}$ está uniformemente acotada en cada subconjunto compacto de $D$.
              Es decir, para cada $K \subset D$, $K$ compacto, existe $M_K > 0$ tal que $|f(z)| \leq M_K$ para toda $f \in \mathcal{F}$ y para todo $z \in K$.
    \end{enumerate}
\end{theorem}

\begin{proof}
    \hfill
    \begin{itemize}
        \item[$\Rightarrow$] Sea $D$ abierto en $\mathbb{C}$ y sea $\mathcal{F}$ una familia de funciones holomorfas en $D$, con $\mathcal{F}$ finitamente normal.
            Supongamos por reducción al absurdo que existe $K \subset D$, $K$ compacto, tal que $\mathcal{F}$ no está uniformemente acotada en $K$.
            Entonces existen $\{z_n\}_{n=1}^\infty$ en $K$ y $\{f_n\}_{n=1}^\infty$ en $\mathcal{F}$ tales que $|f_n(z_n)| \to \infty$.

            Como $\mathcal{F}$ es una familia finitamente normal, existe $\{f_{n_k}\}_{k=1}^\infty$ subsucesión de $\{f_n\}$ tal que $\{f_{n_k}\}$ converge uniformemente en cada subconjunto compacto de $D$ a una función $f$ holomorfa en $D$.
            Como $f$ es continua en $K$ y $K$ es compacto, existe $M > 0$ tal que $|f(z)| \leq M$ para todo $z \in K$.
            Por otro lado, como $f_{n_k} \xrightarrow[k \to \infty]{} f$ uniformemente en $K$, existe $k_0 \in \mathbb{N}$ tal que $k \geq k_0$, $z \in K$ $\Rightarrow$ $|f_{n_k}(z)-f(z)| < 1$.
            Entonces $|f_{n_k}(z)| \leq |f_{n_k}(z)-f(z)| + |f(z)| < 1 + M$, $z \in K$, $k \geq k_0$.
            En particular, $|f_{n_k}(z_{n_k})| < 1 + M$ si $k \geq k_0$.
            Esta es una contradicción.

        \item[$\Leftarrow$] Sea $D$ abierto en $\mathbb{C}$ y sea $\mathcal{F}$ una familia de funciones holomorfas en $D$, uniformemente acotada en cada subconjunto compacto de $D$.
            Tomamos $X_1 = D$ y $X_2 = \mathbb{C}$.

            \begin{enumerate}
                \item Sea $z_0 \in \mathbb{C}$.
                      Dado $\varepsilon > 0$, veamos que existe $\delta > 0$ tal que, si $z_1 \in D$, $|z_1-z_0| < \delta$, $f \in \mathcal{F}$, entonces $|f(z_1) - f(z_0)| < \varepsilon$.
                      Sea $R > 0$ con $\overline{D}(0, R) \subset D$.
                      $\mathcal{F}$ está uniformemente acotada en $\overline{D}(z_0, R)$ y por tanto en $D(z_0, R)$.
                      Sea $K = \overline{D}(z_0, \frac{R}{2})$, que es un subconjunto compacto de $D(z_0, R)$.
                      Por la proposición anterior, existe $A > 0$ tal que
                      $$|f(z_2) - f(z_1)| \leq A|z_2-z_1|, \quad \text{si } z_1, z_2 \in K, f \in \mathcal{F}$$
                      Entonces, si $\delta = \min \left(\frac{\varepsilon}{A}, \frac{R}{2}\right)$, $z_1 \in D$, $|z_1-z_0| < \delta$ y $f \in \mathcal{F}$, entonces $z_1 \in \overline{D}(z_0, \frac{R}{2}) = K$, así que:
                      $$|f(z_1) - f(z_0)| \leq A|z_1-z_0| < A\delta \leq A\frac{\varepsilon}{A} = \varepsilon$$

                \item Sea $z \in D$.
                      El conjunto $\{f(z) : f \in \mathcal{F}\}$ está acotado, ya que $\mathcal{F}$ está uniformemente acotada en $\{z\}$.
                      Por tanto, su clausura es compacta.
            \end{enumerate}

            Entonces, por el teorema de Arzelá-Ascoli, existe una subsucesión $\{f_{n_k}\}_{k=1}^\infty$ de $\{f_n\}$ que converge uniformemente en cada subconjunto compacto de $D$.
            Por tanto, $\mathcal{F}$ es finitamente normal.
    \end{itemize}
\end{proof}

\begin{remark}
    \hfill
    \begin{enumerate}
        \item Sea $D$ un abierto en $\mathbb{C}$.
              Si $\mathcal{F}$ es una familia finitamente normal de funciones holomorfas en $D$, entonces la familia $\mathcal{F}' = \{f' : f \in \mathcal{F}\}$ es finitamente normal.
              En general, si $k \in \mathbb{N}$, la familia $\mathcal{F}^{(k)} = \{f^{(k)} : f \in \mathcal{F}\}$ es finitamente normal.

              Sea $\{g_n\}_{n=1}^\infty$ en $\mathcal{F}'$.
              Entonces $g_n = f_n'$, $f_n \in \mathcal{F}$.
              Existe $\{f_{n_k}\}_{k=1}^\infty$ subsucesión de $\{f_n\}$ que converge uniformemente en cada subconjunto compacto de $D$ a una función $f$ holomorfa en $D$.
              Entonces $g_{n_k} = f_{n_k}' \to f'$ uniformemente en cada subconjunto compacto de $D$.

        \item Sea $D$ abierto en $\mathcal{C}$ y sea $\mathcal{G}$ familia finitamente normal de funciones holomorfas en $D$ con $\mathcal{F} \subset \mathcal{G}$.
              Entonces $\mathcal{F}$ es finitamente normal.

        \item Si $a \in \mathbb{C}$, $R > 0$ y $K \subset D(a, R)$, $K$ compacto, entonces existe $r \in (0, R)$ tal que $K \subset \overline{D}(a, r)$.
    \end{enumerate}
\end{remark}

\begin{example}
    \hfill
    \begin{enumerate}
        \item $\mathcal{F} = \{f : f \text{ es entera y } |f(z)| \leq n \text{ si } |z| = n, n = 1, 2, \dots\}$.
              Sea $K \subset \mathbb{C}$, $K$ compacto, y sea $f \in \mathcal{F}$.
              Existe $n_0 \in \mathbb{N}$ tal que $K \subset \overline{D}(0, n_0)$.
              Además, $|f(z_0)| \leq n_0$ si $|z| = n_0$.
              Por el principio del máximo, $|f(z)| \leq n_0$ si $|z| \leq n_0$.
              En particular, $|f(z)| \leq n_0$ si $z \in K$ y $f \in \mathcal{F}$.
              $\mathcal{F}$ está uniformemente acotada en $K$.
              Por el teorema de Montel, $\mathcal{F}$ es finitamente normal.

        \item $\mathcal{P} = \{f : f \text{ es holomorfa en } \mathbb{D}, f(0) = 1, Re(f(z)) > 0 \; \forall z \in \mathbb{D}\}$.
              Sea $K \subset \mathbb{D}$, $K$ compacto.
              Si $f \in \mathcal{P}$ y $z \in K$,
              $$|f(z)| \leq \frac{1+|z|}{1-|z|}$$
              Existe $R \in (0, 1)$ tal que $K \subset \overline{D}(0, R)$.
              Entonces, si $f \in \mathcal{P}$ y $z \in K$,
              $$|f(z)| \leq \frac{1+|z|}{1-|z|} \leq \frac{1+R}{1-R}$$
              $\mathcal{P}$ está uniformemente acotada en $K$ para todo subconjunto compacto $K$ de $\mathbb{D}$.
              Por el teorema de Montel, $\mathcal{P}$ es finitamente normal.

              \begin{remark}
                  Si quitamos la condición $f(0) = 1$ en $\mathcal{P}$, la familia deja de ser finitamente normal.
                  Por ejemplo, $f_n(z) = n$, $n = 1, 2, \dots$, $\{f_n : n = 1, 2, \dots\} \subset \mathcal{P}$.
                  Si tomamos $K = \{0\}$, $\mathcal{P}$ no está uniformemente acotada en $K$, así que $\mathcal{P}$ no es finitamente normal.
              \end{remark}

              Recordemos que $\mathcal{P} = \{f : f \text{ es holomorfa en } \mathbb{D}, f \prec P\}$, con $P(z) = \frac{1+z}{1-z}$.
              Esto es un caso particular del siguiente ejemplo.

        \item Sea $F$ holomorfa en $\mathbb{D}$ y sea
              $$\mathcal{F}_F = \{f : f \text{ holomorfa en } \mathbb{D}, f \prec F\}$$
              Entonces $\mathcal{F}_F$ es finitamente normal.

        \item Sean $a \in \mathbb{C}$ y $R > 0$.
              Sea $\mathcal{F}$ una familia finitamente normal de funciones holomorfas en $D(a, R)$.
              Para cada $f \in \mathcal{F}$, consideramos el desarrollo de Taylor de $f$ centrado en $a$
              $$f(z) = \sum_{n=0}^\infty a_n(f)(z-a)^n, \quad z \in D(a, R)$$
              Entonces $M_n = \sup_{f \in \mathcal{F}} |a_n(f)| < \infty$ para cada $n$ y la serie de potencias $\sum_{n=0}^\infty M_n(z-a)^n$ tiene radio de convergencia mayor o igual que $R$, y por tanto define una función holomorfa en $D(a, R)$.

              % Demostración

        \item $\mathcal{F} = \{f : f \text{ es holomorfa en } \mathbb{D} \text{ y } \int\int_\mathbb{D} |f(z)|dxdy \leq M\}$, siendo $M > 0$.
              Veamos que $\mathcal{F}$ es finitamente normal.

              Sea $K \subset \mathbb{D}$, $K$ compacto.
              Tomamos $r \in (0, 1)$ con $K \subset D(0, r)$.
              Sea $f \in \mathcal{F}$ y $z \in K$, por la fórmula de Cauchy
              $$f(z) = \frac{1}{2\pi i} \int_{|z| = \rho} \frac{f(\xi)}{\xi-z}d\xi = \frac{1}{2\pi i} \int_{-\pi}^\pi \frac{f(\rho e^{i\theta})}{\rho e^{i\theta} - z}\rho d\rho, \quad r \leq \rho < 1$$

              $$|f(z)| \leq \frac{1}{2\pi} \int_{-\pi}^\pi \frac{|f(\rho e^{i\theta})|}{|\rho e^{i\theta} - z|}\rho d\rho, \quad r \leq \rho < 1$$

              $$\int_{\frac{1+r}{2}}^1 |f(z)|d\rho \leq \frac{1}{2\pi} \int_{\frac{1+r}{2}}^1 \int_{-\pi}^\pi \frac{|f(\rho e^{i\theta})|}{|\rho e^{i\theta} - z|}\rho d\theta d\rho = \frac{1}{2\pi} \int\int_{\frac{1+r}{2} < |w| < 1} \frac{|f(w)|}{|w-z|}dxdy$$
              Como $|w-z| \geq |w| - |z| > \frac{1+r}{2}-r = \frac{1-r}{2} > 0$,
              \begin{align*}
                   & \frac{1}{2\pi} \int\int_{\frac{1+r}{2} < |w| < 1} \frac{|f(w)|}{|w-z|}dxdy \leq \frac{1}{2\pi} \frac{2}{1-r} \int\int_{\frac{1+r}{2} < |w| < 1} |f(w)| dxdy \leq \\
                   & \leq \frac{1}{\pi(1-r)} \int\int_\mathbb{D} |f(w)|dxdy \leq \frac{M}{\pi(1-r)}
              \end{align*}
              Por otro lado,
              $$\int_{\frac{1+r}{2}}^1 |f(z)|d\rho = |f(z)|\left(1 - \frac{1+r}{2}\right) = |f(z)|\frac{1-r}{2}$$
              Entonces:
              $$|f(z)|\frac{1-r}{2} \leq \frac{M}{\pi(1-r)} \Rightarrow |f(z)| \leq \frac{2M}{\pi(1-r)^2}$$
              Por tanto, $\mathcal{F}$ está uniformemente acotada en $K$.
    \end{enumerate}
\end{example}

\begin{theorem}
    Sean $a \in \mathbb{C}$ y $R > 0$.
    Sea $\mathcal{F}$ una familia de funciones holomorfas en $D(a, R)$.
    Las siguientes condiciones son equivalentes:
    \begin{enumerate}
        \item $\mathcal{F}$ es finitamente normal.
        \item Existe una sucesión $\{M_n\}_{n=0}^\infty$ con $M_n \geq 0$ para todo $n$ tal que la serie de potencias $\sum_{n=0}^\infty M_n(z-a)^n$ tiene radio de convergencia mayor o igual que $R$ y tal que, si para cada $f \in \mathcal{F}$,
              $$f(z) = \sum_{n=0}^\infty a_n(f)(z-a)^n, \quad z \in D(a, r)$$
              se tiene que $|a_n(f)| \leq M_n$ para todo $n$ y para toda $f \in \mathcal{F}$.
    \end{enumerate}
\end{theorem}

\begin{proof}
    \hfill
    \begin{itemize}
        \item[$\Rightarrow$] $M_n = \sup_{f \in \mathcal{F}} |a_n(f)|$.
        \item[$\Leftarrow$] Sea $K \subset D(a, R)$, $K$ compacto.
            Existe $r \in (0, R)$ tal que $K \subset \overline{D}(a, r)$.
            Si $z \in K$ y $f \in \mathcal{F}$, se tiene:
            $$|f(z)| = \left|\sum_{n=0}^\infty a_n(f)(z-a)^n\right| \leq \sum_{n=0}^\infty |a_n(f)||z-a|^n \leq \sum_{n=0}^\infty M_n|z-a|^n \leq \sum_{n=0}^\infty M_nr^n < \infty$$
            ya que $\sum_{n=0}^\infty M_n(z-a)^n$ converge para $z = a+r$.
            $\mathcal{F}$ está uniformemente acotada en $K$.
    \end{itemize}
\end{proof}

\section{El teorema de Stieltjes-Vitali}
\begin{theorem}[Teorema de Stieltjes-Vitali]
    Sea $D$ un dominio en $\mathbb{C}$ y $\mathcal{F}$ una familia finitamente normal de funciones holomorfas en $D$.
    Sea $\{f_n\}_{n=1}^\infty$ una sucesión en $\mathcal{F}$.
    Si existe $A \subset D$ tal que $A$ tiene algún punto de acumulación en $D$, para el que existe $\lim\limits_{n \to \infty} f_n(a) \in \mathbb{C}$ para todo $a \in A$, entonces $\{f_n\}$ converge uniformemente en cada subconjunto compacto de $D$.
\end{theorem}

% Demostración

\begin{example}
    Para $x \geq 0$, tenemos que $\left(1 + \frac{x}{n}\right)^n$ es una sucesión creciente y
    $$\lim\limits_{n \to \infty} \left(1 + \frac{x}{n}\right)^n = e^x$$
    Veamos que $\lim\limits_{n \to \infty} \left(1 + \frac{z}{n}\right)^n = e^z$ para todo $z \in \mathbb{C}$, siendo la convergencia uniforme en cada subconjunto compacto de $\mathbb{C}$.

    Sea $D = \mathbb{C}$, $f_n(z) = \left(1 + \frac{z}{n}\right)^n$, $n \in \mathbb{N}$.
    Cada $f_n$ es una función entera.
    Veamos que $\mathcal{F}$ es finitamente normal.

    Sea $K \subset \mathbb{C}$, $K$ compacto.
    Tomamos $R > 0$ con $K \subset \overline{D}(0, R)$.
    Si $k \in K$ y $n \in \mathbb{N}$,
    $$|f_n(z)| = \left|1 + \frac{z}{n}\right|^n \leq \left(1 + \frac{|z|}{n}\right)^n \leq \left(1 + \frac{R}{n}\right)^n \leq e^R$$
    Sea $A = [0, 1]$.
    $A$ tiene puntos de acumulación en $\mathbb{C}$ y $\lim\limits_{n \to \infty} f_n(x) = e^x$ para todo $x \in A$.
    Por el teorema de Stieltjes-Vitali, $\{f_n\}$ converge uniformemente en cada subconjunto compacto de $\mathbb{C}$.
    Sea $f$ el límite, entonces $f$ es entera.
    Si $x \in A$, $\lim\limits_{n \to \infty} f_n(x) = f(x) = e^x$.
    Por el teorema de identidad, $f(z) = e^z$ si $z \in \mathbb{C}$.
\end{example}

\begin{theorem}[Teorema de Lindelöf]
    Sea $f$ holomorfa y acotada en $\mathbb{D}$.
    Sea $\xi \in \partial \mathbb{D}$ y supongamos que existe el límite radial, es decir, $\lim\limits_{r \to 1^-} f(r\xi) = L \in \mathbb{C}$.
    Entonces para todo $\alpha \in (0, \frac{\pi}{2})$ existe el límite tangencial de $f$ en $\xi$, es decir,
    $$\lim\limits_{z \to \xi, z \in S_\alpha(\xi)} f(z) = L$$
    siendo $S_\alpha(\xi)$ el vector de vértice $\xi$ y ángulo $2\alpha$, simétrico con respecto al segmento $[0, \xi]$.
\end{theorem}

\begin{theorem}
    Sea $f$ holomorfa y acotada en $D(1, 1)$.
    Supongamos que existe $\lim\limits_{x \to 0^+} f(x) = L \in \mathbb{C}$.
    Entonces para todo $\alpha \in (0, \frac{\pi}{2})$ existe
    $$\lim\limits_{z \to 0, |\Arg(z)| < \alpha} f(z) = L$$
\end{theorem}

\begin{proof}
    Sea $M > 0$ tal que $|f(z)| \leq M$ si $z \in D(1, 1)$.
    Consideramos la sucesión $\{f_n\}_{n=1}^\infty$, $f_n(z) = f\left(\frac{z}{n}\right)$.
    Cada $f_n$ es holomorfa en $D(1, 1)$.
    La familia $\mathcal{F} = \{f_n : n \in \mathbb{N}\}$ está uniformemente acotada en $D$, porque si $z \in D$ y $n \in \mathbb{N}$ se tiene que $|f_n(z)| = \left|f\left(\frac{z}{n}\right)\right| \leq M$.
    Así que $\mathcal{F}$ es finitamente normal.

    Sea $A = (0, 1)$.
    Si $x \in A$, $\lim\limits_{n \to \infty} f_n(x) = \lim\limits_{n \to \infty} f\left(\frac{x}{n}\right) = L$.
    Por el teorema de Stieltjes-Vitali, $\{f_n\}$ converge uniformemente en cada subconjunto compacto de $D(1, 1)$.
    Sea $g$ el límite, entonces $g$ es holomorfa en $D(1, 1)$ y $g(x) = L$ para todo $x \in A$.
    Por el teorema de identidad, $g(z) = L$ para todo $z \in D(1, 1)$.
    Hemos probado que $f_n \xrightarrow[n \to \infty]{} L$ uniformemente en cada subconjunto compacto de $D$.

    Sea $\alpha \in (0, \frac{\pi}{2})$.
    Sea $K = \left\{z \in \mathbb{C} : \frac{\cos(\alpha)}{2} \leq |z| \leq \cos(\alpha), |\Arg(z)| \leq \alpha\right\}$.
    $K$ es un subconjunto compacto de $D(1, 1)$, así que $f_n \xrightarrow[n \to \infty]{} L$ uniformemente en $K$.
    Es decir, existe $n_0 \in \mathbb{N}$ tal que, si $n \geq n_0$ y $z \in K$, entonces $|f_n(z)-L| < \varepsilon$.

    Sea $\delta = \frac{\cos(\alpha)}{2n_0} > 0$.
    Sea $z$ tal que $0 < |z| < \delta$ y $|\Arg(z)| < \alpha$.
    Observamos que $|z| < \frac{\cos(\alpha)}{2n_0} \leq \frac{\cos(\alpha)}{2}$.
    Tomamos $n_z$ el primer natural para el que $n_z|z| \geq \frac{\cos(\alpha)}{2}$.
    Como $|z| < \frac{\cos(\alpha)}{2n_0} \Leftrightarrow n_0|z| < \frac{\cos(\alpha)}{2}$, entonces $1 \leq n_0 < n_z$.
    Por otro lado,
    $$(n_z-1)|z| < \frac{\cos(\alpha)}{2} \Leftrightarrow n_z|z| - |z| < \frac{\cos(\alpha)}{2} \Leftrightarrow n_z|z| < |z| + \frac{\cos(\alpha)}{2} < \frac{\cos(\alpha)}{2} + \frac{\cos(\alpha)}{2} = \cos(\alpha)$$
    Así que $\frac{\cos(\alpha)}{2} \leq n_z|z| = |n_zz| < \cos(\alpha)$.
    Además, $|\Arg(n_zz)| = |\Arg(z)| < \alpha$.
    Por tanto, $n_zz \in K$.
    Entonces:
    $$|f_{n_k}(n_zz) - L| = |f(z) - L| < \varepsilon$$
\end{proof}

\section{Teoremas de Hurwitz}
\begin{theorem}[Teorema de Rouché]
    Sea $D$ un dominio simplemente conexo en $\mathbb{C}$ y sea $J$ un camino de Jordan en $D$.
    Sean $f$ y $g$ funciones holomorfas en $D$ tales que
    $$|f(z) - g(z)| < |f(z)| \quad \text{si } z \in J$$
    Entonces:
    \begin{enumerate}
        \item $I(J) \subset D$.
        \item Ni $f$ ni $g$ se anulan en $J$.
        \item $f$ y $g$ tienen el mismo número de ceros en $I(J)$.
    \end{enumerate}
\end{theorem}

\begin{theorem}[Primer teorema de Hurwitz]
    Sea $D$ un dominio en $\mathbb{C}$ y sea $\{f_n\}_{n=1}^\infty$ una sucesión de funciones holomorfas y nunca nulas en $D$, que converge uniformemente en cada subconjunto de $D$ a una función $f$.
    Entonces $f$ es nunca nula en $D$ o bien $f$ es idénticamente nula en $D$.
\end{theorem}

\begin{proof}
    Si $f \equiv 0$ en $D$, no hay nada que hacer.
    Supongamos que $f \not\equiv 0$ en $D$.
    Supongamos por reducción al absurdo que existe $a \in D$ con $f(a) = 0$.
    Entonces $a$ es un cero aislado de $f$.
    Podemos tomar $R > 0$ tal que $D(a, 2R) \subset D$ y $f$ no tiene ceros en $D(a, 2R) \setminus \{a\}$.

    Sea $C_R$ la circunferencia $|z-a| = R$.
    Como $f$ no tiene ceros en $C_R$, existe $\alpha > 0$ tal que $|f(z)| > \alpha$ para todo $z \in C_R$.
    Como $f_n \to f$ uniformemente en $C_R$, existe $n_0 \in \mathbb{N}$ tal que $n \geq n_0, z \in C_R \Rightarrow |f_n(z) - f(z)| < \alpha$.
    Entonces, si $n \geq n_0$ y $z \in C_R$, se tiene que
    $$|f_n(z) - f(z)| < \alpha < |f(z)|$$
    Por el teorema de Rouché, $f_n$ y $f$ tienen el mismo número de ceros en $D(a, R)$.
    Pero $f_n$ es nunca nula en $D$, por lo que no tiene ceros en $D(a, R)$, mientras que $f(a) = 0$.
    Esta es una contradicción.
    Entonces $f$ es nunca nula en $D$.
\end{proof}

\begin{theorem}[Segundo teorema de Hurwitz]
    Sea $D$ un dominio en $\mathbb{C}$ y sea $\{f_n\}_{n=1}^\infty$ una sucesión de funciones holomorfas e inyectivas en $D$.
    Si $\{f_n\}$ converge uniformemente a $f$ en cada subconjunto compacto de $D$, entonces $f$ es inyectiva o constante.
\end{theorem}

\begin{proof}
    Sabemos que $f$ es holomorfa en $D$.
    Supongamos que $f$ no es constante.
    Sean $a, b \in D$ con $a \neq b$.
    Veamos que $f(a) \neq f(b)$.
    $D \setminus \{a\}$ es un dominio en $\mathbb{C}$.
    Para cada $n \in \mathbb{N}$, sea $g_n(z) = f_n(z) - f_n(a)$ si $z \in D \setminus \{a\}$.
    Cada $g_n$ es holomorfa y nunca nula en $D \setminus \{a\}$.
    $f_n \to f$ uniformemente en cada subconjunto compacto de $D$.
    Sea $g(z) = f(z) - f(a)$, $z \in D \setminus \{a\}$.
    Entonces $g_n \to g$ uniformemente en cada subconjunto compacto de $D \setminus \{a\}$.
    Por el teorema anterior, $g \equiv 0$ en $D \setminus \{a\}$ o bien $g$ es nunca nula en $D \setminus \{a\}$.

    \begin{enumerate}
        \item Si $g \equiv 0$ en $D \setminus \{a\}$, entonces $f(z) = f(a)$ si $z \in D \setminus \{a\}$ $\Rightarrow$ $f(z) = f(a)$ si $z \in D$.
              $f$ es constante, lo que contradice nuestra hipótesis.
        \item Si $g(z) \neq 0$ si $z \in D \setminus \{a\}$, en particular $g(b) = f(b) - f(a) \neq 0 \Rightarrow f(a) \neq f(b)$.
    \end{enumerate}
\end{proof}
\chapter{Convergencia}
Sea $(\Omega, \mathcal{A}, P)$ un espacio de probabilidad, con $P: \mathcal{A} \to [0, 1]$.
Estudiaremos las sucesiones $\{A_n\}_{n \geq 1}$ con $A_i \in \mathcal{A}$ para todo $i \geq 1$.

\begin{definition}
    Sea $\{A_n\}_n \subset \mathcal{A}$.
    \begin{itemize}
        \item Definimos el límite superior de la sucesión como:
              $$\limsup\limits_{n \to \infty} A_n = \bigcap_{n \geq 1} \bigcup_{m \geq n} A_m \in \mathcal{A}$$
        \item Definimos el límite inferior de la sucesión como:
              $$\liminf\limits_{n \to \infty} A_n = \bigcup_{n \geq 1} \bigcap_{m \geq n} A_m \in \mathcal{A}$$
    \end{itemize}
\end{definition}

\begin{remark}
    $$\liminf\limits_{n \to \infty} A_n \subseteq \limsup\limits_{n \to \infty} A_n$$
\end{remark}

La sucesión $\{A_n\}_n$ converge si:
$$\liminf\limits_{n \to \infty} A_n = \limsup\limits_{n \to \infty} A_n = \lim\limits_{n \to \infty} A_n$$

\begin{definition}
    Sea $\{A_n\}_n \subset \mathcal{A}$.
    $\{A_n\}_n$ es monótona creciente si:
    $$A_1 \subseteq A_2 \subseteq \dots \subseteq A_n \subseteq \dots$$

    En ese caso,
    $$\lim\limits_{n \to \infty} A_n = \bigcup_{n \geq 1} A_n$$
\end{definition}

\begin{definition}
    Sea $\{A_n\}_n \subset \mathcal{A}$.
    $\{A_n\}_n$ es monótona decreciente si:
    $$A_1 \supseteq A_2 \supseteq \dots \supseteq A_n \supseteq \dots$$

    En ese caso,
    $$\lim\limits_{n \to \infty} A_n = \bigcap_{n \geq 1} A_n$$
\end{definition}

\begin{theorem}
    Sea $\{A_n\}_n \subset \mathcal{A}$ monótona.
    Entonces:
    $$P(\lim\limits_{n \to \infty} A_n) = \lim\limits_{n \to \infty} P(A_n)$$
\end{theorem}

\begin{theorem}
    Sea $\{A_n\}_n \subset \mathcal{A}$.
    Entonces:
    \begin{enumerate}
        \item $$P(\limsup\limits_{n \to \infty} A_n) = \lim\limits_{n \to \infty} P(\bigcup_{m \geq n} A_m)$$
        \item $$P(\liminf\limits_{n \to \infty} A_n) = \lim\limits_{n \to \infty} P(\bigcap_{m \geq n} A_m)$$
    \end{enumerate}
\end{theorem}

\begin{theorem}
    Sean $\{A_n\}_n \subset \mathcal{A}$ y $\omega \in \Omega$.
    Entonces:
    \begin{enumerate}
        \item $w \in \limsup\limits_{n \to \infty} A_n$ si y solo si existe una sucesión de índices
              $$n_1 < n_2 < \dots < n_k < \dots$$
              tal que $w \in A_{n_k}$, para $k = 1, 2, \dots$.
        \item $w \in \liminf\limits_{n \to \infty} A_n$ si y solo si existe $n_0 \geq 1$ tal que $w \in A_m$ para todo $m \geq n_0$.
    \end{enumerate}
\end{theorem}

\begin{theorem}[Primer lema de Borel-Cantelli]
    Sea $(\Omega, \mathcal{A}, P)$ un espacio de probabilidad y sea $\{A_n\}_n \subset \mathcal{A}$ tal que $\sum_{n \geq 1} P(A_n) < \infty$.
    Entonces:
    $$P(\limsup\limits_{n \to \infty} A_n) = 0$$
\end{theorem}

Veamos que el recíproco del primer lema de Borel-Cantelli no es cierto.
\begin{example}
    Sea $(\Omega, \mathcal{A}, P)$ el espacio de probabilidad dado por $\Omega = (0, 1)$, $A = \mathcal{B}_\Omega$ y $P$ la medida de Lebesgue.
    Consideramos la sucesión $\{A_n\}_n$ con $A_n = \left\{\left(0, \frac{1}{\sqrt{n}}\right)\right\}$.
    Observamos que:
    $$\limsup\limits_{n \to \infty} A_n = \bigcap_{n \geq 1} \bigcup_{m \geq n} \left(0, \frac{1}{\sqrt{n}}\right) = \emptyset \Rightarrow P(\limsup\limits_{n \to \infty} A_n) = 0$$
    Sin embargo,
    $$\sum_{n=1}^\infty P(A_n) = \sum_{i=1}^\infty \frac{1}{\sqrt{n}} = \infty$$
\end{example}

\begin{theorem}[Segundo lema de Borel-Cantelli]
    Sea $(\Omega, \mathcal{A}, P)$ un espacio de probabilidad y sea $\{A_n\}_n \subset \mathcal{A}$ con $A_i$ independientes tal que $\sum_{n \geq 1} P(A_n) = \infty$.
    Entonces:
    $$P(\limsup\limits_{n \to \infty} A_n) = 1$$
\end{theorem}
\chapter{Inferencia bayesiana}
\section{Teorema de Bayes}
\begin{theorem}[Teorema de Bayes]
    Sea $(\Omega, \mathcal{A}, P)$ un espacio de probabilidad.
    Sea $\{A_1, \dots, A_n\} \subset \mathcal{A}$ una partición de $\Omega$ y sea $B \in \mathcal{A}$ tal que $P(B) > 0$ y del que se conocen $P(B|A_i)$, para $i = 1, \dots, n$.
    Entonces
    $$P(A_i|B) = \frac{P(B|A_i)P(A_i)}{\sum_{j=1}^n P(B|A_j)P(A_j)}, \quad \forall i = 1, \dots, n$$
    donde:
    \begin{itemize}
        \item $P(A_j)$, $j = 1, \dots, n$, se llaman probabilidades a priori.
        \item $P(B|A_j)$, $j = 1, \dots, n$, se llaman verosimilitudes.
        \item $P(A_j|B)$, $j= 1, \dots, n$, se llaman probabilidades a posteriori.
    \end{itemize}

    Esta se conoce como la fórmula de Bayes.
\end{theorem}

\begin{remark}
    Las probabilidades a posteriori son proporcionales al producto de verosimilitudes y probabilidades a priori.
    $$P(A_i|B) \propto P(B|A_i)P(A_i)$$
\end{remark}

\begin{example}
    Una caja contiene una moneda legal $M_1$ y otra con una cara en cada lado $M_2$.
    \begin{enumerate}
        \item Se selecciona una de las dos monedas al azar, se lanza y sale cara.
              ¿Cuál es la probabilidad de que la moneda lanzada sea la legal?

              Definimos los sucesos:
              \begin{itemize}
                  \item $C_i$: en el lanzamiento $i$ sale cara.
                  \item $F_i$: en el lanzamiento $i$ sale cruz.
              \end{itemize}

              \begin{center}
                  \begin{tabular}{| c | c | c |}
                      \hline
                      Probabilidad a priori  & Verosimilitudes            & Probabilidad a posteriori  \\
                      \hline
                      $P(M_1) = \frac{1}{2}$ & $P(C_1|M_1) = \frac{1}{2}$ & $P(M_1|C_1) = \frac{1}{3}$ \\
                      $P(M_2) = \frac{1}{2}$ & $P(C_1|M_2) = 1$           & $P(M_2|C_1) = \frac{2}{3}$ \\
                      \hline
                  \end{tabular}
              \end{center}

        \item Lanzamos de nuevo la moneda elegida y se obtiene otra cara.
              ¿Cuál es la probabilidad de que la moneda lanzada sea la legal?

              Podemos usar el carácter secuencial del teorema de Bayes y usar los resultados anteriores.

              \begin{center}
                  \begin{tabular}{| c | c | c |}
                      \hline
                      Probabilidad a priori      & Verosimilitudes            & Probabilidad a posteriori           \\
                      \hline
                      $P(M_1|C_1) = \frac{1}{3}$ & $P(C_2|M_1) = \frac{1}{2}$ & $P(M_1|C_1 \cap C_2) = \frac{1}{5}$ \\
                      $P(M_2|C_1) = \frac{2}{3}$ & $P(C_2|M_2) = 1$           & $P(M_2|C_1 \cap C_2) = \frac{4}{5}$ \\
                      \hline
                  \end{tabular}
              \end{center}
    \end{enumerate}
\end{example}

\section{Teorema de Bayes generalizado}
\begin{theorem}[Teorema de Bayes generalizado]
    Sean $\vec{x} = (x_1, \dots, x_n)$ una muestra y $\theta$ una variable aleatoria.
    Sea $f_\theta$ la distribución a priori y $f(\vec{x}|\theta)$ la función de verosimilitud.
    Entonces:
    $$f(\theta|\vec{x}) = \frac{f(\vec{x}|\theta)f_\theta(\theta)}{f(\vec{x})}$$
    donde:
    $$f(\vec{x}) = \begin{cases}
            \sum_{i=1}^n f(\vec{x}|\theta_i)f_\theta(\theta_i)    & \text{si es discreta} \\
            \int_\Theta f(\vec{x}, \theta)f_\theta(\theta)d\theta & \text{si es continua}
        \end{cases}$$
\end{theorem}

\begin{remark}
    Para los clásicos, $\theta$ es un parámetro fijo y desconocido.
    En cambio, para los bayesianos $\theta$ es una variable aleatoria.
\end{remark}

% Ejemplo

\section{Familias de distribución conjugadas}
Las familias de distribución conjugadas son aquellas en las que las distribuciones a priori y a posteriori son de la misma familia.

% Ejemplos

\begin{lemma}
    $$A(z-a)^2 + B(z-b)^2 = (A+B)\left(z - \frac{Aa+Bb}{A+B}\right)^2 + \frac{AB}{A+B}(a-b)^2$$
\end{lemma}

% Ejemplo

\begin{definition}
    Sea $X \sim Ga(a, p)$, consideramos $Y = \frac{1}{X}$.
    Entonces $Y \sim GaI(a, p)$ gamma invertida.
    Su función de densidad es:
    $$f_Y(y) = \frac{a^p}{\Gamma(p)} e^{-\frac{a}{y}} y^{-(p+1)}, \quad y > 0$$
\end{definition}

% Ejemplo

\begin{definition}
    Decimos que $(\mu, \tau) \sin NGa(m_0, \tau_0, a_0, p_0)$ normal gamma, con $m_0 \in \mathbb{R}, \tau_0, a_0, p_0 > 0$, si:
    $$\mu|\tau \sim N(m_0, pr = \tau\tau_0) \text{ y } \tau \sim Ga(a_0, p_0), \quad \mu \in \mathbb{R}, \tau > 0$$
    Su función de densidad es:
    $$f(\mu, \tau) = \frac{\sqrt{\tau_0}}{\sqrt{2\pi}} \frac{a_0^{p_0}}{\Gamma(p_0)} \tau^{p_0 - \frac{1}{2}} e^{-\tau \left(a_0 + \frac{\tau_0}{2}(\mu-m_0)^2\right)}$$
\end{definition}

\begin{definition}
    Si $T \sim t_n$ y $X = \mu + \frac{1}{\sqrt{p}}T$, entonces $X \sim t(\mu, p, n)$, donde $\mu$ es la media y $p$ es el parámetro de escala.
    Su función de densidad es:
    $$f_X(x) = \frac{\Gamma\left(\frac{n+1}{2}\right)\sqrt{p}}{\Gamma\left(\frac{1}{2}\right)\Gamma\left(\frac{n}{2}\right)\sqrt{n}} \left(1 + \frac{p}{n}(x-\mu)^2\right)^{-\frac{n+1}{2}}$$
    Verifica que:
    $$E(X) = \mu, \quad V(X) = \frac{1}{p} \frac{n}{n-2}$$
\end{definition}

\begin{remark}
    La distribución $t_1$ se llama distribución de Cauchy.
    Además:
    $$t_n \xrightarrow[n \to \infty]{} N(0, 1)$$
\end{remark}

\begin{theorem}
    Si $(\mu, \tau) \sim NGa(m_0, \tau_0, a_0, p_0)$, entonces:
    $$\mu \sim t\left(m_0, \frac{p_0\tau_0}{a_0}, 2p_0\right)$$
\end{theorem}

\begin{corollary}[Génesis bayesiana de la $t$ de Student]
    $$\begin{cases}
            \mu|\tau \sim N(0, pr = \tau) \\
            \tau \sim Ga\left(\frac{n}{2}, \frac{n}{2}\right)
        \end{cases} \Rightarrow \mu \sim t(0, 1, n) \equiv t_n$$
\end{corollary}

% Ejemplo

\section{Distribuciones a priori no informativas}
\begin{definition}
    La información de Fisher para $\theta$ se define como:
    $$J(\theta) = -E\left(\frac{\partial^2 \log(f(x|\theta))}{\partial\theta^2}\right)$$
\end{definition}

\begin{proposition}[Regla de Jeffreys]
    $$f_\theta(\theta) \propto \sqrt{J(\theta)}$$
\end{proposition}

\begin{remark}
    La reglas de Jeffreys no da densidades en general.
    A aquellas que no son densidades se les llama densidades impropias.
\end{remark}
\chapter{Introducción a la inferencia no paramétrica}
\section{Introducción}
Hasta ahora los tests de hipótesis han sido utilizados para contrastar la veracidad de hipótesis acerca de los parámetros de la distribución de una población.
Sin embargo, en muchas ocasiones, es necesario emitir un juicio estadístico sobre la distribución poblacional en su conjunto.
Los problemas de este tipo que se plantean de manera habitual son los siguientes:
\begin{itemize}
    \item \textbf{Contrastes de bondad de ajuste.}
          Decidir, a la vista de una muestra aleatoria de una población, si puede admitirse que la distribución poblacional coincide con una cierta distribución dada o pertenece a un determinado tipo de distribuciones.
    \item \textbf{Contrastes de homogeneidad.}
          Analizar si varias muestras aleatorias provienen de poblaciones con la misma distribución teórica, de forma que puedan ser utilizadas conjuntamente para inferencias posteriores acerca de esta; o, por el contrario, son muestras de poblaciones con distinta distribución, que no pueden agruparse como información homogénea acerca de una única distribución.
    \item \textbf{Contrastes de independencia.}
          Estudiar, en el caso en que se observen dos o más características de los elementos de la población si las características observadas pueden considerarse independientes, y se puede proceder a su análisis por separado, o, por el contrario, existe relación estadística entre ellas.
\end{itemize}

\section{Contrastes de bondad de ajuste}
\subsection*{Primer caso}
Consideremos una muestra aleatoria $(X_1, \dots, X_n)$ de una variable aleatoria $X$ con distribución desconocida.
Para decidir si es razonable admitir que la distribucion de $X$ viene dada por un determinado modelo de probabilidad $P$, resolvemos el siguiente contraste de hipótesis:
$$\begin{cases}
        H_0: \text{ el modelo de probabilidad de } X \text{ es } P \\
        H_1: \text{ el modelo de probabilidad de } X \text{ no es } P
    \end{cases}$$

\begin{note}
    El modelo $P$ debe estar completamente especificado.
\end{note}

Para contrastar $H_0$ frente a $H_1$ hacemos una partición arbitraria del espacio muestral de la población en $k$ clases $A_1, \dots, A_k$.
Después, para cada $A_i$ consideramos las siguientes frecuencias absolutas:
\begin{itemize}
    \item $O_i$: frecuencia observada en $A_i$, número de elementos de la muestra que están en la clase $A_i$.
    \item $e_i$: frecuencia esperada de la clase $A_i$ si la hipótesis $H_0$ es cierta, $nP(A_i)$.
\end{itemize}

El estadístico que utilizaremos es:
$$\sum_{i=1}^k \frac{(O_i-e_i)^2}{e_i}$$
que tiene aproximadamente cuando $n$ es grande una distribución $\chi^2_{k-1}$ si $H_0$ es cierta.

Para que la aproximación sea razonablemente buena, además de tener una muestra suficientemente grande, es necesario que el valor esperado de cada clase sea suficientemente grande.
Si la muestra procede de $P$, es de esperar que haya valores parecidos para $O_i$ y $e_i$ y, por tanto, este estadístico debería tomar valores próximos a cero.

Rechazaremos $H_0$ a nivel de significación $\alpha$ si:
$$\sum_{i=1}^k \frac{(O_i-e_i)^2}{e_i} > \chi^2_{k-1, 1-\alpha}$$
En caso contrario, aceptaremos $H_0$ a nivel de significación $\alpha$.

\begin{example}
    Después de lanzar un dado 300 veces, se obtienen las siguientes frecuencias:
    \begin{center}
        \begin{tabular}{ | c | c c c c c c | }
            \hline
            Resultado  & 1  & 2  & 3  & 4  & 5  & 6  \\
            \hline
            Frecuencia & 43 & 49 & 56 & 45 & 66 & 41 \\
            \hline
        \end{tabular}
    \end{center}

    Veamos si se puede afirmar que el dado es regular a nivel de significación $\alpha = 0.05$.

    Disponemos de una muestra aleatoria de $n = 300$ lanzamientos de un dado.
    Para decidir si el dado es regular o no, llevamos a cabo un contraste de bondad de ajuste a nivel de significación $\alpha$:
    $$\begin{cases}
            H_0: \text{ El dado es regular: } P(1) = \dots = P(6) = \frac{1}{6} \\
            H_1: \text{ El dado es irregular}
        \end{cases}$$

    La tabla de frecuencias observadas y esperadas es:
    \begin{center}
        \begin{tabular}{| c | c c c c c c |}
            \hline
            $A_i$ & 1  & 2  & 3  & 4  & 5  & 6  \\
            \hline
            $O_i$ & 43 & 49 & 56 & 45 & 66 & 41 \\
            $e_i$ & 50 & 50 & 50 & 50 & 50 & 50 \\
            \hline
        \end{tabular}
    \end{center}
    donde las frecuencias esperadas bajo $H_0$ han sido calculadas como:
    $$e_i = nP(A_i) = 300 \frac{1}{6} = 50$$

    Usamos el estadístico de contraste:
    $$\sum_{i=1}^k \frac{(O_i-e_i)^2}{e_i}$$
    que tiene aproximadamente una distribución $\chi^2_{k-1}$ si $H_0$ es cierta.

    Rechazaremos $H_0$ a nivel de significación $\alpha$ si:
    $$\sum_{i=1}^k \frac{(O_i-e_i)^2}{e_i} > \chi^2_{k-1, 1-\alpha}$$
    En nuestro caso:
    $$\sum_{i=1}^6 \frac{(O_i-e_i)^2}{e_i} = 8.96, \quad \chi^2_{5, 0.95} = 11.07$$

    Por tanto, aceptamos $H_0$ a nivel de significación $\alpha = 0.05$, es decir, aceptamos que el dado es regular a nivel de significación $\alpha = 0.05$.
\end{example}

\begin{example}
    Nos dicen que un programa de ordenador genera observaciones de una distribucion $N(0, 1)$.
    Como no estamos seguros de ello, obtenemos una muestra aleatoria de 450 observaciones mediante dicho programa, obteniendo los siguientes resultados:
    \begin{itemize}
        \item 30 observaciones menores que -2.
        \item 80 observaciones entre -2 y -1.
        \item 140 observaciones entre -1 y 0.
        \item 110 observaciones entre 0 y 1.
        \item 60 observaciones entre 1 y 2.
        \item 30 observaciones mayores que 2.
    \end{itemize}
    Veamos si se puede aceptar que el programa funciona correctamente a nivel de significación $\alpha = 0.01$.

    Disponemos de una muestra aleatoria de $n = 450$ observaciones generadas por el programa.
    Los posibles resultados de estas observaciones se agrupan en seis clases:
    \begin{align*}
        A_1 & = (-\infty, -2) & A_2 & = (-2, -1)    \\
        A_3 & = (-1, 0)       & A_4 & = (0, 1)      \\
        A_5 & = (1, 2)        & A_6 & = (2, \infty)
    \end{align*}
    Para decidir si el programa funciona correctamente o no, llevamos a cabo un contraste de bondad de ajuste a nivel de significación $\alpha$:
    $$\begin{cases}
            H_0: \text{ El programa funciona correctamente: provienen de } N(0, 1) \\
            H_1: \text{ El programa no funciona correctamente}
        \end{cases}$$

    La tabla de frecuencias observadas y esperadas es:
    \begin{center}
        \begin{tabular}{| c | c c c c c c |}
            \hline
            $A_i$    & $(-\infty, -2)$ & $(-2, -1)$ & $(-1, 0)$ & $(0, 1)$ & $(1, 2)$ & $(2, \infty)$ \\
            \hline
            $O_i$    & 30              & 80         & 140       & 110      & 60       & 30            \\
            $P(A_i)$ & 0.0228          & 0.1359     & 0.3413    & 0.3413   & 0.1359   & 0.0228        \\
            $e_i$    & 10.26           & 61.155     & 153.585   & 154.585  & 61.155   & 10.26         \\
            \hline
        \end{tabular}
    \end{center}
    donde las frecuencias esperadas bajo $H_0$ han sido calculadas de la forma:
    $$e_i = nP(A_i) = 450P(A_i)$$
    y los valores $P(A_i)$ se han calculado a partir de la tabla de la función de distribucion de la normal estándar.

    Usamos el estadístico de contraste:
    $$\sum_{i=1}^k \frac{(O_i-e_i)^2}{e_i}$$
    que tiene aproximadamente una distribución $\chi^2_{k-1}$ si $H_0$ es cierta.

    Rechazaremos $H_0$ a nivel de significación $\alpha$ si:
    $$\sum_{i=1}^k \frac{(O_i-e_i)^2}{e_i} > \chi^2_{k-1, 1-\alpha}$$
    En nuestro caso:
    $$\sum_{i=1}^6 \frac{(O_i-e_i)^2}{e_i} = 95.358, \quad \chi^2_{5, 0.99} = 15.086$$

    Por tanto, rechazamos $H_0$ a nivel de significación $\alpha = 0.01$, es decir, podemos afirmar que el programa no funciona correctamente a nivel de significación $\alpha = 0.01$.
\end{example}

\subsection*{Segundo caso}
El contraste de bondad de ajuste se puede plantear también en una situación más general.
Consideremos una muestra aleatoria $(X_1, \dots, X_n)$ de una variable aleatoria $X$ con distribución desconocida.
Para decidir si es razonable admitir que la distribución de $X$ viene dada por algún modelo de probabilidad de una cierta familia $P_\theta$, con $\theta = (\theta_1, \dots, \theta_r)$, resolveremos el siguiente contraste de hipótesis:
$$\begin{cases}
        H_0: \text{ el modelo de probabilidad } X \text{ es de la familia } \{P_\theta: \theta \in \Theta\} \\
        H_0: \text{ el modelo de probabilidad } X \text{ no es de la familia } \{P_\theta: \theta \in \Theta\}
    \end{cases}$$

Para contrastar $H_0$ frente a $H_1$ hacemos nuevamente una partición arbitraria del espacio muestral de la población en $k$ clases $A_1, \dots, A_k$.
Después, para cada $A_i$ consideramos las siguientes frecuencias absolutas:
\begin{itemize}
    \item $O_i$: frecuencia observada en $A_i$, número de elementos de la muestra que están en la clase $A_i$.
    \item $e_i$: frecuencia esperada de la clase $A_i$ si la hipótesis $H_0$ es cierta, $nP_\theta(A_i) \approx nP_{\hat{\theta}}(A_i)$, donde $\hat{\theta}$ es el estimador de máxima verossimilitud de $\theta$.
\end{itemize}

El estadístico que utilizaremos es:
$$\sum_{i=1}^k \frac{(O_i-e_i)^2}{e_i}$$
que tiene aproximadamente cuando $n$ es grande una distribución $\chi^2_{k-r-1}$ si $H_0$ es cierta.

Rechazaremos $H_0$ a nivel de significación $\alpha$ si:
$$\sum_{i=1}^k \frac{(O_i-e_i)^2}{e_i} > \chi^2_{k-r-1, 1-\alpha}$$
En caso contrario, aceptaremos $H_0$ a nivel de significación $\alpha$.

% Ejemplo

\section{Contrastes de homogeneidad}
Supongamos que disponemos de $p$ muestras aleatorias independientes tomadas de $p$ poblaciones sobre una característica común $X$ a todas ellas:
\begin{align*}
    (X_{11}, \dots, X_{1n_1}) \\
    \dots                     \\
    (X_{p1}, \dots, X_{pn_p})
\end{align*}
con $n_1 + \dots + n_p = n$.

Queremos ver si, a la vista de las muestras obtenidas, es razonable admitir que todas las poblaciones tienen una distribucion común, es decir, si son poblaciones homogéneas.
Por tanto, tenemos el contraste:
$$\begin{cases}
        H_0: \text{ Las } p \text{ poblaciones tienen una distribucion común} \\
        H_1: \text{ Las } p \text{ poblaciones no tienen una distribucion común}
    \end{cases}$$

Para contrastar $H_0$ frente a $H_1$ hacemos nuevamente una partición arbitraria del espacio muestral común a las $p$ poblaciones en $k$ clases $A_1, \dots, A_k$.

Después, definimos para la clase $A_i$ y para la muestra de la población $j$-ésima:
\begin{itemize}
    \item $O_{ij}$: frecuencia observada en la clase $A_i$ con la muestra $j$-ésima.
    \item $e_{ij}$: frecuencia esperada en la clase $A_i$ con la muestra $j$-ésima si todas las poblaciones tienen la distribucion común $P$, $n_jP(A_i)$.
\end{itemize}

El estadístico utilizado es:
$$\sum_{j=1}^p \sum_{i=1}^k \frac{(O_{ij}-e_{ij})^2}{e_{ij}}$$
que tiene aproximadamente cuando $n$ es grande una distribucion $\chi^2_{(k-1)(p-1)}$ si $H_0$ es cierta.

Rechazaremos $H_0$ a nivel de significación $\alpha$ si:
$$\sum_{j=1}^p \sum_{i=1}^k \frac{(O_{ij}-e_{ij})^2}{e_{ij}} > \chi^2_{(k-1)(p-1), 1-\alpha}$$
En caso contrario, aceptaremos $H_0$ a nivel de significación $\alpha$.

% Ejemplo y problemas

\section{Contrastes de independencia}
Supongamos que queremos estudiar si dos características $X$ e $Y$ de una población están relacionadas o no.
Para hacer este estudio, obtenemos una muestra aleatoria de $n$ pares de valores de estas características:
$$((X_1, Y_1), \dots, (X_n, Y_n))$$
Queremos ver si, a la vista de la muestra, tiene sentido admitir que $X$ e $Y$ son independientes.
Por tanto, tenemos el contraste:
$$\begin{cases}
        H_0: X \text{ e } Y \text{ son independientes} \\
        H_1: X \text{ e } Y \text{ no son independientes}
    \end{cases}$$

Tomamos una partición arbitraria del espacio muestral en $kp$ clases:
$$A_1 \times B_1, \dots, A_k \times B_p$$
Estas $kp$ clases corresponden a tomar las clases $A_1, \dots, A_k$ para la característica $X$ y las clases $B_1, \dots, B_p$ para la característica $Y$.

Llamamos:
\begin{itemize}
    \item $O_{ij}$: frecuencia observada en la clase $A_i \times B_j$.
    \item $e_{ij}$: frecuencia esperada en la clase $A_i \times B_j$ si la hipótesis nula es cierta, $nP(A_i)P(B_j)$.
\end{itemize}

El estadístico utilizado es:
$$\sum_{j=1}^p \sum_{i=1}^k \frac{(O_{ij}-e_{ij})^2}{e_{ij}}$$
que tiene aproximadamente cuando $n$ es grande una distribucion $\chi^2_{(k-1)(p-1)}$ si $H_0$ es cierta.

\begin{remark}
    Este estadístico coincide con el que utilizábamos en el contraste de homogeneidad.
\end{remark}

Rechazaremos $H_0$ a nivel de significación $\alpha$ si:
$$\sum_{j=1}^p \sum_{i=1}^k \frac{(O_{ij}-e_{ij})^2}{e_{ij}} > \chi^2_{(k-1)(p-1), 1-\alpha}$$
En caso contrario, aceptaremos $H_0$ a nivel de significación $\alpha$.

% Ejemplos
\chapter{El teorema fundamental de la teoría de Galois}
\section{Grupo de Galois}

\begin{example}
    \begin{align*}
         & Gal(\mathbb{Q}(\sqrt[3]{2})/\mathbb{Q}) = \{Id\}                                                                \\
         & Gal(\mathbb{Q}(\sqrt{2})/\mathbb{Q}) = \{ Id, \sigma \}, \text{ donde } \sigma(\sqrt{2}) = -\sqrt{2}            \\
         & Gal(\mathbb{R}/\mathbb{Q}) = \{Id\}                                                                             \\
         & Gal(\mathbb{Q}(x)/\mathbb{Q}) = \{ x \mapsto \frac{ax + b}{cx + d} : a, b, c, d \in \mathbb{Q}, ad-bc \neq 0 \}
    \end{align*}
    Observamos que $\mathbb{R}/\mathbb{Q}$ y $\mathbb{Q}(x)/\mathbb{Q}$ son extensiones infinitas cuyo grupos de Galois son respectivamente finito e infinito.
\end{example}

\begin{lemma}
    Sea $K/F$ algebraica. Entonces todo $F$-homomorfismo $\sigma : K \to K$ es $F$-automorfismo.
\end{lemma}

\begin{theorem}
    Si $K/F$ es una extensión finita, entonces $Gal(K/F)$ es un grupo finito.
\end{theorem}

\section{Subgrupos y cuerpos intermedios}

\begin{theorem}
    Sea $K/F$ una extensión de cuerpos, $E$ un cuerpo intermedio y $H$ un subgrupo de $G = Gal(K/F)$. Entonces:
    $$K^H = \{ \alpha \in K : \sigma(\alpha) = \alpha, \text{ para todo } \sigma \in H \}$$
    es un cuerpo intermedio de $K/F$ y
    $$Gal(K/E) = \{ \sigma \in G : \sigma(\alpha) = \alpha, \text{ para todo } \alpha \in E \}$$
    es un subgrupo de $G$.
\end{theorem}

\begin{example}
    Sea $G$ el grupo de Galois de $\mathbb{Q}(\sqrt{2}, \sqrt{3})$ sobre $\mathbb{Q}$.
    Podemos comprobar que $G$ es un grupo de orden 4.
    Como sus elementos están completamente determinados por su acción en $\sqrt{2}$ y $\sqrt{3}$, podemos describirlos así:
    \begin{align*}
        \sigma_1           & = Id                                           \\
        \sigma_2(\sqrt{2}) & = \sqrt{2}  & \sigma_2(\sqrt{3}) & = -\sqrt{3} \\
        \sigma_3(\sqrt{2}) & = -\sqrt{2} & \sigma_3(\sqrt{3}) & = \sqrt{3}  \\
        \sigma_4(\sqrt{2}) & = -\sqrt{2} & \sigma_4(\sqrt{3}) & = -\sqrt{3}
    \end{align*}
    Observamos que $\sigma_2, \sigma_3, \sigma_4$ son elementos de orden 2 en $G$.
    Luego $G$ es isomorfo al grupo $C_2 \times C_2$.\\
    El subgrupo de $G$ asociado al cuerpo intermedio $\mathbb{Q}(\sqrt{3})$ es:
    $$Gal(\mathbb{Q}(\sqrt{2}, \sqrt{3}) / \mathbb{Q}(\sqrt{3}))$$
    el grupo de $\mathbb{Q}(\sqrt{3})$-automorfismos de $\mathbb{Q}(\sqrt{2}, \sqrt{3})$. Este es $< \sigma_3 >$.\\
    Por otro lado, sea $H$ el subgrupo generado por $\sigma_2 : H = < \sigma_2 >$.
    El correspondiente cuerpo intermedio es:
    \begin{align*}
        \mathbb{Q}(\sqrt{2}, \sqrt{3})^H & = \{ \alpha \in \mathbb{Q}(\sqrt{2}, \sqrt{3}) : \sigma(\alpha) = \alpha, \text{ para todo } \sigma \in H \} \\
                                         & = \{ \alpha \in \mathbb{Q}(\sqrt{2}, \sqrt{3}) : \sigma_2(\alpha) = \alpha \}
    \end{align*}
    Obervamos que $\mathbb{Q}(\sqrt{2}) \subseteq \mathbb{Q}(\sqrt{2}, \sqrt{3})^H$.
    Se puede demostrar que se da la igualdad considerando la acción de $\sigma_2$ en cada elemento de $\mathbb{Q}(\sqrt{2}, \sqrt{3})$, recordando que son de la forma $a + b\sqrt{2} + c\sqrt{3} + d\sqrt{2}\sqrt{3}$, con $a, b, c, d \in \mathbb{Q}$.
\end{example}

\begin{definition}
    Sea $K/F$ una extensión de cuerpos y $H$ un subgrupo de $Gal(K/F)$.
    El cuerpo intermedio $K^H$ se llama subcuerpo de $K$ fijo por $H$.
\end{definition}

\section{Extensiones de Galois}

\begin{definition}
    Una extensión de Galois es una extensión de cuerpos $K/F$ tal que $K^G = F$, donde $G = Gal(K/F)$.
\end{definition}

\begin{example}
    El grupo de Galois de $\mathbb{Q}(\sqrt[3]{2})$ sobre $\mathbb{Q}$ es trivial.
    Luego todos los elementos de $\mathbb{Q}(\sqrt[3]{2})$ son fijos por todos los elementos del grupo de Galois, es decir:
    $$\mathbb{Q}(\sqrt[3]{2})^G = \mathbb{Q}(\sqrt[3]{2})$$
    y por tanto $\mathbb{Q}(\sqrt[3]{2})$ no es una extensión de Galois de $\mathbb{Q}$.
\end{example}

\begin{example}
    Veamos si $\mathbb{Q}(\sqrt[4]{2})/\mathbb{Q}$ es una extensión de Galois.\\
    Su grupo de Galois $G$ tiene dos elementos: la identidad y el elemento $\sigma$ determinado por:
    $$\sigma(\sqrt[4]{2}) = -\sqrt[4]{2}$$
    Observamos que $\sqrt{2} \in \mathbb{Q}(\sqrt[4]{2})$ es fijo por $Id$ y $\sigma$, así que $\mathbb{Q}(\sqrt{2}) \subseteq \mathbb{Q}(\sqrt[4]{2})^G$ y por tanto la extensión no es de Galois.
\end{example}

\begin{theorem}
    Sea $K/F$ una extensión algebraica. Entonces son equivalentes:
    \begin{enumerate}
        \item $K/F$ es una extensión de Galois.
        \item $K/F$ es normal y separable.
    \end{enumerate}
\end{theorem}

\section{El teorema fundamental}

\begin{proposition}
    Sea $K/F$ algebraico. Si $K/F$ es de Galois y $E$ es un cuerpo intermedio, entonces $K/E$ es de Galois.
\end{proposition}

\begin{example}
    La extensión $\mathbb{Q}(i, \sqrt{3}, \sqrt[3]{2})/\mathbb{Q}$ es de Galois pero $\mathbb{Q}(\sqrt[3]{2})/\mathbb{Q}$ no es de Galois.
\end{example}

\begin{theorem}
    Sea $K/F$ una extensión finita. Entonces:
    \begin{enumerate}
        \item $|Gal(K/F)|$ divide a $[K : F]$.
        \item $K/F$ es de Galois si y solo si $|Gal(K/F)| = [K : F]$.
    \end{enumerate}
\end{theorem}

\begin{theorem}
    Sea $K/F$ una extensión finita y $E$ un cuerpo intermedio. Si $K/F$ es de Galois, son equivalentes:
    \begin{enumerate}
        \item $E = \sigma E$ para todo $\sigma \in Gal(K/F)$.
        \item $Gal(K/E)$ es un subgrupo normal de $Gal(K/F)$.
        \item $E/F$ es de Galois.
        \item $E/F$ es una extensión normal.
    \end{enumerate}
\end{theorem}

\begin{theorem}
    Sean $F \subset E \subset K$ extensiones finitas con $E/F$ y $K/F$ de Galois.
    Entonces $Gal(K/E)$ es un subgrupo normal de $Gal(K/F)$ y existe un isomorfismo de grupos:
    $$Gal(K/F)/Gal(K/E) \equiv Gal(E/F)$$
\end{theorem}

\begin{theorem}[Teorema fundamental]
    Sea $K/F$ una extensión finita de Galois.
    Existe una correspondencia uno a uno entre el conjunto de cuerpos intermedios de $K/F$ y el conjunto de subgrupos del grupo de Galois $G = Gal(K/F)$.
    Esta correspondencia viene dada por $E \mapsto Gal(K/E)$ y satisface las siguientes condiciones:
    \begin{enumerate}
        \item Si $F \subseteq E \subseteq K$, entonces $K/E$ es una extensión de Galois y su grupo de Galois $Gal(K/E)$ es un subgrupo de $G = Gal(K/F)$. Además, $$[K : E] = |Gal(K/E)| \text{ y } |Gal(K/F) : Gal(K/E)| = [E : F]$$
        \item $E/F$ es de Galois si y solo si $Gal(K/E)$ es un subgrupo normal de $G = Gal(K/F)$. En este caso, $Gal(E/F)$ es isomorfo a $G/Gal(K/E)$.
    \end{enumerate}
\end{theorem}

\end{document}