\chapter{Funciones armónicas}

\section{Funciones armónicas y funciones holomorfas}
\begin{definition}
    Sea $D$ un abierto en $\mathbb{R}^n$ y sea $f: D \to \mathbb{R}$.
    Decimos que $f$ es armónica en $D$ si $f \in \mathcal{C}^2(D)$ y
    $$\Delta f = \frac{\partial^2f}{\partial x_1^2} + \frac{\partial^2f}{\partial x_2^2} + \dots + \frac{\partial^2f}{\partial x_n^2}$$
\end{definition}

\begin{definition}
    Dado $D$ abierto en $\mathbb{C}$ y $u: D \to \mathbb{R}$, con $u \in \mathcal{C}^2(D)$.
    El laplaciano de $u$ se define como:
    $$\Delta u = \frac{\partial^2u}{\partial x^2} + \frac{\partial^2u}{\partial y^2}$$
\end{definition}

\begin{definition}
    Sea $D$ abierto en $\mathbb{D}$ y sea $u: D \to \mathbb{R}$, con $u \in \mathcal{C}^2(D)$.
    Diremos que $u$ es armónica en $D$ si $\Delta u = 0$.
\end{definition}

\begin{example}
    \hfill
    \begin{itemize}
        \item Las funciones constantes.
              $u(z) = a$, $a \in \mathbb{R}$, es armónica en $\mathbb{C}$.
        \item La función parte real es armónica en $\mathbb{C}$.
              \begin{align*}
                  \Re: \mathbb{C} & \to \mathbb{R}                         \\
                  z               & \mapsto \Re(z) = \frac{z + \bar{z}}{2} \\
                  (x, y)          & \mapsto x
              \end{align*}
        \item La función parte imaginaria es armónica en $\mathbb{C}$.
              \begin{align*}
                  \Im: \mathbb{C} & \to \mathbb{R}                          \\
                  z               & \mapsto \Im(z) = \frac{z - \bar{z}}{2i} \\
                  (x, y)          & \mapsto y
              \end{align*}
        \item Si $D$ es un abierto en $\mathbb{C}$ y $u$ y $v$ son armónicas en $D$, entonces $u+v$ es armónica en $D$.
        \item Si $D$ es un abierto en $\mathbb{C}$, $u$ es armónica en $D$ y $c \in \mathbb{R}$, entonces $cu$ es armónica en $D$.
    \end{itemize}
\end{example}

\begin{theorem}
    Sea $D$ un abierto en $\mathbb{C}$ y sea $f$ una función holomorfa en $D$.
    Entonces $\Re(f)$ e $\Im(f)$ son armónicas en $D$.
\end{theorem}

\begin{proof}
    Sean $u = \Re(f)$ y $v = \Im(f)$, $u, v: D \to \mathbb{R}$, con $u, v \in \mathcal{C}^\infty(D) \subset \mathcal{C}^2(D)$.
    Se verifican las condiciones de Cauchy-Riemann:
    $$\begin{cases}
            u_x = v_y \\
            u_y = -v_x
        \end{cases}$$
    Por tanto:
    \begin{align*}
        \Delta u & = u_{xx} + u_{yy} = v_{yx} - v_{xy} = 0  \\
        \Delta v & = v_{xx} + v_{yy} = -u_{yx} + u_{xy} = 0
    \end{align*}
\end{proof}

El recíproco no es cierto.
Es decir, dado $D$ abierto en $\mathbb{C}$, $u$ y $v$ armónicas en $D$ y $f = u + iv: D \to \mathbb{C}$, no se cumple en general que $f$ sea holomorfa en $D$.

\begin{example}[Contraejemplo]
    Sea $D = \mathbb{C}$ y sean $u(z) = \Re(z)$ y $v(z) = \Im(z)$.
    $u$ y $v$ son armónicas en $\mathbb{C}$.
    Sin embargo, $f(z) = u(z) + iv(z) = \bar{z}$ no es holomorfa.
\end{example}

\begin{definition}
    Sea $D$ un abierto en $\mathbb{C}$ y sea $u$ armónica en $D$.
    Diremos que $v: D \to \mathbb{R}$ es una conjugada armónica de $u$ en $D$ si la función $f = u + iv$ es holomorfa en $D$.
\end{definition}

\begin{properties}
    \hfill
    \begin{itemize}
        \item Si existe una conjugada armónica de $u$ en $D$, entonces es una función armónica en $D$.
        \item Si $u$ es una conjugada armónica de $u$ en $D$, entonces $u + c$ es conjugada armónica de $u$ en $D$ para todo $c \in \mathbb{R}$.
        \item Si $D$ es un dominio en $\mathbb{C}$, $u$ es armónica en $D$ y $v_1, v_2$ son conjugadas armónicas de $u$ en $\mathbb{C}$, entonces existe $c \in \mathbb{R}$ tal que $v_1-v_2 = c$.
              \begin{proof}
                  $f_1 = u + iv_1$ y $f_2 = u + iv_2$ son holomorfas en $D$, así que $f_2 - f_1 = i(v_2-v_1)$ es holomorfa en $D$.
                  Como $\Re(f_2-f_1) = 0$ en $D$ con $D$ dominio, entonces $f_2 - f_1$ es constante.
                  Es decir, existe $c \in \mathbb{R}$ con $f_2-f_1 = ic$.
                  Por tanto:
                  $$i(v_2-v_1) = ic \Rightarrow v_2 - v_1 = c$$
              \end{proof}
    \end{itemize}
\end{properties}

No tiene por qué existir la conjugada armónica.

\begin{example}[Contraejemplo]
    Sea $D = \mathbb{C} \setminus \{0\}$ abierto en $\mathbb{C}$ y sea $u: D \to \mathbb{R}$, $u(z) = \Log|z|$.
    Si escribimos $z = x +iy$,
    $$u(z) = \Log(x^2 + y^2)^{1/2} = \frac{1}{2}\Log(x^2+y^2)$$
    Calculamos sus derivadas parciales:
    \begin{align*}
        u_x(z)    & = \frac{1}{2} \frac{2x}{x^2+y^2} = \frac{x}{x^2+y^2}             & u_y(z) & = \frac{y}{x^2+y^2}           \\
        u_{xx}(z) & = \frac{x^2+y^2-2x^2}{(x^2+y^2)^2} = \frac{y^2-x^2}{(x^2+y^2)^2} & u_{yy} & = \frac{x^2-y^2}{(x^2+y^2)^2}
    \end{align*}
    $\Delta u = 0$, así que $u$ es armónica en $D$.

    Supongamos que $v$ es una conjugada armónica de $u$ en $D$.
    Entonces $v: \mathbb{C} \setminus \{0\} \to \mathbb{R}$, con $f = u + iv$ holomorfa en $\mathbb{C} \setminus \{0\}$.
    Consideramos:
    \begin{align*}
        g = \Log: \mathbb{C} \setminus (-\infty, 0] & \to \mathbb{C}                       \\
        z                                           & \mapsto \Log(z) = \log|z| + i\Arg(z)
    \end{align*}
    $f$ y $g$ son holomorfas en $\mathbb{C} \setminus (-\infty, 0]$, luego $f-g$ es holomorfa en $\mathbb{C} \setminus (-\infty, 0]$.
    Como $\Re(f-g) = 0$, entonces $f-g = ic$, con $c \in \mathbb{R}$.
    Así que $g = f - ic$.
    Sin embargo, $g$ no se extiende de forma continua a $\mathbb{C} \setminus \{0\}$.
    Esto es una contradicción.
\end{example}

Hemos probado que la función $u(z) = \Log|z|$ es armónica en $\mathbb{C} \setminus \{0\}$.

\begin{lemma}
    Sea $D$ un abierto en $\mathbb{C}$ y sea $f$ una función holomorfa y nunca nula en $D$.
    Entonces la función $u = \Log|f|$ es armónica en $D$.
\end{lemma}

\begin{proof}
    Sean $z_0 \in D$ y $R > 0$ tales que $D(z_0, R) \subset D$.
    Como $D(z_0, R)$ es un dominio simplemente conexo en $\mathbb{C}$, existe $g$ una rama de $\log(f)$ en $D(z_0, R)$.
    $g$ es holomorfa en $D(z_0, R)$, con
    $$\Re(g(z)) = \Log|f(z)|, \quad z \in D(z_0, R)$$
    $Re(g) = \Log|f|$ es armónica en $D(z_0, R)$.
\end{proof}

\begin{proposition}
    Sea $D$ un abierto en $\mathbb{C}$ y sea $u$ armónica en $D$.
    Entonces la función $u_x - iu_y$ es holomorfa en $D$.
\end{proposition}

% Demostración

\begin{corollary}
    Sea $D$ abierto en $\mathbb{C}$ y sea $u$ armónica en $D$.
    Entonces $u \in \mathcal{C}^\infty(D)$.
\end{corollary}

% Demostración

\begin{proposition}
    Sea $D$ un dominio simplemente conexo en $\mathbb{C}$ y sea $u$ armónica en $D$.
    Entonces existe $F$ holomorfa en $D$ tal que $\Re(F) = u$ en $D$.
    Además, esta $F$ es única salvo adición de constantes imaginarias.
\end{proposition}

% Demostración

\begin{corollary}
    Sea $D$ un abierto en $\mathbb{C}$ y sea $u$ armónica en $D$.
    Si $z_0 \in D$ y $R > 0$ con $D(z_0, R) \subset D$, entonces existe $F$ holomorfa en $D(z_0, R)$ tal que $\Re(F) = u$ en $D(z_0, R)$.
\end{corollary}

Toda función armónica, localmente, es la parte real de una función holomorfa.

\begin{theorem}
    Sea $D$ un dominio en $\mathbb{C}$.
    Son equivalentes:
    \begin{enumerate}
        \item $D$ es simplemente conexo.
        \item Toda función armónica tiene conjugada armónica.
    \end{enumerate}
\end{theorem}

% Demostración

\begin{theorem}[Propiedad del valor medio]
    Sea $D$ un abierto en $\mathbb{C}$ y sea $u$ armónica en $D$.
    Sea $z_0 \in D$ y $R > 0$ con $D(z_0, R) \subset D$.
    Entonces:
    $$u(z_0) = \frac{1}{2\pi} \int_{-\pi}^\pi u(z_0 + re^{it})dt, \quad 0 \leq 0 < R$$
\end{theorem}

% Demostración

\begin{theorem}[Forma débil del principio del máximo para funciones armónicas]
    Sea $D$ un abierto en $\mathbb{C}$ y sea $u$ armónica en $D$.
    Si $u$ tiene un máximo local en un punto $z_0 \in D$, entonces existe $r > 0$ con $D(z_0, R) \subset D$ tal que $u$ es constante en $D(z_0, R)$.
\end{theorem}

% Demostración

Sea $D$ dominio en $\mathbb{C}$, sean $u$, $v$ armónicas en $D$ y sea $A \subset D$, $A$ con puntos de acumulación en $D$.
$u = v$ en $A$ no implica en general que $u = v$ en $D$.

\begin{example}[Contraejemplo]
    Sea $D = \mathbb{C}$ y sean $u(z) = \Re(z)$ y $v(z) = 0$ armónicas.
    $u = v$ en el eje imaginario, que tiene puntos de acumulación en $\mathbb{C}$, pero $u \neq v$ en $\mathbb{C}$.
\end{example}

\begin{remark}
    Sea $D$ un dominio en $\mathbb{C}$ y sea $u: D \to \mathbb{R}$.
    Si existen $u_x$ y $u_y$ en $D$ y $u_x = u_y = 0$ en $D$, entonces $u$ es constante en $D$.
\end{remark}

\begin{theorem}[Teorema de identidad para funciones armónicas]
    Sea $D$ un dominio en $\mathbb{C}$ y sea $u$ armónica en $D$.
    Si existe $G \subset D$, $G$ abierto y no vacío tal que $u = 0$ en $G$, entonces $u = 0$ en $D$.
\end{theorem}

\begin{proof}
    Sea $f = u_x - iu_y$ holomorfa en $D$.
    Como $u = 0$ en $G$, tenemos que $u_x = u_y = 0$ en $G$.
    Entonces $f = 0$ en $G$.
    Por el teorema de identidad, $f = 0$ en $D$.
    Entonces $u_x = u_y = 0$ en $D$, luego $u$ es constante en $D$.
    Como $u = 0$ en $G$, entonces $u = 0$ en $D$.
\end{proof}

\begin{corollary}
    Sea $D$ un dominio en $\mathbb{C}$ y sean $u$ y $v$ armónicas en $D$.
    Si existe $G \subset D$, $G$ abierto y no vacío tal que $u = v$ en $G$, entonces $u = v$ en $D$.
\end{corollary}

\begin{theorem}[Principio del máximo para funciones armónicas: primera versión]
    Sea $D$ un dominio en $\mathbb{C}$ y sea $u$ armónica en $D$.
    Si $u$ tiene un máximo local en un punto $z_0 \in D$, entonces $u$ es constante en $D$.
\end{theorem}

\begin{proof}
    Existe $R > 0$ con $D(z_0, R) \subset D$ tal que $u$ es constante en $D(z_0, R)$.
    Es decir, $u(z) = c$, $z \in D(z_0, R)$, $c \in \mathbb{R}$.
    Sea $v(z) = c$, $z \in D$.
    $u$ y $v$ son armónicas en $D$ y $u = v$ en $D(z_0, R)$.
    Entonces $u = v$ en $D$, es decir, $u(z) = c$ para todo $z \in D$.
\end{proof}

\begin{notation}
    Si $D$ es un dominio en $\mathbb{C}$, $\bar{D}$ denotará la clausura de $D$ como subconjunto de $\mathbb{C}^\ast$.
    \begin{itemize}
        \item Si $D$ es acotado, entonces $\partial_\infty D = \partial D$.
              $$\bar{D} = D \cup \partial_\infty D = D \cup \partial D$$
        \item Si $D$ no es acotado, entonces $\partial_\infty D = \partial D \cup \{\infty\}$.
              $$\bar{D} = D \cup \partial_\infty D = D \cup \partial D \cup \{\infty\}$$
    \end{itemize}
\end{notation}

\begin{theorem}[Principio del máximo para funciones armónicas: segunda versión]
    Sea $D$ un dominio en $\mathbb{C}$ y sea $u$ armónica en $D$.
    Si existe $M \in \mathbb{R}$ tal que
    $$\limsup_{z \to \chi, z \in D} u(z) \leq M, \quad \forall \chi \in \partial_\infty D$$
    entonces $u(z) \leq M$ para todo $z \in D$.
    Además, si $u(z_0) = M$ para algún $z_0 \in D$, entonces $u$ es constante en $D$.
\end{theorem}