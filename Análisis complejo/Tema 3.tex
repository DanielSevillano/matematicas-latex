\chapter{El teorema de Riemann de la aplicación conforme}

\section{Preliminares}
Recordemos algunos conceptos y resultados.

\begin{definition}
    Sea $D$ un dominio en $\mathbb{C}$ y sea $f$ una función holomorfa en $D$.
    \begin{itemize}
        \item $g$ es una rama de $\sqrt{f}$ en $D$ si $g: D \to \mathbb{C}$ es una función continua tal que $g(z)^2 = f(z)$ para todo $z \in D$.
        \item $g$ es una rama de $\log(f)$ en $D$ si $g: D \to \mathbb{C}$ es una función continua tal que $e^{g(z)} = f(z)$ para todo $z \in D$.
    \end{itemize}
\end{definition}

\begin{proposition}
    Sean $D$ un dominio en $\mathbb{C}$ y $f$ una función holomorfa y nunca nula en $D$.
    \begin{enumerate}
        \item Si $g$ es una rama de $\sqrt{f}$ en $D$, entonces $g$ es holomorfa en $D$ y $g'(z) = \frac{f'(z)}{2g(z)}$ para todo $z \in D$.
        \item Si $g$ es un rama de $\log(f)$ en $D$, entonces $g$ es holomorfa en $D$ y $g'(z) = \frac{f'(z)}{f(z)}$ para todo $z \in D$.
        \item Existe una rama de $\log(f)$ en $D$ si y solo si $\frac{f'}{f}$ tiene primitiva en $D$.
    \end{enumerate}
\end{proposition}

\begin{proposition}
    Sean $D$ un dominio en $\mathbb{C}$ y $f: D \to \mathbb{C}$ una función continua en $D$.
    Entonces $f$ tiene primitiva en $D$ si y solo si $\int_\gamma f(z)dz = 0$ para todo camino cerrado $\gamma$ en $D$.
\end{proposition}

\begin{definition}
    Si $D$ es un dominio en $\mathbb{C}$ y $\Gamma$ es un ciclo en $D$, se dice que $\Gamma$ es homólogo a cero módulo $D$, y se denota $\Gamma \sim 0 (mod D)$, si $n(\Gamma, a) = 0$ para todo $a \in \mathbb{C} \setminus D$.
\end{definition}

\begin{theorem}[Versión general del teorema de Cauchy]
    Sea $D$ un dominio en $\mathbb{C}$ y sea $\Gamma$ un ciclo en $D$.
    Las dos siguientes condiciones son equivalentes:
    \begin{enumerate}
        \item $\Gamma \sim 0 (mod D)$.
        \item $\int_\Gamma f(z)dz = 0$ para toda función $f$ holomorfa en $D$.
    \end{enumerate}
\end{theorem}

\section{Dominios simplemente conexos}
\begin{definition}
    Si $D$ es un dominio en $\mathbb{C}$, se dice que $D$ es simplemente conexo si $\mathbb{C}^\ast \setminus D$ es conexo.
\end{definition}

Hay una serie de caracterizaciones para los dominios simplemente conexos, que se pueden deducir de los resultados anteriores.

\begin{theorem}
    Sea $D$ un dominio en $\mathbb{C}$.
    Las siguientes condiciones son equivalentes:
    \begin{enumerate}
        \item $D$ es simplemente conexo.
        \item Todo ciclo en $D$ es homólogo a cero módulo $D$.
        \item Todo camino cerrado en $D$ es homólogo a cero módulo $D$.
        \item $\int_\Gamma f(z)dz = 0$ para toda $f$ holomorfa en $D$ y para todo ciclo $\Gamma$ en $D$.
        \item $\int_\gamma f(z)dz = 0$ para toda $f$ holomorfa en $D$ y para todo camino cerrado $\gamma$ en $D$.
        \item Toda función holomorfa en $D$ tiene primitiva.
        \item Para toda función $f$ holomorfa y nunca nula en $D$, existe una rama de $\log(f)$ en $D$.
        \item Para toda función $f$ holomorfa y nunca nula en $D$, existe una rama de $\sqrt{f}$ en $D$.
    \end{enumerate}
\end{theorem}

Recordamos que:
\begin{itemize}
    \item Dos dominios $D_1$ y $D_2$ en $\mathbb{C}^\ast$ son conformemente equivalentes si existe una aplicación conforme $f$ de $D_1$ sobre $D_2$.
    \item En el conjunto de los dominios en $\mathbb{C}^\ast$, el ser conformemente equivalentes es una relación de equivalencia.
    \item Si $D_1$ y $D_2$ son dos dominios en $\mathbb{C}^\ast$ que son conformemente equivalentes, entonces $D_1$ es simplemente conexo si y solo si $D_2$ es simplemente conexo.
    \item $\mathbb{C}^\ast$, $\mathbb{C}$ y el disco unidad $\mathbb{D} = \{z \in \mathbb{C} : |z| < 1\}$ son tres dominios simplemente conexos en $\mathbb{C}^\ast$, que no son conformemente equivalentes.
    \item El único dominio en $\mathbb{C}^\ast$ conformemente equivalente a $\mathbb{C}^\ast$ es $\mathbb{C}^\ast$.
\end{itemize}

Vamos a ver que, además de $\mathbb{C}\ast$, $\mathbb{C}$ y $\mathbb{D}$, no hay más dominios simplemente conexos en $\mathbb{C}^\ast$ módulo la relación de equivalencia.
Es decir, si $D$ es un dominio simplemente conexo en $\mathbb{C}^\ast$, entonces $D$ es conformemente equivalente a uno de los tres: $\mathbb{C}^\ast$, $\mathbb{C}$ o $\mathbb{D}$.
Por tanto se tendrá que, si $D$ es un dominio simplemente conexo en $\mathbb{C}$, con $D \neq \mathbb{C}$, entonces $D$ es conformemente equivalente a $\mathbb{D}$.

\begin{definition}
    Sea $D$ un dominio en $\mathbb{C}^\ast$.
    Llamamos automorfismos de $D$ a aquellas aplicaciones conformes de $D$ sobre $D$.
    El conjunto de todos los automorfismos de $D$ se denota $Aut(D)$, y tiene estructura de grupo con la composición.
\end{definition}

Tenemos que
\begin{align*}
    Aut(\mathbb{C}^\ast) & = \mathcal{M}                                                                                                                                                       \\
    Aut(\mathbb{C})      & = \{f_{\alpha, \beta} : f_{\alpha, \beta}(z) = \alpha z + \beta, \alpha, \beta \in \mathbb{C}, \alpha \neq 0\} = \{T \in \mathcal{M} : T(\mathbb{C}) = \mathbb{C}\} \\
    Aut(\mathbb{D})      & = \{\lambda T_a : \lambda \in \mathbb{C}, |\lambda| = 1, a \in \mathbb{D}\} = \{T \in \mathcal{M} : T(\mathbb{D}) = \mathbb{D}\}
\end{align*}

\begin{example}
    Veamos algunos ejemplos de dominios en $\mathbb{C}$ para los que podemos encontrar una aplicación conforme del dominio sobre $\mathbb{D}$.
    \begin{enumerate}
        \item Un disco abierto, $D(a, R)$, $a \in \mathbb{C}, R > 0$.
              \begin{align*}
                  \mathbb{D} & \to D(a, R)    \\
                  z          & \mapsto a + rz
              \end{align*}
        \item Un semiplano.
              \begin{align*}
                  \mathbb{D} & \to \mathbb{H} = \{z \in \mathbb{C} : \Re(z) > 0\} \\
                  z          & \mapsto P(z)
              \end{align*}
              donde $P(z) = \frac{1+z}{1-z}$, es una aplicación conforme.

              Componiendo con una rotación y una traslación, vemos que $\mathbb{D}$ es conformemente equivalente a cualquier semiplano.
              \begin{align*}
                  \mathbb{D} & \to \mathbb{C}              \\
                  z          & \mapsto a + e^{i\theta}P(z)
              \end{align*}
              con $a \in \mathbb{C}$ y $\theta \in \mathbb{R}$.
        \item El exterior de un disco, $\{z \in \mathbb{C} : |z-a| > R\} \cup \{\infty\}$, $a \in \mathbb{C}, R > 0$.
              \begin{align*}
                  D(a, R) & \to \{z \in \mathbb{C} : |z-a| > R\} \cup \{\infty\} \\
                  z       & \mapsto \frac{1}{z}
              \end{align*}
        \item El plano menos una semirrecta, $\mathbb{C} \setminus \{a + re^{i\theta}, r \geq 0\}$, $a \in \mathbb{C}, \theta \in \mathbb{R}$.
              \begin{align*}
                  \mathbb{H} & \to \mathbb{C} \setminus (-\infty, 0] \\
                  z          & \mapsto z^2
              \end{align*}
              es una aplicación conforme.
              Así que
              \begin{align*}
                  \mathbb{H} & \to \mathbb{C} \setminus (-\infty, 0]           \\
                  z          & \mapsto P(z)^2 = \left(\frac{1+z}{1-z}\right)^2
              \end{align*}
              es una aplicación conforme.

              Componiendo con una rotación y una traslación, vemos que $\mathbb{D}$ es conformemente equivalente al plano menos una semirrecta cualquiera.
        \item La función exponencial no es inyectiva.
              $$z = x + iy_0 \mapsto e^z = e^{x + iy_0} = e^x(\cos(y_0) + i\sin(y_0))$$
              Es inyectiva en cualquier banda horizontal abierta de amplitud menor o igual que $2\pi$.
              Por ejemplo,
              $$\exp: \left\{z \in \mathbb{C} : |\Im(z)| < \frac{\pi}{2}\right\} \to \mathbb{H}$$
              es una aplicación conforme.
              Como $\mathbb{H}$ es conformemente equivalente a $\mathbb{D}$, tenemos que esta banda es conformemente equivalente a $\mathbb{D}$.

              Componiendo con el producto por un número real, una rotación y una traslación, vemos que $\mathbb{D}$ es conformemente equivalente a cualquier banda.
        \item Sectores.
              $$\left\{z \in \mathbb{C} : |\Im(z)| < \frac{\alpha}{2}\right\} \xrightarrow{\exp} S$$
              donde $S$ es el sector de vértice 0 y amplitud $\alpha$, es una aplicación conforme.
        \item $\mathbb{D}^+ = \{z \in \mathbb{D} : \Im(z) > 0\}$.
              $$\mathbb{D}^+ \xrightarrow{P} \{z \in \mathbb{C} : \Re(z) > 0, \Im(z) > 0\}$$
              es una aplicación conforme. El dominio $\{z \in \mathbb{C} : \Re(z) > 0, \Im(z) > 0\}$ es un sector, así que es conformemente equivalente a $\mathbb{D}$.
    \end{enumerate}
\end{example}

\section{El teorema de Riemann de la aplicación conforme}
\begin{theorem}[Teorema de Riemann de la aplicación conforme]
    Sea $D$ un dominio simplemente conexo en $\mathbb{C}$ con $D \neq \mathbb{C}$ y sea $z_0 \in D$.
    Entonces existe una única aplicación conforme $f$ de $D$ sobre $\mathbb{D}$ tal que $f(z_0) = 0$ y $f'(z_0) > 0$.
\end{theorem}

\begin{remark}
    \hfill
    \begin{enumerate}
        \item Existen infinitas aplicaciones conformes de $D$ sobre $\mathbb{D}$.
              Basta cambiar el punto $z_0$ o componer con una rotación.
        \item Para la demostración, las condiciones
              \begin{enumerate}
                  \item $D$ simplemente conexo.
                  \item $D \neq \mathbb{C}$.
              \end{enumerate}
              solo las vamos a utilizar para deducir que:
              \begin{itemize}
                  \item $\mathbb{C} \setminus D$ tiene más de un punto.
                  \item Si $h$ es holomorfa y nunca nula en $D$, existe una rama de $\sqrt{h}$ en $D$.
              \end{itemize}
    \end{enumerate}
\end{remark}

\begin{theorem}
    Sea $D$ un dominio en $\mathbb{C}$ tal que:
    \begin{enumerate}
        \item $\mathbb{C} \setminus D$ tiene más de un punto.
        \item Para toda función $h$ holomorfa y nunca nula en $D$, existe una rama de $\sqrt{h}$ en $D$.
    \end{enumerate}
    Sea $z_0 \in D$.
    Entonces existe una única aplicación conforme $f$ de $D$ sobre $\mathbb{D}$ tal que $f(z_0) = 0$ y $f'(z_0) > 0$.
\end{theorem}


% Demostración