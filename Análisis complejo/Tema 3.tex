\chapter{El teorema de Riemann de la aplicación conforme}

\section{Preliminares}
Recordemos algunos conceptos y resultados.

\begin{definition}
    Sea $D$ un dominio en $\mathbb{C}$ y sea $f$ una función holomorfa en $D$.
    \begin{itemize}
        \item $g$ es una rama de $\sqrt{f}$ en $D$ si $g: D \to \mathbb{C}$ es una función continua tal que $g(z)^2 = f(z)$ para todo $z \in D$.
        \item $g$ es una rama de $\log(f)$ en $D$ si $g: D \to \mathbb{C}$ es una función continua tal que $e^{g(z)} = f(z)$ para todo $z \in D$.
    \end{itemize}
\end{definition}

\begin{proposition}
    Sean $D$ un dominio en $\mathbb{C}$ y $f$ una función holomorfa y nunca nula en $D$.
    \begin{enumerate}
        \item Si $g$ es una rama de $\sqrt{f}$ en $D$, entonces $g$ es holomorfa en $D$ y $g'(z) = \frac{f'(z)}{2g(z)}$ para todo $z \in D$.
        \item Si $g$ es un rama de $\log(f)$ en $D$, entonces $g$ es holomorfa en $D$ y $g'(z) = \frac{f'(z)}{f(z)}$ para todo $z \in D$.
        \item Existe una rama de $\log(f)$ en $D$ si y solo si $\frac{f'}{f}$ tiene primitiva en $D$.
    \end{enumerate}
\end{proposition}

\begin{proposition}
    Sean $D$ un dominio en $\mathbb{C}$ y $f: D \to \mathbb{C}$ una función continua en $D$.
    Entonces $f$ tiene primitiva en $D$ si y solo si $\int_\gamma f(z)dz = 0$ para todo camino cerrado $\gamma$ en $D$.
\end{proposition}

\begin{definition}
    Si $D$ es un dominio en $\mathbb{C}$ y $\Gamma$ es un ciclo en $D$, se dice que $\Gamma$ es homólogo a cero módulo $D$, y se denota $\Gamma \sim 0 (mod D)$, si $n(\Gamma, a) = 0$ para todo $a \in \mathbb{C} \setminus D$.
\end{definition}

\begin{theorem}[Versión general del teorema de Cauchy]
    Sea $D$ un dominio en $\mathbb{C}$ y sea $\Gamma$ un ciclo en $D$.
    Las dos siguientes condiciones son equivalentes:
    \begin{enumerate}
        \item $\Gamma \sim 0 (mod D)$.
        \item $\int_\Gamma f(z)dz = 0$ para toda función $f$ holomorfa en $D$.
    \end{enumerate}
\end{theorem}

\section{Dominios simplemente conexos}
\begin{definition}
    Si $D$ es un dominio en $\mathbb{C}$, se dice que $D$ es simplemente conexo si $\mathbb{C}^\ast \setminus D$ es conexo.
\end{definition}

Hay una serie de caracterizaciones para los dominios simplemente conexos, que se pueden deducir de los resultados anteriores.

\begin{theorem}
    Sea $D$ un dominio en $\mathbb{C}$.
    Las siguientes condiciones son equivalentes:
    \begin{enumerate}
        \item $D$ es simplemente conexo.
        \item Todo ciclo en $D$ es homólogo a cero módulo $D$.
        \item Todo camino cerrado en $D$ es homólogo a cero módulo $D$.
        \item $\int_\Gamma f(z)dz = 0$ para toda $f$ holomorfa en $D$ y para todo ciclo $\Gamma$ en $D$.
        \item $\int_\gamma f(z)dz = 0$ para toda $f$ holomorfa en $D$ y para todo camino cerrado $\gamma$ en $D$.
        \item Toda función holomorfa en $D$ tiene primitiva.
        \item Para toda función $f$ holomorfa y nunca nula en $D$, existe una rama de $\log(f)$ en $D$.
        \item Para toda función $f$ holomorfa y nunca nula en $D$, existe una rama de $\sqrt{f}$ en $D$.
    \end{enumerate}
\end{theorem}

Recordamos que:
\begin{itemize}
    \item Dos dominios $D_1$ y $D_2$ en $\mathbb{C}^\ast$ son conformemente equivalentes si existe una aplicación conforme $f$ de $D_1$ sobre $D_2$.
    \item En el conjunto de los dominios en $\mathbb{C}^\ast$, el ser conformemente equivalentes es una relación de equivalencia.
    \item Si $D_1$ y $D_2$ son dos dominios en $\mathbb{C}^\ast$ que son conformemente equivalentes, entonces $D_1$ es simplemente conexo si y solo si $D_2$ es simplemente conexo.
    \item $\mathbb{C}^\ast$, $\mathbb{C}$ y el disco unidad $\mathbb{D} = \{z \in \mathbb{C} : |z| < 1\}$ son tres dominios simplemente conexos en $\mathbb{C}^\ast$, que no son conformemente equivalentes.
    \item El único dominio en $\mathbb{C}^\ast$ conformemente equivalente a $\mathbb{C}^\ast$ es $\mathbb{C}^\ast$.
\end{itemize}

Vamos a ver que, además de $\mathbb{C}\ast$, $\mathbb{C}$ y $\mathbb{D}$, no hay más dominios simplemente conexos en $\mathbb{C}^\ast$ módulo la relación de equivalencia.
Es decir, si $D$ es un dominio simplemente conexo en $\mathbb{C}^\ast$, entonces $D$ es conformemente equivalente a uno de los tres: $\mathbb{C}^\ast$, $\mathbb{C}$ o $\mathbb{D}$.
Por tanto se tendrá que, si $D$ es un dominio simplemente conexo en $\mathbb{C}$, con $D \neq \mathbb{C}$, entonces $D$ es conformemente equivalente a $\mathbb{D}$.

\begin{definition}
    Sea $D$ un dominio en $\mathbb{C}^\ast$.
    Llamamos automorfismos de $D$ a aquellas aplicaciones conformes de $D$ sobre $D$.
    El conjunto de todos los automorfismos de $D$ se denota $Aut(D)$, y tiene estructura de grupo con la composición.
\end{definition}

Tenemos que
\begin{align*}
    Aut(\mathbb{C}^\ast) & = \mathcal{M}                                                                                                                                                       \\
    Aut(\mathbb{C})      & = \{f_{\alpha, \beta} : f_{\alpha, \beta}(z) = \alpha z + \beta, \alpha, \beta \in \mathbb{C}, \alpha \neq 0\} = \{T \in \mathcal{M} : T(\mathbb{C}) = \mathbb{C}\} \\
    Aut(\mathbb{D})      & = \{\lambda T_a : \lambda \in \mathbb{C}, |\lambda| = 1, a \in \mathbb{D}\} = \{T \in \mathcal{M} : T(\mathbb{D}) = \mathbb{D}\}
\end{align*}

\begin{example}
    Veamos algunos ejemplos de dominios en $\mathbb{C}$ para los que podemos encontrar una aplicación conforme del dominio sobre $\mathbb{D}$.
    \begin{enumerate}
        \item Un disco abierto, $D(a, R)$, $a \in \mathbb{C}, R > 0$.
              \begin{align*}
                  \mathbb{D} & \to D(a, R)    \\
                  z          & \mapsto a + rz
              \end{align*}
        \item Un semiplano.
              \begin{align*}
                  \mathbb{D} & \to \mathbb{H} = \{z \in \mathbb{C} : \Re(z) > 0\} \\
                  z          & \mapsto P(z)
              \end{align*}
              donde $P(z) = \frac{1+z}{1-z}$, es una aplicación conforme.

              Componiendo con una rotación y una traslación, vemos que $\mathbb{D}$ es conformemente equivalente a cualquier semiplano.
              \begin{align*}
                  \mathbb{D} & \to \mathbb{C}              \\
                  z          & \mapsto a + e^{i\theta}P(z)
              \end{align*}
              con $a \in \mathbb{C}$ y $\theta \in \mathbb{R}$.
        \item El exterior de un disco, $\{z \in \mathbb{C} : |z-a| > R\} \cup \{\infty\}$, $a \in \mathbb{C}, R > 0$.
              \begin{align*}
                  D(a, R) & \to \{z \in \mathbb{C} : |z-a| > R\} \cup \{\infty\} \\
                  z       & \mapsto \frac{1}{z}
              \end{align*}
        \item El plano menos una semirrecta, $\mathbb{C} \setminus \{a + re^{i\theta}, r \geq 0\}$, $a \in \mathbb{C}, \theta \in \mathbb{R}$.
              \begin{align*}
                  \mathbb{H} & \to \mathbb{C} \setminus (-\infty, 0] \\
                  z          & \mapsto z^2
              \end{align*}
              es una aplicación conforme.
              Así que
              \begin{align*}
                  \mathbb{H} & \to \mathbb{C} \setminus (-\infty, 0]           \\
                  z          & \mapsto P(z)^2 = \left(\frac{1+z}{1-z}\right)^2
              \end{align*}
              es una aplicación conforme.

              Componiendo con una rotación y una traslación, vemos que $\mathbb{D}$ es conformemente equivalente al plano menos una semirrecta cualquiera.
        \item La función exponencial no es inyectiva.
              $$z = x + iy_0 \mapsto e^z = e^{x + iy_0} = e^x(\cos(y_0) + i\sin(y_0))$$
              Es inyectiva en cualquier banda horizontal abierta de amplitud menor o igual que $2\pi$.
              Por ejemplo,
              $$\exp: \left\{z \in \mathbb{C} : |\Im(z)| < \frac{\pi}{2}\right\} \to \mathbb{H}$$
              es una aplicación conforme.
              Como $\mathbb{H}$ es conformemente equivalente a $\mathbb{D}$, tenemos que esta banda es conformemente equivalente a $\mathbb{D}$.

              Componiendo con el producto por un número real, una rotación y una traslación, vemos que $\mathbb{D}$ es conformemente equivalente a cualquier banda.
        \item Sectores.
              $$\left\{z \in \mathbb{C} : |\Im(z)| < \frac{\alpha}{2}\right\} \xrightarrow{\exp} S$$
              donde $S$ es el sector de vértice 0 y amplitud $\alpha$, es una aplicación conforme.
        \item $\mathbb{D}^+ = \{z \in \mathbb{D} : \Im(z) > 0\}$.
              $$\mathbb{D}^+ \xrightarrow{P} \{z \in \mathbb{C} : \Re(z) > 0, \Im(z) > 0\}$$
              es una aplicación conforme. El dominio $\{z \in \mathbb{C} : \Re(z) > 0, \Im(z) > 0\}$ es un sector, así que es conformemente equivalente a $\mathbb{D}$.
    \end{enumerate}
\end{example}

\section{El teorema de Riemann de la aplicación conforme}
\begin{theorem}[Teorema de Riemann de la aplicación conforme]
    Sea $D$ un dominio simplemente conexo en $\mathbb{C}$ con $D \neq \mathbb{C}$ y sea $z_0 \in D$.
    Entonces existe una única aplicación conforme $f$ de $D$ sobre $\mathbb{D}$ tal que $f(z_0) = 0$ y $f'(z_0) > 0$.
\end{theorem}

\begin{remark}
    \hfill
    \begin{enumerate}
        \item Existen infinitas aplicaciones conformes de $D$ sobre $\mathbb{D}$.
              Basta cambiar el punto $z_0$ o componer con una rotación.
        \item Para la demostración, las condiciones
              \begin{enumerate}
                  \item $D$ simplemente conexo.
                  \item $D \neq \mathbb{C}$.
              \end{enumerate}
              solo las vamos a utilizar para deducir que:
              \begin{itemize}
                  \item $\mathbb{C} \setminus D$ tiene más de un punto.
                  \item Si $h$ es holomorfa y nunca nula en $D$, existe una rama de $\sqrt{h}$ en $D$.
              \end{itemize}
    \end{enumerate}
\end{remark}

\begin{theorem}
    Sea $D$ un dominio en $\mathbb{C}$ tal que:
    \begin{enumerate}
        \item $\mathbb{C} \setminus D$ tiene más de un punto.
        \item Para toda función $h$ holomorfa y nunca nula en $D$, existe una rama de $\sqrt{h}$ en $D$.
    \end{enumerate}
    Sea $z_0 \in D$.
    Entonces existe una única aplicación conforme $f$ de $D$ sobre $\mathbb{D}$ tal que $f(z_0) = 0$ y $f'(z_0) > 0$.
\end{theorem}


\begin{proof}
    Sea $\mathcal{F} = \{f : f \text{ es holomorfa e inyectiva en } D, f(D) \subset \mathbb{D}, f(z_0) = 0\}$.
    \begin{enumerate}
        \item Veamos que $\mathcal{F} \neq \emptyset$.
              Por (1), existen $a, b \in \mathbb{C} \setminus D$ con $a \neq b$.
              Sea $\varphi(z) = \frac{z-a}{z-b}$, $z \in D$.
              $$\begin{vmatrix}
                      1 & -a \\
                      1 & -b
                  \end{vmatrix} = -b + a \neq 0 \Rightarrow \varphi \in \mathcal{M}$$
              $\varphi$ es holomorfa e inyectiva en $D$ y $\varphi$ es nunca nula en $D$, porque $a, b \notin D$.
              Por (2), existe $\psi$ rama de $\sqrt{\varphi}$ en $D$.
              $\psi$ es holomorfa e inyectiva en $D$ y $\psi$ es nunca nula.

              Además, se tiene que si $w \in \psi(D)$, entonces $-w \notin \psi(D)$.
              Veámoslo.
              Supongamos que $w \in \psi(D)$ y $-w \in \psi(D)$.
              Entonces:
              \begin{align*}
                  w  & = \psi(z_1), \quad z_1 \in D \\
                  -w & = \psi(z_2), \quad z_2 \in D
              \end{align*}
              $$\psi(z_1)^2 = w^2 = (-w)^2 = \psi(z_2)^2 \Leftrightarrow \varphi(z_1) = \varphi(z_2) \Leftrightarrow z_1 = z_2 \Leftrightarrow w = -w \Leftrightarrow w = 0 \in \psi(D)$$
              Sin embargo, $\psi$ es nunca nula en $D$.

              Tomamos $w_0 \in \psi(D)$.
              Como $\psi(D)$ es abierto, existe $r > 0$ tal que $\overline{D}(0, r) \subset \psi(D)$.
              Entonces, si $z \in D$ se tiene que $\psi(z) \in \psi(D)$ y por tanto $-\psi(z) \notin \psi(D)$, de manera que $-\psi(z) \notin \overline{D}(w_0, r)$.
              Es decir,
              $$|-\psi(z)-w_0| > r \Leftrightarrow |\psi(z)+w_0| > r > 0 \Leftrightarrow \frac{r}{|\psi(z)+w_0} < 1$$
              Sea $h(z) = \frac{r}{\psi(z)+w_0}$, $z \in D$.
              $h$ es holomorfa e inyectiva en $D$ y $|h(z)| < 1$ para todo $z \in D$, luego $h(D) \subset \mathbb{D}$.
              Consideramos la transformación de Möbius $S_{h(z_0)}(z) = \frac{z-h(z_0)}{1-\overline{h(z_0)}z}$.
              Sabemos que $S_{h(z_0)}(\mathbb{D}) = \mathbb{D}$ y $S_{h(z_0)}(h(z_0)) = 0$.
              Por tanto, $f = S_{h(z_0)} \circ h \in \mathcal{F}$.

        \item $\mathcal{F}$ está uniformemente acotada en $D$.
              Por el teorema de Montel, $\mathcal{F}$ es finitamente normal.

        \item Sea $M = \sup_{f \in \mathcal{F}} |f'(z_0)|$, $0 \leq M \leq \infty$.
              Si $f \in \mathcal{F}$, $f$ es holomorfa e inyectiva en $D$, por lo que $f'(z_0) \neq 0$.
              Entonces $M \neq 0$.
              Como $\mathcal{F}' = \{f' : f \in \mathcal{F}\}$ es finitamente normal, entonces está uniformemente acotada en $\{z_0\}$.
              Entonces $\{f'(z_0) : f \in \mathcal{F}\}$ está acotado, así que $M = \sup_{f \in \mathcal{F}} |f'(z_0)| < \infty$.
              Por tanto, $0 < M < \infty$.

              Tomamos una sucesión $\{f_n\}_{n=1}^\infty$ en $\mathcal{F}$ tal que $\lim\limits_{n \to \infty} |f_n'(z_0)| = M$.
              Como $\mathcal{F}$ es finitamente normal, existe $\{f_{n_k}\}_{k=1}^\infty$ subsucesión de $\{f_n\}$ que converge a una función $F$ uniformemente en cada subconjunto comapcto de $D$.
              Entonces $F$ es holomorfa en $D$ y cada $f_{n_k}$ es holomorfa e inyectiva en $D$.
              Por el segundo teorema de Hurwitz, $F$ es inyectiva o constante.
              Como $f_{n_k}' \to F'$ uniformemente en cada subconjunto compacto de $D$, se tiene que $f_{n_k}'(z_0) \to F'(z_0)$, así que $|f_{n_k}'(z_0)| \to |F'(z_0)| = M > 0$.
              Luego $F$ no es constante.
              Entonces $F$ es inyectiva.
              Además, $F(z_0)= \lim\limits_{k \to \infty} f_{n_k}(z_0) = 0$ porque $f_{n_z}(z_0) = 0$.
              Si $z \in D$, $F(z) = \lim\limits_{k \to \infty} f_{n_k}(z)$, así que $|F(z)| = \lim\limits_{k \to \infty} |f_{n_z}(z)| \leq 1$ porque $|f_{n_z}(z)| < 1$.
              Pero si $|F(z)| = 1$ para algún $z \in D$, por el principio del máximo $F$ sería constante, lo cual es imposible.
              Por tanto, $|F(z)| < 1$ para todo $z \in D$, luego $F(D) \subset \mathbb{D}$.
              Entonces $F \in \mathcal{F}$ y $|F'(z_0)| = M$.

        \item Veamos que $F(D) = \mathbb{D}$.
              Supongamos por reducción al absurdo que existe $\alpha \in \mathbb{D} \setminus F(D)$.
              Consideramos $S_\alpha(z) = \frac{z-\alpha}{1-\bar{\alpha}z}$ transformación de Möbius con $S_\alpha(\mathbb{D}) = \mathbb{D}$ y $S_\alpha(\alpha) = 0$.
              Sea $h = S_\alpha \circ F$.
              $h$ es holomorfa e inyectiva en $D$.
              Además, $h$ es nunca nula en $D$ y $h(D) \subset \mathbb{D}$.
              Por (2), existe $g$ una rama de $\sqrt{h}$ en $D$, es decir, $g^2 = h$ en $D$.
              $g$ es holomorfa, inyectiva y nunca nula en $D$, con $g(D) \subset \mathbb{D}$.
              Sea $G = S_{g(z_0)} \circ g$.
              $G$ es holomorfa e inyectiva en $D$, con $G(D) \subset \mathbb{D}$ y $G(z_0) = 0$.
              Por tanto, $G \in \mathcal{F}$.

              Calculemos $|G'(z_0)|$.
              $$G'(z_0) = g'(z_0)S_{g(z_0)}'(g(z_0))$$
              En primer lugar, hallamos la derivada de $S_a$.
              $$S_a'(z) = \frac{1-\bar{a}z + (z-a)\bar{a}}{(1-\bar{a}z)^2} = \frac{1-|a|^2}{(1-\bar{a}z)^2}$$
              Observamos que $S_a'(a) = \frac{1}{1-|a|^2}$ y $S_a'(0) = 1-|a|^2$.
              Así que:
              $$G'(z_0) = g'(z_0)\frac{1}{1-|g(z_0)|^2} \Rightarrow |G'(z_0)| = \frac{|g'(z_0)|}{1-|g(z_0)|^2}$$
              Como $g^2 = h$ en $D$, también tenemos que $2gg' = h$ en $D$.
              Luego $|g(z_0)|^2 = |h(z_0)|$ y $2|g(z_0)||g'(z_0)| = |h'(z_0)|$.
              Entonces:
              $$|G'(z_0)| = \frac{|h'(z_0)|}{2|g(z_0)|}\frac{1}{1-|g(z_0)|^2} = \frac{|h'(z_0)|}{2\sqrt{|h(z)|}}\frac{1}{1-|h(z_0)|}$$
              Calculamos también:
              \begin{align*}
                  h'(z_0) & = F'(z_0)S_\alpha'(F(z_0)) = F'(z_0)S_\alpha'(0) = F'(z_0)(1-|\alpha|^2)   \\
                  h(z_0)  & = S_\alpha(F(z_0)) = S_\alpha(0) = -\alpha \Rightarrow |h(z_0)| = |\alpha|
              \end{align*}
              Por tanto:
              $$|G'(z_0)| = \frac{|F'(z_0)|(1-|\alpha|^2)}{2\sqrt{|\alpha|}(1-|\alpha|)} = M\frac{1-|\alpha|^2}{2\sqrt{|\alpha|}(1-|\alpha|)} = M\frac{1+|\alpha|}{2\sqrt{|\alpha|}}$$

              Veamos que $\frac{1+|\alpha|}{2\sqrt{|\alpha|}} > 1$.
              \begin{align*}
                   & \frac{1+|\alpha|}{2\sqrt{|\alpha|}} > 1 \Leftrightarrow 1+|\alpha| > 2\sqrt{|\alpha|} \Leftrightarrow 1+\alpha - 2\sqrt{|\alpha|} > 0 \Leftrightarrow (1-\sqrt{|\alpha|})^2 > 0 \Leftrightarrow 1-\sqrt{|\alpha|} \neq 0 \Leftrightarrow \\
                   & \Leftrightarrow \sqrt{|\alpha|} \neq 1 \Leftrightarrow |\alpha| \neq 1
              \end{align*}
              Como $\alpha \in \mathbb{D}$, la desigualdad se cumple.
              Por tanto, $|G'(z_0)| > M$, con $G \in \mathcal{F}$ y $M = \sup_{f \in \mathcal{F}} |f'(z_0)|$, luego llegamos a contradicción.
              Entonces, $F(D) = \mathbb{D}$.

        \item Tenemos $F \in \mathcal{F}$, $|F'(z_0)| = M$ y $F(D) = \mathbb{D}$.
              $F$ es una aplicación conforme de $D$ sobre $\mathbb{D}$ con $F(z_0) = 0$.
              Falta que $F'(z_0) > 0$.

              Queremos encontrar $\lambda \in \mathbb{C}$ con $|\lambda| = 1$ tal que $f = \lambda F$ verifique que $f'(z_0) > 0$.
              $$f'(z_0) = \lambda F'(z_0) > 0 \Leftrightarrow f'(z_0) = |\lambda||F'(z_0)| = |F'(z_0)| = M \Rightarrow \lambda = \frac{M}{F'(z_0)}$$
              Sea $\lambda = \frac{M}{F'(z_0)} \in \mathbb{C}$, con $|\lambda| = \frac{M}{|F'(z_0)|} = 1$, $F'(z_0) \neq 0$.
              Sea $f = \lambda F$.
              $f$ es holomorfa e inyectiva en $D$, con $f(D) = \mathbb{D}$, $f(z_0) = 0$ y $f'(z_0) = \lambda F'(z_0) = \frac{M}{F'(z_0)}F'(z_0) = M > 0$.

        \item Veamos que esta aplicación conforme es única.
              Supongamos $f_1, f_2: D \to \mathbb{D}$ aplicación conforme, con $f_j(z_0) = 0$, $f_j'(z_0) > 0$.
              Sea $g = f_1 \circ f_2^{-1}$.
              $$g: \mathbb{D} \xrightarrow{f_2^{-1}} D \xrightarrow{f_1} \mathbb{D}$$
              $g$ es una aplicación conforme de $\mathbb{D}$ sobre $\mathbb{D}$, así que es de la forma
              $$g(z) = \lambda T_a(z) = \lambda \frac{z+a}{1+\bar{a}z}, \quad \lambda \in \mathbb{C}, \; |\lambda| = 1, \; a \in \mathbb{D}$$
              Como $g(0) = 0$,
              $$g(0) = \lambda a = 0 \Rightarrow a = 0 \Rightarrow g(z) = \lambda z$$
              Como $g \circ f_2 = f_1$ en $D$,
              $$f_2'(z_0)g'(f_2(z_0)) = f_1'(z_0) \Leftrightarrow f_2'(z_0)g'(0) = f_1'(z_0) \Leftrightarrow g'(0) = \frac{f_1'(z_0)}{f_2'(z_0)} > 0$$
              Como $g'(0) = \lambda > 0$ y $|\lambda| = 1$, entonces $\lambda = 1$.
              Por tanto, $g(z) = z \Leftrightarrow f_1 = f_2$.
    \end{enumerate}
\end{proof}

\subsection*{Otros enunciados equivalentes}
\begin{theorem}[Teorema de Riemann]
    Sea $D$ un dominio simplemente conexo en $\mathbb{C}$ con $D \neq \mathbb{C}$ y sea $z_0 \in D$.
    Entonces existe una única aplicación conforme de $\mathbb{D}$ sobre $D$ tal que $f(0) = z_0$ y $f'(0) > 0$.
\end{theorem}


\begin{theorem}[Teorema de Riemann: enunciado equivalente]
    Sea $D$ un dominio simplemente conexo en $\mathbb{C}$ con $D \neq \mathbb{C}$ y sea $z_0 \in D$.
    Entonces existe un único $R > 0$ tal que existe una aplicación conforme $f$ de $D$ sobre $D(0, R)$ con $f(z_0) = 0$ y $f'(z_0) = 1$.
    Además, esta $f$ es única.

    A este número $R$ se le denomina radio conforme interior a $D$ en $z_0$ y se denota $r(D, z_0)$.
\end{theorem}

\begin{proof}
    Este enunciado es equivalente al teorema de Riemann.
    Si $D$ es un dominio simplemente conexo en $\mathbb{C}$ con $D \neq \mathbb{C}$ y $z_0 \in D$.
    \begin{enumerate}
        \item Si $f: D \to \mathbb{D}$ aplicación conforme, $f(z_0) = 0$ y $f'(z_0) > 0$, entonces si $R = \frac{1}{f'(z_0)} > 0$ se tiene que $g = Rf: D \to D(0, R)$ es una aplicación conforme con $g(z_0) = 0$ y $g'(z_0) = 1$.
        \item Si $R > 0$, $g: D \to D(0, R)$ es una aplicación conforme con $g(z_0) = 0$ y $g'(z_0) = 1$, entonces $f = \frac{1}{R}g: D \to \mathbb{D}$ es una aplicación conforme con $f(z_0) = 0$ y $f'(z_0) = \frac{1}{R}g'(z_0) = \frac{1}{R} > 0$.
    \end{enumerate}
\end{proof}

\begin{remark}
    Hemos visto que cualquier dominio $D$ simplemente conexo en $\mathbb{C}$ con $D \neq \mathbb{C}$ es conformemente equivalente a $\mathbb{D}$.
    Entonces, si $D_1$ y $D_2$ son dominios simplemente conexos en $\mathbb{C}$ con $D_1, D_2 \neq \mathbb{C}$, $D_1$ y $D_2$ son conformemente equivalentes.
\end{remark}

En $\mathbb{C}^\ast$ tenemos el siguiente resultado.

\begin{theorem}
    Sea $D$ un dominio simplemente conexo en $\mathbb{C}^\ast$ tal que $\mathbb{C}^\ast \setminus D$ tiene más de un punto.
    Entonces $D$ es conformemente equivalente a $\mathbb{D}$.
\end{theorem}

\begin{proof}
    \hfill
    \begin{itemize}
        \item Si $D \subset \mathbb{C}$, entonces $D$ es un dominio simplemente conexo en $\mathbb{C}$ con $D \neq \mathbb{C}$, ya que $\mathbb{C}^\ast \setminus D$ tiene más de un punto.
              Entonces $D$ es conformemente equivalente a $\mathbb{D}$.
        \item Si $\infty \in D$, entonces tomamos $a, b \in \mathbb{C}^\ast \setminus D$ con $a \neq b$.
              Entonces $a, b \in \mathbb{C} \setminus D$.
              Sea $T(z) = \frac{1}{z-a}$, $z \in \mathbb{C} \setminus \{a\}$, con $T(a) = \infty$ y $T(\infty) = 0$.
              $T: \mathbb{C}^\ast \to \mathbb{C}\ast$ es una aplicación conforme.
              Entonces $D \xrightarrow{T} T(D) = D'$ es una aplicación conforme y $D'$ es un dominio simplemente conexo en $\mathbb{C}^\ast$.

              Como $a, b \notin D$, $T(a) = \infty \notin D'$, así que $D'$ es un dominio en $\mathbb{C}$.
              Además, $T(b) \notin D'$ con $T(b) \in \mathbb{C}$, luego $D \neq \mathbb{C}$.
              $D'$ es un dominio simplemente conexo en $\mathbb{C}$, con $D' \neq \mathbb{C}$.
              Por tanto, $D'$ es conformemente equivalente a $\mathbb{D}$.
              Como $D'$ es conformemente equivalente a $D$, entonces $D$ es conformemente equivalente a $\mathbb{D}$.
    \end{itemize}
\end{proof}