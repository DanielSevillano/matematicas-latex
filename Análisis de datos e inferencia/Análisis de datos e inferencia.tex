\documentclass{report}
\usepackage[spanish]{babel}
\usepackage{amssymb, amsmath, amsthm, hyperref}

\title{Análisis de datos e inferencia}
\author{}

\newtheorem{theorem}{Teorema}[chapter]
\newtheorem{corollary}[theorem]{Corolario}
\newtheorem{lemma}[theorem]{Lema}
\newtheorem{proposition}[theorem]{Proposición}

\theoremstyle{remark}
\newtheorem*{remark}{Observación}
\theoremstyle{remark}
\newtheorem*{note}{Nota}
\theoremstyle{remark}
\newtheorem*{notation}{Notación}

\theoremstyle{definition}
\newtheorem{definition}{Definición}[chapter]
\theoremstyle{definition}
\newtheorem*{properties}{Propiedades}
\theoremstyle{definition}

\begin{document}
\maketitle
\tableofcontents

\chapter{Modelo de regresión lineal simple}
\section{Introducción}

La regresión lineal es un modelo matemático que nos permite establecer la relación de dependencia entre una variable dependiente $Y$ y una variable independiente $X$.

Nos interesan las relaciones de la forma $y = f(x) + u$, donde $u$ es una variable aleatoria a la que llamamos perturbación.
En el caso de la regresión lineal simple, el modelo será de la forma $$y = \beta_0 + \beta_1x + u$$
con $\beta_0$ y $\beta_1$ parámetros.
Llamamos intercepto a $\beta_0$ y pendiente a $\beta_1$.

\section{Modelo e hipótesis}
Sea $X$ una variable aleatoria cuantitativa, $Y$ una variable aleatoria continua y $(x_1, y_1), (x_2, y_2), \dots (x_n, y_n)$ un conjunto de datos.
Entonces el modelo de regresión lineal simple es
$$y_i = \beta_0 + \beta_1x_i + u_i, \quad i = 1, \dots n$$

\subsection*{Hipótesis del modelo}
\begin{enumerate}
    \item $E(u_i) = 0, \quad \forall i = 1, \dots n$.
    \item $Var(u_i) = \sigma^2, \quad \forall i = 1, \dots n$ (homocedasticidad)
    \item $u_i \sim N(0, \sigma^2), \quad \forall i = 1, \dots n$ (normalidad)
    \item $E(u_i u_j) = 0, \quad \forall i \neq j$ (independencia)
\end{enumerate}

\begin{note}
    En realidad, la cuarta hipótesis es de incorrelación ($Cov(u_i, u_j) = 0$).
    $$Cov(u_i, u_j) = E(u_i u_j) - E(u_i)E(u_j) = E(u_i u_j)$$
    Sin embargo, bajo normalidad la incorrelación y la independencia son equivalentes.
\end{note}

Podemos escribir las mismas hipótesis en términos de $y_i, \quad \forall i = 1, \dots n$.
\begin{enumerate}
    \item $E(y_i | x_i) = E(\beta_0 + \beta_1x_i + u_i) = \beta_0 + \beta_1x_i, \quad \forall i = 1, \dots n$ (linealidad)
    \item $Var(y_i | x_i) = Var(\beta_0 + \beta_1x_i + u_i) = \sigma^2, \quad \forall i = 1, \dots n$ (homocedasticidad)
    \item $y_i | x_i \sim N(\beta_0 + \beta_1x_i, \sigma^2), \quad \forall i = 1, \dots n$ (normalidad)
    \item $Cov(y_i, y_j) = 0, \quad \forall i \neq j$ (independencia)
\end{enumerate}

Podemos dar un significado real a $\beta_0$ y $\beta_1$:
\begin{itemize}
    \item $\beta_0$ es el valor medio de la variable $Y$ cuando $x_i$ toma el valor 0.
          $$E(y_i | x_i = 0) = \beta_0, \quad i = 1, \dots n$$
    \item $\beta_1$ es la variación media que experimenta la variable $Y$ cuando $x_i$ aumenta en una unidad.
          $$E(y_i | x_i+1) - E(y_i | x_i) = \beta_1, \quad i = 1, \dots n$$
\end{itemize}

\section{Estimación de los parámetros}
Queremos estimar $\beta_0$, $\beta_1$ y $\sigma^2$.
Con los estimadores $\hat{\beta_0}$ y $\hat{\beta_1}$ podemos estimar $$\hat{E(y_i | x_i)} = \hat{\beta_0} + \hat{\beta_1}x_i, \quad i = 1, \dots n$$.

\subsection{Método de máxima verosimilitud}
$y_i | x_i \sim N(\beta_0 + \beta_1x_i, \sigma^2), \quad i = 1, \dots, n$, así que podemos encontrar estimadores de máxima verosimilitud para los parámetros y para $\sigma^2$.

Usando el método de máxima verosimilitud llegamos las ecuaciones normales de la regresión:
$$\begin{cases}
        \frac{\partial \log{L}}{\partial \beta_0} = \frac{1}{\sigma^2} \sum_{i=1}^n (y_i - \beta_0 - \beta_1x_i) = 0 \\
        \frac{\partial \log(L)}{\partial \beta_1} = \frac{1}{\sigma^2} \sum_{i=1}^n x_i(y_i - \beta_0 - \beta_1x_i) = 0
    \end{cases}$$

\begin{notation}
    $\hat{y_i} = \hat{E(y_i | x_i)} = \hat{\beta_0} + \hat{\beta_1}x_i$
\end{notation}

Si definimos el error o residuo como $e_i = y_i - \hat{y_i} = y_i - \hat{\beta_0} - \hat{\beta_1}x_i$, podemos escribir las ecuaciones normales de regresión de la siguiente forma:
$$\begin{cases}
        \sum_{i=1}^n e_i = 0 \\
        \sum_{i=1}^n x_ie_i = 0
    \end{cases}$$

Resolviendo este sistema, obtenemos los estimadores:
\begin{align*}
    \hat{\beta_1}  & = \frac{s_{XY}}{s_X^2}                                                                                 \\
    \hat{\beta_0}  & = \bar{y} - \frac{s_{XY}}{s_X^2}\bar{x}                                                                \\
    \hat{\sigma^2} & = \frac{1}{n} \sum_{i=1}^n (y_i - \hat{\beta_0} - \hat{\beta_1}x_i)^2 = \frac{1}{n} \sum_{i=1}^n e_i^2
\end{align*}

La ecuación de la recta resultante es:
$$\hat{y_i} = \bar{y} + \frac{s_{XY}}{s_X^2}(x_i - \bar{x})$$

\subsection{Estimación por mínimos cuadrados}
Queremos minimizar la suma de los cuadrados de los errores $\sum_{i=1}^n e_i^2$, donde $e_i = y_i - \hat{y_i}$.
Para ello minimizamos la función $M(\beta_0, \beta_1) = \sum_{i=1}^n (y_i - \beta_0 - \beta_1x_i)^2$.
$$\begin{cases}
        \frac{\partial M}{\partial \beta_0}(\beta_0, \beta_1) = -2\sum(y_i - \beta_0 - \beta_1x_i) = 0 \\
        \frac{\partial M}{\partial \beta_1}(\beta_0, \beta_1) = -2\sum x_i(y_i - \beta_0 - \beta_1x_i) = 0
    \end{cases}$$

Simplificando obtenemos las ecuaciones normales de la regresión, como antes.
Así que los estimadores de $\beta_0$ y $\beta_1$ por máxima verosimilitud coinciden con los estimadores por mínimos cuadrados.

\subsection{Estimación de la varianza}
Partiendo del estimador $\bar{\sigma^2} = \frac{1}{n} \sum_{i=1}^n e_i^2$ obtenido previamente, podemos llegar a una expresión equivalente:
$$\hat{\sigma^2} = s_Y^2 - \frac{s_{XY}^2}{s_X^2}$$

Veamos si este estimador es insesgado calculando su esperanza.
$$E(\hat{\sigma^2}) = E(\frac{\sum_{i=1}^n e_i^2}{n}) = \frac{1}{n}E(\sum_{i=1}^n e_i^2) = \frac{1}{n}\sigma^2(n-2)$$

\begin{note}
    $\frac{\sum_{i=1}^n e_i^2}{\sigma^2} \sim \chi^2_{n-2}, \quad E(\frac{\sum_{i=1}^n e_i^2}{\sigma^2}) = n-2$
\end{note}

Observamos que este estimador no es insesgado. Consideramos entonces:
$$s_R^2 = \frac{1}{n-2}\sum_{i=1}^n e_i^2$$
Este sí es un estimador insesgado de $\sigma^2$ y le llamamos varianza residual.
Tenemos la relación $s_R^2 = \frac{n}{n-2} \hat{\sigma^2}$.

\section{Propiedades de los estimadores}
Podemos escribir $\hat{\beta_1}$ de la forma:
$$\hat{\beta_1} = \sum_{i=1}^n w_iy_i, \quad w_i = \frac{x_i - \bar{x}}{ns_X^2}$$
Por las hipótesis del modelo, $y_i$ son normales e independientes, luego $\hat{\beta_1} \sim N$.
Podemos calcular:

\begin{itemize}
    \item $E(\hat{\beta_1}) = \beta_1$ (estimador insesgado)
    \item $V(\hat{\beta_1}) = \frac{\sigma^2}{ns_X^2}$
\end{itemize}
Por tanto, $\hat{\beta_1} \sim N(\beta_1, \frac{\sigma^2}{ns_X^2})$.

De forma análoga, podemos escribir:
$$\hat{\beta_0} = \sum_{i=1}^n (\frac{1}{n} - \bar{x}w_i)$$
Como las $y_i$ son normales e independientes, $\hat{\beta_0} \sim N$.
Calculamos:

\begin{itemize}
    \item $E(\hat{\beta_0}) = \beta_0$ (estimador insesgado)
    \item $V(\hat{\beta_0}) = \frac{\sigma^2}{n} (1 + \frac{\bar{x}^2}{s_X^2})$
\end{itemize}
Por tanto, $\hat{\beta_0} \sim N(\beta_0, \frac{\sigma^2}{n} (1 + \frac{\bar{x}^2}{s_X^2}))$.

En cuanto a $s_R^2$, sabemos que $\frac{1}{\sigma^2} \sum_{i_1}^n e_i^2 \sim \chi^2_{n-2}$.
Obtenemos que:
\begin{itemize}
    \item $E(s_R^2) = \sigma^2$
    \item $V(s_R^2) = \frac{2}{n-2} (\sigma^2)^2$
\end{itemize}

\section{Intervalos de confianza para los parámetros}
\subsection*{Intervalos de confianza para $\beta_1$}
\subsubsection*{Caso 1: $\sigma^2$ conocida}
Sabemos que $\hat{\beta_1} \sim N(\beta_1, \frac{\sigma^2}{ns_X^2})$. Entonces:
$$\frac{\hat{\beta_1} - \beta_1}{\frac{\sigma}{\sqrt{ns_X^2}}} \sim N(0, 1)$$
Por tanto, el intervalo de confianza para $\beta_1$ a nivel de significación $\alpha$ es:
$$IC_{1-\alpha}(\beta_1) = \left( \hat{\beta_1} - z_{1-\frac{\alpha}{2}} \frac{\sigma}{\sqrt{ns_X^2}}, \hat{\beta_1} + z_{1-\frac{\alpha}{2}} \frac{\sigma}{\sqrt{ns_X^2}} \right)$$
donde $z_{1-\frac{\alpha}{2}}$ es el percentil de orden $(1 - \frac{\alpha}{2}) 100\%$ de una variable aleatoria $Z \sim N(0, 1)$.

\subsubsection*{Caso 2: $\sigma^2$ desconocida}
$$\begin{cases}
        \frac{\hat{\beta_1} - \beta_1}{\frac{\sigma}{\sqrt{ns_X^2}}} \sim N(0, 1) \\
        \frac{\sum_{i=1}^n e_i^2}{\sigma^2} = \frac{(n-2)s_R^2}{\sigma^2} \sim \chi^2_{n-2}
    \end{cases} \Rightarrow
    \frac{\frac{\hat{\beta_1} - \beta_1}{\frac{\sigma}{\sqrt{ns_X^2}}}}{\sqrt{\frac{(n-2)s_R^2}{\sigma^2} \frac{1}{n-2}}} = \frac{\hat{\beta_1} - \beta_1}{\frac{s_R}{\sqrt{ns_X^2}}} \sim t_{n-2}$$
Luego el intervalo de confianza para $\beta_1$ a nivel de significación $\alpha$ es:
$$IC_{1-\alpha}(\beta_1) = \left( \hat{\beta_1} - t_{n-2, 1-\frac{\alpha}{2}} \frac{s_R}{\sqrt{ns_X^2}}, \hat{\beta_1} + t_{n-2, 1-\frac{\alpha}{2}} \frac{s_R}{\sqrt{ns_X^2}} \right)$$
donde $s_R = +\sqrt{s_R^2}$ y $t_{n-2, 1-\frac{\alpha}{2}}$ es el percentil de orden $(1 - \frac{\alpha}{2}) 100\%$ de una variable aleatoria $T \sim t_{n-2}$.

\subsection*{Intervalos de confianza para $\beta_0$}
\subsubsection*{Caso 1: $\sigma^2$ conocida}
Sabemos que $\hat{\beta_0} \sim N\left( \beta_0, \frac{\sigma^2}{n} \left( 1 + \frac{\bar{x}}{s_X^2} \right) \right)$.
Entonces, el intervalo de confianza para $\beta_0$ a nivel de significación $\alpha$ es:
$$IC_{1-\alpha}(\beta_0) = \left( \hat{\beta_0} - z_{1-\frac{\alpha}{2}} \frac{\sigma}{\sqrt{n}} \sqrt{1 + \frac{\bar{x}^2}{s_X^2}}, \hat{\beta_0} + z_{1-\frac{\alpha}{2}} \frac{\sigma}{\sqrt{n}} \sqrt{1 + \frac{\bar{x}^2}{s_X^2}} \right)$$
donde $z_{1-\frac{\alpha}{2}}$ es el percentil de orden $(1 - \frac{\alpha}{2}) 100\%$ de una variable aleatoria $Z \sim N(0, 1)$.

\subsubsection*{Caso 2: $\sigma^2$ desconocida}
Razonando de forma análoga al caso de $\beta_1$, tenemos que:
$$\frac{\hat{\beta_0} - \beta_0}{\frac{s_R}{n} \sqrt{1 + \frac{\bar{x}^2}{s_X^2}}}$$
Por tanto, el intervalo de confianza para $\beta_0$ a nivel de significación $\alpha$ es:
$$IC_{1-\alpha}(\beta_0) = \left( \hat{\beta_0} - t_{n-2, 1-\frac{\alpha}{2}} \frac{s_R}{\sqrt{n}} \sqrt{1 + \frac{\bar{x}^2}{s_X^2}}, \hat{\beta_0} + t_{n-2, 1-\frac{\alpha}{2}} \frac{s_R}{\sqrt{n}} \sqrt{1 + \frac{\bar{x}^2}{s_X^2}} \right)$$
donde $t_{n-2, 1-\frac{\alpha}{2}}$ es el percentil de orden $(1 - \frac{\alpha}{2}) 100\%$ de una variable aleatoria $T \sim t_{n-2}$.

\subsection*{Intervalos de confianza para $\sigma^2$}
Sabemos que $\frac{\sum_{i=1}^n e_i^2}{\sigma^2} = \frac{(n-2)s_R^2}{\sigma^2} \sim \chi^2_{n-2}$.
Queremos que $P(a < \sigma^2 < b) = 1-\alpha$.
\begin{align*}
    P(a < \sigma^2 < b) & = P\left( \frac{1}{b} < \frac{1}{\sigma^2} < \frac{1}{a} \right) =                          \\
                        & = P\left( \frac{(n-2)s_R^2}{b} < \frac{(n-2)s_R^2}{\sigma^2} < \frac{(n-2)s_R^2}{a} \right)
\end{align*}
Luego:
$$\begin{cases}
        \frac{(n-2)s_R^2}{b} = \chi^2_{n-2, \frac{\alpha}{2}} \Rightarrow b = \frac{(n-2)s_R^2}{\chi^2_{n-2, \frac{\alpha}{2}}} \\
        \frac{(n-2)s_R^2}{a} = \chi^2_{n-2, 1 - \frac{\alpha}{2}} \Rightarrow a = \frac{(n-2)s_R^2}{\chi^2_{n-2, 1 - \frac{\alpha}{2}}}
    \end{cases}$$
Por tanto, el intervalo de confianza para $\sigma^2$ a nivel de significación $\alpha$ tiene por extremos $a$ y $b$, es decir:
$$IC_{1-\alpha}(\sigma^2) = (a, b)$$

\section{Contraste de la regresión}
Consideramos el siguiente contraste de hipótesis:
$$\begin{cases}
        H_0 : \beta_1 = 0 \Leftrightarrow E(y|x) = \beta_0 \\
        H_1 : \beta_1 \neq 0 \Leftrightarrow E(y|x) = \beta_0 + \beta_1x
    \end{cases}$$
Fijamos el nivel de significación $\alpha$.
Podemos resolverlo de cuatro formas distintas.

\subsection*{Intervalos de confianza}
Sea $IC_{1-\alpha}(\beta_1)$ el intervalo de confianza para $\beta_1$ a nivel de significación $\alpha$.
Entonces:
\begin{itemize}
    \item Aceptamos $H_0$ a nivel de significación $\alpha$ si $0 \in IC_{1-\alpha}(\beta_1)$.
    \item Rechazamos $H_0$ a nivel de significación $\alpha$ en caso contrario.
\end{itemize}

\subsection*{Estadístico $T$}
Sabemos que $\frac{\hat{\beta_1} - \beta_1}{\frac{s_R}{\sqrt{ns_X^2}}} \sim t_{n-2}$.
Entonces $T = \frac{\hat{\beta_1}}{\frac{s_R}{\sqrt{ns_X^2}}} \sim t_{n-2}$ si $H_0$ es cierto.

Tomamos un $t_{exp}$.
\begin{itemize}
    \item Si $t_{exp} \in (-t_{n-2, 1-\frac{\alpha}{2}}, t_{n-2, 1-\frac{\alpha}{2}})$, o equivalentemente $|t_{exp}| \leq t_{n-2, 1-\frac{\alpha}{2}}$, aceptamos $H_0$ a nivel de significación $\alpha$.
    \item En caso contrario, rechazamos $H_0$ a nivel de significación $\alpha$.
\end{itemize}

\subsection*{Valor $p$}
Sea $p$ el valor $p$ o $p$-valor de la distribución. Entonces:
\begin{itemize}
    \item Si $p \geq \alpha$, aceptamos $H_0$ a nivel de significación $\alpha$.
    \item En caso contrario, rechazamos $H_0$ a nivel de significación $\alpha$.
\end{itemize}

\subsection*{Tabla ANOVA}
Partimos de que podemos escribir:
$$\sum_{i=1}^n (y_i - \bar{y})^2 = \sum_{i=1}^n (y_i - \hat{y_i})^2 + \sum_{i=1}^n (\hat{y_i} - \bar{y})^2$$
Definimos:
\begin{itemize}
    \item Variabilidad total:
          $$VT = \sum_{i=1}^n (y_i - \bar{y})^2 = ns_Y^2$$
    \item Variabilidad no explicada:
          $$VNE = \sum_{i=1}^n (y_i - \hat{y_i})^2 = \sum_{i=1}^n e_i^2 = (n-2)s_R^2 = n\hat{\sigma^2} = n(s_Y^2 - \hat{\beta_1}^2s_X^2)$$
    \item Variabilidad explicada:
          $$VE = \sum_{i=1}^n (\hat{y_i} - \bar{y})^2 = VT - VNE = n\hat{\beta_1}^2 s_X^2$$
\end{itemize}
Observamos que:
$$\frac{VNE}{\sigma^2} = \frac{\sum_{i=1}^n e_i^2}{\sigma^2} \sim \chi^2_{n-2}$$
Además, como $\frac{\hat{\beta_1} - \beta_1}{\frac{\sigma}{\sqrt{ns_X^2}}} \sim N(0,1)$, entonces:
$$\frac{(\hat{\beta_1} - \beta_1)^2}{\frac{\sigma^2}{ns_X^2}} \sim \chi^2_1$$
Luego $\frac{\hat{\beta_1}^2}{\frac{\sigma^2}{ns_X^2}} = \frac{n\hat{\beta_1}^2s_X^2}{\sigma^2} = \frac{VE}{\sigma^2} \sim \chi^2_1$ si $H_0$ es cierta.

Consideramos ahora $F = \frac{\frac{VE}{\sigma^2} / 1}{\frac{VNE}{\sigma^2} / (n-2)} = \frac{VE}{s_R^2}$.
Observamos que $F \sim F_{1, n-2}$ si $H_0$ es cierta.

Tomamos un $F_{exp}$.
\begin{itemize}
    \item Aceptamos $H_0$ a nivel de significación $\alpha$ si $F_{exp} \leq F_{1, n-2, 1-\alpha}$.
    \item En caso contrario, rechazamos $H_0$ a nivel de significación $\alpha$.
\end{itemize}

La tabla ANOVA es de la forma:
\begin{center}
    \begin{tabular}{ c | c | c | c }
        Fuentes & Suma de cuadrados                   & Grados de libertad & Cocientes         \\
        \hline
        $VE$    & $\sum_{i=1}^n(\hat{y_i}-\bar{y})^2$ & 1                  & $\frac{VE}{1}$    \\
        $VNE$   & $\sum_{i=1}^n(y_i - \hat{y_i})^2$   & $n-2$              & $\frac{VNE}{n-2}$ \\
        $VT$    & $\sum_{i=1}^n(y_i - \bar{y})^2$     & $n-1$
    \end{tabular}
\end{center}
También se incluyen columnas para $F_{exp}$ y $p$-valor.

\begin{remark}
    Existe la siguiente relación entre $t_{exp}$ y $F_exp$:
    $$t_{exp}^2 = F_{exp}$$
\end{remark}

\section{Evaluación del ajuste}
Existen dos coeficientes para evaluar el ajuste del modelo: el coeficiente de correlación lineal y el coeficiente de determinación.

\subsection*{Coeficiente de correlación lineal}
El coeficiente de correlación lineal se define como:
$$r = \frac{s_{XY}}{s_X s_Y}, \quad -1 \leq r \leq 1$$
\begin{itemize}
    \item Si $r = 1$, se tiene dependencia lineal exacta positiva.
    \item Si $r = -1$, se tiene dependencia lineal exacta negativa.
    \item Si $r = 0$, las variables están incorreladas linealmente.
\end{itemize}
Se dice que el ajuste es bueno si $|r|$ es cercano a 1.
Si por el contrario $r$ se aproxima a 0, entonces las variables no tienen relación lineal.

\subsection*{Coeficiente de determinación}
El coeficiente de determinación se define como:
$$R^2 = \frac{VE}{VT}, \quad 0 \leq R^2 \leq 1$$
\begin{itemize}
    \item Si $R^2 = 1$ entonces $VE = VT$ luego $VNE = \sum_{i=1}^n (y_i - \hat{y_i})^2 = \sum_{i=1}^n e_i^2 = 0$.
          Por tanto, $e_i = 0$ para todo $i = 1, \dots n$, así que el ajuste lineal es exacto.
    \item Si $R^2 = 0$ entonces $VE = 0$, luego $VT = VNE$.
          Así que el ajuste lineal es pésimo.
\end{itemize}

\begin{theorem}
    El coeficiente de determinación coincide con el coeficiente de correlación lineal al cuadrado.
    Es decir, $$r^2 = R^2$$

    \begin{proof}
        $$R^2 = \frac{VE}{VT} = \frac{n\hat{\beta_1}^2s_X^2}{ns_Y^2} = \frac{\left(\frac{s_{XY}}{s_X^2}\right)^2 s_X^2}{s_Y^2} = \frac{s_{XY}^2}{s_X^2 s_Y^2} = r^2$$
    \end{proof}
\end{theorem}

\section{Predicción}

\subsection{Estimación de las medias condicionadas}
Llamamos $m_0 = E(y | x=x_0) = \beta_0 + \beta_1x_0$.
Observamos que $m_0$ es un parámetro que podemos estimar de la forma:
$$\hat{m_0} = \hat{E(y | x=x_0)} = \hat{\beta_0} + \hat{\beta_1}x_0$$

\begin{theorem}
    $$\hat{m_0} \sim N\left( m_0, \frac{\sigma^2}{n} \left(1+\frac{(x_0-\bar{x})^2}{s_X^2}\right) \right)$$
\end{theorem}

\subsubsection*{Intervalos de confianza para $m_0$}
Podemos calcular los intervalos de confianza para $m_0$ con nivel de confianza de $100(1-\alpha)\%$.

Si $\sigma^2$ es conocida,
$$\left( \hat{m_0} - z_{1-\frac{\alpha}{2}} \frac{\sigma}{\sqrt{n}} \sqrt{1 + \frac{(x_0-\bar{x})^2}{s_X^2}}, \hat{m_0} + z_{1-\frac{\alpha}{2}} \frac{\sigma}{\sqrt{n}} \sqrt{1 + \frac{(x_0-\bar{x})^2}{s_X^2}} \right)$$

Si $\sigma^2$ es desconocida,
$$\left( \hat{m_0} - t_{n-2, 1-\frac{\alpha}{2}} \frac{s_R}{\sqrt{n}} \sqrt{1 + \frac{(x_0-\bar{x})^2}{s_X^2}}, \hat{m_0} + t_{n-2, 1-\frac{\alpha}{2}} \frac{s_R}{\sqrt{n}} \sqrt{1 + \frac{(x_0-\bar{x})^2}{s_X^2}} \right)$$

\subsection{Predicción de una observación futura}
Dado un conjunto de datos $(x_1, y_1), \dots, (x_n, y_n)$ y dado $x_0$ queremos predecir:
$$y_0 = \beta_0 + \beta_1 x_0 + u_0$$
$u_0$ es una variable aleatoria independiente de $u_1, \dots, u_n$ con $u_0 \sim N(0, \sigma^2)$.
Observamos que $y_0$ es una variable aleatoria, a diferencia de la estimación $$\hat{y_0} = \hat{\beta_0} + \hat{\beta_1}x_0 = \hat{m_0}$$

Consideramos el error:
$$e_0 = y_0 - \hat{y_0} = \beta_0 + \beta_1x_0 + u_0 - (\hat{\beta_0} + \hat{\beta_1}x_0)$$
que también es una variable aleatoria.

\begin{theorem}
    $$e_0 \sim \left( 0, \sigma^2 \left(1+\frac{1}{n}+\frac{(x_0-\bar{x})^2}{ns_X^2} \right) \right)$$
\end{theorem}

\subsubsection*{Intervalos de pronóstico para $y_0$}
Podemos calcular los intervalos de pronóstico $IP_{1-\alpha}(y_0)$ para $y_0$ con contenido probabilístico $1-\alpha$.

Si $\sigma^2$ es conocida,
$$\left( \hat{y_0} - z_{1-\frac{\alpha}{2}} \sigma \sqrt{1 + \frac{1}{n} + \frac{(x_0-\bar{x})^2}{ns_X^2}}, \hat{y_0} + z_{1-\frac{\alpha}{2}} \sigma \sqrt{1 + \frac{1}{n} + \frac{(x_0-\bar{x})^2}{ns_X^2}} \right)$$

Si $\sigma^2$ es desconocida,
$$\left( \hat{y_0} - t_{n-2, 1-\frac{\alpha}{2}} s_R \sqrt{1 + \frac{1}{n} + \frac{(x_0-\bar{x})^2}{ns_X^2}}, \hat{y_0} + t_{n-2, 1-\frac{\alpha}{2}} s_R \sqrt{1 + \frac{1}{n} + \frac{(x_0-\bar{x})^2}{ns_X^2}} \right)$$

\section{Análisis de residuos y observaciones atípicas e influyentes}
\subsection*{Residuos}
El residuo de un dato es la diferencia entre su valor y la predicción mediante el modelo.
$$e_i = y_i - \hat{y_i}, \quad \forall i = 1, \dots n$$
El análisis de los residuos puede darnos información sobre el ajuste del modelo.

\subsection*{Observaciones atípicas}
Una observación atípica es un valor que es numéricamente distinto al resto de los datos.
Visualmente, es un dato que se sale del patrón.
Las observaciones atípicas pueden ser indicativas de errores de observación o errores en el modelo.
Un error de observación se debe a datos que pertenecen a una población diferente del resto de muestras, mientras que un error en el modelo puede ser debido a que la muestra depende una variable desconocida que no se han tenido en cuenta.

\subsection*{Observaciones influyentes}
Una observación influyente $(x_A, y_A)$ es una observación atípica cuya exclusión produce un cambio drástico en la recta de regresión.
Puede ser causada por un error de observación o por un modelo incorrecto.
Algunas posibles causas de que el modelo sea incorrecto son:
\begin{itemize}
    \item La relación entre $x$ e $y$ no es lineal cerca de $x_A$.
    \item La varianza aumenta mucho con $x$.
    \item Una variable desconocida ha tomado un valor distinto en $x_A$.
\end{itemize}

\subsection*{Puntos palanca}
Los puntos palanca son observaciones con un valor alto de $p_i$.
Estos tienen la capacidad de alterar en gran medida la recta de regresión.

\section{Transformaciones}
Cuando el diagrama de dispersión entre las dos variables o el de los residuos presenta indicios de incumplimiento de alguna hipótesis básica, entonces hay que abandonar el modelo inicial por uno menos simple o bien aplicar alguna transformación a los datos.

\chapter{Modelo de regresión lineal múltiple}
\section{Modelo e hipótesis}
Sean $X_1, \dots, X_n$ variables explicativas, $Y$ una variable aleatoria continua y $(x_{11}, x_{21}, \dots, x_{k1}, y_1), \dots, (x_{1n}, x_{2n}, \dots, x_{kn}, y_n)$ un conjunto de datos.
Entonces el modelo de regresión lineal múltiple es:
$$y_i = \beta_0 + \beta_1 x_{1i} + \beta_2 x_{2i} + \dots + \beta_k x_{ki} + u_i, \quad i = 1, \dots, n$$

\subsection*{Hipótesis del modelo}
\begin{itemize}
    \item $E(u_i) = 0, \quad \forall i = 1, \dots n$.
    \item $Var(u_i) = \sigma^2, \quad \forall i = 1, \dots n$ (homocedasticidad)
    \item $u_i \sim N(0, \sigma^2), \quad \forall i = 1, \dots n$ (normalidad)
    \item $E(u_i u_j) = 0, \quad \forall i \neq j$ (independencia)
    \item $n > k+1$
    \item No existen relaciones lineales entre los $X_i$ (ausencia de multicolinealidad)
\end{itemize}
El modelo se puede escribir de forma matricial.
Definimos:
$$\vec{y} = \begin{pmatrix}
        y_1 \\ \vdots \\ y_n
    \end{pmatrix} \in \mathbb{R}^n, \quad
    \vec{\beta} = \begin{pmatrix}
        \beta_0 \\ \vdots \\ \beta_k
    \end{pmatrix} \in \mathbb{R}^{k+1}, \quad
    \vec{u} = \begin{pmatrix}
        u_1 \\ \vdots \\ u_n
    \end{pmatrix} \in \mathbb{R}^n$$
$$X = \begin{pmatrix}
        1      & x_{11} & x_{21} & \dots & x_{k1} \\
        1      & x_{12} & x_{22} & \dots & x_{k2} \\
        \vdots & \vdots & \vdots &       & \vdots \\
        1      & x_{1n} & x_{2n} & \dots & x_{kn}
    \end{pmatrix} \in \mathcal{M}_{n \times (k+1)}$$
Entonces el modelo es equivalente a:
$$\vec{y} = X\vec{\beta} + \vec{u}$$
Las hipótesis del modelo se pueden reescribir como:
\begin{itemize}
    \item $\vec{u} \sim N_n(\vec{0}, \sigma^2 I_n)$
    \item $n > k+1$
    \item Ausencia de multicolinealidad.
\end{itemize}
Podemos escribir las mismas hipótesis iniciales en términos de $y_i, \quad \forall 1, \dots, n$.
\begin{enumerate}
    \item $E(y_i | x_{1i}, \dots, x_{ki}) = \beta_0 + \beta_1x_{1i} + \dots + \beta_kx_{ki}, \quad \forall i = 1, \dots n$ (linealidad)
    \item $Var(y_i | x_{1i}, \dots, x_{ki}) = \sigma^2, \quad \forall i = 1, \dots n$ (homocedasticidad)
    \item $y_i | x_{1i}, \dots, x_{ki} \sim N(\beta_0 + \beta_1x_{1i} + \dots + \beta_kx{ki}, \sigma^2), \quad \forall i = 1, \dots n$ (normalidad)
    \item $Cov(y_i, y_j) = 0, \quad \forall i \neq j$ (independencia)
    \item $n > k+1$
    \item No existen relaciones lineales entre los $X_i$ (ausencia de multicolinealidad)
\end{enumerate}
Escritas para el modelo en forma matricial quedan:
\begin{itemize}
    \item $\vec{y} \sim N_n(\vec{x}\vec{\beta}, \sigma^2 I_n)$
    \item $n > k+1$
    \item $rg(X) = k+1$
\end{itemize}

\section{Estimación de los parámetros}
Queremos estimar $\beta_0, \beta_1, \dots, \beta_k$, o análogamente $\hat{\vec{\beta}}$, y $\sigma^2$.
Con los estimadores $\hat{\beta_0}, \hat{\beta_1}, \dots, \hat{\beta_k}$ podemos estimar $$\hat{E(y_i | x_{1i}, \dots, x_{ki})} = \hat{\beta_0} + \hat{\beta_1}x_{1i} + \dots + \hat{\beta_k}x_{ki}, \quad i = 1, \dots n$$.

Procedemos mediante el método de mínimos cuadrados.
La función a minimizar es:
$$M(\beta_0, \dots, \beta_k) = \sum_{i=1}^n (y_i - \beta_0 - \beta_1x_{1i} - \dots - \beta_kx_{ki})^2$$
Planteamos las ecuaciones:
$$\begin{cases}
        \frac{\partial M}{\partial \beta_0} (\beta_0, \dots, \beta_k) = -2\sum_{i=1}^n (y_i - \beta_0 - \beta_1x_{1i} - \dots - \beta_kx_{ki}) \\
        \frac{\partial M}{\partial \beta_k} (\beta_0, \dots, \beta_k) = -2\sum_{i=1}^n x_{ji}(y_i - \beta_0 - \beta_1x_{1i} - \dots - \beta_kx_{ki}), \quad k \geq 1
    \end{cases}$$
Estas son las ecuaciones normales de la regresión.

Resolviendo este sistema, llegamos a que $M$ alcanza el mínimo si:
$$X^t\vec{y} = X^tX\hat{\vec{\beta}}$$

Por la hipótesis 6, $X^tX$ tiene inversa así que podemos escribir:
$$\hat{\vec{\beta}} = (X^tX)^{-1}X^t\vec{y}$$

Para estimar la varianza $\sigma^2$ usaremos la varianza residual:
$$s_R^2 = \frac{e_i^2}{n-k-1}$$

\begin{note}
    $$\frac{\sum_{i=1}^n e_i^2}{\sigma^2} \sim \chi^2_{n-(k+1)}$$
\end{note}

\section{Propiedades de los estimadores}
Sobre el estimador $\hat{\vec{\beta}}$, sabemos que:
$$\begin{cases}
        \hat{\vec{\beta}} = (X^tX)^{-1}X^t\vec{y} \\
        \hat{y} \sim N_n(X\vec{\beta}, \sigma^2 I_n)
    \end{cases} \Rightarrow \hat{\vec{\beta}} \sim N_{k+1}(\vec{\beta}, \sigma^2(X^tX)^{-1})$$

\begin{note}
    $$\begin{cases}
            \vec{x} \sim N_n(\vec{\mu}, \Sigma) \\
            \vec{y} = A\vec{x}
        \end{cases} \Rightarrow \vec{y} \sim N_k(A\vec{\mu}, A\Sigma A^t)$$
\end{note}

En cuanto a $s_R^2$,
$$\begin{cases}
        E(\frac{\sum_{i=1}^n e_i^2}{\sigma^2}) = n-k-1 \\
        Var(\frac{\sum_{i=1}^n e_i^2}{\sigma^2}) = 2(n-k-1)
    \end{cases} \Rightarrow \begin{cases}
        E(s_R^2) = \sigma^2 \\
        Var(s_R^2) = \frac{2(\sigma^2)^2}{n-k-1}
    \end{cases}$$

\end{document}