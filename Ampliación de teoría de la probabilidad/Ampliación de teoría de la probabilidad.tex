\documentclass{report}
\usepackage[spanish]{babel}
\usepackage{amssymb, amsmath, amsthm, mathtools, hyperref}

\title{Ampliación de teoría de la probabilidad}
\author{}

\newtheorem{theorem}{Teorema}[chapter]
\newtheorem{corollary}[theorem]{Corolario}
\newtheorem{lemma}[theorem]{Lema}
\newtheorem{proposition}[theorem]{Proposición}

\theoremstyle{remark}
\newtheorem*{remark}{Observación}
\theoremstyle{remark}
\newtheorem*{note}{Nota}
\theoremstyle{remark}
\newtheorem*{notation}{Notación}

\theoremstyle{definition}
\newtheorem{definition}{Definición}[chapter]
\theoremstyle{definition}
\newtheorem*{properties}{Propiedades}
\theoremstyle{definition}
\newtheorem*{example}{Ejemplo}
\theoremstyle{definition}
\newtheorem*{exercise}{Ejercicio}

\begin{document}
\maketitle
\tableofcontents

\chapter{Función de distribución}
\section{Introducción}
Sea $(\Omega, \mathcal{A}, P)$ espacio de probabilidad, donde $\mathcal{A} \subset P(\Omega)$ es una $\sigma$-álgebra y $P: \mathcal{A} \to [0, 1]$ es una medida de probabilidad.

Una variable aleatoria es una función $X: \Omega \to \mathbb{R}$ tal que
$$X^{-1}(B) = \{ \omega \in \Omega : X(\omega) \in B \} \in \mathcal{A}$$
para todo $B \in \mathcal{B}$ $\sigma$-álgebra de Borel.
$X$ induce una medida en $(\mathbb{R}, \mathcal{B})$:
$$P_X: \mathcal{B} \to [0, 1], \quad P_X(B) = P(X^{-1}(B))$$

\begin{definition}
    Sea $(\Omega, \mathcal{A}, P)$ espacio de probabilidad.
    Sean $X: \Omega \to \mathbb{R}$ una variable aleatoria en $(\Omega, \mathcal{A}, P)$ y $P_X$ la medida de probabilidad inducida por $X$ en $(\mathbb{R}, \mathcal{B})$.
    La función de distribución asociada a $X$ es:
    $$F: \mathbb{R} \to [0, 1], \quad F(a) = P_X((-\infty, a]) = P(X \leq a)$$
\end{definition}

\begin{note}
    Variables aleatorias distintas pueden tener la misma función de distribución.
\end{note}

\section{Propiedades}
Sea $F$ la función de distribución asociada a una variable aleatoria $X$.
Entonces:
\begin{itemize}
    \item $F$ es creciente.
    \item $\lim\limits_{x \to -\infty} F(x) = F(-\infty) = 0, \quad \lim\limits_{x \to \infty} F(x) = F(\infty) = 1$
    \item $F$ es continua por la derecha: $$\lim\limits_{h \to 0^+} F(x+h) = F(x), \quad \forall x \in \mathbb{R}$$
    \item Existe $\lim\limits_{h \to 0^-} F(x+h) = F(x^-) = F(x) - P_X(\{x\})$
\end{itemize}

\begin{theorem}[Teorema de correspondencia]
    Si $F: \mathbb{R} \to [0, 1]$ es una función que verifica:
    \begin{itemize}
        \item $F$ es creciente.
        \item $F(-\infty) = 0, \quad F(\infty) = 1$
        \item $F$ es continua por la derecha.
    \end{itemize}
    Entonces existe una única medida de probabilidad $P_F$ en $(\mathbb{R}, \mathcal{B})$ tal que $F$ es su función de distribución.
    Es decir, tal que $F(a) = P_F((-\infty, a])$.
\end{theorem}

\begin{definition}
    Sea $F$ función de distribución.
    El conjunto de discontinuidad de $F$ se define como:
    $$C(F) = \{ x \in \mathbb{R} : F(x) = F(x^-) \}$$
    También se puede definir el conjunto de puntos de discontinuidad de $F$ como:
    $$D(F) = \{ x \in \mathbb{R} : F(x) - F(x^-) > 0 \}$$
\end{definition}

\begin{remark}
    $D(F) = \bar{C(F)}$.
\end{remark}

\begin{proposition}
    $D(F)$ es a lo sumo numerable.
\end{proposition}

\begin{corollary}
    $C(F)$ es denso en $\mathbb{R}$.
\end{corollary}

\begin{proposition}
    Sean $F$ y $G$ funciones de distribución tales que $F(x) = G(x)$ para todo $x \in E \subset \mathbb{R}$, con $E$ denso en $\mathbb{R}$.
    Entonces $F(x) = G(x)$ para todo $x \in \mathbb{R}$.
\end{proposition}

\begin{definition}
    La función de masa de probabilidad se define como:
    $$p : \mathbb{R} \to [0, 1], \quad p(x) = P_F(\{x\}) = F(x) - F(x^-)$$
\end{definition}

\begin{definition}
    Sea $X$ variable aleatoria con función de distribución $F$ y función de masa $p$.
    Entonces:
    \begin{itemize}
        \item $X$ es discreta cuando $\sum_{x \in D(F)} p(x) = 1$.
        \item $X$ es continua cuando $p(x) = 0$ para todo $x \in \mathbb{R}$.
    \end{itemize}
    En otro caso, $X$ es mixta.
\end{definition}

\begin{definition}
    Sea $X: \Omega \to \mathbb{R}$ variable aleatoria.
    $X$ es singular si existe $B \in \mathcal{B}$ con $m(B) = 0$ tal que $P_X(B) = 1$.
\end{definition}

\begin{remark}
    Las variables aleatorias discretas son singulares.
\end{remark}

\begin{definition}
    Sea $X$ variable aleatoria.
    $X$ es absolutamente continua si para cualquier $B \in \mathcal{B}$ con $m(B) = 0$ se tiene que $P_X(B) = 0$.
\end{definition}

\begin{theorem}
    Sea $F$ función de distribución.
    $F$ es absolutamente continua si y solo si existe una funcoón medible $f$ no negativa y finita tal que
    $$F(b) - F(a) = \int_a^b f(x) dx, \quad \forall a < b$$
    La función $f$ se llama función de densidad.
\end{theorem}

\begin{remark}
    $F$ es continua cuando no hay saltos y absolutamente continua cuando tiene una densidad.
\end{remark}

\begin{theorem}[Mixtura de distribuciones]
    Toda función de distribución $F$ se puede descomponer de la forma:
    $$F = \alpha F_d + (1-\alpha)F_c, \quad 0 \leq \alpha \leq 1$$
    donde $F_d$ es la función de distribución de una variable aleatoria discreta y $F_c$ de una continua.
\end{theorem}

\begin{example}
    Consideramos la función de distribución:
    $$F(x) = \begin{cases}
            0                         & \text{si } x < 0        \\
            \frac{x^2}{16}            & \text{si } 0 \leq x < 2 \\
            \frac{1}{4}               & \text{si } 2 \leq x < 4 \\
            \frac{x}{4} - \frac{5}{8} & \text{si } 4 \leq x < 5 \\
            1 - \frac{5}{4x}          & \text{si } x \geq 5     \\
        \end{cases}$$
    Estudiamos sus puntos de discontinuidad y la probabilidad en ellos.
    $$D(F) = \{4, 5\}, \quad \begin{cases}
            p(4) = F(4) - F(4^-) = \frac{1}{8} \\
            p(5) = F(5) - F(5^-) = \frac{1}{8}
        \end{cases}
        \Rightarrow \alpha = \frac{1}{4}$$
    Luego la función de distribución discreta es:
    $$F_d(x) = \begin{cases}
            0                                             & \text{si } x < 4        \\
            \frac{\frac{1}{8}}{\frac{1}{4}} = \frac{1}{2} & \text{si } 4 \leq x < 5 \\
            1                                             & \text{si } x \geq 5
        \end{cases}$$
    Por último podemos calcular la función de distribución continua:
    $$F_c(x) = \frac{1}{1-\frac{1}{4}} \left( F(x) - \frac{1}{4}F_d(x) \right) = \begin{cases}
            0                           & \text{si } x < 0        \\
            \frac{x^2}{12}              & \text{si } 0 \leq x < 2 \\
            \frac{1}{3}                 & \text{si } 2 \leq x < 4 \\
            \frac{5x}{24} - \frac{1}{6} & \text{si } 4 \leq x < 5 \\
            1 - \frac{5}{3x}            & \text{si } x \geq 5
        \end{cases}$$
\end{example}

\begin{lemma}
    Sea $F$ función de distribución.
    Entonces:
    \begin{itemize}
        \item Existe $F'$ en casi todo punto y es no negativa y finita.
        \item $\int_a^b F'(x) dx \leq F(b) - F(a), \quad a<b$
        \item Siendo $F_{ac}(x) = \int_{-\infty}^x F'(t) dt$ y $F_s(x) = F(x) - F_{ac}(x)$, entonces $F'_{ac}(x) = F'(x)$ en casi todo punto y $F'_s(x) = 0$.
    \end{itemize}
\end{lemma}

\begin{theorem}[Descomposición de Lebesgue]
    Toda función de distribución $F$ se puede descomponer de la forma:
    $$F = \beta F_{ac} + (1-\beta)F_s$$
    con $F_{ac}$ función de distribución absolutamente continua y $F_s$ singular.
\end{theorem}

\begin{remark}
    Se pueden aplicar ambas descomposiciones (continua-discreta y Lebesgue) a una función de distribución $F$.
    $$F = \beta F_{ac} + (1-\beta) \left( \alpha F_{d} + (1-\alpha)F_{cs} \right)$$
\end{remark}

\begin{definition}[Esperanza]
    Sea $X$ variable aleatoria en $(\Omega, \mathcal{A}, P)$.
    La esperanza de $X$ se define como:
    $$E(X) = \int_\Omega X dP = \int_\mathbb{R} x dF(x)$$
\end{definition}

\begin{remark}
    \hfill
    \begin{itemize}
        \item Si $F$ es absolutamente continua,
              $$E(X) = \int_\mathbb{R} xf(x) dx$$
        \item Si $F$ es discreta,
              $$E(X) = \sum_{x \in D(F)} xp(x)$$
    \end{itemize}
\end{remark}

\section{Convolución en funciones de distribución}

\begin{definition}
    Sean $F$ y $G$ funciones de distribución.
    Definimos la convolución de $F$ y $G$ como la función definida por:
    $$(F \ast G)(z) = \int_\mathbb{R} F(z-y) dG(y), \quad z \in R$$
\end{definition}

\begin{note}
    La convolución es conmutativa con funciones medibles no negativas.
\end{note}

\begin{proposition}
    $F \ast G$ es una función de distribución.
\end{proposition}

\begin{theorem}
    Sean $X$ e $Y$ variables aleatorias independientes con funciones de distribución $F_X$ y $F_Y$ respectivamente.
    Entonces $F_X \ast F_Y$ es la función de distribución de la variable aleatoria $X+Y$.
\end{theorem}

\begin{theorem}
    Si $F$ es absolutamente continua con densidad $f$, entonces $F \ast G$ es absolutamente continua con densidad
    $$(f \ast G)(z) = \int_{-\infty}^\infty f(z-y)dG(y)$$
\end{theorem}

\begin{theorem}
    Si $F$ y $G$ son absolutamente continuas con densidades $f$ y $g$ respectivamente, entonces $F \ast G$ es absolutamente continua con densidad $f \ast g$.
\end{theorem}

\begin{example}
    Sean $X, Y \sim U([0, 1])$. Sus funciones de distribución son:
    $$F_X(x) = \begin{cases}
            0 & \text{si } x < 0        \\
            x & \text{si } 0 \leq x < 1 \\
            1 & \text{si } x \geq 1
        \end{cases}, \quad
        F_Y(y) = \begin{cases}
            0 & \text{si } y < 0        \\
            y & \text{si } 0 \leq y < 1 \\
            1 & \text{si } y \geq 1
        \end{cases}$$
    Sea $Z = X+Y$ y $z \in \mathbb{R}$.
    Como $X$ e $Y$ son absolutamente continuas, $Z$ es absolutamente continua.
    Queremos calcular:
    $$F_Z(z) = (F_X \ast F_Y)(z) = \int_\mathbb{R} F_X(z-y)dF_y(y) = \int_\mathbb{R} F_X(z-y) f_Y(y) dy$$
    Consideramos todos los casos:
    \begin{itemize}
        \item Si $z < 0$ entonces $z-y<0$ para todo $y \in [0, 1]$.
              Luego $F_X(z-y) = 0$, así que $F_Z(z) = 0$.
        \item Si $0 \leq z < 1$ distinguimos dos casos:
              \begin{itemize}
                  \item Si $0 \leq y < z$ entonces $0 < z-y < 1$, así que $F_X(z-y) = z-y$.
                  \item Si $z \leq y < 1$ entonces $z-y < 0$, luego $F_X(z-y) = 0$.
              \end{itemize}
              $$F_Z(z) = \int_0^z (z-y) dy + \int_z^1 0 dy = \frac{z^2}{2}$$
        \item Si $1 \leq z < 2$ de nuevo distinguimos dos casos:
              \begin{itemize}
                  \item Si $0 \leq y < z-1$ entonces $z-y \geq 1$, luego $F_X(z-y) = 1$.
                  \item Si $z-1 \leq y < 1$ entonces $0 \leq z-1 < 1$, así que $F_X(z-y) = z-y$.
              \end{itemize}
              $$F_Z(z) = \int_0^{z-1} 1 dy + \int_{z-1}^1 (z-y) dy = 2z - \frac{z^2}{2} - 1$$
        \item Si $z \geq 2$ entonces $z-y \geq 1$ para todo $y \in [0, 1]$.
              Luego $F_X(z-y) = 1$, de forma que $F_Z(z) = \int_0^1 1 dx = 1$.
    \end{itemize}
    Por tanto:
    $$F_Z(z) = \begin{cases}
            0                      & \text{si } z < 0        \\
            \frac{z^2}{2}          & \text{si } 0 \leq z < 1 \\
            2z - \frac{z^2}{2} - 1 & \text{si } 1 < z \leq 2 \\
            1                      & \text{si } x \geq 2
        \end{cases}$$
\end{example}

\begin{example}
    Sea $X \sim U([-1, 1])$ y sea $Y$ absolutamente continua con densidad:
    $$f_Y(y) = \begin{cases}
            \frac{y+2}{4} & \text{si } -2 \leq y < 0 \\
            \frac{2-y}{4} & \text{si } 0 \leq y < 2  \\
            0             & \text{en el resto}
        \end{cases}$$
    Sabemos que $X$ es absolutamente continua con función de densidad:
    $$f_X(x) = \begin{cases}
            \frac{1}{2} & \text{si } -1 \leq x \leq 1 \\
            0           & \text{en el resto}
        \end{cases}$$
    Sea $Z = X+Y$.
    Como $X$ e $Y$ son absolutamente continuas, $Z$ es absolutamente continua con función de densidad $f \ast g$.
    $$(f_X \ast f_Y)(z) = (f_Y \ast f_X)(z) = \int_{-\infty}^\infty f_Y(z-x) f_X(x) dx$$
    Sabemos que $S_X = [-1 ,1]$ y $S_Y = [-2, 2]$, así que $S_Z = [-3, 3]$.
    Consideramos los casos:
    \begin{itemize}
        \item Si $z < -3$ entonces $z-x < 2$ para todo $x \in [-1, 1]$.
              Luego $f_Y(z-x) = 0$, así que $f_Z(z) = 0$.
        \item Si $-3 \leq z < -1$ distinguimos dos casos:
              \begin{itemize}
                  \item Si $-1 \leq x < z+2$ entonces $-2 \leq z-x < 0$, así que $f_Y(z-x) = \frac{z-x+2}{4}$.
                  \item Si $z+2 \leq x < 1$ entonces $z-x < -2$, luego $f_Z(z) = 0$.
              \end{itemize}
              $$f_Z(z) = \int_{-1}^{z+2} \frac{z-x+2}{4} \frac{1}{2} dx = \frac{(z+3)^2}{16}$$
        \item Si $-1 \leq z < 1$ distinguimos dos casos:
              \begin{itemize}
                  \item Si $-1 \leq x < z$ entonces $0 \leq z-x < 2$, así que $f_Y(z-x) = \frac{2-z+x}{4}$.
                  \item Si $z \leq x < 1$ entonces $-2 \leq z-x < 0$, luego $f_Y(z-x) = \frac{z-x+2}{4}$.
              \end{itemize}
              $$f_Z = \int_{-1}^z \frac{2-z+x}{4} \frac{1}{2} dx + \int_z^1 \frac{z-x+2}{4} \frac{1}{2} dx = \frac{3-z^2}{8}$$
        \item Si $1 \leq z < 3$ distinguimos dos casos:
              \begin{itemize}
                  \item Si $-1 \leq x < z-2$ entonces $z-x \geq 2$, luego $f_Z(z) = 0$.
                  \item Si $z-2 \leq x < 1$ entonces $0 \leq z-x < 2$, así que $f_Y(z-x) = \frac{2-z+x}{4}$.
              \end{itemize}
              $$f_Z(z) = \int_{z-2}^1 \frac{2-z+x}{4} \frac{1}{2} dx = \frac{(z-3)^2}{16}$$
        \item Si $z \geq 3$ entonces $z-x \geq 2$ para todo $x \in [-1, 1]$, así que $f_Z(z) = 0$.
    \end{itemize}
    Por tanto:
    $$f_Z(z) = \begin{cases}
            \frac{(z+3)^2}{16} & \text{si } -3 \leq z < -1 \\
            \frac{3-z^2}{8}    & \text{si } -1 \leq z < 1  \\
            \frac{(z-3)^2}{16} & \text{si } 1 \leq z < 3   \\
            0                  & \text{en el resto}
        \end{cases}$$
\end{example}

\begin{example}
    Sean $X, Y$ variables aleatorias con funciones de distribución:
    $$F_X(x) = \begin{cases}
            0             & \text{si } x < -2        \\
            \frac{x+2}{4} & \text{si } -2 \leq x < 1 \\
            1             & \text{si } x \geq 1
        \end{cases}, \quad
        F_Y(y) = \begin{cases}
            0           & \text{si } y < 0        \\
            \frac{y}{2} & \text{si } 0 \leq y < 2 \\
            1           & \text{si } y \geq 2
        \end{cases}$$
    Observamos que $Y$ es absolutamente continua y $X$ es mixta, así que $Z = X+Y$ es absolutamente continua.
    Queremos calcular la función de densidad de $Z$.
    Como $F_X$ es discontinua en 1,
    $$f_Z(z) = \int_{-\infty}^\infty f_Y(z-x) dF_x(x) = \int_{-2}^1 f_Y(z-x) f_X(x) dx + f_Y(z-1)p_x(1)$$
    \begin{note}
        Para no lidiar con discontinuidades, también se podría calcular:
        $$f_Z(z) = \int_{-\infty}^\infty f_X(z-y) dF_y(y) = \int_{-\infty}^\infty f_X(z-y) f_Y(y) dy$$
    \end{note}
    Calculamos las funciones de densidad, pseudodensidad y masa:
    $$f_Y(y) = \begin{cases}
            \frac{1}{2} & \text{si } 0 \leq y \leq 2 \\
            0           & \text{en el resto}
        \end{cases}$$
    $$f_X(x) = \begin{cases}
            \frac{1}{4} & \text{si } -2 \leq x < 1 \\
            0           & \text{en el resto}
        \end{cases}, \quad
        p_X(x) = \begin{cases}
            \frac{1}{4} & \text{si } x = 1   \\
            0           & \text{en el resto}
        \end{cases}$$
    $S_X = [-2, 1]$ y $S_Y = [0, 2]$, así que $S_Z = [-2, 3]$.
    Consideramos los casos:
    \begin{itemize}
        \item Si $z < -2$ entonces $z-x < 0$ para todo $x \in [-2, 1]$.
              Luego $f_Y(z-x) = 0$, así que $f_Z(z) = 0$.
        \item Si $-2 \leq z < 0$ entonces $0 \leq z-x \leq 2$, así que $f_Y(z-x) = \frac{1}{2}$ y $f_X(x) = \frac{1}{4}$.
              $$f_Z(z) = \int_{-2}^z \frac{1}{2} \frac{1}{4} dx = \frac{z+2}{8}$$
        \item Si $0 \leq z \leq 1$, $f_Y(z-1) = 0$.
              Distinguimos tres casos:
              \begin{itemize}
                  \item Si $-2 \leq x < z-2$ entonces $z-x > 2$.
                        Luego $f_Y(z-x) = 0$, así que $f_Z(z) = 0$.
                  \item Si $z-2 \leq x < z$ entonces $0 \leq z-x < 2$.
                        Así que $f_Y(z-x) = \frac{1}{2}$ y $f_X(x) = \frac{1}{4}$.
                  \item Si $z \leq x \leq 1$ entonces $z-x \leq 0$.
                        Luego $f_Y(z-x) = 0$, así que $f_Z(z) = 0$.
              \end{itemize}
              $$f_Z(z) = \int_{-2}^{z-2} 0 dx + \int_{z-2}^z \frac{1}{2} \frac{1}{4} dx + \int_z^1 0 dx + 0p_x(1) = \frac{1}{4}$$
        \item Si $1 < z \leq 3$, $f_Y(z-1) = \frac{1}{2}$ y $p_X(1) = \frac{1}{4}$.
              Distinguimos dos casos:
              \begin{itemize}
                  \item Si $-2 \leq x < z-2$ entonces $z-x \geq 2$, así que $f_Y(z-x) = 0$.
                  \item Si $z-2 \leq x < 1$ entonces $0 \leq z-x < 2$, luego $f_Y(z-x) = \frac{1}{2}$ y $f_X(x) = \frac{1}{4}$.
              \end{itemize}
              $$f_Z(z) = \int_{z-2}^1 \frac{1}{2} \frac{1}{4} dx + \frac{1}{2} \frac{1}{4} = \frac{4-z}{8}$$
        \item Si $z \geq 3$ entonces $f_Y(z-x) = 0$, así que $f_Z(z) = 0$.
    \end{itemize}
    Por tanto:
    $$f_Z(z) = \begin{cases}
            \frac{z+2}{8} & \text{si } -2 \leq z < 0   \\
            \frac{1}{4}   & \text{si } 0 \leq z \leq 1 \\
            \frac{4-z}{8} & \text{si } 1 < z \leq 3    \\
            0             & \text{en el resto}
        \end{cases}$$
\end{example}

\begin{exercise}
    Sean $X$ y $Y$ variables aleatorias con funciones de distribución:
    $$F_X(x) = \begin{cases}
            0             & \text{si } x < -1        \\
            \frac{x+1}{3} & \text{si } -1 \leq x < 1 \\
            1             & \text{si } x > 1
        \end{cases}, \quad
        F_Y(y) = \begin{cases}
            0            & \text{si } y < 0        \\
            \frac{2y}{7} & \text{si } 0 \leq y < 2 \\
            \frac{5}{7}  & \text{si } 2 \leq y < 4 \\
            1            & \text{si } y \geq 4
        \end{cases}$$
    Observamos que $X$ es una variable aleatoria mixta con funciones de pseudodensidad y de masa:
    $$f_X(x) = \begin{cases}
            \frac{1}{3} & \text{si } -1 \leq x < 1 \\
            0           & \text{en el resto}
        \end{cases}, \quad
        p_X(x) = \begin{cases}
            \frac{1}{3} & \text{si } x = 1   \\
            0           & \text{en el resto}
        \end{cases}$$
    $Y$ también es mixta con pseudodensidad y masa:
    $$f_Y(y) = \begin{cases}
            \frac{2}{7} & \text{si } 0 \leq y < 2 \\
            0           & \text{en el resto}
        \end{cases}, \quad
        p_Y(y) = \begin{cases}
            \frac{1}{7} & \text{si } y = 2   \\
            \frac{2}{7} & \text{si } y = 4   \\
            0           & \text{en el resto}
        \end{cases}$$
    Sea $Z = X+Y$, queremos calcular $F_Z$.
    Observamos que $S_X = [-1,1) \cup \{1\} = [-1,1]$ y $S_Y = [0,2) \cup \{2\} \cup \{4\} = [0,2] \cup \{4\}$.
    Por tanto, $S_Z = [-1, 5]$.
    Calculamos:
    $$F_Z = \int_{-\infty}^\infty F_Y(z-x)dF_x(x) = \int_{-1}^1 F_Y(z-x)f_X(x)dx + F_Y(z-1)p_X(1)$$
    Consideramos los casos:
    \begin{itemize}
        \item Si $z < -1$, $F_Z(z) = 0$.
        \item Si $-1 \leq z < 1$, consideramos tres casos:
              \begin{itemize}
                  \item Si $-1 \leq x < z$ entonces $0 \leq z-x < 2$, luego $F_Y(z-x) = \frac{2(z-x)}{7}$.
                  \item Si $z \leq x < 1$ entonces $z-x < 0$, así que $F_Z(z) = 0$.
                  \item Si $x = 1$ entonces $z-1 < 0$, luego $F_Z(z) = 0$.
              \end{itemize}
              $$F_Z(z) = \int_{-1}^z \frac{2(z-x)}{7}\frac{1}{3}dx = \frac{(z+1)^2}{21}$$
        \item Si $1 \leq z < 3$, consideramos tres casos:
              \begin{itemize}
                  \item Si $-1 \leq x < z-2$ entonces $2 \leq z-x < 4$, así que $F_Y(z-x) = \frac{5}{7}$.
                  \item Si $z-2 \leq x < 1$ entonces $0 \leq z-x < 2$, luego $F_Y(z-x) = \frac{2(z-x)}{7}$.
                  \item Si $x = 1$ entonces $0 \leq z-1 < 2$, así que $F_Y(z-1) = \frac{2(z-1)}{z}$.
              \end{itemize}
              $$F_Z(z) = \int_{-1}^{z-2} \frac{5}{7}\frac{1}{3}dx + \int_{z-2}^1 \frac{2(z-x)}{7}\frac{1}{3} + \frac{2(z-1)}{7}\frac{1}{3} = \frac{-z^2+9z-4}{21}$$
        \item Si $3 \leq z < 5$, consideramos tres casos:
              \begin{itemize}
                  \item Si $-1 \leq x < z-4$ entonces $z-x \geq 4$, luego $F_Y(z-x) = 1$.
                  \item Si $z-4 \leq x < 1$ entonces $2 \leq z-x < 4$, así que $F_Y(z-x) = \frac{5}{7}$.
                  \item Si $x = 1$ entonces $2 \leq z-1 < 4$, luego $F_Y(z-x) = \frac{5}{7}$.
              \end{itemize}
              $$F_Z(z) = \int_{-1}^{z-4} 1 \frac{1}{3}dx + \int_{z-4}^1 \frac{5}{7}\frac{1}{3}dx + \frac{5}{7}\frac{1}{3} = \frac{2z+9}{21}$$
        \item Si $z \geq 5$, $F_Z(z) = 1$.
    \end{itemize}
    Por tanto:
    $$F_Z(z) = \begin{cases}
            0                    & \text{si } z < -1        \\
            \frac{(z+1)^2}{21}   & \text{si } -1 \leq z < 1 \\
            \frac{-z^2+9z-4}{21} & \text{si } 1 \leq z < 3  \\
            \frac{2z+9}{21}      & \text{si } 3 \leq z < 5  \\
            1                    & \text{si } z \geq 5
        \end{cases}$$
\end{exercise}

\section{Convergencia en distribución}

Sea $n \in \mathbb{N}$ y sea $X_n$ una variable aleatoria en $(\Omega_n, \mathcal{A}_n, P_n)$.
$\{X_n\}_n$ tiene una sucesión asociada $\{F_n\}_n$ de funciones de distribución.

\begin{definition}
    Sean $F$ y $F_n$ funciones de distribución.
    Decimos que la sucesión $\{F_n\}_n$ converge a $F$ débilmente cuando
    $$\lim\limits_{n \to \infty} F_n(x) = F(x), \quad \forall x \in C(F)$$
    Se escribe $F_n \xrightarrow{d} F$.
\end{definition}

\begin{example}
    Sea $X_n \sim \delta(\frac{1}{n})$, es decir, $P(X_n = \frac{1}{n}) = 1$.
    Su función de distribución es:
    $$F_n(x) = \begin{cases}
            0 & \text{si } x < \frac{1}{n}    \\
            1 & \text{si } x \geq \frac{1}{n}
        \end{cases}$$
    Calculamos el límite puntual de la sucesión:
    $$F(x) = \lim\limits_{n \to \infty} F_n(x) = \begin{cases}
            0 & \text{si } x \leq 0 \\
            1 & \text{si } x > 0
        \end{cases}$$
    Aunque $\{F_n\}_n$ converge puntualmente a $F$, observamos que $F$ no es continua por la derecha.
    Así que $F$ no es función de distribución y no puede ser el límite débil de la sucesión.
    Definimos:
    $$G(x) = \begin{cases}
            0 & \text{si } x < 0    \\
            1 & \text{si } x \geq 0
        \end{cases}$$
    $G$ es función de distribución y además $\lim\limits_{n \to \infty} F_n(x) = G(x)$ para todo $x \in C(G) = \mathbb{R} \setminus \{0\}$, luego $F_n \xrightarrow{d} G$.
    Es función de distribución de una variable aleatoria $\delta(0)$.
\end{example}

\begin{example}
    Sea $Y_n \sim \delta(-\frac{1}{n})$.
    Procedemos de forma análoga al ejemplo anterior.
    Su función de distribución es:
    $$F_{Y_n}(y) = \begin{cases}
            0 & \text{si } x < -\frac{1}{n}    \\
            1 & \text{si } x \geq -\frac{1}{n}
        \end{cases}$$
    y su límite puntual es:
    $$F_Y(y) = \lim\limits_{n \to \infty} F_{Y_n}(y) = \begin{cases}
            0 & \text{si } x < 0    \\
            1 & \text{si } x \geq 0
        \end{cases}$$
    En este caso el límite puntual $F_Y$ sí es función de distribución, así que el límite puntual coincide con el límite débil.
    $$F_{Y_n} \xrightarrow{d} F_Y$$
\end{example}

\begin{theorem}
    El límite débil de una sucesión de funciones de distribución es único en caso de existir.
\end{theorem}

\begin{example}
    Sea $X_n \sim U([-\frac{1}{n}, \frac{1}{n}])$.
    Su función de distribución es:
    $$F_n(x) = \begin{cases}
            0              & \text{si } x < -\frac{1}{n}                  \\
            \frac{nx+1}{2} & \text{si } -\frac{1}{n} \leq x < \frac{1}{n} \\
            1              & \text{si } x \geq \frac{1}{n}
        \end{cases}$$
    El límite puntual es:
    $$F(x) = \begin{cases}
            0           & \text{si } x < 0 \\
            \frac{1}{2} & \text{si } x = 0 \\
            1           & \text{si } x > 0
        \end{cases}$$
    $F$ no es función de distribución porque no es continua por la derecha en $x = 0$.
    Definimos entonces:
    $$G(x) = \begin{cases}
            0 & \text{si } x < 0    \\
            1 & \text{si } x \geq 0
        \end{cases}$$
    $G$ es función de distribución y $\lim\limits_{n \to \infty} F_n(x) = G(x)$ para todo $x \in C(G)$, así que $F_n \xrightarrow{d} G$.
\end{example}

\begin{example}
    Consideramos la sucesión de funciones de distribución $\{F_n\}_n$, con
    $$F_n(x) = \begin{cases}
            0                          & \text{si } x < 0        \\
            \frac{1}{2} + \frac{x}{2n} & \text{si } 0 \leq x < n \\
            1                          & \text{si } x \geq n
        \end{cases}$$
    Su límite puntual es:
    $$F(x) = \begin{cases}
            0           & \text{si } x < 0    \\
            \frac{1}{2} & \text{si } x \geq 0
        \end{cases}$$
    Podemos descomponer $F_n$ como mixtura de distribuciones de la forma $F_n = \alpha F_n^d + (1-\alpha)F_n^c$.
    Calculamos el valor de $\alpha$:
    $$\alpha = \sum_{x \in D(F_n)} p(x) = p(0) = \frac{1}{2}$$
    Por tanto, las funciones de distribución son:
    $$F_n^d(x) = \frac{1}{\alpha} \sum_{t \leq x, t \in D(F_n)} p(t) = \begin{cases}
            0 & \text{si } x < 0    \\
            1 & \text{si } x \geq 0
        \end{cases} \xrightarrow[n \to \infty]{}
        F^d(x) = \begin{cases}
            0 & \text{si } x < 0    \\
            1 & \text{si } x \geq 0
        \end{cases}$$
    $$F_n^c(x) = \frac{1}{1-\alpha}(F_n(x) - \alpha F_n^d(x)) = \begin{cases}
            0           & \text{si } x < 0        \\
            \frac{x}{n} & \text{si } 0 \leq x < n \\
            1           & \text{si } x \geq n
        \end{cases} \xrightarrow[n \to \infty]{}
        F^c(x) = 0$$
    Observamos que $F = \alpha F^d + (1-\alpha)F^c$.
\end{example}

\begin{definition}
    La sucesión de variables aleatorias $\{X_n\}_n$ converge en distribución a otra variable aleatoria $X$ cuando $F_n \xrightarrow{d} F$, siendo $F_n$ y $F$ las funciones de distribución asociadas a $X_n$ y $X$, respectivamente.
    Se escribe $X_n \xrightarrow{d} X$.
\end{definition}

\begin{exercise}
    Sea $(\Omega, \mathcal{A}, P)$ espacio de probabilidad, donde:
    $$\Omega = [0, 3], \quad \mathcal{A} = \{ B \cap [0, 3] : B \in \mathcal{B} \}$$
    $$P: \mathcal{A} \to [0, 1], \quad P(A) = \begin{cases}
            \frac{1}{6}    & \text{si } A = \{0\}        \\
            0              & \text{si } A \subset (0, 1) \\
            \frac{m(A)}{6} & \text{si } A \subset [1, 3) \\
            \frac{1}{2}    & \text{si } A = \{3\}
        \end{cases}$$
    Sobre este espacio definimos para $n \in \mathbb{N}$ las variables aleatorias:
    $$X_n: \Omega \to \mathbb{R}, \quad X_n(\omega) = \begin{cases}
            \frac{\omega - 1}{n\omega + 1} & \text{si } 0 \leq \omega < 1 - \frac{1}{n}               \\
            2                              & \text{si } 1 - \frac{1}{n} \leq \omega < 3 - \frac{1}{n} \\
            n(\omega - 3)                  & \text{si } 3 - \frac{1}{n} \leq \omega \leq 3
        \end{cases}$$
    $\{X_n\}_n$ es una sucesión de variables aleatorias.

    \begin{enumerate}
        \item \textbf{Determinar los puntos de discontinuidad y sus masas.}
              Sabemos que $x$ es punto de discontinuidad de $F_n$ si $F_n(x) - F_n(x^-) > 0$, es decir, $P_{X_n}(\{x\}) > 0$.
              En primer lugar estudiamos las imágenes por $X_n$ de los puntos con masa en la definición de $P$.
              En este caso, estos puntos son 0 y 3, con imágenes $X_n(0) = -1$ y $X_n(3) = 0$.
              \begin{align*}
                  P_{X_n}(-1) & = P(X_n^{-1}(-1)) = P(\{0\} \cup \{3-\frac{1}{n}\}) = \frac{1}{6} \\
                  P_{X_n}(0)  & = P(X_n^{-1}(0)) = P(\{3\}) = \frac{1}{2}
              \end{align*}
              Además, estudiamos los puntos cuya imagen inversa es un intervalo, es decir, donde $X_n$ es constante.
              $$P_{X_n}(2) = P([1-\frac{1}{n}, 3-\frac{1}{n})) = P([1, 3-\frac{1}{n})) = \frac{2n-1}{6n}$$
              Así que $D = \{-1, 0, 2\}$.

        \item \textbf{Calcular la función de distribución $F_n$ asociada a $X_n$.}
              Recordamos que la función de distribución $F_n$ asociada a una variable aleatoria $X_n$ se define como:
              $$F_n(x) = P(X_n \leq x) = P(\{ \omega \in \Omega : X(\omega) \leq x \})$$
              Distringuimos varios casos:
              \begin{itemize}
                  \item Si $x < -1$, $F_n(x) = P(\emptyset) = 0$.
                  \item Si $-1 \leq x < -\frac{1}{n^2}$,
                        $$F_n(x) = P(X_n \leq x) = P([0, \alpha] \cup [3-\frac{1}{n}, \beta])$$
                        donde:
                        \begin{align*}
                            X_n(\alpha) & = x \Leftrightarrow \frac{\alpha-1}{n\alpha + 1} = x \Leftrightarrow \alpha = \frac{x+1}{1-nx} \\
                            X_n(\beta)  & = x \Leftrightarrow n(\beta - 3) = x \Leftrightarrow \beta = \frac{x+3n}{n}
                        \end{align*}
                        Así que
                        $$F_n(x) = P([0, \frac{x+1}{1-nx}] \cup [3-\frac{1}{n}, \frac{x+3n}{n}]) = \frac{x+1+n}{6}$$
                  \item Si $-\frac{1}{n^2} \leq x < 0$,
                        $$F_n(x) =  P(\{0\}) + P([0, 1-\frac{1}{n})) + P([3-\frac{1}{n}, \frac{x+3n}{n}]) = \frac{x+1+n}{6}$$
                  \item Si $0 \leq x < 2$,
                        $$F_n(x) =  P(\{0\}) + P([0, 1-\frac{1}{n})) + P([3-\frac{1}{n}, 3)) + P(\{3\}) = \frac{4n+1}{6n}$$
                  \item Si $X \geq 2$, $F_n(x) = 1$.
              \end{itemize}
              Por tanto,
              $$F_n(x) = \begin{cases}
                      0               & \text{si } x < -1        \\
                      \frac{x+1+n}{6} & \text{si } -1 \leq x < 0 \\
                      \frac{4n+1}{6n} & \text{si } 0 \leq x < 2  \\
                      1               & \text{si } x \geq 2
                  \end{cases}$$

        \item \textbf{Analizar la convergencia en distribución de $\{X_n\}$}
              Sabemos que $X_n \xrightarrow{d} X$ cuando $F_n \xrightarrow{d} F$.
              Tomamos el límite puntual en $\{F_n\}$.
              $$F(x) = \lim\limits_{n \to \infty} F_n(x) = \begin{cases}
                      0           & \text{si } x < -1        \\
                      \frac{1}{6} & \text{si } -1 \leq x < 0 \\
                      \frac{2}{3} & \text{si } 0 \leq x < 2  \\
                      1           & \text{si } x \geq 2
                  \end{cases}$$
              $F$ es continua por la derecha y $D(F) = \{-1, 0, 2\}$ con $p_X(-1) = \frac{1}{6}$, $p_X(0) = \frac{1}{2}$ y $p_X(2) = \frac{1}{3}$.
              Observamos que estas masas coinciden con los límites cuando $n \to \infty$ de las masas de los puntos de discontinuidad de $X_n$.
              Por tanto, $F_n \xrightarrow{d} F$.
    \end{enumerate}
\end{exercise}

\begin{lemma}
    Sean $F_n$ y $F$ funciones de distribución.
    $F_n \xrightarrow{d} F$ si y solo si
    $$\limsup\limits_{n \to \infty} F_n(x) \leq F(x) \quad \text{y} \quad \liminf\limits_{n \to \infty} F_n(x) \geq F(x^-) \quad \forall x \in \mathbb{R}$$
\end{lemma}

\begin{theorem}[Helly-Bray]
    Sean $F_n$ y $F$ funciones de distribución.
    $F_n \xrightarrow{d} F$ si y solo si
    $$\lim\limits_{n \to \infty} \int_\mathbb{R} g(x)dF_n(x) = \int_\mathbb{R} g(x)dF(x)$$
    para toda función $g$ real, continua y acotada.
\end{theorem}

\begin{remark}
    En general, el teorema de Helly-Bray no implica que $E(X_n) \xrightarrow[n \to \infty]{} E(X)$ porque $g(x) = x$ no siempre está acotada.
\end{remark}

\begin{example}
    Sean $n \in \mathbb{N}$, consideramos la función de distribución:
    $$F_n(x) = \begin{cases}
            0                          & \text{si } x < -1                 \\
            \frac{x+1}{2n}             & \text{si } -1 \leq x < 0          \\
            \frac{1}{2n} + \frac{1}{4} & \text{si } 0 \leq x < \frac{2}{n} \\
            \frac{x}{4} + \frac{1}{2}  & \text{si } \frac{2}{n} \leq x < 2 \\
            1                          & \text{si } x \geq 2
        \end{cases}$$
    Su límite puntual es:
    $$G(x) = \lim\limits_{n \to \infty} F_n(x) = \begin{cases}
            0                         & \text{si } x < 0     \\
            \frac{1}{4}               & \text{si } x = 0     \\
            \frac{x}{4} + \frac{1}{2} & \text{si } 0 < x < 2 \\
            1                         & \text{si } x \geq 2
        \end{cases}$$
    Observamos que $G$ no es función de distribución.
    Definimos entonces:
    $$F(x) = \begin{cases}
            0                         & \text{si } x < 0        \\
            \frac{x}{4} + \frac{1}{2} & \text{si } 0 \leq x < 2 \\
            1                         & \text{si } x \geq 2
        \end{cases}$$
    $F$ es función de distribución así que $F_n \xrightarrow{d} F$.

    Sea $g(x) = I_{(0,2]}(x)$, veamos si $E(g(X_n)) \xrightarrow[n \to \infty]{} E(g(X))$.
    \begin{align*}
        E(g(X_n)) & = E(I_{(0,2]}(X_n)) = \int_\mathbb{R} I_{(0,2]}(x) dF_n(x) = \int_0^2 dF_n(x) =         \\
                  & = F_n(2) - F_n(0) = \frac{3}{4} - \frac{1}{2n} \xrightarrow[n \to \infty]{} \frac{3}{4} \\
        E(g(X))   & = \int_0^2 dF(x) = F(2) - F(0) = 1 - \frac{1}{2} = \frac{1}{2}
    \end{align*}
    Observamos que $E(g(X_n))$ no tiende a $E(g(X))$ con $n \to \infty$.
    No se cumplen las hipótesis del teorema de Helly-Bray porque $g$ no es continua.

    Sea ahora $g(x) = x$.
    Queremos calcular $E(g(X_n)) = \int_\mathbb{R} x dF_n(x)$.
    Como $F_n$ es mixta, hallamos primero las masas de sus puntos de discontinuidad y su función de pseudodensidad:
    $$D(F_n) = \{0, \frac{2}{n}\}, \quad p(0) = \frac{1}{4}, \quad p(\frac{2}{n}) = \frac{1}{4}$$
    $$f_n(x) = F_n'(x) = \begin{cases}
            \frac{1}{2n} & \text{si } -1 \leq x < 0          \\
            \frac{1}{4}  & \text{si } \frac{2}{n} \leq x < 2 \\
            0            & \text{en el resto}
        \end{cases}$$
    $$E(g(X_n)) = 0 \frac{1}{4} + \frac{2}{n} \frac{1}{4} + \int_{-1}^0 \frac{x}{2n}dx + \int_\frac{2}{n}^2 \frac{x}{4}dx = \frac{1}{2} + \frac{1}{4n} - \frac{1}{2n^2} \xrightarrow[n \to \infty]{} \frac{1}{2}$$
    Procedemos de forma análoga para $F$.
    $$D(F) = \{0\}, \quad p(0) = \frac{1}{2}, \quad
        f(x) = \begin{cases}
            \frac{1}{4} & \text{si } 0 \leq x < 2 \\
            0           & \text{en el resto}
        \end{cases}$$
    $$E(g(X)) = 0 \frac{1}{2} + \int_0^2 \frac{x}{4}dx = \frac{1}{2}$$
    Luego en este caso $E(g(X_n)) \xrightarrow[n \to \infty]{} E(g(X))$.
    Se verifica el teorema de Helly-Bray porque $g(x) = x$ está acotada en los soportes de $f_n$ y $f$, que son acotados.
\end{example}

\begin{example}
    Sea $n \in \mathbb{R}$, consideramos la función de distribución:
    $$F_n(x) = \begin{cases}
            0             & \text{si } x < 0        \\
            \frac{n-1}{n} & \text{si } 0 \leq x < n \\
            1             & \text{si } x \geq n
        \end{cases}$$
    Es claro que:
    $$F_n \xrightarrow{d} \delta(0) = \begin{cases}
            0 & \text{si } x < 0    \\
            1 & \text{si } x \geq 0
        \end{cases}$$
    Sea $g(x) = x$, procedemos igual que en el ejemplo anterior.
    $$D(F_n) = \{0, n\}, \quad p(0) = \frac{n-1}{n}, \quad p(n) = \frac{1}{n}$$
    \begin{align*}
        E(g(X_n)) & = 0 \frac{n-1}{n} + n \frac{1}{n} = 1 \xrightarrow[n \to \infty]{} 1 \\
        E(g(X))   & = E(\delta(0)) = 0 \neq 1
    \end{align*}
    No se cumplen las hipótesis del teorema porque $g$ no está acotada en el soporte, ya que no está acotado.
\end{example}

\begin{definition}
    Una función $F$ es función de distribución impropia si verifica:
    \begin{itemize}
        \item $F$ es creciente.
        \item $F$ es continua por la derecha.
        \item Existe $\lim\limits_{h \to 0^-} F(x+h) = F(x^-)$ para todo $x \in \mathbb{R}$.
        \item $F(-\infty) > 0$ o $F(\infty) < 1$.
    \end{itemize}
\end{definition}

\begin{definition}
    Sea $\{F_n\}$ una sucesión de funciones de distribución y sea $F$ una función de distribución propia o impropia.
    Decimos que $\{F_n\}$ converge de forma vaga o vagamente a $F$ si:
    $$\lim\limits_{n \to \infty} F_n(x) = F(x), \quad \forall x \in C(F)$$
    Se escribe $F_n \xrightarrow{v} F$.
\end{definition}

\begin{remark}
    Convergencia débil implica convergencia vaga.
\end{remark}

\begin{example}
    Sea $n \in \mathbb{N}$, consideramos la función de distribución:
    $$F_n(x) = \begin{cases}
            0                          & \text{si } x < 0        \\
            \frac{1}{2} + \frac{x}{2n} & \text{si } 0 \leq x < n \\
            1                          & \text{si } x \geq n
        \end{cases}$$
    El límite puntual de $\{F_n\}$ es:
    $$G(x) = \begin{cases}
            0           & \text{si } x < 0    \\
            \frac{1}{2} & \text{si } x \geq 0
        \end{cases}$$
    $G$ es una función de distribución impropia.
    Por tanto, $F_n \xrightarrow{v} G$.
\end{example}

\begin{example}
    Consideramos la función de distribución:
    $$F_0(x) = \begin{cases}
            0 & \text{si } x < 2    \\
            1 & \text{si } x \geq 2
        \end{cases}$$
    Definimos:
    $$F_n(x) = F_0(x+n) = \begin{cases}
            0 & \text{si } x+n < 2    \\
            1 & \text{si } x+n \geq 2
        \end{cases}$$
    El límite puntual de $F_n(x)$ es $F(x) = 1$, que es una función de distribución impropia.
    Por tanto, $F_n \xrightarrow{v} F$.
\end{example}

\begin{exercise}
    Sea $(\Omega, \mathcal{A}, P)$ espacio de probabilidad con:
    $$\Omega = \mathbb{N} \cup \{0\}, \quad \mathcal{A} = \mathcal{P}(\Omega), \quad P(\omega) = e^{-\lambda} \frac{\lambda^\omega}{w!}, \quad w \in \mathbb{R}, \lambda > 0$$
    Consideramos la sucesión de variables aleatorias:
    $$X_n: \Omega \to \mathbb{R}, \quad X_n(\omega) = e^{n\omega}$$
    Observamos que $x = e^{n\omega} \Leftrightarrow \omega = -\frac{\log(x)}{n}$, con $x > 0$.
    $$P(X_n^{-1}(x)) = P\left(-\frac{\log(x)}{n}\right) = \begin{cases}
            e^{-\lambda} \frac{\lambda^{-\frac{\log(x)}{n}}}{\left(-\frac{\log(x)}{n}\right)!} & \text{si } \frac{-\log(x)}{n} \in \mathbb{N} \cup \{0\} \\
            0                                                                                  & \text{en el resto}
        \end{cases}$$
    Calculamos la función de distribución de $X_n$:
    \begin{align*}
        F_n(x) & = P(X_n^{-1}((-\infty,x])) = P(X_n^{-1}([0,x])) = P(X_n^{-1}([0,e^{-nk}])) =                                                                                        \\
               & = P(\{ \omega \in \Omega : \omega \geq k \}) = 1 - P(\{ \omega \in \Omega : \omega < k \}) = 1 - \sum_{\omega=0}^{k-1} e^{-\lambda}\frac{\lambda^\omega}{\omega!} = \\
               & = \begin{cases}
                       0                                                                    & \text{si } x <= 0                                              \\
                       1 - \sum_{\omega=0}^{k-1} e^{-\lambda}\frac{\lambda^\omega}{\omega!} & \text{si } e^{-nk} \leq x < e^{-n(k-1)}, \quad k = 1, 2, \dots \\
                       1                                                                    & \text{si } x \geq 1
                   \end{cases}                             \\
               & = \begin{cases}
                       0                           & \text{si } x <= 0                  \\
                       \vdots                                                           \\
                       1 - e^{-\lambda}(1+\lambda) & \text{si } e^{-2n} \leq x < e^{-n} \\
                       1 - e^{-\lambda}            & \text{si } e^{-n} \leq x < 1       \\
                       1                           & \text{si } x \geq 1
                   \end{cases}
    \end{align*}
    Observamos que $F_n(x)$ tiene como límite puntual:
    $$G(x) = \begin{cases}
            0              & \text{si } x <= 0    \\
            1-e^{-\lambda} & \text{si } 0 < x < 1 \\
            1              & \text{si } x \geq 1
        \end{cases}$$
    $G$ no es función de distribución.
    Podemos definir:
    $$F(x) = \begin{cases}
            0              & \text{si } x < 0      \\
            1-e^{-\lambda} & \text{si } 0 <= x < 1 \\
            1              & \text{si } x \geq 1
        \end{cases}$$
    Como $F$ sí es función de distribución, $F_n \xrightarrow[d]{} F$.
\end{exercise}

\end{document}