\chapter{Resolución de sistemas y ecuaciones diferenciales lineales con coeficientes constantes}
\section{Exponencial de una matriz cuadrada}
\begin{definition}
    Dada $A \in \mathcal{M}_n(\mathbb{R})$, se llama matriz exponencial de $A$ y se escribe $e^A$ a la matriz de $\mathcal{M}_n(\mathbb{R})$ definida por:
    $$e^A = \sum_{k=0}^\infty \frac{1}{k!}A^k$$
\end{definition}

\begin{proposition}[Propiedades de la exponencial]
    \hfill
    \begin{enumerate}
        \item Si $\Theta \in \mathcal{M}_n(\mathbb{R})$ es la matriz nula, entonces $e^\Theta = I_n$.
        \item Si $I \in \mathcal{M}_n(\mathbb{R})$ es la matriz identidad, $e^I = eI$.
        \item Si $A, B \in \mathcal{M}_n(\mathbb{R})$ y $A$ y $B$ conmutan, entonces $Ae^B = e^BA$
        \item Si $A, B \in \mathcal{M}_n(\mathbb{R})$ y $A$ y $B$ conmutan, entonces $e^{A+B} = e^Ae^B$
        \item Si $A \in \mathcal{M}_n(\mathbb{R})$, entonces $e^A$ es invertible y su inversa es $(e^A)^{-1} = e^{-A}$.
    \end{enumerate}
\end{proposition}

\section{Determinación de una matriz fundamental}
\begin{theorem}
    Supongamos que $I \subset \mathbb{R}$ es intervalo no degenerado, que $B \in \mathcal{C}^1(I, \mathcal{M}_n(\mathbb{R}))$ y que $B$ y $B'$ conmutan para todo $t \in I$.
    Entonces, fijado $t_0 \in I$, la aplicación:
    \begin{align*}
        \Phi_{t_0}: I & \to \mathcal{M}_n(\mathbb{R})             \\
        t_0           & \mapsto \Phi_{t_0}(t) = e^{B(t) - B(t_0)}
    \end{align*}
    es la matriz fundamental canónica de $x' = B'(t)x$ en $t_0$.
\end{theorem}

\begin{corollary}
    Si $A \in \mathcal{M}_n(\mathbb{R})$ y $b \in \mathcal{C}(I, \mathbb{R}^n)$ entonces, para cada $t_0 \in I$ y cada $x^0 \in \mathbb{R}^n$, el problema de Cauchy
    $$(P): \begin{cases}
            x' = Ax + b \\
            x(t_0) = x^0
        \end{cases}$$
    tiene solución única en $I$.

    Además, esta solución viene dada por:
    $$\varphi(t) = e^{(t-t_0)A}x^0 + \int_{t_0}^t e^{(t-s)A}b(s)ds, \quad t \in I$$
\end{corollary}

\section{Cálculo de la exponencial de una matriz}
\begin{enumerate}
    \item Si $A = diag(\lambda_1, \dots, \lambda_n)$, entonces:
          $$e^{tA} = diag(e^{t\lambda_1}, \dots, e^{t\lambda_n})$$
    \item Si $A$ es semejante a una matriz diagonal $D$, es decir, existe $P$ invertible tal que $A = PDP^{-1}$, entonces:
          $$e^{tA} = Pe^{tD}P^{-1}$$
    \item Si $A$ es diagonal por bloques, es decir, $A = diag(A_1, \dots, A_l)$ siendo cada $A_j \in \mathcal{M}_{r_j}(\mathbb{R})$ con $r_1 + \dots + r_l = n$, entonces:
          $$e^{tA} = diag(e^{tA_1}, \dots, e^{tA_l})$$
    \item Si $A$ es un bloque de Jordan real del tipo
          $$A = D_r(\lambda) = \begin{pmatrix}
                  \lambda & 1       & \dots  & 0       & 0       \\
                  0       & \lambda & \dots  & 0       & 0       \\
                  \vdots  & \vdots  & \ddots & \vdots  & \vdots  \\
                  0       & 0       & \dots  & \lambda & 1       \\
                  0       & 0       & \dots  & 0       & \lambda
              \end{pmatrix}$$
          entonces:
          $$e^{tA} = e^{t\lambda}I_r\sum_{k=0}^{r-1}{\frac{t^k}{k!}N_r^k} = e^{t\lambda}\begin{pmatrix}
                  1      & t      & \frac{t^2}{2!} & \dots  & \frac{t^{r-1}}{(r-1)!} \\
                  0      & 1      & t              & \dots  & \frac{t^{r-2}}{(r-2)!} \\
                  \vdots & \vdots & \vdots         & \ddots & \vdots                 \\
                  0      & 0      & 0              & \dots  & t                      \\
                  0      & 0      & 0              & \dots  & 1
              \end{pmatrix}$$
    \item Si $A$ es un bloque de Jordan complejo del tipo
          $$A = E_r(\mu) = \begin{pmatrix}
                  \vec{\mu} & \vec{1}   & \dots  & \vec{0}   & \vec{0}   \\
                  \vec{0}   & \vec{\mu} & \dots  & \vec{0}   & \vec{0}   \\
                  \vdots    & \vdots    & \ddots & \vdots    & \vdots    \\
                  \vec{0}   & \vec{0}   & \dots  & \vec{\mu} & \vec{1}   \\
                  \vec{0}   & \vec{0}   & \dots  & \vec{0}   & \vec{\mu}
              \end{pmatrix}$$
          con $\mu = \alpha + i\beta$, $\beta > 0$, y donde:
          $$\vec{\mu} = \begin{pmatrix}
                  \alpha & \beta  \\
                  -\beta & \alpha
              \end{pmatrix}, \quad
              \vec{0} = \begin{pmatrix}
                  0 & 0 \\
                  0 & 0
              \end{pmatrix}, \quad
              \vec{1} = \begin{pmatrix}
                  1 & 0 \\
                  0 & 1
              \end{pmatrix}$$
          entonces:
          $$e^{tA} = e^{t\vec{\mu}} \begin{pmatrix}
                  \vec{1} & t\vec{1} & \frac{t^2}{2!}\vec{1} & \dots  & \frac{t^{r-1}}{(r-1)!}\vec{1} \\
                  \vec{0} & \vec{1}  & t\vec{1}              & \dots  & \frac{t^{r-2}}{(r-2)!}\vec{1} \\
                  \vdots  & \vdots   & \vdots                & \ddots & \vdots                        \\
                  \vec{0} & \vec{0}  & \vec{0}               & \dots  & t\vec{1}                      \\
                  \vec{0} & \vec{0}  & \vec{0}               & \dots  & \vec{1}
              \end{pmatrix}$$
          donde
          $$e^{t\mu} = e^{t\alpha}\begin{pmatrix}
                  \cos(t\beta)  & \sin(t\beta) \\
                  -\sin(t\beta) & \cos(t\beta)
              \end{pmatrix}$$
\end{enumerate}

\begin{theorem}[Forma canónica de Jordan real]
    Si $A \in \mathcal{M}_n(\mathbb{R})$, entonces existen una matriz de Jordan $J \in \mathcal{M}_n(\mathbb{R})$ y una matriz $P \in \mathcal{M}_n(\mathbb{R})$ invertible tales que:
    $$AP = PJ \Leftrightarrow A = PJP^{-1}$$
    Más concretamente,
    \begin{itemize}
        \item Para cada $\lambda \in \mathbb{R}$ autovalor de $A$ con multiplicidad $m(\lambda)$, $J$ contiene tantas cajas de Jordan del tipo $D_r(\lambda)$ como indique $\dim(\ker(A - \lambda I))$ y la suma de los tamaños de estas cajas es $m(\lambda)$.
        \item Para cada $\mu \in \mathbb{C}$ autovalor de $A$ con multiplicidad $m(\mu)$, $J$ contiene tantas cajas de Jordan del tipo $E_r(\mu)$ como indique $\dim(\ker(A - \lambda I))$ y la suma de los tamaños de estas cajas es $m(\mu)$.
    \end{itemize}
\end{theorem}

\section{Formas canónicas de Jordan}
\subsection*{Formas canónicas de Jordan reales asociadas a matrices $2 \times 2$}
Sea $A \in \mathcal{M}_2(\mathbb{R})$.
Su polinomio característico es $p(\lambda) = (A - \lambda I)$, que tiene coeficientes reales y es de grado 2, así que tiene dos raíces $\lambda_1$ y $\lambda_2$.
\begin{enumerate}
    \item Supongamos que $\lambda_1, \lambda_2 \in \mathbb{R}$, con $\lambda_1 \neq \lambda_2$.
          Sean $P^1$ y $P^2$ autovectores asociados a $\lambda_1$ y $\lambda_2$ respectivamente, entonces:
          $$J = \begin{pmatrix}
                  \lambda_1 & 0         \\
                  0         & \lambda_2
              \end{pmatrix}, \quad
              P = \begin{pmatrix}
                  P^1 | P^2
              \end{pmatrix}$$
    \item Supongamos que $\lambda_1 = \lambda_2 = \lambda \in \mathbb{R}$.
          Entonces $\dim(\ker(A - \lambda I)) \in \{1, 2\}$.
          \begin{enumerate}
              \item Supongamos que $\dim(\ker(A - \lambda I)) = 2$.
                    Sean $P^1$ y $P^2$ autovectores linealmente independientes asociados a $\lambda$, entonces:
                    $$J = \begin{pmatrix}
                            \lambda & 0       \\
                            0       & \lambda
                        \end{pmatrix}, \quad
                        P = \begin{pmatrix}
                            P^1 | P^2
                        \end{pmatrix}$$
              \item Supongamos que $\dim(\ker(A - \lambda I)) = 1$.
                    Entonces solo hay un bloque de Jordan asociado a $\lambda$ de tamaño 2, es decir,
                    $$J = \begin{pmatrix}
                            \lambda & 1       \\
                            0       & \lambda
                        \end{pmatrix}, \quad
                        P = \begin{pmatrix}
                            P^1 | P^2
                        \end{pmatrix}$$
                    donde:
                    \begin{itemize}
                        \item $P^1$ es autovector de $A$ asociado a $\lambda$.
                        \item $(A - \lambda I)P^2 = P^1$, es decir, $P^2$ es solución de $(A - \lambda I)X = P^1$.
                    \end{itemize}
          \end{enumerate}
    \item Supongamos que $\lambda_1, \lambda_2 \in \mathbb{C}$, es decir, son de la forma $\lambda_1 = \lambda = \alpha + i\beta$, $\lambda_2 = \bar{\lambda} = \alpha - i\beta$.
          Sea $W = P^1 + iP^2$ un autovector complejo de $A$ asociado a $\lambda$, entonces:
          $$J = \begin{pmatrix}
                  \alpha & \beta  \\
                  -\beta & \alpha
              \end{pmatrix}, \quad
              P = (P^1 | P^2)$$
\end{enumerate}

\subsection*{Formas canónicas de Jordan reales asociadas a matrices $3 \times 3$}
Sea $A \in \mathcal{M}_3(\mathbb{R})$.
Su polinomio característico es $p(\lambda) = (A - \lambda I)$, que tiene coeficientes reales y es de grado 3, así que tiene tres raíces $\lambda_1$, $\lambda_2$ y $\lambda_3$.
\begin{enumerate}
    \item Supongamos que $\lambda_1, \lambda_2, \lambda_3 \in \mathbb{R}$ distintos.
          Sean $P^1$, $P^2$ y $P^3$ autovectores asociados a $\lambda_1$, $\lambda_2$ y $\lambda_3$ respectivamente, entonces:
          $$J = \begin{pmatrix}
                  \lambda_1 & 0         & 0         \\
                  0         & \lambda_2 & 0         \\
                  0         & 0         & \lambda_3
              \end{pmatrix}, \quad
              P = \begin{pmatrix}
                  P^1 | P^2 | P^3
              \end{pmatrix}$$
    \item Supongamos que $\lambda_1, \lambda_2, \lambda_3 \in \mathbb{R}$ con $\lambda_2 = \lambda_3 = \lambda$ y $\lambda_1 \neq \lambda$.
          Entonces $\dim(\ker(A - \lambda I)) \in \{1, 2\}$.
          \begin{enumerate}
              \item Supongamos que $\dim(\ker(A - \lambda I)) = 2$.
                    Sea $P^1$ autovector asociado a $\lambda_1$ y sean $P^2$ y $P^3$ autovectores linealmente independientes asociados a $\lambda$, entonces:
                    $$J = \begin{pmatrix}
                            \lambda_1 & 0       & 0       \\
                            0         & \lambda & 0       \\
                            0         & 0       & \lambda
                        \end{pmatrix}, \quad
                        P = \begin{pmatrix}
                            P^1 | P^2 | P^3
                        \end{pmatrix}$$
              \item Supongamos que $\dim(\ker(A - \lambda I)) = 1$.
                    Entonces:
                    $$J = \begin{pmatrix}
                            \lambda_1 & 0       & 0       \\
                            0         & \lambda & 1       \\
                            0         & 0       & \lambda
                        \end{pmatrix}, \quad
                        P = \begin{pmatrix}
                            P^1 | P^2 | P^3
                        \end{pmatrix}$$
                    donde:
                    \begin{itemize}
                        \item $P^1$ es autovector de $A$ asociado a $\lambda_1$.
                        \item $P^2$ es autovector de $A$ asociado a $\lambda$.
                        \item $(A - \lambda I)P^3 = P^2$, es decir, $P^3$ es solución de $(A - \lambda I)X = P^2$.
                    \end{itemize}
          \end{enumerate}
    \item Supongamos que $\lambda_1 = \lambda_2 = \lambda_3 = \lambda \in \mathbb{R}$.
          Entonces $\dim(\ker(A - \lambda I)) \in \{1, 2, 3\}$.
          \begin{enumerate}
              \item Supongamos que $\dim(\ker(A - \lambda I)) = 3$.
                    Sean $P^1$, $P^2$ y $P^3$ autovectores linealmente independientes asociados a $\lambda$, entonces:
                    $$J = \begin{pmatrix}
                            \lambda & 0       & 0       \\
                            0       & \lambda & 0       \\
                            0       & 0       & \lambda
                        \end{pmatrix}, \quad
                        P = \begin{pmatrix}
                            P^1 | P^2 | P^3
                        \end{pmatrix}$$
              \item Supongamos que $\dim(\ker(A - \lambda I)) = 2$.
                    Entonces:
                    $$J = \begin{pmatrix}
                            \lambda & 0       & 0       \\
                            0       & \lambda & 1       \\
                            0       & 0       & \lambda
                        \end{pmatrix}, \quad
                        P = \begin{pmatrix}
                            P^1 | P^2 | P^3
                        \end{pmatrix}$$
                    donde:
                    \begin{itemize}
                        \item $P^1$ y $P^2$ son autovectores linealmente independientes asociados a $\lambda$.
                        \item $(A - \lambda I)P^3 = P^2$, es decir, $P^3$ es solución de $(A - \lambda I)X = P^2$.
                    \end{itemize}
              \item Supongamos que $\dim(\ker(A - \lambda I)) = 1$.
                    Entonces:
                    $$J = \begin{pmatrix}
                            \lambda_1 & 1       & 0       \\
                            0         & \lambda & 1       \\
                            0         & 0       & \lambda
                        \end{pmatrix}, \quad
                        P = \begin{pmatrix}
                            P^1 | P^2 | P^3
                        \end{pmatrix}$$
                    donde:
                    \begin{itemize}
                        \item $P^1$ es autovector de $A$ asociado a $\lambda$.
                        \item $(A - \lambda I)P^2 = P^1$, es decir, $P^2$ es solución de $(A - \lambda I)X = P^1$.
                        \item $(A - \lambda I)P^3 = P^2$, es decir, $P^3$ es solución de $(A - \lambda I)X = P^2$.
                    \end{itemize}
          \end{enumerate}
    \item Supongamos que $\lambda_1 \in \mathbb{R}$ y $\lambda_2, \lambda_3 \in \mathbb{C}$, es decir, son de la forma $\lambda_2 = \lambda = \alpha + i\beta$, $\lambda_3 = \bar{\lambda} = \alpha - i\beta$.
          Entonces:
          $$J = \begin{pmatrix}
                  \lambda_1 & 0      & 0      \\
                  0         & \alpha & \beta  \\
                  0         & -\beta & \alpha
              \end{pmatrix}, \quad
              P = \begin{pmatrix}
                  P^1 | P^2 | P^3
              \end{pmatrix}$$
          donde:
          \begin{itemize}
              \item $P^1$ es autovector de $A$ asociado a $\lambda_1$.
              \item $P^2 + iP^3$ es autovector complejo de $A$ asociado a $\lambda$.
          \end{itemize}
\end{enumerate}

\section{Ecuaciones diferenciales lineales de orden superior}
Las ecuaciones diferenciales lineales ordinarias de coeficientes constantes de orden $n$ son de la forma:
$$(E): y^{(n)} + a_1y^{(n-1)} + \dots + a_ny = a_0(t)$$
donde $a_0 \in \mathcal{C}(I, \mathbb{R})$ y $a_1, \dots, a_n \in \mathbb{R}$.

El sistema asociado es $(S): x' = Ax + b$, donde
$$A = \begin{pmatrix}
        0      & 1       & \dots  & 0      \\
        \vdots & \vdots  & \ddots & \vdots \\
        0      & 0       & \dots  & 1      \\
        -a_n   & a_{n-1} & \dots  & -a_1
    \end{pmatrix}, \quad
    b(t) = \begin{pmatrix}
        0      \\
        \vdots \\
        0      \\
        a_0(t)
    \end{pmatrix}$$

La solución de $(E)$ con dato inicial $y(t_0) = y_1^0$, $y'(t_0) = y_2^0$ viene dada por la primera componente de la solución de $(S)$.

El polinomio característico de $A$ es
$$p(\lambda) = (-1)^n(\lambda^n + a_1\lambda^{n-1} + ... + a_{n-2}\lambda^2 + a _{n-1}\lambda + a_n)$$
luego su ecuación característica es:
$$\lambda^n + a_1\lambda^{n-1} + ... + a_{n-2}\lambda^2 + a _{n-1}\lambda + a_n = 0$$

Ha de verificarse que
$$rg(A - \lambda I) + \dim(\ker(A - \lambda I)) = n$$
Como $rg(A - \lambda I) = n-1$, entonces $\dim(Ker(A - \lambda I)) = 1$.
Es decir, hay una caja de Jordan asociada a $\lambda$.
\begin{itemize}
    \item Si $\lambda \in \mathbb{R}$, la caja tiene tamaño $m(\lambda$) y es de la forma:
          $$D_{m(\lambda)}(\lambda) =
              \begin{pmatrix}
                  \lambda & 1       & \dots  & 0       & 0       \\
                  0       & \lambda & \dots  & 0       & 0       \\
                  \vdots  & \vdots  & \ddots & \vdots  & \vdots  \\
                  0       & 0       & \dots  & \lambda & 1       \\
                  0       & 0       & \dots  & 0       & \lambda
              \end{pmatrix}$$
    \item Si $\lambda = \mu = \alpha + i\beta \in \mathbb{C}$ con $\beta > 0$, la caja tiene tamaño $2m(\mu)$ y es de la forma:
          $$E_{m(\mu)}(\mu) = \begin{pmatrix}
                  \vec{\mu} & \vec{1}   & \dots  & \vec{0}   & \vec{0}   \\
                  \vec{0}   & \vec{\mu} & \dots  & \vec{0}   & \vec{0}   \\
                  \vdots    & \vdots    & \ddots & \vdots    & \vdots    \\
                  \vec{0}   & \vec{0}   & \dots  & \vec{\mu} & \vec{1}   \\
                  \vec{0}   & \vec{0}   & \dots  & \vec{0}   & \vec{\mu}
              \end{pmatrix}$$
          donde:
          $$\vec{\mu} = \begin{pmatrix}
                  \alpha & \beta  \\
                  -\beta & \alpha
              \end{pmatrix}, \quad
              \vec{0} = \begin{pmatrix}
                  0 & 0 \\
                  0 & 0
              \end{pmatrix}, \quad
              \vec{1} = \begin{pmatrix}
                  1 & 0 \\
                  0 & 1
              \end{pmatrix}$$
\end{itemize}

Supongamos que $\lambda_1, \dots, \lambda_l$ son los autovalores reales de $A$ con multiplicidades respectivas $m_1, \dots, m_n$ y que $\mu_1, \dots, \mu_s$ son los autovalores complejos de $A$ con parte imaginaria positiva y con multiplicidades respectivas $v_1, \dots, v_s$.
Entonces la forma canónica de Jordan real asociada a $A$ es:
$$J = diag(D_{m_1}(\lambda_1), \dots, D_{m_l}(\lambda_l), E_{v_1}(\mu_1), \dots, E_{v_s}(\mu_s))$$
Además, existe una matriz $P$ invertible tal que $A = PJP^{-1}$.
Por tanto:
$$e^{tA} = P \; diag(e^{tD_{m_1}(\lambda_1)}, \dots, e^{tD_{m_l}(\lambda_l)}, e^{tE_{v_1}(\mu_1)}, \dots, e^{tE_{v_s}(\mu_s)}) \; P^{-1}$$

Observamos que los elementos de $e^{tA}$ son combinaciones lineales de los elementos de la colección:
\begin{align*}
    \mathcal{F} = & \{t^ke^{\lambda_jt}, 1 \leq j \leq l, 0 \leq k \leq m_j-1\}                   \\
                  & \cup \{t^ke^{\alpha_j}t\cos(\beta_jt), 1 \leq j \leq s, 0 \leq k \leq v_j-1\} \\
                  & \cup \{t^ke^{\alpha_j}t\sin(\beta_jt), 1 \leq j \leq s, 0 \leq k \leq v_j-1\}
\end{align*}

\begin{theorem}
    $\mathcal{F}$ es sistema fundamental de soluciones de:
    $$(E_H): y^{(n)} + a_1y^{(n-1)} + \dots + a_ny = 0$$
\end{theorem}

\begin{theorem}[Método de los coeficientes indeterminados]
    Consideremos:
    $$(E_H): y^{(n)} + a_1y^{(n-1)} + \dots + a_ny = a_0(t)$$
    con $a_1, \dots,a_n \in \mathbb{R}$ y $a_0 \in \mathcal{C}(I, \mathbb{R})$.
    \begin{itemize}
        \item \textbf{Caso complejo.}
              Supongamos que $a_0(t) = g(t)e^{\lambda t}$, siendo $g$ un polinomio complejo de grado a lo sumo $d$ y $\lambda \in \mathbb{C}$.
              Entonces, una solución particular de $(E)$ es del tipo:
              $$\varphi_p(t) = t^{m(\lambda)}Q(t)e^{\lambda t}$$
              donde $Q$ es un polinomio de grado a lo sumo $d$ y $m(\lambda)$ es la multiplicidad de $\lambda$ como autovalor del polinomio característico de $(E_H)$.
        \item \textbf{Caso real.}
              Supongamos que $a_0(t) = e^{\alpha t}(q_1(t)\cos(\beta t) + q_2(t)\sin(\beta t))$, siendo $q_1$ y $q_2$ polinomios reales de grado a lo sumo $d$ y $\alpha, \beta \in \mathbb{R}$.
              Entonces, una solución particular de $(E)$ es del tipo:
              $$\varphi_p(t) = t^{m(\mu)}e^{\alpha t}(Q_1(t)\cos(\beta t) + Q_2(t)\sin(\beta t))$$
              donde $Q_1$ y $Q_2$ son polinomios reales de grado a lo sumo $d$ y $m(\mu)$ es la multiplicidad de $\mu = \alpha + i\beta$ como autovalor de $(E_H)$.
    \end{itemize}
\end{theorem}