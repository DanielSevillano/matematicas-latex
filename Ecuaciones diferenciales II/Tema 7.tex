\chapter{Resolución de sistemas y ecuaciones diferenciales lineales con coeficientes constantes}
\begin{definition}[Álgebra de Banach]
    Un espacio de Banach $(X, \|.\|)$ es álgebra de Banach cuando también hay definida en $X$ una operación multiplicación interna "$\cdot$" que hace que $(X, +, \cdot)$ es anillo (con unidad, no necesariamente conmutativo) y que es compatible con la norma $\|.\|$ en $X$ en el sentido de que si $z, w \in X$ entonces:
    $$\|z \cdot w \| \leq \|z\| \cdot \|w\|$$
    En las álgebras de Banach $(X,\|.\|)$ podemos definir la exponencial:
    $$e^{A} = \sum_{k=0}^{\infty}{\frac{A^k}{k!}}, \ \ \ A \in X$$
\end{definition}

\begin{remark}
    La convergencia de esta serie es absoluta:
    $$\sum_{k=0}^{\infty}{\left\| \frac{A^k}{k!}\right\|} \leq \sum_{k=0}^{\infty}{\frac{\|A\|^k}{k!}} = e^{\|A\|}$$
\end{remark}

\begin{proposition}[Producto de series en álgebras de Banach]
    Sea $(X, \|.\|)$ espacio de Banach y sean $\sum_{k=0}^{\infty}{A_k}$ y $\sum_{k=0}^{\infty}{B_k}$ series absolutamente convergentes. Entonces la serie $\sum_{k=0}^{\infty}{C_k}$ donde $C_k = \sum_{j=0}^{k}{A_kB_{k-j}}$ (producto de Cauchy), es absolutamente convergente y
    $$\sum_{k=0}^{\infty}{C_k} = \left( \sum_{k=0}^{\infty}{A_k} \right) \cdot \left( \sum_{k=0}^{\infty}{B_k} \right)$$
\end{proposition}

\begin{properties}
    \hfill
    \begin{enumerate}
        \item Si $A \in X$, $\|e^A\| \leq e^{\|A\|}$.
        \item $e^{0_X} = \sum_{n=0}^\infty \frac{0_X^n}{n!} = I + 0_X + 0_X + \dots = I$.
        \item $e^I = I^0 + I^1 + \frac{I^2}{2!} + \dots = (1 + 1 + \frac{1}{2!} + \dots)I = eI$.
        \item Si $A \in X$,
              $$Ae^A = A\sum_{k=0}^\infty \frac{A^k}{k!} = \sum_{k=0}^\infty \frac{A^{k+1}}{k!} = \sum_{k=0}^\infty \frac{A^k}{k!} A = e^AA$$
        \item Si $A, B \in X$ y $A$ y $B$ conmutan,
              $$Ae^B = A\sum_{k=0}^\infty \frac{B^k}{k!} = \sum_{k=0}^\infty \frac{AB^k}{k!} = \sum_{k=0}^\infty \frac{B^k}{k!}A = e^BA$$
        \item Si $A, B \in X$ y $A$ y $B$ conmutan,
              $$e^{A+B} = e^Ae^B$$
        \item Si $A \in X$, entonces $e^A$ es invertible y su inverso es $(e^A)^{-1} = e^{-A}$.
    \end{enumerate}
\end{properties}

\begin{theorem}
    Supongamos que $I \subset \mathbb{R}$ es intervalo no degenerado, que $B \in \mathcal{C}^1(I, \mathcal{M}_n(\mathbb{R}))$ y que $B$ y $B'$ conmutan para todo $t \in I$.
    Entonces, fijado $t_0 \in I$, la aplicación:
    \begin{align*}
        \Phi_{t_0}: I & \to \mathcal{M}_n(\mathbb{R})             \\
        t_0           & \mapsto \Phi_{t_0}(t) = e^{B(t) - B(t_0)}
    \end{align*}
    es la matriz fundamental canónica de $x' = B'(t)x$ en $t_0$.
\end{theorem}

\begin{corollary}
    Si $A \in \mathcal{M}_n(\mathbb{R})$ y $b \in \mathcal{C}(I, \mathbb{R}^n)$ entonces, para cada $t_0 \in I$ y cada $x^0 \in \mathbb{R}^n$, el problema de Cauchy
    $$(P): \begin{cases}
            x' = Ax + b \\
            x(t_0) = x^0
        \end{cases}$$
    tiene solución única en $I$.

    Además, esta solución viene dada por:
    $$\varphi(t) = e^{(t-t_0)A}x^0 + \int_{t_0}^t e^{(t-s)A}b(s)ds, \quad t \in I$$
\end{corollary}

\begin{definition}[Autovalor y autovector]
    Sea $A$ una matriz cuadrada de orden $n$.
    Decimos que $\lambda \in \mathbb{C}$ es autovalor de $A$ si existe $v \in \mathbb{R}^n$ no nulo tal que $Av = \lambda v$.
    En tal caso, $v$ recibe el nombre de autovector de $A$ asociado a $\lambda$.

    Tratándose de una matriz cuadrada,
    $$Av = \lambda v \Leftrightarrow (A-\lambda I)X = 0 \Leftrightarrow |A-\lambda I| = 0$$
    Es decir, el conjunto de autovalores de $A$ es el conjunto de raíces del polinomio característica de $A$:
    $$\rho(\lambda) = |A - \lambda I|$$

    Llamamos ecuación característica de $A$ a $\rho(\lambda) = |A - \lambda I| = 0$.
\end{definition}

\begin{theorem}[Teorema fundamental del álgebra]
    Si $\rho(X) \in \mathbb{C}[X]$ es no constante, entonces tiene al menos una raíz compleja.
\end{theorem}

\begin{theorem}[Forma canónica de Jordan compleja]
    Si $A \in \mathcal{M}_n(\mathbb{C})$, entonces existen esencialmente una única matriz Jordan $J \in \mathcal{M}_n(\mathbb{C})$ y una matriz $P \in \mathcal{M}_n(\mathbb{C})$ invertible tales que:
    $$AP = PJ \Leftrightarrow A = PJP^{-1}$$
\end{theorem}