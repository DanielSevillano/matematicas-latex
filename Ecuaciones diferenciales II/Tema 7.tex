\chapter{Resolución de sistemas y ecuaciones diferenciales lineales con coeficientes constantes}
\section{Exponencial de una matriz cuadrada}
\begin{definition}
    Dada $A \in \mathcal{M}_n(\mathbb{R})$, se llama matriz exponencial de $A$ y se escribe $e^A$ a la matriz de $\mathcal{M}_n(\mathbb{R})$ definida por:
    $$e^A = \sum_{k=0}^\infty \frac{1}{k!}A^k$$
\end{definition}

\begin{proposition}[Propiedades de la exponencial]
    \hfill
    \begin{enumerate}
        \item Si $\Theta \in \mathcal{M}_n(\mathbb{R})$ es la matriz nula, entonces $e^\Theta = I_n$.
        \item Si $I \in \mathcal{M}_n(\mathbb{R})$ es la matriz identidad, $e^I = eI$.
        \item Si $A, B \in \mathcal{M}_n(\mathbb{R})$ y $A$ y $B$ conmutan, entonces $Ae^B = e^BA$
        \item Si $A, B \in \mathcal{M}_n(\mathbb{R})$ y $A$ y $B$ conmutan, entonces $e^{A+B} = e^Ae^B$
        \item Si $A \in \mathcal{M}_n(\mathbb{R})$, entonces $e^A$ es invertible y su inversa es $(e^A)^{-1} = e^{-A}$.
    \end{enumerate}
\end{proposition}

\section{Determinación de una matriz fundamental}
\begin{theorem}
    Supongamos que $I \subset \mathbb{R}$ es intervalo no degenerado, que $B \in \mathcal{C}^1(I, \mathcal{M}_n(\mathbb{R}))$ y que $B$ y $B'$ conmutan para todo $t \in I$.
    Entonces, fijado $t_0 \in I$, la aplicación:
    \begin{align*}
        \Phi_{t_0}: I & \to \mathcal{M}_n(\mathbb{R})             \\
        t_0           & \mapsto \Phi_{t_0}(t) = e^{B(t) - B(t_0)}
    \end{align*}
    es la matriz fundamental canónica de $x' = B'(t)x$ en $t_0$.
\end{theorem}

\begin{corollary}
    Si $A \in \mathcal{M}_n(\mathbb{R})$ y $b \in \mathcal{C}(I, \mathbb{R}^n)$ entonces, para cada $t_0 \in I$ y cada $x^0 \in \mathbb{R}^n$, el problema de Cauchy
    $$(P): \begin{cases}
            x' = Ax + b \\
            x(t_0) = x^0
        \end{cases}$$
    tiene solución única en $I$.

    Además, esta solución viene dada por:
    $$\varphi(t) = e^{(t-t_0)A}x^0 + \int_{t_0}^t e^{(t-s)A}b(s)ds, \quad t \in I$$
\end{corollary}

\section{Cálculo de la exponencial de una matriz}
\begin{enumerate}
    \item Si $A = diag(\lambda_1, \dots, \lambda_n)$, entonces:
          $$e^{tA} = diag(e^{t\lambda_1}, \dots, e^{t\lambda_n})$$
    \item Si $A$ es semejante a una matriz diagonal $D$, es decir, existe $P$ invertible tal que $A = PDP^{-1}$, entonces:
          $$e^{tA} = Pe^{tD}P^{-1}$$
    \item Si $A$ es diagonal por bloques, es decir, $A = diag(A_1, \dots, A_l)$ siendo cada $A_j \in \mathcal{M}_{r_j}(\mathbb{R})$ con $r_1 + \dots + r_l = n$, entonces:
          $$e^{tA} = diag(e^{tA_1}, \dots, e^{tA_l})$$
\end{enumerate}