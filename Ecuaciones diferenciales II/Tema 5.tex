\chapter{Prolongaciones de soluciones y soluciones maximales}
\begin{definition}[Soluciones estrictamente prolongables y soluciones maximales]
    Una solución $x: I \to \mathbb{R}^n$ de una ecuación diferencial o de un problema de Cauchy se dice que es estrictamente prolongable cuando existe otra solución $y: J \to \mathbb{R}^n$ tal que:
    $$I \subsetneqq J, \quad y_{|_I} = x$$
    Cuando esto sucede, se dice que $y: J \to \mathbb{R}^n$ es una prolongación estricta de $x: I \to \mathbb{R}^n$.
    Una solución que no admite prolongación estricta se dice que es no prolongable o que es maximal.
\end{definition}

\begin{definition}[Soluciones estrictamente prolongables lateralmente]
    \hfill
    \begin{itemize}
        \item Una solución $x: I \to \mathbb{R}^n$ de $(P)$ se dice que es estrictamente prolongable a la derecha cuando existe otra solución $y: J \to \mathbb{R}^n$ de $(P)$ con $I \subset J$ tal que:
              $$I^+ = \{t \in I: t \geq t_0\} \subsetneqq J^+ = \{t \in J: t \geq t_0\} \quad \text{y} \quad y|_I = x$$
              Cuando esto sucede, se dice que $y: J \to \mathbb{R}^n$ es una prolongación estricta a la derecha de $x: I \to \mathbb{R}^n$.
              Una solución de $(P)$ que no admite prolongación estricta a la derecha se dice que no es prolongable a la derecha.
        \item Una solución $x: I \to \mathbb{R}^n$ de $(P)$ se dice que es estrictamente prolongable a la izquierda cuando existe otra solución $y: J \to \mathbb{R}^n$ de $(P)$ con $I \subset J$ tal que:
              $$I^- = \{t \in I: t \leq t_0\} \subsetneqq J^- = \{t \in J: t \leq t_0\} \quad \text{y} \quad y|_I = x$$
              Cuando esto sucede, se dice que $y: J \to \mathbb{R}^n$ es una prolongación estricta a la izquierda de $x: I \to \mathbb{R}^n$.
              Una solución de $(P)$ que no admite prolongación estricta a la izquierda se dice que no es prolongable a la izquierda.
    \end{itemize}
\end{definition}

\section{Existencia y unicidad de soluciones no prolongables}
\begin{theorem}[Existencia y unicidad de soluciones no prolongables]
    Si $f$ es continua en $D$ se verifica:
    \begin{enumerate}
        \item Si $(t_0, x^0)$ es un punto interior a $D$, entonces $(P)$ tiene al menos una solución que no es prolongable definida en un intervalo que contiene al punto $t_0$ en el interior.
              Si además $f \in Lip_{Loc}(x, D)$, esta solución maximal es única.
        \item Si existen $a > 0$ y $b > 0$ tales que $Q = [t_0, t_0 + a] \times \bar{B}(x^0; b) \subset D$, entonces $(P)$ posee al menos una solución lateral a la derecha que no es prolongable a la derecha.
        \item Si existen $a > 0$ y $b > 0$ tales que $Q = [t_0 - a, t_0] \times \bar{B}(x^0; b) \subset D$, entonces $(P)$ posee al menos una solución lateral a la izquierda que no es prolongable a la izquierda.
    \end{enumerate}
\end{theorem}

\begin{remark}
    Si $D$ es abierto y $f$ es de clase $\mathcal{C}^1$ en $D$, entonces $(P)$ tiene una única solución maximal, que está definida en un intervalo que contiene a $t_0$ en su interior.
\end{remark}

\section{Soluciones maximales con gráficas contenidas en compactos}
\begin{theorem}
    Sea el problema de valor inicial:
    $$(P) \begin{cases}
            x'(t) = f(t, x(t)) \\
            x(t_0) = x^0
        \end{cases}$$
    donde $f: D \to \mathbb{R}^n$, $D \subset \mathbb{R} \times \mathbb{R}^n$, $n \geq 1$ y $(t_0, x^0) \in D$.
    Sea $x: I \to \mathbb{R}^n$ una solución maximal de $(P)$ y sea $\Gamma$ su gráfica.

    Supongamos que existe un conjunto $K$ compacto en $\mathbb{R} \times \mathbb{R}^n$ tal que $\Gamma \subset K \subset D$ y $f$ es continua en $K$.
    Entonces:
    \begin{enumerate}
        \item $I$ es un intervalo compacto, es decir, $I = [a, b]$.
        \item Los puntos $(a, x(a))$ y $(b, x(b))$ están en la frontera de $K$.
    \end{enumerate}
\end{theorem}

\section{Puntos límites y el lema de Wintner}
\begin{definition}[Puntos límites]
    Sean $n \geq 1$ y $x: [t_0, t_1) \to \mathbb{R}^n$ una función tal que $t_1 < \infty$.
    Sea $x^1 \in \mathbb{R}^n$.
    Se dice que $(t_1, x^1)$ es un punto límite de la gráfica de $x$ para $t \to t_1$ cuando existe una sucesión $(s_m)$ en el intervalo $[t_0, t_1)$ tal que $(s_m, x(s_m)) \to (t_1, x^1)$ en $\mathbb{R} \times \mathbb{R}^n$.
\end{definition}

\begin{proposition}
    Sean $n \geq 1$, $t_1 < \infty$, $x: [t_0, t_1) \to \mathbb{R}^n$ una función, $\Gamma$ su gráfica y $\|.\|$ una norma en $\mathbb{R}^n$.
    Se verifica una y solamente una de las dos siguientes situaciones:
    \begin{itemize}
        \item $\lim\limits_{t \to t_1} \|x(t)\| = \infty$
        \item $\Gamma$ tiene al menos un punto límite para $t \to t_1$.
    \end{itemize}
\end{proposition}

\begin{theorem}[Lema de Wintner]
    Sea $x: [t_0, t_1) \to \mathbb{R}^n$, siendo $t_1 < \infty$, una solución de la ecuación diferencial $x'(t) = f(t, x(t))$, con gráfica $\Gamma$ contenida en $D \subset \mathbb{R} \times \mathbb{R}^n$ y sea $f: D \to \mathbb{R}^n$ una función continua en $D$.
    Sea $(t_1, x^1)$ un punto límite de $\Gamma$ para $t \to t_1$ y supongamos que se verifica la siguiente condición:
    \begin{center}
        Existe un entorno $U$ de $(t_1, x^1)$ tal que $f$ es acotada de $U \cap D$.
    \end{center}

    Entonces $\lim\limits_{t \to t_1} x(t) = x^1$.
\end{theorem}

\section{Soluciones maximales con gráficas contenidas en abiertos}
\begin{theorem}
    Sean $A$ un abierto en $\mathbb{R} \times \mathbb{R}^n$, $n \geq 1$, $f: A \to \mathbb{R}^n$ una función continua en $A$ y $\|.\|$ una norma en $\mathbb{R}^n$.
    Si $x: I \to \mathbb{R}^n$ es una solución no prolongable de la ecuación diferencial $x'(t) = f(t, x(t))$ con gráfica $\Gamma$ contenida en $A$, se verifica:
    \begin{enumerate}
        \item El intervalo $I$ es abierto.
        \item Si $I$ tiene un extremo finito $\alpha$, entonces $\lim\limits_{t \to \alpha} \|x(t)\| = \infty$ o bien cualquier punto límite de $\Gamma$ para $t \to \alpha$ está en la frontera de $A$.
    \end{enumerate}
\end{theorem}

\section{Soluciones maximales de las ecuaciones diferenciales autónomas}
\begin{theorem}[Teorema fundamental de las ecuaciones autónomas]
    Supongamos que $g \in \mathcal{C}^1(\mathbb{R}, \mathbb{R})$.
    Sea $x: I \to \mathbb{R}$ una solución no prolongable de la ecuación $X' = g(x)$ y sea $t_0 \in \dot{I}$.
    Se verifica lo siguiente:
    \begin{enumerate}
        \item El intervalo $I$ es abierto.
        \item Si $x$ es acotada en $I^+ = [t_0, \infty)$, existe $\lim\limits_{t \to \infty} x(t) = a$, siendo $a \in \mathbb{R}$, y la función constante dada por $y(t) \equiv a$ es solución de la ecuación $x' = g(x)$.
        \item Si $x$ es acotada en $I^- = (-\infty, t_0]$, existe $\lim\limits_{t \to -\infty} x(t) = b$, siendo $b \in \mathbb{R}$, y la función constante dada por $y(t) \equiv b$ es solución de la ecuación $x' = g(x)$.
    \end{enumerate}
\end{theorem}