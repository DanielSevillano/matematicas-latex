\chapter{Inferencia bayesiana}
\section{Teorema de Bayes}
\begin{theorem}[Teorema de Bayes]
    Sea $(\Omega, \mathcal{A}, P)$ un espacio de probabilidad.
    Sea $\{A_1, \dots, A_n\} \subset \mathcal{A}$ una partición de $\Omega$ y sea $B \in \mathcal{A}$ tal que $P(B) > 0$ y del que se conocen $P(B|A_i)$, para $i = 1, \dots, n$.
    Entonces
    $$P(A_i|B) = \frac{P(B|A_i)P(A_i)}{\sum_{j=1}^n P(B|A_j)P(A_j)}, \quad \forall i = 1, \dots, n$$
    donde:
    \begin{itemize}
        \item $P(A_j)$, $j = 1, \dots, n$, se llaman probabilidades a priori.
        \item $P(B|A_j)$, $j = 1, \dots, n$, se llaman verosimilitudes.
        \item $P(A_j|B)$, $j= 1, \dots, n$, se llaman probabilidades a posteriori.
    \end{itemize}

    Esta se conoce como la fórmula de Bayes.
\end{theorem}

\begin{remark}
    Las probabilidades a posteriori son proporcionales al producto de verosimilitudes y probabilidades a priori.
    $$P(A_i|B) \propto P(B|A_i)P(A_i)$$
\end{remark}