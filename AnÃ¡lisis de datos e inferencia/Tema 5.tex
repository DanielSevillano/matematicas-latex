\chapter{Introducción a la inferencia no paramétrica}
\section{Introducción}
Hasta ahora los tests de hipótesis han sido utilizados para contrastar la veracidad de hipótesis acerca de los parámetros de la distribución de una población.
Sin embargo, en muchas ocasiones, es necesario emitir un juicio estadístico sobre la distribución poblacional en su conjunto.
Los problemas de este tipo que se plantean de manera habitual son los siguientes:
\begin{itemize}
    \item \textbf{Contrastes de bondad de ajuste.}
          Decidir, a la vista de una muestra aleatoria de una población, si puede admitirse que la distribución poblacional coincide con una cierta distribución dada o pertenece a un determinado tipo de distribuciones.
    \item \textbf{Contrastes de homogeneidad.}
          Analizar si varias muestras aleatorias provienen de poblaciones con la misma distribución teórica, de forma que puedan ser utilizadas conjuntamente para inferencias posteriores acerca de esta; o, por el contrario, son muestras de poblaciones con distinta distribución, que no pueden agruparse como información homogénea acerca de una única distribución.
    \item \textbf{Contrastes de independencia.}
          Estudiar, en el caso en que se observen dos o más características de los elementos de la población, si las características observadas pueden considerarse independientes, y se puede proceder a su análisis por separado; o, por el contrario, existe relación estadística entre ellas.
\end{itemize}

\section{Contrastes de bondad de ajuste}
\subsection*{Primer caso}
Consideremos una muestra aleatoria $(X_1, \dots, X_n)$ de una variable aleatoria $X$ con distribución desconocida.
Para decidir si es razonable admitir que la distribución de $X$ viene dada por un determinado modelo de probabilidad $P$, resolvemos el siguiente contraste de hipótesis:
$$\begin{cases}
        H_0: \text{ El modelo de probabilidad de } X \text{ es } P \\
        H_1: \text{ El modelo de probabilidad de } X \text{ no es } P
    \end{cases}$$

\begin{note}
    El modelo $P$ debe estar completamente especificado.
\end{note}

Para contrastar $H_0$ frente a $H_1$ hacemos una partición arbitraria del espacio muestral de la población en $k$ clases $A_1, \dots, A_k$.
Después, para cada $A_i$ consideramos las siguientes frecuencias absolutas:
\begin{itemize}
    \item $O_i$: frecuencia observada en $A_i$, número de elementos de la muestra que están en la clase $A_i$.
    \item $e_i$: frecuencia esperada de la clase $A_i$ si la hipótesis $H_0$ es cierta, $nP(A_i)$.
\end{itemize}

El estadístico que utilizaremos es:
$$\sum_{i=1}^k \frac{(O_i-e_i)^2}{e_i}$$
que tiene aproximadamente cuando $n$ es grande una distribución $\chi^2_{k-1}$ si $H_0$ es cierta.

Para que la aproximación sea razonablemente buena, además de tener una muestra suficientemente grande, es necesario que el valor esperado de cada clase sea suficientemente grande.
Si la muestra procede de $P$, es de esperar que haya valores parecidos para $O_i$ y $e_i$ y, por tanto, este estadístico debería tomar valores próximos a cero.

Rechazaremos $H_0$ a nivel de significación $\alpha$ si:
$$\sum_{i=1}^k \frac{(O_i-e_i)^2}{e_i} > \chi^2_{k-1, 1-\alpha}$$
En caso contrario, aceptaremos $H_0$ a nivel de significación $\alpha$.

\begin{example}
    Después de lanzar un dado 300 veces, se obtienen las siguientes frecuencias:
    \begin{center}
        \begin{tabular}{ | l | c c c c c c | }
            \hline
            Resultado  & 1  & 2  & 3  & 4  & 5  & 6  \\
            \hline
            Frecuencia & 43 & 49 & 56 & 45 & 66 & 41 \\
            \hline
        \end{tabular}
    \end{center}

    Veamos si se puede afirmar que el dado es regular a nivel de significación $\alpha = 0.05$.

    Disponemos de una muestra aleatoria de $n = 300$ lanzamientos de un dado.
    Para decidir si el dado es regular o no, llevamos a cabo un contraste de bondad de ajuste a nivel de significación $\alpha$:
    $$\begin{cases}
            H_0: \text{ El dado es regular: } P(1) = \dots = P(6) = \frac{1}{6} \\
            H_1: \text{ El dado es irregular}
        \end{cases}$$

    La tabla de frecuencias observadas y esperadas es:
    \begin{center}
        \begin{tabular}{| c | c c c c c c |}
            \hline
            $A_i$ & 1  & 2  & 3  & 4  & 5  & 6  \\
            \hline
            $O_i$ & 43 & 49 & 56 & 45 & 66 & 41 \\
            $e_i$ & 50 & 50 & 50 & 50 & 50 & 50 \\
            \hline
        \end{tabular}
    \end{center}
    donde las frecuencias esperadas bajo $H_0$ han sido calculadas como:
    $$e_i = nP(A_i) = 300 \frac{1}{6} = 50$$

    Usaremos el estadístico de contraste:
    $$\sum_{i=1}^k \frac{(O_i-e_i)^2}{e_i}$$
    que tiene aproximadamente una distribución $\chi^2_{k-1}$ si $H_0$ es cierta.

    Rechazaremos $H_0$ a nivel de significación $\alpha$ si:
    $$\sum_{i=1}^k \frac{(O_i-e_i)^2}{e_i} > \chi^2_{k-1, 1-\alpha}$$
    En nuestro caso:
    $$\sum_{i=1}^6 \frac{(O_i-e_i)^2}{e_i} = 8.96, \quad \chi^2_{5, 0.95} = 11.07$$

    Por tanto, aceptamos $H_0$ a nivel de significación $\alpha = 0.05$, es decir, aceptamos que el dado es regular a nivel de significación $\alpha = 0.05$.
\end{example}

\begin{example}
    Nos dicen que un programa de ordenador genera observaciones de una distribución $N(0, 1)$.
    Como no estamos seguros de ello, obtenemos una muestra aleatoria de 450 observaciones mediante dicho programa, obteniendo los siguientes resultados:
    \begin{itemize}
        \item 30 observaciones menores que -2.
        \item 80 observaciones entre -2 y -1.
        \item 140 observaciones entre -1 y 0.
        \item 110 observaciones entre 0 y 1.
        \item 60 observaciones entre 1 y 2.
        \item 30 observaciones mayores que 2.
    \end{itemize}
    Veamos si se puede aceptar que el programa funciona correctamente a nivel de significación $\alpha = 0.01$.

    Disponemos de una muestra aleatoria de $n = 450$ observaciones generadas por el programa.
    Los posibles resultados de estas observaciones se agrupan en seis clases:
    \begin{align*}
        A_1 & = (-\infty, -2) & A_2 & = (-2, -1)    \\
        A_3 & = (-1, 0)       & A_4 & = (0, 1)      \\
        A_5 & = (1, 2)        & A_6 & = (2, \infty)
    \end{align*}
    Para decidir si el programa funciona correctamente o no, llevamos a cabo un contraste de bondad de ajuste a nivel de significación $\alpha$:
    $$\begin{cases}
            H_0: \text{ El programa funciona correctamente: provienen de } N(0, 1) \\
            H_1: \text{ El programa no funciona correctamente}
        \end{cases}$$

    La tabla de frecuencias observadas y esperadas es:
    \begin{center}
        \begin{tabular}{| c | c c c c c c |}
            \hline
            $A_i$    & $(-\infty, -2)$ & $(-2, -1)$ & $(-1, 0)$ & $(0, 1)$ & $(1, 2)$ & $(2, \infty)$ \\
            \hline
            $O_i$    & 30              & 80         & 140       & 110      & 60       & 30            \\
            $P(A_i)$ & 0.0228          & 0.1359     & 0.3413    & 0.3413   & 0.1359   & 0.0228        \\
            $e_i$    & 10.26           & 61.155     & 153.585   & 154.585  & 61.155   & 10.26         \\
            \hline
        \end{tabular}
    \end{center}
    donde las frecuencias esperadas bajo $H_0$ han sido calculadas de la forma:
    $$e_i = nP(A_i) = 450P(A_i)$$
    y los valores $P(A_i)$ se han calculado a partir de la tabla de la función de distribución de la normal estándar.

    Usaremos el estadístico de contraste:
    $$\sum_{i=1}^k \frac{(O_i-e_i)^2}{e_i}$$
    que tiene aproximadamente una distribución $\chi^2_{k-1}$ si $H_0$ es cierta.

    Rechazaremos $H_0$ a nivel de significación $\alpha$ si:
    $$\sum_{i=1}^k \frac{(O_i-e_i)^2}{e_i} > \chi^2_{k-1, 1-\alpha}$$
    En nuestro caso:
    $$\sum_{i=1}^6 \frac{(O_i-e_i)^2}{e_i} = 95.358, \quad \chi^2_{5, 0.99} = 15.086$$

    Por tanto, rechazamos $H_0$ a nivel de significación $\alpha = 0.01$, es decir, podemos afirmar que el programa no funciona correctamente a nivel de significación $\alpha = 0.01$.
\end{example}

\subsection*{Segundo caso}
El contraste de bondad de ajuste se puede plantear también en una situación más general.
Consideremos una muestra aleatoria $(X_1, \dots, X_n)$ de una variable aleatoria $X$ con distribución desconocida.
Para decidir si es razonable admitir que la distribución de $X$ viene dada por algún modelo de probabilidad de una cierta familia $P_\theta$, con $\theta = (\theta_1, \dots, \theta_r)$, resolveremos el siguiente contraste de hipótesis:
$$\begin{cases}
        H_0: \text{ El modelo de probabilidad } X \text{ es de la familia } \{P_\theta: \theta \in \Theta\} \\
        H_1: \text{ El modelo de probabilidad } X \text{ no es de la familia } \{P_\theta: \theta \in \Theta\}
    \end{cases}$$

Para contrastar $H_0$ frente a $H_1$ hacemos nuevamente una partición arbitraria del espacio muestral de la población en $k$ clases $A_1, \dots, A_k$.
Después, para cada $A_i$ consideramos las siguientes frecuencias absolutas:
\begin{itemize}
    \item $O_i$: frecuencia observada en $A_i$, número de elementos de la muestra que están en la clase $A_i$.
    \item $e_i$: frecuencia esperada de la clase $A_i$ si la hipótesis $H_0$ es cierta, $nP_\theta(A_i) \approx nP_{\hat{\theta}}(A_i)$, donde $\hat{\theta}$ es el estimador de máxima verosimilitud de $\theta$.
\end{itemize}

El estadístico que utilizaremos es:
$$\sum_{i=1}^k \frac{(O_i-e_i)^2}{e_i}$$
que tiene aproximadamente cuando $n$ es grande una distribución $\chi^2_{k-r-1}$ si $H_0$ es cierta.

Rechazaremos $H_0$ a nivel de significación $\alpha$ si:
$$\sum_{i=1}^k \frac{(O_i-e_i)^2}{e_i} > \chi^2_{k-r-1, 1-\alpha}$$
En caso contrario, aceptaremos $H_0$ a nivel de significación $\alpha$.

\begin{example}
    En el transcurso de dos horas, el número de llamadas por minuto recibidas en una centralita fue el siguiente:
    \begin{center}
        \begin{tabular}{| l | c c c c c c c |}
            \hline
            Número de llamadas por minuto & 0 & 1  & 2  & 3  & 4  & 5  & 6 \\
            \hline
            Frecuencia                    & 6 & 18 & 32 & 35 & 17 & 10 & 2 \\
            \hline
        \end{tabular}
    \end{center}
    Veamos si se puede aceptar que el número de llamadas por minuto sigue una distribución de Poisson.

    Disponemos de una muestra de $n = 120$ minutos, en los cuales registramos el número de llamadas que se han producido.
    El número de llamadas por minutos lo clasificamos en las siguientes clases:
    $$\{0\}, \quad \{1\}, \quad \{2\}, \quad \{3\}, \quad \{4\}, \quad \{\geq 5\}$$
    Las dos últimas clases las hemos agrupado para evitar frecuencias demasiado bajas.

    Sea $X$ el número de llamadas por minuto, realizamos el siguiente contraste:
    $$\begin{cases}
            H_0: X \sim \text{ Poisson}  \\
            H_1: X \nsim \text{ Poisson} \\
        \end{cases}$$

    La tabla de frecuencias observadas y esperadas es:
    \begin{center}
        \begin{tabular}{| c | c c c c c c |}
            \hline
            $A_i$    & $\{0\}$ & $\{1\}$ & $\{2\}$ & $\{3\}$ & $\{4\}$ & $\{\geq 5\}$ \\
            \hline
            $O_i$    & 6       & 18      & 32      & 35      & 17      & 12           \\
            $P(A_i)$ & 0.0743  & 0.19318 & 0.2510  & 0.2176  & 0.1414  & 0.1226       \\
            $e_i$    & 8.92    & 23.17   & 30.12   & 26.11   & 16.97   & 14.71        \\
            \hline
        \end{tabular}
    \end{center}
    donde las frecuencias observadas bajo $H_0$ han sido calculadas como:
    $$e_i = nP(A_i) = 120P(A_i)$$
    y los valores de $P(A_i)$ se han calculado a partir de una distribución de Poisson de parámetro $\hat{\lambda} = \bar{x} = 2.6$.

    Usaremos el estadístico de contraste:
    $$\sum_{i=1}^k \frac{(O_i-e_i)^2}{e_i}$$
    que tiene aproximadamente una distribución $\chi^2_{k-r-1}$ si $H_0$ es cierta.

    Rechazaremos $H_0$ a nivel de significación $\alpha$ si:
    $$\sum_{i=1}^k \frac{(O_i-e_i)^2}{e_i} > \chi^2_{k-r-1, 1-\alpha}$$
    En nuestro caso:
    $$\sum_{i=1}^6 \frac{(O_i-e_i)^2}{e_i} = 5.75, \quad \chi^2_{4, 0.95} = 9.488$$

    Por tanto, aceptamos $H_0$ a nivel de significación $\alpha = 0.05$, es decir, podemos aceptar que el número de llamadas por minuto sigue una distribución de Poisson a nivel de significación $\alpha = 0.05$.
\end{example}

\section{Contrastes de homogeneidad}
Supongamos que disponemos de $p$ muestras aleatorias independientes tomadas de $p$ poblaciones sobre una característica común $X$ a todas ellas:
\begin{align*}
    (X_{11}, \dots, X_{1n_1}) \\
    \dots                     \\
    (X_{p1}, \dots, X_{pn_p})
\end{align*}
con $n_1 + \dots + n_p = n$.

Queremos ver si, a la vista de las muestras obtenidas, es razonable admitir que todas las poblaciones tienen una distribución común, es decir, si son poblaciones homogéneas.
Por tanto, tenemos el contraste:
$$\begin{cases}
        H_0: \text{ Las } p \text{ poblaciones tienen una distribución común} \\
        H_1: \text{ Las } p \text{ poblaciones no tienen una distribución común}
    \end{cases}$$

Para contrastar $H_0$ frente a $H_1$ hacemos nuevamente una partición arbitraria del espacio muestral común a las $p$ poblaciones en $k$ clases $A_1, \dots, A_k$.

Después, definimos para la clase $A_i$ y para la muestra de la población $j$-ésima:
\begin{itemize}
    \item $O_{ij}$: frecuencia observada en la clase $A_i$ con la muestra $j$-ésima.
    \item $e_{ij}$: frecuencia esperada en la clase $A_i$ con la muestra $j$-ésima si todas las poblaciones tienen la distribución común $P$, $n_jP(A_i)$.
\end{itemize}

El estadístico utilizado es:
$$\sum_{j=1}^p \sum_{i=1}^k \frac{(O_{ij}-e_{ij})^2}{e_{ij}}$$
que tiene aproximadamente cuando $n$ es grande una distribución $\chi^2_{(k-1)(p-1)}$ si $H_0$ es cierta.

Rechazaremos $H_0$ a nivel de significación $\alpha$ si:
$$\sum_{j=1}^p \sum_{i=1}^k \frac{(O_{ij}-e_{ij})^2}{e_{ij}} > \chi^2_{(k-1)(p-1), 1-\alpha}$$
En caso contrario, aceptaremos $H_0$ a nivel de significación $\alpha$.

\begin{example}
    Una fábrica de automóviles quiere averiguar si la preferencia de modelo tiene relación con el sexo de los clientes.
    Se toman dos muestras aleatorias de 1000 hombres y 1000 mujeres, observándose las siguientes preferencias:
    \begin{center}
        \begin{tabular}{| c | c c |}
            \hline
            Modelo & Hombre & Mujer \\
            \hline
            $A$    & 340    & 350   \\
            $B$    & 400    & 270   \\
            $C$    & 260    & 380   \\
            \hline
        \end{tabular}
    \end{center}
    Veamos si son homogéneas las preferencias entre hombres y mujeres a nivel de significación $\alpha = 0.01$.

    Disponemos de una muestra aleatoria de 1000 mujeres y otra de 1000 hombres, clasificadas según sus preferencias por los modelos $A$, $B$ y $C$.
    El número total de datos es $n = 2000$.

    Para decidir si las preferencias en las dos poblaciones son homogéneas, planteamos el siguiente contraste de homogeneidad:
    $$\begin{cases}
            H_0: \text{ Las preferencias son homogéneas} \\
            H_1: \text{ Las preferencias no son homogéneas}
        \end{cases}$$

    La tabla de frecuencias observadas es:
    \begin{center}
        \begin{tabular}{| c | c c | c |}
            \hline
            $O_{ij}$ & Hombres & Mujeres &      \\
            \hline
            $A$      & 340     & 350     & 690  \\
            $B$      & 400     & 270     & 670  \\
            $C$      & 260     & 380     & 640  \\
            \hline
                     & 1000    & 1000    & 2000 \\
            \hline
        \end{tabular}
    \end{center}

    Las frecuencias esperadas se calculan de la forma:
    $$e_{ij} = \frac{(\sum_{i=1}^k O_{ij})(\sum_{j=1}^p O_{ij})}{n}$$
    obteniéndose la siguiente tabla de frecuencias esperadas:
    \begin{center}
        \begin{tabular}{| c | c c |}
            \hline
            $e_{ij}$ & Hombres & Mujeres \\
            \hline
            $A$      & 345     & 345     \\
            $B$      & 335     & 335     \\
            $C$      & 320     & 320     \\
            \hline
        \end{tabular}
    \end{center}

    Usaremos el estadístico de contraste:
    $$\sum_{j=1}^p \sum_{i=1}^k \frac{(O_{ij}-e_{ij})^2}{e_{ij}}$$
    que tiene aproximadamente una distribución $\chi^2_{(k-1)(p-1)}$ si $H_0$ es cierta.

    Rechazaremos $H_0$ a nivel de significación $\alpha$ si:
    $$\sum_{j=1}^p \sum_{i=1}^k \frac{(O_{ij}-e_{ij})^2}{e_{ij}} > \chi^2_{(k-1)(p-1), 1-\alpha}$$
    En nuestro caso:
    $$\sum_{j=1}^3 \sum_{i=1}^3 \frac{(O_{ij}-e_{ij})^2}{e_{ij}} = 47.87, \quad \chi^2_{2, 0.99} = 9.210$$

    Por tanto, rechazamos $H_0$ a nivel de significación $\alpha = 0.01$, es decir, podemos concluir que no son homogéneas las preferencias entre hombres y mujeres a nivel de significación $\alpha = 0.01$.
\end{example}

% Problema

\begin{exercise}
    Para probar la efectividad de una vacuna contra cierta enfermedad, se realizó un experimento observando el comportamiento de 100 personas vacunadas y 100 personas sin vacunar.
    Los resultados fueron los siguientes:
    \begin{center}
        \begin{tabular}{| c | c c |}
            \hline
                         & Enfermaron & No enfermaron \\
            \hline
            Vacunados    & 4          & 96            \\
            No vacunados & 6          & 94            \\
            \hline
        \end{tabular}
    \end{center}
    Queremos contrastar la efectividad de la vacuna con los datos obtenidos.

    Para ello consideramos el contraste:
    $$\begin{cases}
            H_0: \text{Las poblaciones son homogéneas, luego la vacuna no es efectiva} \\
            H_1: \text{La vacuna es efectiva}
        \end{cases}$$

    La tabla de frecuencias observadas es:
    \begin{center}
        \begin{tabular}{| c | c c | c |}
            \hline
            $O_{ij}$     & Enfermaron & No enfermaron &     \\
            \hline
            Vacunados    & 4          & 96            & 100 \\
            No vacunados & 6          & 94            & 100 \\
            \hline
                         & 10         & 190           & 200 \\
            \hline
        \end{tabular}
    \end{center}

    Las frecuencias esperadas se calculan de la forma:
    $$e_{ij} = \frac{(\sum_{i=1}^k O_{ij})(\sum_{j=1}^p O_{ij})}{n}$$
    obteniéndose la siguiente tabla de frecuencias esperadas:
    \begin{center}
        \begin{tabular}{| c | c c |}
            \hline
            $e_{ij}$     & Enfermaron & No enfermaron \\
            \hline
            Vacunados    & 5          & 95            \\
            No vacunados & 5          & 95            \\
            \hline
        \end{tabular}
    \end{center}

    Usaremos el estadístico de contraste:
    $$\sum_{j=1}^p \sum_{i=1}^k \frac{(O_{ij}-e_{ij})^2}{e_{ij}}$$
    que tiene aproximadamente una distribución $\chi^2_{(k-1)(p-1)}$ si $H_0$ es cierta.

    Rechazaremos $H_0$ a nivel de significación $\alpha$ si:
    $$\sum_{j=1}^p \sum_{i=1}^k \frac{(O_{ij}-e_{ij})^2}{e_{ij}} > \chi^2_{(k-1)(p-1), 1-\alpha}$$
    En nuestro caso:
    $$\sum_{j=1}^2 \sum_{i=1}^2 \frac{(O_{ij}-e_{ij})^2}{e_{ij}} = 0.4211, \quad \chi^2_{1, 0.95} = 3.8415$$

    Por tanto, aceptamos $H_0$ a nivel de significación $\alpha = 0.05$, es decir, podemos concluir que la vacuna no es eficaz a nivel de significación $\alpha = 0.05$.
\end{exercise}

% Problema

\section{Contrastes de independencia}
Supongamos que queremos estudiar si dos características $X$ e $Y$ de una población están relacionadas o no.
Para hacer este estudio, obtenemos una muestra aleatoria de $n$ pares de valores de estas características:
$$((X_1, Y_1), \dots, (X_n, Y_n))$$
Queremos ver si, a la vista de la muestra, tiene sentido admitir que $X$ e $Y$ son independientes.
Por tanto, tenemos el contraste:
$$\begin{cases}
        H_0: X \text{ e } Y \text{ son independientes} \\
        H_1: X \text{ e } Y \text{ no son independientes}
    \end{cases}$$

Tomamos una partición arbitraria del espacio muestral en $kp$ clases:
$$A_1 \times B_1, \dots, A_k \times B_p$$
Estas $kp$ clases corresponden a tomar las clases $A_1, \dots, A_k$ para la característica $X$ y las clases $B_1, \dots, B_p$ para la característica $Y$.

Llamamos:
\begin{itemize}
    \item $O_{ij}$: frecuencia observada en la clase $A_i \times B_j$.
    \item $e_{ij}$: frecuencia esperada en la clase $A_i \times B_j$ si la hipótesis nula es cierta, $nP(A_i)P(B_j)$.
\end{itemize}

El estadístico utilizado es:
$$\sum_{j=1}^p \sum_{i=1}^k \frac{(O_{ij}-e_{ij})^2}{e_{ij}}$$
que tiene aproximadamente cuando $n$ es grande una distribución $\chi^2_{(k-1)(p-1)}$ si $H_0$ es cierta.

\begin{remark}
    Este estadístico coincide con el que utilizábamos en el contraste de homogeneidad.
\end{remark}

Rechazaremos $H_0$ a nivel de significación $\alpha$ si:
$$\sum_{j=1}^p \sum_{i=1}^k \frac{(O_{ij}-e_{ij})^2}{e_{ij}} > \chi^2_{(k-1)(p-1), 1-\alpha}$$
En caso contrario, aceptaremos $H_0$ a nivel de significación $\alpha$.

\begin{example}
    Se desea evaluar la efectividad de una nueva vacuna antigripal. Para ello, se decide suministrar dicha vacuna, de manera voluntaria y gratuita, a una pequeña comunidad.
    La vacuna se administra en dos dosis, separadas por un período de dos semanas, de forma que alguna de las personas han recibido una sola dosis, otras han recibido las dos y otras personas no han recibido ninguna.
    La siguiente tabla muestra los resultados que se registraron durante la siguiente primavera en 1000 habitantes de la comunidad elegidos al azar:
    \begin{center}
        \begin{tabular}{| c | c c c |}
            \hline
                     & No vacunados & Una dosis & Dos dosis \\
            \hline
            Gripe    & 24           & 9         & 13        \\
            No gripe & 289          & 100       & 565       \\
            \hline
        \end{tabular}
    \end{center}
    Veamos si proporcionan estos datos suficiente evidencia estadística para indicar una dependencia entre la clasificación respecto a la vacuna y la protección frente a la gripe a nivel de significación $\alpha = 0.05$.

    Disponemos de una muestra aleatoria de $n = 1000$ habitantes clasificados según dos características: número de dosis recibidas y protección frente a la gripe.

    Para decidir si existe dependencia entre estas dos características, planteamos un contraste de independencia:
    $$\begin{cases}
            H_0: \text{ Las dos características son independientes} \\
            H_1: \text{ Existe dependencia entre las dos características}
        \end{cases}$$

    La tabla de frecuencias observadas es:
    \begin{center}
        \begin{tabular}{| c | c c c | c |}
            \hline
            $O_{ij}$ & No vacunados & Una dosis & Dos dosis &      \\
            \hline
            Gripe    & 24           & 9         & 13        & 46   \\
            No gripe & 289          & 100       & 565       & 954  \\
            \hline
                     & 313          & 109       & 578       & 1000 \\
            \hline
        \end{tabular}
    \end{center}

    Las frecuencias esperadas se calculan de la forma:
    $$e_{ij} = \frac{(\sum_{i=1}^k O_{ij})(\sum_{j=1}^p O_{ij})}{n}$$
    obteniéndose la siguiente tabla de frecuencias esperadas:
    \begin{center}
        \begin{tabular}{| c | c c c |}
            \hline
            $e_{ij}$ & No vacunados & Una dosis & Dos dosis \\
            \hline
            Gripe    & 14.4         & 5         & 26.6      \\
            No gripe & 298.6        & 104       & 551.4     \\
            \hline
        \end{tabular}
    \end{center}

    Usaremos el estadístico de contraste:
    $$\sum_{j=1}^p \sum_{i=1}^k \frac{(O_{ij}-e_{ij})^2}{e_{ij}}$$
    que tiene aproximadamente una distribución $\chi^2_{(k-1)(p-1)}$ si $H_0$ es cierta.

    Rechazaremos $H_0$ a nivel de significación $\alpha$ si:
    $$\sum_{j=1}^p \sum_{i=1}^k \frac{(O_{ij}-e_{ij})^2}{e_{ij}} > \chi^2_{(k-1)(p-1), 1-\alpha}$$
    En nuestro caso:
    $$\sum_{j=1}^3 \sum_{i=1}^2 \frac{(O_{ij}-e_{ij})^2}{e_{ij}} = 17.35, \quad \chi^2_{2, 0.95} = 5.991$$

    Por tanto, rechazamos $H_0$ a nivel de significación $\alpha = 0.05$, es decir, podemos concluir que existe dependencia entre el número de dosis recibidas y la protección frente a la gripe a nivel de significación $\alpha = 0.05$.
\end{example}

% Problema