\chapter{Funciones enteras. Crecimiento y distribución de los ceros}
\begin{theorem}[Teorema de Liouville]
    Si $f$ es una función entera y acotada, entonces $f$ es constante.
\end{theorem}

\begin{theorem}[Generalización del teorema de Liouville]
    Sea $f$ entera tal que existen $\alpha > 0$, $c > 0$ y $R_0 > 0$ con $|f(z)| \leq c|z|^\alpha$ para todo $|z| \geq R_0$.
    Entonces $f$ es un polinomio de grado $E(\alpha)$ y por tanto tiene $E(\alpha)$ ceros.
\end{theorem}

\begin{proof}
    Sea $f(z) = \sum_{n=0}^\infty a_nz^n$.
    Sea $R > 0$,
    $$a_n = \frac{f^{(n)}(0)}{n!} = \frac{1}{2\pi i} \int_{|z|=R} \frac{f(z)}{z^{n+1}}dz \leq \max_{|z|=R} |f(z)|\frac{1}{R^n} \leq c\frac{|z|^\alpha}{R} = cR^{\alpha-n} \xrightarrow[R \to \infty, n > E(\alpha)]{} 0$$
    Por tanto, $a_n = 0$ para todo $n > E(\alpha)$.
\end{proof}

\begin{remark}
    \hfill
    \begin{enumerate}
        \item Parece que si el crecimiento está controlado, el número de ceros también lo está.
        \item Al revés esto no ocurre.
              Por ejemplo, con la exponencial.
    \end{enumerate}
\end{remark}

\section{La fórmula de Jensen}
La fórmula de Jensen permite controlar el número de ceros de una función holomorfa sabiendo restricciones sobre su crecimiento.

\begin{lemma}
    Sea $s > 0$.
    Entonces
    $$\frac{1}{2\pi} \int_{-\pi}^\pi \log|1+se^{i\theta}|d\theta = \log^+(s)$$
    donde
    $$\log^+(s) = \begin{cases}
            \log(s) & \text{si } s \geq 1  \\
            0       & \text{si } 0 < s < 1
        \end{cases}$$
\end{lemma}

% Demostración

\begin{theorem}[Fórmula de Jensen]
    Sea $f$ una función holomorfa en $D(0, R)$ con $f(0) \neq 0$ y tal que tiene ceros $\{a_n\}$ de modo que
    $$|a_1| \leq |a_2| \leq |a_3| \leq \dots$$
    Sea $\rho \in (0, R)$ y $n(\rho, f) = \#\{a_n : |a_n| \leq \rho\}$.
    Entonces
    $$\frac{1}{2\pi} \int_{-\pi}^\pi \log|f(\rho e^{i\theta})|d\theta = \log|f(0)| + \sum_{k=1}^{n(\rho, f)} \log\left(\frac{\rho}{|a_k|}\right)$$
\end{theorem}

% Demostración

\begin{definition}
    Se llama función contadora de Nevanlinna a la cantidad
    $$N(\rho, f) = \sum_{\{n : |a_n| \leq \rho\}} \log\left(\frac{\rho}{|a_n|}\right)$$
\end{definition}

\begin{remark}
    \hfill
    \begin{enumerate}
        \item $$N(\rho, f) = \sum_{\{n : |a_n| \leq \rho\}} \log\left(\frac{\rho}{|a_n|}\right) = \sum_{k=1}^{n(\rho, f)} \frac{\rho}{|a_k|} = \sum_{k=1}^\infty \log^+\left(\frac{\rho}{|a_k|}\right)$$
        \item $$N(\rho, f) = \int_0^\rho \frac{n(t, f)}{t}dt$$
              % Demostración
    \end{enumerate}
\end{remark}

Existen diversas generalizaciones de la fórmula de Jensen.

\begin{theorem}
    Sea $f(z) = c_Nz^N + c_{N+1}z^{N+1} + \dots$ una función holomorfa en $D(0, R)$ y sea $\rho \in [0, R)$ tal que $\{a_n\}_{n=1}^\infty$ son los ceros de $f$ en $D(0, R)$ y además están ordenados respetando la multiplicidad.
    Entonces:
    $$\frac{1}{2\pi} \int_{-\pi}^\pi \log|f(\rho e^{i\theta})|d\theta = \log|c_N\rho^N| + \sum_{k=1}^{n(\rho, f)} \log\left(\frac{\rho}{|a_k|}\right)$$
\end{theorem}

% Demostración

Vamos a ver cómo aplicar la fórmula de Jensen para ver que si el crecimiento de $f$ está controlado entonces sus ceros también lo están.

\begin{definition}
    Llamamos módulo máximo de $f$ a
    $$M_\infty(\rho, f) = \sup_{|z|=\rho} |f(z)|$$
\end{definition}

\begin{remark}
    Si $f$ es holomorfa,
    $$M_\infty(\rho, f) = \max_{|z|=\rho} |f(z)| = \max_{|z|\leq\rho} |f(z)|$$
\end{remark}

\begin{theorem}
    Sea $f$ holomorfa en $D(0, R)$ tal que $f(0) \neq 0$ y $0 \leq s \leq \rho < R$.
    Entonces:
    $$n(s, f)\log\left(\frac{\rho}{s}\right) \leq \log(M_\infty(\rho, f)) - \log|f(0)|$$
\end{theorem}

\begin{proof}
    \begin{align*}
        n(s, f)\log\left(\frac{\rho}{s}\right) & = n(s, f)\int_s^\rho \frac{1}{t}dt \leq \int_s^\rho \frac{n(t, f)}{t}dt \leq \int_0^\rho \frac{n(t, f)}{t}dt =           \\
                                               & = \frac{1}{2\pi} \int_{-\pi}^\pi \log|f(\rho e^{i\theta})|d\theta - \log|f(0)| \leq \log(M_\infty(\rho, f)) - \log|f(0)|
    \end{align*}
\end{proof}

\begin{example}
    Sea $f$ entera con $|f(0)| = 1$ y sea $\rho = es$.
    Entonces:
    $$n(s, f) \leq \log(M_\infty(es, f))$$
\end{example}

\begin{remark}
    Si $f(z) = c_Nz^N + \dots$, el teorema anterior se puede escribir de la forma:
    $$n(s, f)\log\left(\frac{\rho}{s}\right) \leq \log(M_\infty(\rho, f)) - \log|c_N\rho^N|$$
\end{remark}

\section{La fórmula de Poisson-Jensen}
\begin{theorem}
    Sea $0 < R \leq \infty$ y sea $f$ holomorfa en $D(0, R)$.
    Sean $s \in (0, R)$ y $z \in D(0, s)$, entonces:
    $$\frac{1}{2\pi} \int_{-\pi}^\pi \log|f(se^{i\theta})\frac{s^2-|z|^2}{|s-ze^{-i\theta}|^2}d\theta = \log|f(z)| + \sum_{\{|a_n| \leq s, a_n \text{ cero de } f\}} \log^+\left|\frac{s^2-\bar{a}_nz}{s(z-a_n)}\right|$$
\end{theorem}