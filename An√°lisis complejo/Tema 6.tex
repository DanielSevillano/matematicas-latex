\chapter{El teorema de factorización de Weierstrass}
Si $P(z)$ es un polinomio con ceros $z_1, \dots, z_n$, entonces podemos factorizar $P(z)$ como
$$P(z) = c\prod_{k=1}^n (z-z_k)$$
El objetivo es factorizar una función holomorfa usando sus ceros.

\section{Funciones holomorfas sin ceros o con finitos ceros}
\begin{theorem}
    Sea $D$ un dominio simplemente conexo y sea $f$ una función holomorfa en $D$ sin ceros.
    Entonces existe $g$ holomorfa en $D$ tal que $f = e^g$.
\end{theorem}

\begin{theorem}
    Sea $D$ un dominio simplemente conexo y sea $f$ una función holomorfa en $D$ con un número finito de ceros $z_1, \dots, z_n$.
    Entonces existe $g$ holomorfa en $D$ tal que
    $$f(z) = e^{g(z)} \prod_{n=1}^N (z-z_n)$$
\end{theorem}

\begin{proof}
    Sea
    $$h(z) = \frac{f(z)}{(z-z_1)\dots(z-z_N)}$$
    Solucionando las singularidades evitables, $h$ es holomorfa en $D$ y sin ceros.
    Entonces por el teorema anterior existe $g$ holomorfa en $D$ tal que
    $$e^{g(z)} = h(z) = \frac{f(z)}{(z-z_1)\dots(z-z_N)} \Rightarrow f(z) = e^{g(z)} \prod_{n=1}^N (z-z_n)$$
\end{proof}

Desde otro punto de vista, sea $D$ un dominio simplemente conexo y $\{z_\alpha\}_{\alpha \in \mathcal{F}} \subset D$, podemos plantearnos si existe $f$ holomorfa en $D$ tal que $f$ tiene ceros $\{z_\alpha\}_{\alpha \in \mathcal{F}}$.

Si $\{z_\alpha\}_{\alpha \in \mathcal{F}}$ tiene un punto de acumulación en $D$ entonces, por el teorema de identidad de Weierstrass, $f \equiv 0$.
Nos interesa el caso en el que $\{z_n\}_{n=1}^\infty$ es numerable y sin puntos de acumulación en $D$.

Sea $D = \mathbb{C}$ y sea $\{z_n\}_{n=1}^\infty$ numerable y sin puntos de acumulación.
Habría que definir $\prod_{n=1}^\infty (z-z_n)$, por ejemplo de la forma
$$\prod_{n=1}^\infty (z-z_n) = \lim_{N \to \infty} \prod_{n=1}^N (z-z_n)$$
Como $\{z_n\}_{n=1}^\infty$ es infinito, entonces necesariamente $|z_n| \to \infty$.
En caso contrario, $\{z_n\}$ tendría un punto de acumulación en $\mathbb{C}$.
Si fijamos $z \in \mathbb{C}$, existe $n_0 \in \mathbb{N}$ tal que si $n \geq n_0$ entonces $|z-z_n| > 2$, así que
$$\lim_{N \to \infty} \prod_{n=n_0}^\infty |z-z_n| = \infty$$

\section{Productos infinitos}
Sea $\{a_n\}_{n=1}^\infty$ una sucesión de números complejos.
Queremos darle sentido a $\prod_{n=1}^\infty a_n$.
Por ejemplo, si $P_N = \prod_{n=1}^N a_n$, podemos definir
$$\prod_{n=1}^\infty a_n = \lim_{N \to \infty} P_N = \lim_{N \to \infty} \prod_{n=1}^N a_n$$
Sin embargo, esta definición plantea algunos problemas.
\begin{enumerate}
    \item Si tenemos una multiplicación de números complejos cuyo resultado es 0, queremos que uno de ellos sea cero.
          Sin embargo, si $a_n = \frac{1}{n}$, entonces $P_N = \prod_{n=1}^N \frac{1}{n} = \frac{1}{N!}$ y $\lim\limits_{n \to \infty} P_N = 0$, con $a_n \neq 0$ para todo $n \in \mathbb{N}$.

    \item Queremos que la convergencia depende de la cola.
          Sin embargo, con esta definición depende de un número finito de términos.

          Sea $a_1 = a$ y $a_n = n$ para $n \geq 2$, entonces $\lim\limits_{N \to \infty} P_N = 0$ porque $P_N = 0$ para todo $N \in \mathbb{N}$.
          Sin embargo, si $a_n = n+1$ para todo $n \geq 2$, entonces $\lim\limits_{N \to \infty} P_N = \infty$.
\end{enumerate}

\begin{definition}
    Sea $\{a_n\}_{n=1}^\infty$ una sucesión de números complejos.
    Diremos que el producto infinito asociado a $\{a_n\}_{n=1}^\infty$, que denotamos por $\prod_{n=1}^\infty a_n$, converge si:
    \begin{enumerate}
        \item Existe $n_0 \in \mathbb{N}$ tal que $a_n \neq 0$ para todo $n \geq n_0$.
        \item Existe $\lim\limits_{N \to \infty} \prod_{n=1}^N a_n$ y además es distinto de cero.
    \end{enumerate}
    Si converge, entonces $\prod_{n=1}^\infty a_n = \lim\limits_{N \to \infty} \prod_{n=1}^N a_n$.
\end{definition}

\begin{example}
    \hfill
    \begin{enumerate}
        \item Sea $a_n = \frac{1}{n}$, $n \in \mathbb{N}$.
              Observamos que $a_n \neq 0$ para todo $n \in \mathbb{N}$.
              Sin embargo,
              $$\lim_{N \to \infty} \prod_{n=1}^N a_n = \lim_{N \to \infty} \frac{1}{N!} = 0$$
              Por tanto, $\prod_{n=1}^\infty a_n$ no converge.

        \item Sea $a_1 = 0$ y $a_n = 1 - \frac{1}{n}$, $n \geq 2$.
              Se verifica que $a_n \neq 0$ para $n \geq 2$.
              Ahora bien,
              $$\prod_{n=2}^N \left(1-\frac{1}{n}\right) = \prod_{n=2}^N \frac{n-1}{n} = \frac{1}{2}\frac{2}{3}\dots\frac{N-1}{N} = \frac{1}{N} \xrightarrow[N \to \infty]{} 0$$
              Por tanto, $\prod_{n=1}^\infty a_n$ no converge.

        \item Sea $a_1 = 0$ y $a_n = 1 - \frac{1}{n^2}$, $n \geq 2$.
              Es claro que $a_n \neq 0$ para $n \geq 2$.
              Además,
              \begin{align*}
                   & \prod_{n=2}^N \left(1-\frac{1}{n^2}\right) = \prod_{n=2}^N \frac{n^2-1}{n^2} = \prod_{n=2}^N \frac{(n-1)(n+1)}{n^2} = \left(\prod_{n=2}^N \frac{n-1}{n}\right)\left(\prod_{n=2}^N \frac{n+1}{n}\right) = \\
                   & = \frac{1}{N}\frac{N+1}{2} \to \frac{1}{2}
              \end{align*}
              Por tanto, $\prod_{n=1}^\infty a_n$ converge.
    \end{enumerate}
\end{example}

\begin{theorem}
    Sea $\{a_n\}_{n=1}^\infty$ una sucesión de números complejos.
    Entonces $\prod_{n=1}^\infty a_n$ converge si ocurre lo siguiente.
    \begin{enumerate}
        \item El conjunto $\{a_n : a_n = 0\}$ es finito.
        \item Si existe $M \in \mathbb{N}$ tal que $a_n \neq 0$ para todo $n \geq M$, entonces existe $\lim\limits_{N \to \infty} \prod_{n=M}^N a_n$ y es distinto de 0.
    \end{enumerate}
\end{theorem}

\begin{theorem}
    Sea $\{a_n\}_{n=1}^\infty$ tal que $\prod_{n=1}^\infty a_n$ converge.
    Entonces:
    \begin{enumerate}
        \item $\lim\limits_{n \to \infty} a_n = 1$.
        \item $\prod_{n=1}^\infty a_n = 0$ si y solo si existe $n_0 \in \mathbb{N}$ tal que $a_{n_0} = 0$.
        \item Sea $N \in \mathbb{N}$.
              Entonces $\prod_{n=1}^\infty a_{n+N}$ converge y además
              $$\prod_{n=1}^\infty a_n = \left(\prod_{n=1}^N a_n\right) \left(\prod_{n=1}^\infty a_{n+N}\right)$$
        \item Sea $\{b_n\}_{n=1}^\infty$ tal que $\prod_{n=1}^\infty b_n$ converge.
              Entonces $\prod_{n=1}^\infty a_nb_n$ converge y además
              $$\prod_{n=1}^\infty a_nb_n = \left(\prod_{n=1}^\infty a_n\right) \left(\prod_{n=1}^\infty b_n\right)$$
    \end{enumerate}
\end{theorem}

\begin{proof}
    \hfill
    \begin{enumerate}
        \item Como $\prod_{n=1}^\infty a_n$ converge, existe $n_0 \in \mathbb{N}$ tal que $a_n \neq 0$ para todo $n \geq n_0$ y $\lim\limits_{N \to \infty} \prod_{n=n_0}^N a_n = q$, con $q > 0$.
              Si $M > n_0$,
              $$a_M = \frac{\prod_{n=n_0}^M a_n}{\prod_{n=n_0}^{M-1} a_n} \xrightarrow[M \to \infty]{} \frac{q}{q} = 1$$

        \item Supongamos que existe $n_0 \in \mathbb{N}$ tal que $a_{n_0} = 0$.
              Como $\prod_{n=1}^\infty a_n$ converge, entonces
              $$\prod_{n=1}^\infty a_n = \lim_{N \to \infty} \prod_{n=1}^N a_n = \lim_{N \to \infty} P_N \to 0$$
              Si $N > n_0$, entonces $P_N = a_1\dots a_{n_0-1}0a_{n_0+1}\dots a_N = 0$.

              Recíprocamente, si $a_n \neq 0$ para todo $n \in \mathbb{N}$, entonces $\prod_{n=1}^\infty a_n \neq 0$ y $\prod_{n=1}^\infty a_n$ converge.

        \item Tomemos $N \in \mathbb{N}$.
              Si $\prod_{n=1}^\infty a_n$ converge, entonces existe $n_0 \in \mathbb{N}$ tal que $a_n \neq 0$ para todo $n \geq n_0$ y $\lim\limits_{M \to \infty} \prod_{n=n_0}^M a_n = q$ con $q \neq 0$.
              Entonces $a_{n+N} \neq 0$ para todo $n \geq n_0$ y además
              $$\prod_{n=n_0}^M a_{n+N} = \prod_{n=n_0+N}^{M+N} a_n = \frac{\prod_{n=n_0}^{M+N} a_n}{a_{n_0}\dots a_{n_0+N-1}} \to \frac{q}{a_{n_0}\dots a_{n_0+N-1}} \neq 0$$
              Por tanto, $\prod_{n=1}^\infty a_{n+N}$ converge.

        \item $\prod_{n=1}^\infty a_n$ y $\prod_{n=1}^\infty b_n$ convergen.
              Entonces:
              \begin{itemize}
                  \item Existe $n_a \in \mathbb{N}$ tal que $a_n \neq 0$ para todo $n \geq n_a$ y $\lim\limits_{N \to \infty} \prod_{n=n_a}^N a_n = l_a \neq 0$.
                  \item Existe $n_b \in \mathbb{N}$ tal que $b_n \neq 0$ para todo $n \geq n_b$ y $\lim\limits_{N \to \infty} \prod_{n=n_b}^N b_n = l_b \neq 0$.
              \end{itemize}
              Sea $n_0 = \max\{n_a, n_b\}$.
              Entonces $a_nb_n \neq 0$ para todo $n \geq n_0$.
              $$\lim_{N \to \infty} \prod_{n=n_0}^N a_nb_n \to c \neq 0$$
              Como los productos convergen,
              $$\prod_{n=1}^\infty a_nb_n = \lim_{N \to \infty} \left(\prod_{n=1}^N a_n\right) \left(\prod_{n=1}^N b_n\right) = \left(\lim_{N \to \infty} \prod_{n=1}^N a_n\right) \left(\lim_{N \to \infty} \prod_{n=1}^N b_n\right) = \left(\prod_{n=1}^\infty a_n\right) \left(\prod_{n=1}^\infty b_n\right)$$
    \end{enumerate}
\end{proof}

\begin{theorem}
    Sea $\{a_n\}_{n=1}^\infty \subset \mathbb{C}$, entonces son equivalentes:
    \begin{enumerate}
        \item $\prod_{n=1}^\infty a_n$ converge.
        \item Existe $n_0 \in \mathbb{N}$ tal que $a_n \neq 0$ para todo $n \geq n_0$ y $\sum_{n=n_0}^\infty \Log(a_n)$ converge.
    \end{enumerate}
\end{theorem}

\begin{proof}
    \hfill
    \begin{itemize}
        \item[$\Rightarrow$] Como $\prod_{n=1}^\infty a_n$ converge, existe $n_0 \in \mathbb{N}$ tal que $a_n \neq 0$ para todo $n \geq n_0$ y además $\prod_{n=n_0}^N \to q$ con $q \neq 0$.
            Sea $S_N = \sum_{n=n_0}^N \Log(a_n)$.
            Entonces:
            $$e^{S_N} = e^{\sum_{n=1}^N \Log(a_n)} = \prod_{n=n_0}^N a_n = q_N \Rightarrow S_N \in \log(q_n) \Rightarrow S_N = \Log(a_n) + 2\pi k_Ni, \quad k_N \in \mathbb{Z}$$
            Distinguimos dos casos:
            \begin{enumerate}
                \item Supongamos que $q \notin (-\infty, 0)$.
                      Como $q_N \to q$, entonces existe $N_0 \in \mathbb{N}$ tal que $q_N \notin (-\infty, 0]$ para todo $N \geq N_0$.
                      Así que $\lim\limits_{N \to \infty} \Log(q_N) = \Log(q)$.
                      $$\begin{cases}
                              S_{N+1}-S_N = \Log(q_{N+1})-\Log(q_N) + 2\pi(k_{N+1}-k_N)i \\
                              S_{N+1}-S_N = \Log(a_{N+1}) \xrightarrow[N \to \infty]{} 0
                          \end{cases} \Rightarrow \lim_{N \to \infty} (k_{N+1}-k_N) = 0$$
                      Como $k \in \mathbb{Z}$, entonces existe $j \in \mathbb{N}$ tal que $k_N = k$ para todo $N \geq j$.

                \item Supongamos que $q \in (-\infty, 0)$.
                      Definimos una nueva sucesión $\tilde{a}_{n_0} = -a_{n_0}$ y $\tilde{a}_n = a_n$ para $n > n_0$.
                      $$\tilde{q}_N = \prod_{n=n_0}^N \tilde{a}_n = -\prod_{n=n_0}^N a_n \to -q > 0$$
                      Por el caso anterior. $\tilde{S}_N = \sum_{n=n_0}^N \Log(\tilde{a}_n)$ converge.
                      Equivalentemente, $S_N$ converge.
            \end{enumerate}

        \item[$\Leftarrow$] Falta ver que $\prod_{n=n_0}^N a_n \to q \neq 0$.
            Sabemos que $S_N = \sum_{n=n_0}^N \Log(a_n) \to p$ y $q_N = e^{S_N}$.
            Tomando límites,
            $$\lim_{N \to \infty} q_N = \lim_{N \to \infty} e^{S_N} = e^{\lim_{N \to \infty} S_N} = e^p = q \neq 0$$
    \end{itemize}
\end{proof}

\begin{corollary}
    Sea $\{a_n\}_{n=1}^\infty \subset \mathbb{C} \setminus \{0\}$, entonces son equivalentes:
    \begin{enumerate}
        \item $\prod_{n=1}^\infty a_n$ converge.
        \item $\sum_{n=1}^\infty \Log(a_n)$ converge.
    \end{enumerate}
    Además,
    $$\prod_{n=1}^\infty a_n = e^{\sum_{n=1}^\infty \Log(a_n)}$$
\end{corollary}

Queremos encontrar una noción de convergencia absoluta.
Para las series sabemos lo siguiente.

\begin{theorem}
    Sea $\{a_n\}_{n=1}^\infty \subset \mathbb{C}$, entonces son equivalentes:
    \begin{enumerate}
        \item $\sum_{n=1}^\infty a_n$ converge absolutamente.
        \item Dada $\sigma: \mathbb{N} \to \mathbb{N}$ permutación, $\sum_{n=1}^\infty a_{\sigma(n)}$ converge y es igual a $\sum_{n=1}^\infty a_n$.
        \item Sea $\{A_n\}$ una partición de $\mathbb{N}$ con infinitos elementos.
              Entonces $\sum_{k \in A_n} a_k$ converge y además
              $$\sum_{n=1}^\infty a_n = \sum_{n=1}^\infty \left(\sum_{k \in A_n} a_n\right)$$
    \end{enumerate}
\end{theorem}

Como consecuencia, si $\sum_{n=1}^\infty a_n$ converge absolutamente, en particular converge.

Uniendo los teoremas anteriores, obtenemos el siguiente resultado.

\begin{theorem}
    Sea $\{a_n\}_{n=1}^\infty \subset \mathbb{C} \setminus \{0\}$.
    Entonces son equivalentes:
    \begin{enumerate}
        \item $\sum_{n=1}^\infty \Log(a_n)$ converge absolutamente.
        \item Dada $\sigma: \mathbb{N} \to \mathbb{N}$ permutación, $\prod_{n=1}^\infty a_{\sigma(n)}$ converge y su valor es
              $$\prod_{n=1}^\infty a_n = e^{\sum_{n=1}^\infty \Log(a_n)}$$
        \item Sea $\{A_n\}$ una partición de $\mathbb{N}$ con infinitos elementos.
              Entonces $\prod_{k \in A_n} a_k$ converge y además
              $$\prod_{n=1}^\infty a_n = \prod_{n=1}^\infty \left(\prod_{k \in A_n} a_n\right)$$
    \end{enumerate}
\end{theorem}

\begin{lemma}
    Sea $\{a_n\}_{n=1}^\infty \subset \mathbb{C} \setminus \{0\}$, entonces son equivalentes:
    \begin{enumerate}
        \item $\sum_{n=1}^\infty \Log(a_n)$ converge absolutamente.
        \item $\sum_{n=1}^\infty (1-a_n)$ converge absolutamente.
    \end{enumerate}
\end{lemma}

\begin{proof}
    Empecemos por $\frac{1}{1-z} = \sum_{n=0}^\infty z^n$, entonces
    $$\Log(1-z) = -\sum_{n=0}^\infty \frac{z^{n+1}}{n+1}$$
    Además,
    $$\Log(1+z) = \sum_{n=0}^\infty \frac{(-1)^nz^{n+1}}{n+1}$$
    Queremos analizar
    $$\left|\frac{\Log(1+z)}{z} - 1\right| = \left|\sum_{n=0}^\infty \frac{(-1)^nz^{n+1}}{n+1} - 1\right| = \left|\sum_{n=1}^\infty \frac{(-1)^nz^n}{n+1}\right| \leq \frac{1}{2}\sum_{n=1}^\infty |z|^n = \frac{1}{2}\frac{|z|}{1-|z|}$$
    Si $|z| < \frac{1}{2}$, entonces
    $$\left|\frac{\Log(1+z)}{z} - 1\right| = \frac{1}{2}\frac{|z|}{1-|z|} \leq \frac{1}{2}$$
    Así que
    $$\left|\left|\frac{\Log(1+z)}{z}\right| - 1\right| \leq \left|\frac{\Log(1+z)}{z} - 1\right| \leq \frac{1}{2} \Rightarrow \frac{1}{2} \leq \left|\frac{\Log(1+z)}{z}\right| \leq \frac{3}{2}$$
    Por último, sea $z = w-1$, entonces si $|w-1| = |z| \leq \frac{1}{2}$,
    $$\frac{1}{2} \leq \left|\frac{\Log(w)}{1-w}\right| \leq \frac{3}{2}$$
    Basta ver que existe $n_0 \in \mathbb{N}$ tal que $|1-a_n| < \frac{1}{2}$.
    \begin{itemize}
        \item[$\Leftarrow$] Como $\sum_{n=1}^\infty (1-a_n)$ converge absolutamente, $a_n \to 1$.
            Por tanto, existe $n_0 \in \mathbb{N}$ tal que $|1-a_n| \leq \frac{1}{2}$ para todo $n \geq n_0$.
        \item[$\Rightarrow$] Análogo.
    \end{itemize}
\end{proof}

\begin{theorem}
    Sea $\{a_n\}_{n=1}^\infty \subset \mathbb{C}$, entonces son equivalentes:
    \begin{enumerate}
        \item $\sum_{n=1}^\infty (1-a_n)$ converge absolutamente.
        \item Existe $n_0 \in \mathbb{N}$ tal que para todo $n \geq n_0$ se tiene que $|1-a_n| \leq \frac{1}{2}$ y $\sum_{n=n_0}^\infty \Log(a_n)$ converge uniformemente.
        \item Para cada permutación $\sigma: \mathbb{N} \to \mathbb{N}$ se tiene que $\prod_{n=1}^\infty a_{\sigma(n)}$ converge y además
              $$\prod_{n=1}^\infty a_{\sigma(n)} = \prod_{n=1}^\infty a_n$$
        \item Sea $\{A_n\}$ una partición de $\mathbb{N}$ con infinitos elementos.
              Entonces $\prod_{k \in A_n} a_k$ converge y además
              $$\prod_{n=1}^\infty a_n = \prod_{n=1}^\infty \left(\prod_{k \in A_n} a_n\right)$$
    \end{enumerate}
\end{theorem}

\begin{definition}[Convergencia absoluta]
    Sea $\{a_n\}_{n=1}^\infty \subset \mathbb{C}$, su producto infinito asociado converge absolutamente si $\sum_{n=1}^\infty (1-a_n)$ converge absolutamente.
\end{definition}

\begin{remark}
    \hfill
    \begin{enumerate}
        \item Si $\prod_{n=1}^\infty a_n$ converge absolutamente, entonces converge.
        \item Si $\prod_{n=1}^\infty a_n$ converge absolutamente, entonces $\prod_{n=1}^\infty \left(1 + |1-a_n|\right)$ converge absolutamente.
              \begin{proof}
                  $\prod_{n=1}^\infty \left(1 + |1-a_n|\right)$ converge absolutamente si y solo si $\prod_{n=1}^\infty \left(1 - 1 - |1-a_n|\right) = -\prod_{n=1}^\infty |1-a_n|$ converge absolutamente.
              \end{proof}
    \end{enumerate}
\end{remark}

\section{Funciones holomorfas definidas por productos infinitos}
\begin{theorem}
    Sea $\{f_n\}_{n=1}^\infty$ una sucesión de funciones holomorfas en un dominio $D$.
    Entonces son equivalentes:
    \begin{enumerate}
        \item $\sum_{n=1}^\infty (1-f_n)$ es absoluta y uniformemente convergente en compactos de $D$.
        \item Para cada compacto $K$ en $D$ existe $n_K \in \mathbb{N}$ tal que para todo $n \geq n_K$ se tiene que $|1-f_n(z)| \leq \frac{1}{2}$ y $\sum_{n=n_K}^\infty \Log(f_n)$ converge absoluta y uniformemente en $K$.
    \end{enumerate}
\end{theorem}

\begin{lemma}
    Sea $\{u_k\}_{k=1}^\infty \subset \mathbb{C}$.
    Entonces
    $$\left|\prod_{k=1}^n (1+u_k)-1\right| \leq \prod_{k=1}^n (1+|u_k|)-1, \quad \forall n \in \mathbb{N}$$
\end{lemma}

\begin{proof}
    Razonamos por inducción.
    \begin{itemize}
        \item Para $n = 1$,
              $$|1+u-1-1| = |u_1| \leq |u_1|$$

        \item Supongamos que es cierto para $n$ y veamos para el caso $n+1$.
              \begin{align*}
                   & \left|\prod_{k=1}^{n+1} (1+u_k) - 1\right| = \left|(1+u_{n+1})\prod_{k=1}^{n+1} (1+u_k) - 1\right| = \left|(1+u_{n+1})\left(\prod_{k=1}^\infty (1+u_k) - 1\right) + u_{n+1}\right| \leq \\
                   & \leq |1+u_{n+1}|\left|\prod_{k=1}^\infty (1+u_k) - 1\right| + |u_{n+1}| \leq |1+u_{n+1}|\left|\prod_{k=1}^\infty (1+|u_k|) - 1\right| + |u_{n+1}| \leq                                  \\
                   & \leq (1 + |u_{n+1}|)\left|\prod_{k=1}^\infty (1+|u_k|) - 1\right| + |u_{n+1}| = \prod_{k=1}^{n+1} (1+|u_k|) - 1
              \end{align*}
    \end{itemize}
\end{proof}

\begin{theorem}
    Sea $\{f_n\}_{n=1}^\infty$ una sucesión de funciones holomorfas en $D$.
    Si $\sum_{n=1}^\infty (1-f_n)$ converge absoluta y uniformemente en compactos de $D$, entonces:
    \begin{enumerate}
        \item $\prod_{k=1}^n$ converge uniformemente en cada compacto de $D$ a una función holomorfa $P$ que denotamos por $\prod_{n=1}^\infty f_n$.
        \item La convergencia de $\prod_{k=1}^n f_k$ no depende de reordenaciones.
        \item Para cada $z_0 \in D$ tenemos que $ord(z_0, P) = \sum_{k=1}^\infty ord(z_0, f_k)$, donde $ord(z_0, f)$ es el orden de $z_0$ como cero de $f$.
        \item Las derivadas logarítmicas $\frac{P'}{P}$ y $\frac{f_n'}{f_n}$ existen como funciones meromorfas.
              Además,
              $$\frac{P'}{P} = \sum_{k=1}^\infty \frac{f_k'}{f_k}$$
              Si $R > 0$ y $\overline{D(z_0, R)} \subset D$ existe $n_0 \in \mathbb{N}$ tal que $f_n$ no tiene ceros en $\overline{D(z_0, R)}$ para todo $n \geq n_0$ y $\sum_{n=n_0+1}^\infty \frac{f_n'}{f_n}$ converge absoluta y uniformemente en $\overline{D(z_0, R)}$.
    \end{enumerate}
\end{theorem}

\begin{proof}
    \hfill
    \begin{enumerate}
        \item Sea $K$ compacto de $D$ y sea $P_n = \prod_{k=1}^n f_k$.
              Basta ver que $P_n$ es uniformemente de Cauchy.

              En primer lugar, veamos que $P_n$ es uniformemente acotado en $K$.
              Como $\sum_{n=1}^\infty (1-f_n)$ converge absoluta y uniformemente en $K$, entonces existe $c_K$ tal que $\sum_{n=1}^\infty |1-f_n(z)| < c_K$.
              Entonces:
              \begin{align*}
                  |P_n| & = \left|\prod_{k=1}^n f_k\right| = \left|\prod_{k=1}^n (1+f_k-1)\right| \leq \prod_{k=1}^n (1+|1-f_k|) \leq \prod_{k=1}^n e^{|1-f_k|} = e^{\sum_{k=1}^n |1-f_k|} \leq \\
                        & \leq e^{\sum_{k=1}^\infty |1-f_k|} < e^{c_K}
              \end{align*}

              Ahora, veamos que $\{f_n\}$ es uniformemente de Cauchy en $K$.
              Sea $\varepsilon > 0$, existe $\delta > 0$ de modo que si $|x| < \delta$ entonces $e^{c_K}|e^x-1| < \varepsilon$.
              Como $\sum_{n=1}^\infty (1-f_n)$ es de Cauchy uniformemente en $K$, existe $n_K$ tal que si $m \leq n \leq n_K$, entonces $\sum_{k=n}^m |1-f_k| < \delta$.
              Por tanto, si $m \geq n \geq n_K$,
              \begin{align*}
                  |P_m - P_n| & = \left|\prod_{k=1}^m f_k - \prod_{k=1}^n f_k\right| = |P_n|\left|\left(\prod_{k=n+1}^m f_k\right) - 1\right| \leq e^{c_K}\left|\prod_{k=n+1}^m (1+f_k-1) - 1\right| \leq \\
                              & \leq e^{c_K}\left(\prod_{k=n+1}^m (1+|1-f_k|) - 1\right) \leq e^{c_K}\left(e^{\sum_{k=n+1}^m |1-f_k|}-1\right) < \varepsilon
              \end{align*}

        \item Es consecuencia de (1).

        \item Sabemos que existe $n_0 \in \mathbb{N}$ tal que $\sum_{n=n_0}^\infty |1-f_n(z)| < \frac{1}{2}$ para todo $z \in \overline{D(z_0, R')}$, con $\overline{D(z_0, R)} \subset D(z_0, R')$.
              Entonces, en particular $|1-f_n(z)| < \frac{1}{2}$ para todo $n \geq n_0+1$.
              Así que $f_n$ no se anula en $\overline{D(z_0, R')}$ para todo $n \geq n_0+1$.
              Sea entonces $g_n = \prod_{k=n_0+1}^n f_k$.
              Sabemos que $g_n$ converge uniformemente en $\overline{D(z_0, R')}$ a una función $g$.
              \begin{align*}
                  |g_n(z)-1| & = \left|\left(\prod_{k=n_0+1}^n f_k(z)\right)-1\right| = \left|\prod_{k=n_0+1}^n (1+f_k(z)-1)-1\right| \leq                              \\
                             & \leq \prod_{k=n_0+1}^n (1+|f_k(z)-1|)-1 \leq e^{\sum_{k=n_0}^n |1-f_k(z)|} - 1 \leq e^{1/2}-1, \quad \forall z \in \overline{D(z_0, R')}
              \end{align*}
              Por tanto, $g$ no tiene ceros en $\overline{D(z_0, R')}$.
              Entonces
              $$P = f_1\dots f_{n_0}\prod_{n=n_0+1}^\infty f_n = f_1\dots f_{n_0}g$$
              Como $g$ es no nula,
              $$ord(z_0, P) = \sum_{n=1}^{n_0} ord(z_0, f_n) + ord(z_0, g) = \sum_{n=1}^\infty ord(z_0, f_n)$$

        \item De la expresión anterior tenemos que
              $$\frac{P'}{P} = \sum_{n=1}^{n_0} \frac{f_n'}{f_n} + \frac{g'}{g}$$
              Falta ver que $\sum_{n=n_0+1}^N \frac{f_n'}{f_n} \to \frac{g'}{g}$.
              Como $\sum_{n=n_0+1}^N f_k = \frac{g_N'}{g_N}$, podemos ver equivalentemente que $\frac{g_N'}{g_N} \to \frac{g'}{g}$ uniformemente en $\overline{D(0, R')}$.
              \begin{align*}
                  \left|\frac{g_N'(z)}{g_N(z)} - \frac{g'(z)}{g(z)}\right| & = \left|\frac{g_N'(z)g(z)-g'(z)g_N(z)}{g_N(z)g(z)}\right| =                              \\
                                                                           & = \left|\frac{g_N'(z)g(z) - g(z)g'(z) + g(z)g'(z) - g'(z)g_N(z)}{g_N(z)g(z)}\right| \leq \\
                                                                           & \leq \frac{|g(z)||g_N'(z)-g'(z)| + |g'(z)||g_N(z)-g(z)|}{|g_N(z)||g(z)|}
              \end{align*}
              Sabemos que
              \begin{align*}
                  ||g_n(z)|-1| \leq |g_n(z)-1| < e^{1/2}-1 & \Leftrightarrow 2-e^{1/2} < |g_n(z)| < e^{1/2} \\
                  ||g(z)|-1| \leq |g(z)-1| < e^{1/2}-1     & \Leftrightarrow 2-e^{1/2} < |g(z)| < e^{1/2}   \\
              \end{align*}
              Además, existe $c_K = \max_{z \in \overline{D(z_0, R')}} (|g(z)|+|g'(z)|)$.
              Por tanto,
              $$\left|\frac{g_N'(z)}{g_N(z)}-\frac{g'(z)}{g(z)}\right| \to 0$$
              uniformemente en $\overline{D(z_0, R')}$.
    \end{enumerate}
\end{proof}

\begin{example}
    \hfil
    \begin{enumerate}
        \item $\prod_{n=1}^\infty \left(1-\frac{z}{n^2}\right)$.
              Veamos que converge uniformemente en compactos de $\mathbb{C}$.

              Sea $K$ compacto de $\mathbb{C}$.
              Entonces existe $R > 0$ tal que $K \subset \overline{D(0, R)}$.
              $$\sum_{n=1}^\infty \left|1-\left(1-\frac{z}{n^2}\right)\right| = \sum_{n=1}^\infty \frac{|z|}{n^2} \leq R\sum_{n=1}^\infty \frac{1}{n^2} \leq Rc$$
              Por el teorema anterior, el producto converge uniformemente en $K$.
              Además, $\prod_{n=1}^\infty \left(1-\frac{z}{n^2}\right)$ se anula en $\{n^2 : n \in \mathbb{N}\}$.

        \item Busquemos una función entera que se anule en $\mathbb{Z}$ y cuyos ceros tengan orden 1.
              Consideramos $P(z) = z\prod_{n=1}^\infty \left(1-\frac{z^2}{n^2}\right)$.
              Sea $K$ un compacto en $\mathbb{C}$, existe $R > 0$ tal que $K \subset \overline{D(0, R)}$.
              $$\sum_{n=1}^\infty \left|\frac{z^2}{n^2}\right| = \sum_{n=1}^\infty \frac{|z|^2}{n^2} \leq R^2c$$
              Por tanto, converge uniformemente en $K$.

              La función $z \mapsto \sin(\pi z)$ tiene las mismas características que buscábamos en $P$.
              Por tanto, la función
              $$z \mapsto \frac{\sin(\pi z)}{P(z)} = \frac{\sin(\pi z)}{z\prod_{n=1}^\infty \left(1-\frac{z^2}{n^2}\right)}$$
              es una función holomorfa sin ceros y se puede factorizar de la forma
              $$\frac{\sin(\pi z)}{z\prod_{n=1}^\infty \left(1-\frac{z^2}{n^2}\right)} = e^{\varphi(z)}$$
    \end{enumerate}
\end{example}

\section{El teorema de factorización de Weierstrass}
Sea $\{z_\alpha\}_{\alpha \in \mathcal{F}}$ una colección de puntos de $\mathbb{C}$.
Nuestro objetivo era encontrar una función entera que se anule en $\{z_\alpha\}_{\alpha \in \mathcal{F}}$.
\begin{itemize}
    \item Si $\{z_k\}_{k=1}^N$ es finita, esta función es $\prod_{n=1}^N (z-z_n)$.
    \item Si $\{z_\alpha\}_{\alpha \in \mathcal{F}}$ tiene un punto de acumulación, entonces la función tiene que ser nula.
\end{itemize}
En el caso restante, $\{z_k\}_{k=1}^\infty$ tiene $\lim\limits_{k \to \infty} |z_k| = \infty$.
Consideramos la expresión
$$\prod_{k=1}^\infty \left(1 - \frac{z}{z_k}\right)$$
Si $z \in \overline{D(0, R)}$ con $R > 0$,
$$\sum_{k=1}^\infty \left|1-\left(1-\frac{z}{z_k}\right)\right| = \sum_{k=1}^\infty \frac{|z|}{|z_k|} \leq R\sum_{k=1}^\infty \frac{1}{|z_k|}$$

\begin{theorem}
    Sea $\{z_k\}_{k=1}^\infty \subset \mathbb{C} \setminus \{0\}$ tal que $\sum_{k=1}^\infty \frac{1}{|z_k|}$ converge y $\lim\limits_{k \to \infty} |z_k| = \infty$.
    Entonces $\prod_{k=1}^\infty \left(1-\frac{z}{z_k}\right)$ converge absoluta y uniformemente en compactos de $\mathbb{C}$ y además tiene como ceros $\{z_k\}_{k=1}^\infty$.
\end{theorem}

\begin{remark}
    La función $z^N\prod_{n=1}^\infty \left(1-\frac{z}{z_n}\right)$ tiene como ceros $\{z_n\}_{n=1}^\infty$ y además el 0 es un cero de multiplicidad $N$.
\end{remark}

Falta por ver qué ocurre cuando $\sum_{k=1}^\infty \frac{1}{|z_k|}$ no converge, donde $\{z_k\}_{k=1}^\infty$ con $\lim\limits_{k \to \infty} |z_k| = \infty$.
Consideramos
$$\prod_{k=1}^\infty \left(1-\frac{z}{z_k}\right)e^{q_k\left(\frac{z}{z_k}\right)}$$

Veamos un razonamiento intuitivo.
Que el producto absoluto converja absoluta y uniformemente en compactos de $\mathbb{C}$ es análogo a que converja la serie
$$\sum_{k=1}^\infty \left|\Log\left(\left(1-\frac{z}{z_k}\right)e^{q_k\left(\frac{z}{z_k}\right)}\right)\right| = \sum_{k=1}^\infty \left|\Log\left(1-\frac{z}{z_k}\right) + q_k\left(\frac{z}{z_k}\right)\right|$$
Si conseguimos que
$$\sup_{|z|\leq R} \left|\Log\left(1-\frac{z}{z_k}\right) + q_k\left(\frac{z}{z_k}\right)\right| < M_K(R)$$
y además $\sum_{k=1}^\infty M_K(R) < \infty$, por el criterio de la mayorante de Weierstrass se tiene la convergencia.
Para $k \geq k_0$, tenemos que
$$\frac{|z|}{|z_k|} \leq \frac{R}{2R} = \frac{1}{2} < 1$$
Sea $w = \frac{z}{z_k}$,
$$|\Log(1-w)+q_k(w)| = \left|-\Log\left(\frac{1}{1-w}\right) + q_k(w)\right| = \left|\sum_{n=1}^\infty \frac{w^n}{n} - q_k(w)\right|$$
Entonces, si $q_k(w) = \sum_{n=1}^k \frac{w^n}{n}$,
$$\left|\sum_{n=1}^\infty \frac{w^n}{n} - q_k(w)\right| = \left|\sum_{n=k+1}^\infty \frac{w^n}{n}\right| \leq \frac{1}{k+1}\sum_{n=k+1}^\infty |w|^n = \frac{|w|^{k+1}}{k+1}\frac{1}{1-|w|}$$
Tenemos que $\frac{1}{1-|w|} < 2$.
Por tanto, queremos que
$$\sum_{k=k_0}^\infty \frac{|w|^{k+1}}{k+1} < \infty$$
Esta serie siempre converge.
En lugar de considerar $q_k$ podemos tomar $q_{p_k}$, donde $\{p_k\}_{k=1}^\infty \subset \mathbb{N} \cup \{0\}$.
Basta tomar $p_k$ tal que
$$\sum_{k=k_0}^\infty \frac{|w|^{p_k+1}}{p_k+1} < \infty$$

\begin{definition}[Factores primos de Weierstrass]
    Los factores primos de Weierstrass son
    \begin{align*}
        E_0(z) & = 1-z                                                                      \\
        E_p(z) & = (1-z)\exp\left(\sum_{k=1}^p \frac{z^k}{k}\right), \quad p \in \mathbb{N}
    \end{align*}
    donde $\sum_{k=1}^p \frac{z^k}{k}$ es el polinomio de Taylor de orden $p$ de $\log\left(\frac{1}{1-z}\right)$.
\end{definition}

\begin{remark}
    Para todo $p \in \mathbb{N} \cup \{0\}$, $E_p(0) = 1$, $E_p(1) = 0$ y $E_p$ es entera.
\end{remark}

\begin{lemma}
    Sea $p \in \mathbb{N} \cup \{0\}$ y $|z| < 1$, entonces $|1-E_p(z)| \leq |z|^{p+1}$.
\end{lemma}

\begin{proof}
    Si $p = 0$,
    $$|1-E_0(z)| = |1-(1-z)| = |z|$$
    Si $p \in \mathbb{N}$, como $E_p$ es una función entera,
    $$E_p(z) = \sum_{n=0}^\infty a_nz^n$$
    Como $E_p(0) = 1$, entonces $a_0 = 1$.
    Así que
    $$E_p(z) = 1 + \sum_{n=1}^\infty a_nz^n$$
    Además, como $E_p(1) = 0$, entonces $\sum_{n=1}^\infty a_n = -1$.
    De la expresión anterior,
    $$-\sum_{n=1}^\infty a_nz^n = 1-E_p(z) = 1-(1-z)\exp\left(\sum_{k=1}^p \frac{z^k}{k}\right)$$
    Derivando,
    \begin{align*}
        -\sum_{n=1}^\infty na_nz^{n-1} & = \exp\left(\sum_{k=1}^n \frac{z^k}{k}\right) - (1-z)\exp\left(\sum_{k=1}^p \frac{z^k}{k}\right)\left(\sum_{k=1}^p z^{k-1}\right) =                             \\
                                       & =\exp\left(\sum_{k=1}^p \frac{z^k}{k}\right)\left(1-(1-z)\sum_{k=0}^{p-1} z^k\right) = \exp\left(\sum_{k=1}^p \frac{z^k}{k}\right)z^p =                         \\
                                       & = z^p\left(1 + \sum_{k=1}^p \frac{z^k}{k} + \frac{\left(\sum_{k=1}^p \frac{z^k}{k}\right)^2}{2!} + \dots\right) = z^p + A_{p+1}z^{p+1} + A_{p+2}z^{p+2} + \dots
    \end{align*}
    Por tanto, $a_n = 0$ para todo $1 \leq n \leq p$.
    Además,
    $$-(p+1)a_{p+1} = 1 \Rightarrow a_{p+1} = -\frac{1}{p+1}$$
    En general, para todo $n > p+1$,
    $$na_n = -A_{n-1} < 0 \Rightarrow a_n < 0$$
    Así que
    \begin{align*}
        |1-E_p(z)| & = \left|\sum_{n=1}^\infty a_nz^n\right| = \left|\sum_{n=p+1}^\infty a_nz^n\right| = |z|^{p+1}\left|\sum_{n=p+1}^\infty a_nz^{n-(p+1)}\right| \leq \\
                   & \leq |z|^{p+1} \sum_{n=p+1}^\infty |a_n||z|^{n-(p+1)} \leq |z|^{p+1} \sum_{n=p+1}^\infty |a_n| = |z|^{p+1}\left(-\sum_{n=p+1}^\infty a_n\right) = \\
                   & = |z|^{p+1} \left(-\sum_{n=1}^\infty a_n\right) = |z|^{p+1}
    \end{align*}
\end{proof}

\begin{remark}
    Hemos visto que
    $$E_p(z) = 1 - \frac{z^{p+1}}{p+1} + \sum_{n=p+2}^\infty a_nz^n$$
\end{remark}

\begin{theorem}[Teorema de factorización de Weierstrass: primera versión]
    Sea $\{a_n\} \subset \mathbb{C} \setminus \{0\}$ con $\lim\limits_{n \to \infty} |a_n| = \infty$.
    Entonces existe una sucesión de números $\{p_n\} \subset \mathbb{N} \cup 0$ de modo que
    $$\prod_{n=1}^\infty E_{p_n}\left(\frac{z}{a_n}\right)$$
    converge absoluta y uniformemente en compactos de $\mathbb{C}$ y además define una función entera cuyos ceros son $\{a_n\}_{n=1}^\infty$.
\end{theorem}

\begin{proof}
    Sea $K$ compacto de $\mathbb{C}$, entonces existe $R > 0$ tal que $K \subset \overline{D(0, R)}$.
    Como $\{a_n\} \to \infty$, existe $n_0 \in \mathbb{N}$ tal que para todo $n \geq n_0$ se tiene que $\frac{|z|}{|a_n|} \leq \frac{R}{2R} < 1$.
    Por el lema anterior,
    $$\left|1-E_{p_n}\left(\frac{z}{a_n}\right)\right| \leq \left|\frac{z}{a_n}\right|^{p_n+1} \leq \left(\frac{1}{2}\right)^{p_n+1}$$
    Si consideramos $p_n = n$, por el criterio de la mayorante de Weierstrass tenemos que
    $$\sum_{n=n_0}^\infty \left|1-E_n\left(\frac{z}{a_n}\right)\right|$$
    converge uniformemente en $K$.
    Además,
    $$E_n\left(\frac{a_n}{a_n}\right) = E_n(1) = 0$$
\end{proof}

\begin{theorem}[Teorema de factorización de Weierstrass: segunda versión]
    Sea $f$ una función entera tal que en $z = 0$ tiene un cero de orden $N$ y los demás ceros de $f$ son $\{a_n\}_{n=1}^\infty \subset \mathbb{C} \setminus \{0\}$.
    Entonces existe $\{p_n\}_{n=1}^\infty \subset \mathbb{N} \cup \{0\}$ y una función entera $g$ tal que
    $$f(z) = e^{g(z)}z^N\prod_{n=1}^\infty E_{p_n}\left(\frac{z}{a_n}\right)$$
\end{theorem}

\begin{proof}
    Suponemos $\{a_n\}$ conjunto infinito.
    Por el teorema anterior, existe una función entera que se anula en $\{a_n\}_{n=1}^\infty$.
    Entonces
    $$\frac{f(z)}{z^N\prod_{n=1}^\infty E_{p_n}\left(\frac{z}{a_n}\right)}$$
    es entera y no se anula en $\mathbb{C}$.
    Por tanto, existe $g$ entera tal que
    $$\frac{f(z)}{z^N\prod_{n=1}^\infty E_{p_n}\left(\frac{z}{a_n}\right)} = e^{g(z)}$$
\end{proof}

\begin{remark}
    \hfill
    \begin{enumerate}
        \item La factorización no es única.
              Si la sucesión $\{p_n\}_{n=1}^\infty$ da una descomposición, entonces una sucesión $\{q_n\}_{n=1}^\infty$ tal que $p_n \leq q_n$ para todo $n \in \mathbb{N}$ también sirve.

        \item Si $\{a_n\}_{n=1}^\infty \subset \mathbb{C} \setminus \{0\}$ y existe $p$ tal que $\sum_{n=1}^\infty \frac{1}{|a|^{p+1}}$ converge, entonces
              $$\prod_{n=1}^\infty E_p\left(\frac{z}{a_n}\right)$$
              converge absoluta y uniformemente en compactos de $\mathbb{C}$.
              \begin{proof}
                  Sea $K$ compacto de $\mathbb{C}$, entonces existe $R > 0$ tal que $K \subset \overline{D(0, R)}$.
                  Como $\lim\limits_{n \to \infty} |a_n| = \infty$, existe $n_0 \in \mathbb{N}$ tal que para todo $n \geq n_0$ se tiene que
                  $$|a_n| \geq 2R \Rightarrow \left|\frac{z}{a_n}\right| \leq \frac{1}{2} < 1$$
                  Por el lema anterior,
                  $$\sum_{n=n_0}^\infty \left|1-E_p\left(\frac{z}{a_n}\right)\right| \leq \sum_{n=n_0}^\infty \left|\frac{z}{a_n}\right|^{p+1} \leq R^{p+1}\sum_{n=n_0}^\infty \frac{1}{|a_n|^{p+1}} < \infty$$
              \end{proof}
    \end{enumerate}
\end{remark}

\section{Exponente de convergencia y género de una sucesión}
A partir de las observaciones anteriores, introducimos la siguiente definición.

\begin{definition}
    Sea $\{a_n\}_{n=1}^\infty \subset \mathbb{C} \setminus \{0\}$ tal que $\lim\limits_{n \to \infty} |a_n| = \infty$.
    Definimos el exponente de convergencia de la sucesión como
    $$\sigma = \inf\left\{s \in \mathbb{R} : \sum_{n=1}^\infty \frac{1}{|a_n|^s} < \infty\right\}$$
\end{definition}

\begin{remark}
    \hfill
    \begin{enumerate}
        \item Si $s \in \mathbb{R}$ satisface que $\sum_{n=1}^\infty \frac{1}{|a_n|^s} < \infty$, entonces si $t > s$ se cumple que $\sum_{n=1}^\infty \frac{1}{|a_n|^t} < \infty$.
        \item Diremos que $\inf\emptyset = +\infty$, es decir, $\sigma = +\infty$ cuando ningún $s \in \mathbb{R}$ satisface que $\sum_{n=1}^\infty \frac{1}{|a_n|^s} < \infty$.
        \item Si $\sigma < \infty$,
              \begin{align*}
                   & \sum_{n=1}^\infty \frac{1}{|a_n|^s} < \infty \text{ si } s \in (\sigma, +\infty) \\
                   & \sum_{n=1}^\infty \frac{1}{|a_n|^s} = \infty \text{ si } s \in (-\infty, \sigma)
              \end{align*}
    \end{enumerate}
\end{remark}

\begin{definition}
    Sea $\{a_n\}_{n=1}^\infty \subset \mathbb{C} \setminus \{0\}$ tal que $\lim\limits_{n \to \infty} |a_n| = \infty$ y $\sigma < \infty$.
    Entonces el género de la sucesión es el menor entero $p$ tal que
    $$\sum_{n=1}^\infty \frac{1}{|a_n|^{p+1}} < \infty$$
\end{definition}

\begin{remark}
    \hfill
    \begin{enumerate}
        \item $\sigma > 0$ siempre.
              Además, $\sum_{n=1}^\infty \frac{1}{|a_n|^0} = \infty$ así que $p \geq 0$.
        \item $p \leq \sigma \leq p+1$.
    \end{enumerate}
\end{remark}

Si $\{a_n\}$ es finita entonces diremos que $\sigma = p = 0$.

\begin{example}
    \hfill
    \begin{enumerate}
        \item Sea $a_n = n+1$.
              $$\sum_{n=1}^\infty \frac{1}{(n+1)^s} < \infty \Leftrightarrow s < 1 \Rightarrow \sigma = 1, p = 1$$
        \item Sea $a_n = (n+1)^2$.
              $$\sum_{n=1}^\infty \frac{1}{(n+1)^{2s}} < \infty \Leftrightarrow s > \frac{1}{2} \Rightarrow \sigma = \frac{1}{2}, p = 0$$
        \item Sea $a_n = (n+1)\log^2(n+1)$.
              $$\sum_{n=1}^\infty \frac{1}{(n+1)^s\log^{2s}(n+1)} < \infty \Leftrightarrow s \geq 1 \Rightarrow \sigma = 1, p = 0$$
              Para el caso $s = 1$,
              $$\sum_{n=1}^\infty \frac{1}{(n+1)\log^2(n+1)} \sim \int_1^\infty \frac{1}{(x+1)\log^2(x+1)}dx = \left.-\frac{1}{\log(x+1)}\right]_1^\infty = \frac{1}{\log(2)}$$
        \item Sea $a_n = \log(n+1)$.
              $$\sum_{n=1}^\infty \frac{1}{\log^s(n+1)} = \infty \; \forall s \in \mathbb{R} \Rightarrow \sigma = \infty$$
        \item Sea $a_n = 2^n$.
              $$\sum_{n=1}^\infty \frac{1}{2^{ns}} = \sum_{n=1}^\infty \frac{1}{(2^s)^n} < \infty \; \forall s > 0 \Rightarrow \sigma = 0, p = 0$$
    \end{enumerate}
\end{example}

\section{Factorización canónica de una función entera}
Con las definiciones anteriores vamos a tener el siguiente teorema.

\begin{theorem}[Teorema de factorización canónica]
    Sea $f$ una función entera con $z = 0$ un cero de orden $N$ y $\{a_n\}_{n=1}^\infty \subset \mathbb{C} \setminus \{0\}$ los demás ceros de $f$.
    Si $\lim\limits_{n \to \infty} |a_n| = \infty$ y $\sigma < \infty$, entonces
    $$f(z) = e^{g(z)}z^N\prod_{n=1}^\infty E_p\left(\frac{z}{a_n}\right)$$
    donde $p$ es el género de $\{a_n\}_{n=1}^\infty$.
\end{theorem}

\begin{proof}
    Sea $K$ compacto de $\mathbb{C}$, entonces existe $R > 0$ tal que $K \subset \overline{D(0, R)}$.
    Queremos ver que $\prod_{n=1}^\infty E_p\left(\frac{z}{a_n}\right)$ converge absoluta y uniformemente en $K$.
    Basta ver que $\sum_{n=1}^\infty \left|1-E_p\left(\frac{z}{a_n}\right)\right|$ converge uniformemente en $K$.
    Como $|a_n| \to \infty$, existe $n_0 \in \mathbb{N}$ tal que $|a_n| \geq 2R$ para todo $n \geq n_0$.
    Entonces, como $p$ es el género de $\{a_n\}_{n=1}^\infty$,
    $$\sum_{n=n_0}^\infty \left|1-E_p\left(\frac{z}{a_n}\right)\right| \leq \sum_{n=n_0}^\infty \frac{|z|^{p+1}}{|a_n|^{p+1}} \leq R^{p+1} \sum_{n=n_0}^\infty \frac{1}{|a_n|^{p+1}} < \infty$$
\end{proof}

\section{Factorización de funciones holomorfas en un dominio}
\begin{theorem}
    Sea $D$ un dominio en $\mathbb{C}$ y sea $\{a_n\}_{n=1}^\infty \subset D \setminus \{0\}$ sin puntos de acumulación en $D$.
    Entonces existe una función holomorfa en $D$ que se anula en $\{a_n\}_{n=1}^\infty$.
\end{theorem}

\begin{proof}
    Si $\{a_n\}_{N=1}^\infty$ es finito, entonces dicha función es $\prod_{n=1}^N (z-a_n)$.
    Si $\{a_n\}_{n=1}^\infty$, entonces definimos $\delta_n = \dist(a_n, \mathbb{C} \setminus D)$.
    Consideramos primero dos casos.
    \begin{enumerate}
        \item Existe $N_0 \in \mathbb{N}$ tal que $|a_n|\delta_n \geq 1$ para todo $n \geq N_0$.
              Veamos que $\lim\limits_{n \to \infty} |a_n| = \infty$.
              Supongamos por reducción al absurdo que $\lim\limits_{n \to \infty} |a_n| \neq \infty$.
              Entonces existe una subsucesión $\{a_{n_k}\}_{k=1}^\infty$ acotada.
              Así que existe una subsucesión $\{a_m\}_{m=1}^\infty$ de $\{a_{n_k}\}$ convergente a $a_0 \in \mathbb{C} \setminus D$.
              Si $D = \mathbb{C}$, esto es imposible.
              En otro caso,
              $$1 \leq |a_m|\delta_m \leq |a_m||a_m-w|, \quad \forall w \in \mathbb{C} \setminus D$$
              Tomando límites,
              $$1 \leq |a_0||a_0-w|, \quad \forall w \in \mathbb{C} \setminus D$$
              Como necesariamente $|a_0| \neq 0$, entonces
              $$|a_0-w| \geq \frac{1}{|a_0|} \Rightarrow a_0 \notin \mathbb{C} \setminus D \Rightarrow a_0 \in D$$
              Esto es una contradicción.

        \item Existe $N_0 \in \mathbb{N}$ tal que $|a_n|\delta_n \leq 1$ para todo $n \geq N_0$.
              Veamos por reducción al absurdo que $\delta_n \xrightarrow[n \to \infty]{} 0$.
              Supongamos que existen una subsucesión $\{\delta_{n_k}\}_{k=1}^\infty$ y $\varepsilon > 0$ tales que $\delta_{n_k} \geq \varepsilon > 0$ para todo $k \in \mathbb{N}$.
              Entonces existe una subsucesión $\{\delta_m\}_{m=1}^\infty$ de $\{\delta_{n_k}\}$ que converge a un cierto $\delta_0 \geq \varepsilon$.
              Como además $|a_m| < \frac{1}{\varepsilon}$, existe una subsucesión $\{a_l\}_{l=1}^\infty$ de $\{a_m\}$ que converge a $a_0$.
              Así que
              $$\varepsilon \leq \delta_l \leq |a_l-w|, \quad \forall w \in \mathbb{C} \setminus \{0\}$$
              Tomando límites,
              $$\varepsilon \leq \delta_0 \leq |a_0-w|, \quad \forall w \in \mathbb{C} \setminus \{0\} \Rightarrow a_0 \in D$$
              Esto contradice que $\{a_n\}$ no tenga puntos de acumulación.

              Construyamos ahora una función holomorfa $f$ con ceros en $\{a_n\}$.
              Para cada $a_n$ existe un $b_n \in \mathbb{C} \setminus D$ de modo que $|a_n-b_n| \leq \frac{3}{2}\delta_n$.
              Entonces definimos
              $$\prod_{n=1}^\infty E_n\left(\frac{a_n-b_n}{z-b_n}\right)$$
              cuyos ceros son $\{a_n\}_{n=1}^\infty$.

              Sea $K$ un compacto de $D$ y sea $d = \dist(K, \mathbb{C} \setminus D)$.
              Observamos que $d > 0$.
              Como $\delta_n \to 0$, podemos tomar $n_0 \in \mathbb{N}$ tal que $\delta_n < \frac{d}{2}$ para todo $n \geq n_0$.
              $$|a_n-b_n| \leq \frac{3}{2}\delta_n < \frac{3}{4}d \leq \frac{3}{4}|z-b_n|, \quad \forall z \in K$$
              Así que
              $$\sum_{n=n_0}^\infty \left|1-E_n\left(\frac{a_n-b_n}{z-b_n}\right)\right| \leq \sum_{n=n_0}^\infty \left|\frac{a_n-b_n}{z-b_n}\right|^{n+1} \leq \sum_{n=n_0}^\infty \left(\frac{3}{4}\right)^{n+1} < \infty$$
              Por tanto,
              $$\prod_{n=1}^\infty E_n\left(\frac{a_n-b_n}{z-b_n}\right)$$
              converge absoluta y uniformemente en compactos de $D$.
    \end{enumerate}

    En general, la sucesión $\{a_n\}_{n=1}^\infty$ se puede separar en $\{b_n\}_{n=1}^\infty$ y $\{c_n\}_{n=1}^\infty$, con $\{b_n\}$ en el caso 1 y $\{c_n\}$ en el caso 2.
    Por el caso 1, existe $f$ holomorfa en $D$ con ceros $\{b_n\}_{n=1}^\infty$.
    De igual forma, por el caso 2 existe $g$ holomorfa en $D$ con ceros $\{c_n\}_{n=1}^\infty$.
    Consideramos $h = fg$.
    $h$ es holomorfa en $D$ con ceros $\{a_n\}_{n=1}^\infty$.
\end{proof}