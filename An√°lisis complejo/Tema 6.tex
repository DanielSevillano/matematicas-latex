\chapter{Funciones enteras. Crecimiento y distribución de los ceros}
\begin{theorem}[Teorema de Liouville]
    Si $f$ es una función entera y acotada, entonces $f$ es constante.
\end{theorem}

\begin{theorem}[Generalización del teorema de Liouville]
    Sea $f$ entera tal que existen $\alpha > 0$, $c > 0$ y $R_0 > 0$ con $|f(z)| \leq c|z|^\alpha$ para todo $|z| \geq R_0$.
    Entonces $f$ es un polinomio de grado $E(\alpha)$ y por tanto tiene $E(\alpha)$ ceros.
\end{theorem}

\begin{proof}
    Sea $f(z) = \sum_{n=0}^\infty a_nz^n$.
    Sea $R > 0$,
    $$a_n = \frac{f^{(n)}(0)}{n!} = \frac{1}{2\pi i} \int_{|z|=R} \frac{f(z)}{z^{n+1}}dz \leq \max_{|z|=R} |f(z)|\frac{1}{R^n} \leq c\frac{|z|^\alpha}{R} = cR^{\alpha-n} \xrightarrow[R \to \infty, n > E(\alpha)]{} 0$$
    Por tanto, $a_n = 0$ para todo $n > E(\alpha)$.
\end{proof}

\begin{remark}
    \hfill
    \begin{enumerate}
        \item Parece que si el crecimiento está controlado, el número de ceros también lo está.
        \item Al revés esto no ocurre.
              Por ejemplo, con la exponencial.
    \end{enumerate}
\end{remark}

\section{Fórmula de Jensen}
La fórmula de Jensen permite controlar el número de ceros de una función holomorfa sabiendo restricciones sobre su crecimiento.

\begin{lemma}
    Sea $s > 0$.
    Entonces
    $$\frac{1}{2\pi} \int_{-\pi}^\pi \log|1+se^{i\theta}|d\theta = \log^+(s)$$
    donde
    $$\log^+(s) = \begin{cases}
            \log(s) & \text{si } s \geq 1  \\
            0       & \text{si } 0 < s < 1
        \end{cases}$$
\end{lemma}

% Demostración

\begin{theorem}[Fórmula de Jensen]
    Sea $f$ una función holomorfa en $D(0, R)$ con $f(0) \neq 0$ y tal que tiene ceros $\{a_n\}$ de modo que
    $$|a_1| \leq |a_2| \leq |a_3| \leq \dots$$
    Sea $\rho \in (0, R)$ y $n(\rho, f) = \#\{a_n : |a_n| \leq \rho\}$.
    Entonces
    $$\frac{1}{2\pi} \int_{-\pi}^\pi \log|f(\rho e^{i\theta})|d\theta = \log|f(0)| + \sum_{k=1}^{n(\rho, f)} \log\left(\frac{\rho}{|a_k|}\right)$$
\end{theorem}

% Demostración

\begin{definition}
    Se llama función contadora de Nevanlinna a la cantidad
    $$N(\rho, f) = \sum_{\{n : |a_n| \leq \rho\}} \log\left(\frac{\rho}{|a_n|}\right)$$
\end{definition}

\begin{remark}
    \hfill
    \begin{enumerate}
        \item $$N(\rho, f) = \sum_{\{n : |a_n| \leq \rho\}} \log\left(\frac{\rho}{|a_n|}\right) = \sum_{k=1}^{n(\rho, f)} \frac{\rho}{|a_k|} = \sum_{k=1}^\infty \log^+\left(\frac{\rho}{|a_k|}\right)$$
        \item $$N(\rho, f) = \int_0^\rho \frac{n(t, f)}{t}dt$$
    \end{enumerate}
\end{remark}