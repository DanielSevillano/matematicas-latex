\chapter{Funciones armónicas}

\section{Funciones armónicas y funciones holomorfas}
\begin{definition}
    Sea $D$ un abierto en $\mathbb{R}^n$ y sea $f: D \to \mathbb{R}$.
    Decimos que $f$ es armónica en $D$ si $f \in \mathcal{C}^2(D)$ y
    $$\Delta f = \frac{\partial^2f}{\partial x_1^2} + \frac{\partial^2f}{\partial x_2^2} + \dots + \frac{\partial^2f}{\partial x_n^2}$$
\end{definition}

\begin{definition}
    Dado $D$ abierto en $\mathbb{C}$ y $u: D \to \mathbb{R}$, con $u \in \mathcal{C}^2(D)$.
    El laplaciano de $u$ se define como:
    $$\Delta u = \frac{\partial^2u}{\partial x^2} + \frac{\partial^2u}{\partial y^2}$$
\end{definition}

\begin{definition}
    Sea $D$ abierto en $\mathbb{D}$ y sea $u: D \to \mathbb{R}$, con $u \in \mathcal{C}^2(D)$.
    Diremos que $u$ es armónica en $D$ si $\Delta u = 0$.
\end{definition}

\begin{example}
    \hfill
    \begin{itemize}
        \item Las funciones constantes.
              $u(z) = a$, $a \in \mathbb{R}$, es armónica en $\mathbb{C}$.
        \item La función parte real es armónica en $\mathbb{C}$.
              \begin{align*}
                  \Re: \mathbb{C} & \to \mathbb{R}                         \\
                  z               & \mapsto \Re(z) = \frac{z + \bar{z}}{2} \\
                  (x, y)          & \mapsto x
              \end{align*}
        \item La función parte imaginaria es armónica en $\mathbb{C}$.
              \begin{align*}
                  \Im: \mathbb{C} & \to \mathbb{R}                          \\
                  z               & \mapsto \Im(z) = \frac{z - \bar{z}}{2i} \\
                  (x, y)          & \mapsto y
              \end{align*}
        \item Si $D$ es un abierto en $\mathbb{C}$ y $u$ y $v$ son armónicas en $D$, entonces $u+v$ es armónica en $D$.
        \item Si $D$ es un abierto en $\mathbb{C}$, $u$ es armónica en $D$ y $c \in \mathbb{R}$, entonces $cu$ es armónica en $D$.
    \end{itemize}
\end{example}

\begin{theorem}
    Sea $D$ un abierto en $\mathbb{C}$ y sea $f$ una función holomorfa en $D$.
    Entonces $\Re(f)$ e $\Im(f)$ son armónicas en $D$.
\end{theorem}

\begin{proof}
    Sean $u = \Re(f)$ y $v = \Im(f)$, $u, v: D \to \mathbb{R}$, con $u, v \in \mathcal{C}^\infty(D) \subset \mathcal{C}^2(D)$.
    Se verifican las condiciones de Cauchy-Riemann:
    $$\begin{cases}
            u_x = v_y \\
            u_y = -v_x
        \end{cases}$$
    Por tanto:
    \begin{align*}
        \Delta u & = u_{xx} + u_{yy} = v_{yx} - v_{xy} = 0  \\
        \Delta v & = v_{xx} + v_{yy} = -u_{yx} + u_{xy} = 0
    \end{align*}
\end{proof}

El recíproco no es cierto.
Es decir, dado $D$ abierto en $\mathbb{C}$, $u$ y $v$ armónicas en $D$ y $f = u + iv: D \to \mathbb{C}$, no se cumple en general que $f$ sea holomorfa en $D$.

\begin{example}[Contraejemplo]
    Sea $D = \mathbb{C}$ y sean $u(z) = \Re(z)$ y $v(z) = \Im(z)$.
    $u$ y $v$ son armónicas en $\mathbb{C}$.
    Sin embargo, $f(z) = u(z) + iv(z) = \bar{z}$ no es holomorfa.
\end{example}

\begin{definition}
    Sea $D$ un abierto en $\mathbb{C}$ y sea $u$ armónica en $D$.
    Diremos que $v: D \to \mathbb{R}$ es una conjugada armónica de $u$ en $D$ si la función $f = u + iv$ es holomorfa en $D$.
\end{definition}

\begin{properties}
    \hfill
    \begin{itemize}
        \item Si existe una conjugada armónica de $u$ en $D$, entonces es una función armónica en $D$.
        \item Si $u$ es una conjugada armónica de $u$ en $D$, entonces $u + c$ es conjugada armónica de $u$ en $D$ para todo $c \in \mathbb{R}$.
        \item Si $D$ es un dominio en $\mathbb{C}$, $u$ es armónica en $D$ y $v_1, v_2$ son conjugadas armónicas de $u$ en $\mathbb{C}$, entonces existe $c \in \mathbb{R}$ tal que $v_1-v_2 = c$.
              \begin{proof}
                  $f_1 = u + iv_1$ y $f_2 = u + iv_2$ son holomorfas en $D$, así que $f_2 - f_1 = i(v_2-v_1)$ es holomorfa en $D$.
                  Como $\Re(f_2-f_1) = 0$ en $D$ con $D$ dominio, entonces $f_2 - f_1$ es constante.
                  Es decir, existe $c \in \mathbb{R}$ con $f_2-f_1 = ic$.
                  Por tanto:
                  $$i(v_2-v_1) = ic \Rightarrow v_2 - v_1 = c$$
              \end{proof}
    \end{itemize}
\end{properties}

No tiene por qué existir la conjugada armónica.

\begin{example}[Contraejemplo]
    Sea $D = \mathbb{C} \setminus \{0\}$ abierto en $\mathbb{C}$ y sea $u: D \to \mathbb{R}$, $u(z) = \Log|z|$.
    Si escribimos $z = x +iy$,
    $$u(z) = \Log(x^2 + y^2)^{1/2} = \frac{1}{2}\Log(x^2+y^2)$$
    Calculamos sus derivadas parciales:
    \begin{align*}
        u_x(z)    & = \frac{1}{2} \frac{2x}{x^2+y^2} = \frac{x}{x^2+y^2}             & u_y(z) & = \frac{y}{x^2+y^2}           \\
        u_{xx}(z) & = \frac{x^2+y^2-2x^2}{(x^2+y^2)^2} = \frac{y^2-x^2}{(x^2+y^2)^2} & u_{yy} & = \frac{x^2-y^2}{(x^2+y^2)^2}
    \end{align*}
    $\Delta u = 0$, así que $u$ es armónica en $D$.

    Supongamos que $v$ es una conjugada armónica de $u$ en $D$.
    Entonces $v: \mathbb{C} \setminus \{0\} \to \mathbb{R}$, con $f = u + iv$ holomorfa en $\mathbb{C} \setminus \{0\}$.
    Consideramos:
    \begin{align*}
        g = \Log: \mathbb{C} \setminus (-\infty, 0] & \to \mathbb{C}                       \\
        z                                           & \mapsto \Log(z) = \log|z| + i\Arg(z)
    \end{align*}
    $f$ y $g$ son holomorfas en $\mathbb{C} \setminus (-\infty, 0]$, luego $f-g$ es holomorfa en $\mathbb{C} \setminus (-\infty, 0]$.
    Como $\Re(f-g) = 0$, entonces $f-g = ic$, con $c \in \mathbb{R}$.
    Así que $g = f - ic$.
    Sin embargo, $g$ no se extiende de forma continua a $\mathbb{C} \setminus \{0\}$.
    Esto es una contradicción.
\end{example}

Hemos probado que la función $u(z) = \Log|z|$ es armónica en $\mathbb{C} \setminus \{0\}$.

\begin{lemma}
    Sea $D$ un abierto en $\mathbb{C}$ y sea $f$ una función holomorfa y nunca nula en $D$.
    Entonces la función $u = \Log|f|$ es armónica en $D$.
\end{lemma}

\begin{proof}
    Sean $z_0 \in D$ y $R > 0$ tales que $D(z_0, R) \subset D$.
    Como $D(z_0, R)$ es un dominio simplemente conexo en $\mathbb{C}$, existe $g$ una rama de $\log(f)$ en $D(z_0, R)$.
    $g$ es holomorfa en $D(z_0, R)$, con
    $$\Re(g(z)) = \Log|f(z)|, \quad z \in D(z_0, R)$$
    $Re(g) = \Log|f|$ es armónica en $D(z_0, R)$.
\end{proof}

\begin{proposition}
    Sea $D$ un abierto en $\mathbb{C}$ y sea $u$ armónica en $D$.
    Entonces la función $u_x - iu_y$ es holomorfa en $D$.
\end{proposition}

\begin{corollary}
    Sea $D$ abierto en $\mathbb{C}$ y sea $u$ armónica en $D$.
    Entonces $u \in \mathcal{C}^\infty(D)$.
\end{corollary}

\begin{proof}
    Sea $u$ armónica en $D$.
    Entonces $f = u_x - iu_y$ es holomorfa en $D$.
    Así que $u_x, u_y \in \mathcal{C}^\infty(D)$.
    Como además $u \in \mathcal{C}^2(D)$, entonces $u \in \mathcal{C}^\infty(D)$.
\end{proof}

\begin{proposition}
    Sea $D$ un dominio simplemente conexo en $\mathbb{C}$ y sea $u$ armónica en $D$.
    Entonces existe $F$ holomorfa en $D$ tal que $\Re(F) = u$ en $D$.
    Además, esta $F$ es única salvo adición de constantes imaginarias.
\end{proposition}

\begin{proof}
    Sea $f = u_x - iu_y$, que es holomorfa en $D$.
    Como $D$ es simplemente conexo, existe $g$ primitiva de $f$ en $D$.
    $g$ es holomorfa en $D$ y $g' = f$ en $D$.
    Escribimos $g = U + iV$, con $U = \Re(f)$ y $V = \Im(f)$.
    Como $g$ es holomorfa, $U_x = V_y$ y $U_y = -V_x$.
    Además, $g' = U_x + iV_x = u_x - iu_y$, así que $U_x = u_x$ y $V_x = -u_y$.

    Sea $F = u + iV$, $F: D \to \mathbb{C}$, con $u \in \mathcal{C}^2(D)$.
    $V$ es armónica en $D$, $V \in \mathcal{C}^\infty(D)$.
    $F$ es diferenciable en sentido real.
    Además,
    $$\begin{cases}
            u_x = U_x = V_y \\
            u_y = -V_x
        \end{cases}$$
    Luego $F$ es holomorfa en $D$, con $\Re(F) = u$ en $D$.

    Veamos que $F$ es única salvo adición de constantes imaginarias.
    \begin{itemize}
        \item Si $c \in \mathbb{R}$, está claro que si $G = F + ic$, entonces $G$ es holomorfa en $D$ y $\Re(G) = u$ en $D$.
        \item Si $G$ es holomorfa en $D$ y $\Re(G) = u$ en $D$.
              Entonces $F-G$ es holomorfa y $\Re(F-G) = 0$ en $D$.
              Por tanto $F-G$ es constante, así que existe $c \in \mathbb{R}$ con $F-G = ic \Leftrightarrow G = F - ic$.
    \end{itemize}
\end{proof}

\begin{corollary}
    Sea $D$ un abierto en $\mathbb{C}$ y sea $u$ armónica en $D$.
    Si $z_0 \in D$ y $R > 0$ con $D(z_0, R) \subset D$, entonces existe $F$ holomorfa en $D(z_0, R)$ tal que $\Re(F) = u$ en $D(z_0, R)$.
\end{corollary}

Toda función armónica, localmente, es la parte real de una función holomorfa.

\begin{theorem}
    Sea $D$ un dominio en $\mathbb{C}$.
    Son equivalentes:
    \begin{enumerate}
        \item $D$ es simplemente conexo.
        \item Toda función armónica tiene conjugada armónica.
    \end{enumerate}
\end{theorem}

\begin{proof}
    \hfill
    \begin{itemize}
        \item[$\Rightarrow$] Lo sabemos.
        \item[$\Leftarrow$] Sea $f$ holomorfa y nunca nula en $D$.
            Veamos que existe un rama de $\log(f)$ en $D$.
            Sea $u = \Log|f|$, que es una función armónica en $D$.
            Sea $v$ una conjugada armónica de $u$ en $D$.
            Entonces $F = u + iv$ es holomorfa en $D$.
            Sea $h = fe^{-F}$ holomorfa en $D$.
            Como $|e^F| = |e^{u+iv}| = e^u = e^{\Log|f|} = |f|$, entonces
            $$|h| = \frac{|f|}{|e^F|} = 1$$
            Entonces $h$ es constante.
            Es decir, existe $\xi = e^{ic}$ con $c \in \mathbb{R}$ tal que $h(z) = \xi$ para todo $z \in D$.
            $$h = \frac{f}{e^F} \Rightarrow f = \xi e^F = e^{ic}e^F = e^{F + ic}$$
            La función $F + ic$ es una rama de $\log(f)$ en $D$.

            Toda función holomorfa y nunca nula en $D$ tiene una rama del logaritmo.
            Luego $D$ es simplemente conexo.
    \end{itemize}
\end{proof}

\begin{theorem}[Propiedad del valor medio]
    Sea $D$ un abierto en $\mathbb{C}$ y sea $u$ armónica en $D$.
    Sea $z_0 \in D$ y $R > 0$ con $D(z_0, R) \subset D$.
    Entonces:
    $$u(z_0) = \frac{1}{2\pi} \int_{-\pi}^\pi u(z_0 + re^{it})dt, \quad 0 \leq 0 < R$$
\end{theorem}

\begin{proof}
    Si $r = 0$ es trivial.
    Supongamos $0 < r < R$.
    Existe $F$ holomorfa en $D(z_0, R)$ con $\Re(F) = u$ en $D(z_0, R)$.
    Por la fórmula de Cauchy,
    $$F(z_0) = \frac{1}{2\pi i} \int_{|z-z_0|=r} \frac{F(z)}{z-z_0}dz = \frac{1}{2\pi i} \int_{-\pi}^\pi \frac{F(z_0 + re^{it})}{z_0 + re^{it} - z_0}rie^{it}dt = \frac{1}{2\pi} \int_{-\pi}^\pi F(z_0 + re^{it})dt$$
    Tomando parte real,
    $$u(z_0) = \Re\left(\frac{1}{2\pi} \int_{-\pi}^\pi F(z_0 + re^{it})dt\right) = \frac{1}{2\pi} \int_{-\pi}^\pi \Re(F(z_0 + re^{it}))dt = \frac{1}{2\pi} \int_{-\pi}^\pi u(z_0 + Re^{it})dt$$
\end{proof}

\begin{theorem}[Forma débil del principio del máximo para funciones armónicas]
    Sea $D$ un abierto en $\mathbb{C}$ y sea $u$ armónica en $D$.
    Si $u$ tiene un máximo local en un punto $z_0 \in D$, entonces existe $r > 0$ con $D(z_0, R) \subset D$ tal que $u$ es constante en $D(z_0, R)$.
\end{theorem}

\begin{proof}
    Sea $r > 0$ con $D(z_0, r) \subset D$.
    Existe $F$ holomorfa en $D(z_0, r)$ tal que $\Re(F) = u$ en $D(z_0, r)$.
    Sea $f = e^F$, que es holomorfa en $D(z_0, r)$.
    En $D(z_0, r)$ tenemos que:
    $$|f| = |e^F| = e^{\Re(F)} = e^u$$
    Existe $r_1$ con $0 < r_1 < r$, tal que $u(z) \leq u(z_0)$ si $z \in D(z_0, r_1)$.
    Entonces
    $$|e^{F(z)}| = e^{u(z)} \leq e^{u(z_0)} = |e^{F(z_0)}|$$
    $|e^F|$ tiene un máximo local en $z_0$.
    Por el principio del máximo, tenemos que $e^F$ es constante en $D(z_0, r)$.
    Entonces $|e^F| = e^u$ es constante, así que $u$ es constante en $D(z_0, r)$.
\end{proof}

Sea $D$ dominio en $\mathbb{C}$, sean $u$, $v$ armónicas en $D$ y sea $A \subset D$, $A$ con puntos de acumulación en $D$.
$u = v$ en $A$ no implica en general que $u = v$ en $D$.

\begin{example}[Contraejemplo]
    Sea $D = \mathbb{C}$ y sean $u(z) = \Re(z)$ y $v(z) = 0$ armónicas.
    $u = v$ en el eje imaginario, que tiene puntos de acumulación en $\mathbb{C}$, pero $u \neq v$ en $\mathbb{C}$.
\end{example}

\begin{remark}
    Sea $D$ un dominio en $\mathbb{C}$ y sea $u: D \to \mathbb{R}$.
    Si existen $u_x$ y $u_y$ en $D$ y $u_x = u_y = 0$ en $D$, entonces $u$ es constante en $D$.
\end{remark}

\begin{theorem}[Teorema de identidad para funciones armónicas]
    Sea $D$ un dominio en $\mathbb{C}$ y sea $u$ armónica en $D$.
    Si existe $G \subset D$, $G$ abierto y no vacío tal que $u = 0$ en $G$, entonces $u = 0$ en $D$.
\end{theorem}

\begin{proof}
    Sea $f = u_x - iu_y$ holomorfa en $D$.
    Como $u = 0$ en $G$, tenemos que $u_x = u_y = 0$ en $G$.
    Entonces $f = 0$ en $G$.
    Por el teorema de identidad, $f = 0$ en $D$.
    Entonces $u_x = u_y = 0$ en $D$, luego $u$ es constante en $D$.
    Como $u = 0$ en $G$, entonces $u = 0$ en $D$.
\end{proof}

\begin{corollary}
    Sea $D$ un dominio en $\mathbb{C}$ y sean $u$ y $v$ armónicas en $D$.
    Si existe $G \subset D$, $G$ abierto y no vacío tal que $u = v$ en $G$, entonces $u = v$ en $D$.
\end{corollary}

\begin{theorem}[Principio del máximo para funciones armónicas: primera versión]
    Sea $D$ un dominio en $\mathbb{C}$ y sea $u$ armónica en $D$.
    Si $u$ tiene un máximo local en un punto $z_0 \in D$, entonces $u$ es constante en $D$.
\end{theorem}

\begin{proof}
    Existe $R > 0$ con $D(z_0, R) \subset D$ tal que $u$ es constante en $D(z_0, R)$.
    Es decir, $u(z) = c$, $z \in D(z_0, R)$, $c \in \mathbb{R}$.
    Sea $v(z) = c$, $z \in D$.
    $u$ y $v$ son armónicas en $D$ y $u = v$ en $D(z_0, R)$.
    Entonces $u = v$ en $D$, es decir, $u(z) = c$ para todo $z \in D$.
\end{proof}

\begin{notation}
    Si $D$ es un dominio en $\mathbb{C}$, $\bar{D}$ denotará la clausura de $D$ como subconjunto de $\mathbb{C}^\ast$.
    \begin{itemize}
        \item Si $D$ es acotado, entonces $\partial_\infty D = \partial D$.
              $$\bar{D} = D \cup \partial_\infty D = D \cup \partial D$$
        \item Si $D$ no es acotado, entonces $\partial_\infty D = \partial D \cup \{\infty\}$.
              $$\bar{D} = D \cup \partial_\infty D = D \cup \partial D \cup \{\infty\}$$
    \end{itemize}
\end{notation}

\begin{theorem}[Principio del máximo para funciones armónicas: segunda versión]
    Sea $D$ un dominio en $\mathbb{C}$ y sea $u$ armónica en $D$.
    Si existe $M \in \mathbb{R}$ tal que
    $$\limsup_{z \to \xi, z \in D} u(z) \leq M, \quad \forall \xi \in \partial_\infty D$$
    entonces $u(z) \leq M$ para todo $z \in D$.
    Además, si $u(z_0) = M$ para algún $z_0 \in D$, entonces $u$ es constante en $D$.
\end{theorem}

\begin{proof}
    Sea $K = \sup_{z \in D} u(z) \in \mathbb{R} \cup \{\infty\}$.
    Hay que probar que $K \leq M$.
    Existe $\{z_n\}_{n=1}^\infty$ en $D$ tal que $\lim\limits_{n \to \infty} u(z_n) = K$.
    Podemos suponer, pasando si es necesario a una subsucesión, que existe $\lim\limits_{n \to \infty} z_n = z^\ast \in \mathbb{C}^\ast$.
    Entonces $z^\ast \in \bar{D} = D \cup \partial_\infty D$.
    \begin{itemize}
        \item Si $z^\ast \in \partial_\infty D$, entonces
              $$K \leq \limsup_{z \to z^\ast. z \in D} u(z) \leq M$$
        \item Si $z^\ast \in D$, como $\lim\limits_{n \to \infty} z_n = z^\ast$, se tiene que $\lim\limits_{n \to \infty} u(z_n) = u(z^\ast) = K$.
              Luego
              $$u(z) \leq u(z^\ast) \quad \forall z \in D$$
              Por la primera versión del principio del máximo, $u$ es constante en $D$, es decir, $u(z) = K$ para todo $z \in D$.
              Aplicando la hipótesis a un punto $\xi$ cualquiera de $\partial_\infty D$, vemos que $K \leq M$.
              Observamos que $\partial_\infty D \neq \emptyset$.
              Si $\partial_\infty D = \emptyset$, entonces $\bar{D} = D$, luego $D$ sería abierto y cerrado en $\mathbb{C}^\ast$.
              Por tanto, $D = \emptyset$ o $D = \mathbb{C}^\ast$, lo que es imposible.
    \end{itemize}
\end{proof}

\begin{theorem}[Principio del mínimo para funciones armónicas]
    Sea $D$ un dominio en $\mathbb{C}$ y sea $u$ armónica en $D$.
    \begin{enumerate}
        \item Si $u$ tiene un mínimo local en un punto $z_0 \in D$, entonces $u$ es constante en $D$.
        \item Si existe $m \in \mathbb{R}$ tal que
              $$\liminf_{z \to \xi, z \in D} u(z) \geq m, \quad \forall \xi \in \partial_infty D$$
              entonces $u(z) \geq m$ para todo $z \in D$.
              Además, si $u(z_0) = m$ para algún $z_0 \in D$, entonces $u$ es constante en $D$.
    \end{enumerate}
\end{theorem}

\begin{corollary}
    Sea $D$ un dominio en $\mathbb{C}$ y sea $u: \bar{D} \to \mathbb{R}$ armónica en $D$ y continua en $\bar{D}$.
    Entonces:
    \begin{align*}
        \max_{z \in \bar{D}} u(z) & = \max_{z \in \partial_\infty D} u(z) \\
        \min_{z \in \bar{D}} u(z) & = \min_{z \in \partial_\infty D} u(z)
    \end{align*}
\end{corollary}

\begin{corollary}
    Sea $D$ un dominio en $\mathbb{C}$ y sea $u: \bar{D} \to \mathbb{R}$ armónica en $D$ y continua en $\bar{D}$.
    Si $u = 0$ en $\partial_\infty D$, entonces $u = 0$ en $D$.
\end{corollary}

\begin{corollary}
    Sea $D$ un dominio en $\mathbb{C}$ y sean $u, v: \bar{D} \to \mathbb{R}$ armónicas en $D$ y continuas en $\bar{D}$.
    Si $u = v$ en $\partial_\infty D$, entonces $u = v$ en $D$.
\end{corollary}

\begin{remark}
    Sea $D$ un dominio en $\mathbb{C}$ y sea $u: \bar{D} \to \mathbb{R}$ armónica en $D$ y continua en $\bar{D}$.
    Entonces los valores de $u$ en $D$ están completamente determinados por los valores de $u$ en $\partial_\infty D$.
\end{remark}

\section{El problema de Dirichlet para el disco unidad}
Esto da lugar a la siguiente cuestión.
Sea $D$ un dominio en $\mathbb{C}$ y sea $f: \partial_\infty D \to \mathbb{R}$ continua.
Queremos estudiar si existe $u: \bar{D} \to \mathbb{R}$ armónica en $D$ y continua en $\bar{D}$ tal que $u = f$ en $D$.
Esto es el problema de Dirichlet en $D$ con valores frontera $f$.
Si existe una solución del problema de Dirichlet en $D$ con valores frontera $f$, esta es única.

Si $D$ es un dominio en $\mathbb{C}$, se dice que $D$ es regular para el problema de Dirichlet si para toda $f: \partial_\infty D \to \mathbb{R}$ continua existe la solución del problema de Dirichlet con valores frontera $f$.
Queremos resolver el problema de Dirichlet en el disco unidad.

Sea $u: \bar{\mathbb{D}} \to \mathbb{R}$ armónica en $\mathbb{D}$ y continua en $\bar{\mathbb{D}}$.
Vamos a expresar $u(z)$ para $z \in \mathbb{D}$ en función de $u(\xi)$, con $\xi \in \partial\mathbb{D}$.
Por la propiedad del valor medio, si $0 < r < 1$,
$$u(0) = \frac{1}{2\pi} \int_{-\pi}^\pi u(re^{it})dt$$
Si consideramos las funciones
$$u_r(t) = u(re^{it}), \quad t \in [-\pi, \pi]$$
vemos que $u_r \xrightarrow[r \to 1^-]{} u_1$ uniformemente porque $u$ es uniformemente continua en $\bar{\mathbb{D}}$.
Así que:
$$u(0) = \frac{1}{2\pi} \int_{-\pi}^\pi u_r(t)dt \xrightarrow[r \to 1^-]{} \frac{1}{2\pi} \int_{-\pi}^\pi u_1(t)dt \Rightarrow u(0) = \frac{1}{2\pi} \int_{-\pi}^\pi u(e^{it})dt$$

\begin{lemma}
    Sean $G_1$ y $G_2$ dos abiertos en $\mathbb{C}$, sea $f$ holomorfa en $G_1$ con $f(G_1) \subset G_2$ y sea $u$ armónica en $G_2$.
    Entonces $u \circ f$ es armónica en $G_1$.
\end{lemma}

\begin{proof}
    $u \circ f \in \mathcal{C}^2(G_1)$.
    Se puede comprobar que $\Delta(u \circ f) = 0$.
    De otra manera, sea $z_0 \in G_1$, entonces $f(z_0) \in G_2$.
    Tomamos $R > 0$ con $D(f(z_0), R) \subset G_2$.
    Como $f$ es continua en $z_0$, existe $\delta > 0$ tal que $D(z_0, \delta) \subset G_1$ y $f(D(z_0, \delta)) \subset D(f(z_0), R)$.

    Como $u$ es armónica en $D(f(z_0), R)$, existe $F$ holomorfa en $D(f(z), R)$ tal que $u = \Re(F)$ en $D(f(z_0), R)$.
    $$D(z_0, \delta) \xrightarrow{f} D(f(z_0), R) \xrightarrow{F} \mathbb{C}$$
    $F \circ f$ es holomorfa en $D(z_0, \delta)$, así que $\Re(f \circ F)$ es armónica en $D(z_0, \delta)$.
    Si $z \in D(z_0, \delta)$, tenemos que $f(z) \in D(f(z_0), R)$.
    $$\Re(F \circ f)(z) = \Re(F(f(z))) = u(f(z))$$
    Entonces $\Re(F \circ f) = u \circ f$.
    Por tanto $u \circ f$ es armónica en $D(z_0, \delta)$.
    Entonces $u \circ f$ es armónica en $D$.

    Sea $u: \bar{\mathbb{D}} \to \mathbb{R}$ armónica en $\mathbb{D}$ y continua en $\bar{\mathbb{D}}$.
    Hemos visto que:
    $$u(0) = \frac{1}{2\pi} \int_{-\pi}^\pi u(e^{it})dt$$
    Sea $a \in \mathbb{D}$.
    Consideramos $T_a(z) = \frac{z+a}{1+\bar{a}z}$, con $T_a(0) = a$ y $T_a(\bar{\mathbb{D}}) = \bar{\mathbb{D}}$.
    Sea $v = u \circ T_a$.
    $$\bar{\mathbb{D}} \xrightarrow{T_a} \bar{\mathbb{D}} \xrightarrow{u} \mathbb{R}$$
    $v$ es continua en $\bar{\mathbb{D}}$ y armónica en $\mathbb{D}$.
    $$u(a) = v(0) \frac{1}{2\pi} \int_{-\pi}^\pi v(e^{it})dt = \frac{1}{2\pi} \int_{-\pi}^\pi u(T_a(e^{it}))dt$$
    Hacemos el cambio de variable:
    $$T_a(e^{it}) = e^{i\varphi} \Leftrightarrow e^{it} = T_a^{-1}(e^{i\varphi}) = S_a(e^{i\varphi})$$
    Derivando,
    $$ie^{it}dt = ie^{i\varphi}S_a'(e^{i\varphi})d\varphi \Leftrightarrow dt = \frac{e^{i\varphi}S_a'(e^{i\varphi})}{S_a(e^{i\varphi})}d\varphi$$
    Recordamos que $S_a'(z) = \frac{1-|a|^2}{(1-\bar{a}z)^2}$, así que $S_a'(e^{i\varphi}) = \frac{1-|a|^2}{(1-\bar{a}e^{i\varphi})^2}$ y entonces:
    $$dt = \frac{e^{i\varphi}(1-|a|^2)}{(1-\bar{a}e^{i\varphi})^2}\frac{1-\bar{a}e^{i\varphi}}{e^{i\varphi}-a}d\varphi = \frac{1-|a|^2}{(1-\bar{a}e^{i\varphi})(1-ae^{i\varphi})}d\varphi = \frac{1-|a|^2}{|1-ae^{-i\varphi}|^2}d\varphi$$
    Entonces la integral queda de la forma:
    $$u(a) = \frac{1}{2\pi} \int_{-\pi + \theta_0}^{\pi + \theta_0} u(e^{i\varphi})\frac{1-|a|^2}{|1-ae^{-i\varphi}|^2}d\varphi = \frac{1}{2\pi} \int_{-\pi}^\pi u(e^{it})\frac{1-|a|^2}{|1-ae^{-it}|^2}dt$$
\end{proof}

Sea $u: \bar{\mathbb{D}} \to \mathbb{R}$ armónica en $\mathbb{D}$ y continua en $\bar{\mathbb{D}}$.
Hemos visto que:
$$u(0) = \frac{1}{2\pi} \int_{-\pi}^\pi u(e^{it})dt$$
Sea $a \in \mathbb{D}$.
Consideramos $T_a(z) = \frac{z+a}{1+\bar{a}z}$, con $T_a(0) = a$ y $T(\bar{\mathbb{D}}) = \bar{\mathbb{D}}$.
Sea $v = u \circ T_a$.
$$\bar{\mathbb{D}} \xrightarrow{T_a} \bar{\mathbb{D}} \xrightarrow{u} \mathbb{R}$$
$v$ es continua en $\bar{\mathbb{D}}$ y armónica en $\mathbb{D}$.
$$u(a) = v(0) = \frac{1}{2\pi} \int_{-\pi}^\pi v(e^{it})dt = \frac{1}{2\pi} \int_{-\pi}^\pi u(T_a(e^{it}))dt$$
Hacemos el cambio de variable:
$$T_a(e^{it}) = e^{i\varphi} \Leftrightarrow e^{it} = T_a^{-1}(e^{i\varphi}) = S_a(e^{i\varphi})$$
Derivando,
$$ie^{it}dt = ie^{i\varphi}S_a'(e^{i\varphi})d\varphi \Leftrightarrow dt = \frac{e^{i\varphi}S_a'(e^{i\varphi})}{S_a(e^{i\varphi})}d\varphi$$
Recordamos que $S_a(z) = \frac{z-a}{1-\bar{a}z}$ y $S_a'(z) = \frac{1-|a|^2}{(1-\bar{a}z)^2}$, así que $S_a'(e^{i\varphi}) = \frac{1-|a|^2}{(1-\bar{a}e^{i\varphi})^2}$ y entonces:
$$dt = \frac{e^{i\varphi}(1-|a|^2)}{(1-\bar{a}e^{i\varphi})^2}\frac{1-\bar{a}e^{i\varphi}}{e^{i\varphi}-a}d\varphi = \frac{1-|a|^2}{(1-\bar{a}e^{i\varphi})(1-ae^{-i\varphi})}d\varphi = \frac{1-|a|^2}{|1-ae^{-i\varphi}|^2}d\varphi$$
Entonces la integral queda de la forma:
$$u(a) = \frac{1}{2\pi} \int_{-\pi + \theta_0}^{\pi + \theta_0} u(e^{i\varphi})\frac{1-|a|^2}{|1-ae^{-i\varphi}|^2}d\varphi = \frac{1}{2\pi} \int_{-\pi}^\pi u(e^{it})\frac{1-|a|^2}{|1-ae^{-it}|^2}dt$$

\begin{theorem}
    Sea $u: \bar{\mathbb{D}} \to \mathbb{R}$ armónica en $\mathbb{D}$ y continua en $\bar{\mathbb{D}}$.
    Entonces
    $$u(z) = \frac{1}{2\pi} \int_{-\pi}^\pi u(e^{it})\frac{1-|z|^2}{|1-ze^{-it}|^2}dt, \quad \forall z \in \mathbb{D}$$
\end{theorem}

Si $f: \partial_\infty D \to \mathbb{R}$ es una función continua, sabemos que la solución del problema de Dirichlet en $\mathbb{D}$ con valores frontera $f$, de existir, es única.
Además, tendría que ser la siguiente:
$$u: \bar{\mathbb{D}} \to \mathbb{R}, \quad
    u(z) = \begin{cases}
        \dfrac{1}{2\pi} \int_{-\pi}^\pi f(e^{it})\dfrac{1-|z|^2}{|1-ze^{-it}|^2}dt & \text{si } z \in \mathbb{D}         \\
        u(z) = f(z)                                                                & \text{si } z \in \partial\mathbb{D}
    \end{cases}$$

\section{La integral de Poisson}
Una primera expresión del núcleo de Poisson es
$$P(z, e^{it}) = \frac{1-|z|^2}{|1-ze^{-it}|^2}, \quad z \in \mathbb{D}, \; t \in \mathbb{R}$$
Si $z = re^{i\theta}$, $0 \leq r < 1$, $\theta \in \mathbb{R}$,
$$P(z, e^{it}) = \frac{1-r^2}{|1-re^{i\theta}e^{-it}|^2} = \frac{1-r^2}{|1-re^{i(\theta-t)}|^2}$$
De esta forma llegamos a una segunda expresión del núcleo de Poisson:
$$P_r(t) = \frac{1-r^2}{|1-re^{it}|^2}, \quad 0 \leq r < 1, \; t \in \mathbb{R}$$
Podemos escribir el denominador de otra forma:
$$|1-re^{it}|^2 = (1-re^{it})(1-re^{-it}) = 1 - re^{-it} - re^{it} + r^2 = 1 - r(e^{it}+e^{-it}) + r^2 = 1 - 2r\cos(t) + r^2$$
Luego
$$P_r(t) = \frac{1-r^2}{1 - 2r\cos(t) + r^2}, \quad 0 \leq < 1, \; t \in \mathbb{R}$$

\begin{remark}
    \hfill
    \begin{enumerate}
        \item Si $0 \leq r < 1$, $\theta \in \mathbb{R}$, $t \in \mathbb{R}$ y $z = re^{i\theta}$, entonces $P(z, e^{it}) = P_r(\theta-t)$.
        \item Si $0 \leq r < 1$, $t \in \mathbb{R}$ y $z = re^{it}$, entonces $P_r(t) = P(z, e^{i0}) = P(z, 1)$.
    \end{enumerate}
\end{remark}

\begin{properties}
    \hfill
    \begin{enumerate}
        \item $P(z, e^{it}) > 0$ si $z \in \mathbb{D}$ y $t \in \mathbb{R}$.
              Equivalentemente, $P_r(t) > 0$ si $0 \leq r < 1$ y $t \in \mathbb{R}$.

        \item Fijando $t \in \mathbb{R}$, la función $z \in \mathbb{D} \mapsto P(z, e^{it}) > 0$ es armónica en $\mathbb{D}$.
              \begin{proof}
                  \hfill
                  \begin{itemize}
                      \item Si $t = 0$ y $z \in \mathbb{D}$,
                            \begin{align*}
                                P(z, e^{i0}) & = \frac{1-|z|^2}{|1-z|^2} = \frac{1-z\bar{z}}{(1-z)(1-\bar{z})} = \frac{1-z+z-z\bar{z}}{(1-z)(1-\bar{z})} = \frac{1}{1-\bar{z}} + \frac{z}{1-z} = \\
                                             & = \frac{1}{1-\bar{z}} + \frac{z-1+1}{1-z} = \frac{1}{1-\bar{z}} - 1 + \frac{1}{1-z} = -1 + \Re\left(\frac{1}{1-z}\right) =                        \\
                                             & = \Re\left(-1 + \frac{2}{1-z}\right) = \Re\left(\frac{1+z}{1-z}\right)
                            \end{align*}
                            La función $z \in \mathbb{D} \mapsto P(z, e^{i0}) = \Re\left(\frac{1+z}{1-z}\right)$ es armónica en $\mathbb{D}$.

                      \item Si $t \in \mathbb{R}$ y $z \in \mathbb{D}$, tomamos $w = ze^{-it} \in \mathbb{D}$, con $|w| = |z|$.
                            Entonces:
                            $$P(z, e^{it}) = \frac{1-|z|^2}{|1-ze^{-it}|^2} = \frac{1-|w|^2}{|1-w|^2} = \Re\left(\frac{1+w}{1-w}\right) = \Re\left(\frac{1+ze^{-it}}{1-ze^{-it}}\right)$$
                            La función $z \in \mathbb{D} \mapsto P(z, e^{it})$ es armónica en $\mathbb{D}$.
                  \end{itemize}
              \end{proof}

        \item Si $z \in \mathbb{D}$ y $t \in \mathbb{R}$,
              $$P(z, e^{it}) = \Re\left(\frac{1+ze^{-it}}{1-ze^{-it}}\right)$$
              Si $0 \leq r < 1$ y $t \in \mathbb{R}$,
              $$P_r(t) = \Re\left(\frac{1+re^{it}}{1-re^{it}}\right)$$

        \item Fijado $r \in [0, 1)$, tenemos la función
              \begin{align*}
                  P_r: \mathbb{R} & \to \mathbb{R}                                                               \\
                  t               & \mapsto P_r(t) = \frac{1-r^2}{|1-re^{it}|^2} = \frac{1-r^2}{1+r^2-2r\cos(t)}
              \end{align*}
              que es continua, positiva, periódica de periodo $2\pi$, par y decreciente en $[0, \pi]$.
              Observamos que:
              \begin{align*}
                  P_r(0)               & = \frac{1-r^2}{(1-r)^2} = \frac{1+r}{1-r} \xrightarrow[r \to 1^-]{} \infty \\
                  P_r(\pi) = P_r(-\pi) & = \frac{1-r^2}{(1+r)^2} = \frac{1-r}{1+r} \xrightarrow[r \to 1^-]{} 0
              \end{align*}
              Tenemos entonces la desigualdad:
              $$\frac{1-r}{1+r} \leq P_r(t) \leq \frac{1+r}{1-r}, \quad 0 \leq r < 1, \; t \in \mathbb{R}$$
              Además, si en el teorema anterior tomamos $u \equiv 1$ y $z = r$, tenemos:
              $$1 = u(r) = \frac{1}{2\pi} \int_{-\pi}^\pi P_r(\theta - t)dt = \frac{1}{2\pi} \int_{-\pi}^\pi P_r(-t)dt = \frac{1}{2\pi} \int_{-\pi}^\pi P_r(t)dt$$
              Por tanto,
              $$\frac{1}{2\pi} \int_{-\pi}^\pi P_r(t) = 1$$

        \item Para cada $t \in [-\pi, 0) \cup (0, \pi]$ se tiene que $P_r(t) \xrightarrow[r \to 1^-]{} 0$.
              Además, si $\delta \in (0, \pi)$, se tiene que $P_r(t) \xrightarrow[r \to 1^-]{} 0$ uniformemente en $[-\pi, -\delta] \cup [\delta, \pi]$.

              Si $t \in [-\pi, 0) \cup (0, \pi]$, $\cos(t) \neq 1$, así que:
              $$\lim_{r \to 1^-} P_r(t) = \frac{0}{2-2\cos(t)} = 0$$
              Si $\delta \in [0, \pi)$, entonces:
              $$P_r(t) \leq P_r(\delta) \xrightarrow[r \to 1^-]{} 0$$
    \end{enumerate}
\end{properties}

\begin{definition}
    Sea $f: \partial\mathbb{D} \to \mathbb{R}$ continua.
    Se define la integral de Poisson de $f$, y se denota $P[f]$, como la función
    \begin{align*}
        P[f]: \mathbb{D} & \to \mathbb{R}                                                           \\
        z                & \mapsto P[f](z) = \frac{1}{2\pi} \int_{-\pi}^\pi f(e^{it})P(z, e^{it})dt
    \end{align*}
    Equivalentemente, si $z = re^{i\theta}$ con $0 \leq r < 1$ y $\theta \in \mathbb{R}$, entonces
    $$P[f](z) = P[f](re^{it}) = \frac{1}{2\pi} \int_{-\pi}^\pi f(e^{it})P_r(\theta - t)dt$$
\end{definition}

\begin{theorem}
    El disco unidad $\mathbb{D}$ es un dominio regular para el problema de Dirichlet.
    En concreto, si $f: \partial\mathbb{D} \to \mathbb{R}$ es continua, entonces la solución del problema de Dirichlet en $\mathbb{D}$ con valores frontera $f$ es la función:
    \begin{align*}
        u: \bar{\mathbb{D}} & \to \mathbb{R}                       \\
        u(z)                & = \begin{cases}
                                    P[f](z) & z \in \mathbb{D}         \\
                                    f(z)    & z \in \partial\mathbb{D}
                                \end{cases}
    \end{align*}
    Es decir, $u$ es armónica en $\mathbb{D}$, continua en $\bar{\mathbb{D}}$ y $u = f$ en $\partial\mathbb{D}$.
\end{theorem}

\begin{proof}
    \hfill
    \begin{enumerate}
        \item $P[f]$ es armónica en $\mathbb{D}$.
              Si $z \in \mathbb{D}$, tenemos que
              \begin{align*}
                  P[f](z) & = \frac{1}{2\pi} \int_{-\pi}^\pi f(e^{it})P(z, e^{it})dt = \frac{1}{2\pi} \int_{-\pi}^\pi f(e^{it})\Re\left(\frac{1+ze^{-it}}{1-ze^{-it}}\right)dt =                                \\
                          & = \frac{1}{2\pi} \int_{-\pi}^\pi \Re\left(f(e^{it})\frac{1+ze^{-it}}{1-ze^{-it}}\right)dt = \Re\left(\frac{1}{2\pi} \int_{-\pi}^\pi f(e^{it})\frac{1+ze^{-it}}{1-ze^{-it}}dt\right)
              \end{align*}
              Veamos que $\frac{1}{2\pi} \int_{-\pi}^\pi f(e^{it})\frac{1+ze^{-it}}{1-ze^{-it}}dt$ es una función holomorfa en $\mathbb{D}$ expresándola como serie de potencias.

              Si $z \in \mathbb{D}$,
              $$\frac{1+z}{1-z} = \frac{1}{1-z} + \frac{z}{1-z} = \sum_{n=0}^\infty z^n + \sum_{n=0}^\infty z^{n+1} = 1 + 2\sum_{n=1}^\infty z^n$$
              converge uniformemente en cada subconjunto compacto de $\mathbb{D}$.
              Así que:
              $$\frac{1+ze^{-it}}{1-ze^{-it}} = 1 + 2\sum_{n=1}^\infty z^ne^{-int}, \quad t \in [-\pi, \pi]$$
              con convergencia uniforme.
              Entonces:
              \begin{align*}
                  P[f](z) & = \Re\left(\frac{1}{2\pi} \int_{-\pi}^\pi f(e^{it})\left(1 + 2\sum_{n=1}^\infty z^ne^{-int}\right)dt\right) =                                          \\
                          & = \Re\left(\frac{1}{2\pi} \int_{-\pi}^\pi \left(f(e^{it}) + 2\sum_{n=1}^\infty f(e^{it})z^ne^{-int}\right)dt\right) =                                  \\
                          & = \Re\left(\frac{1}{2\pi} \int_{-\pi}^\pi f(e^{it})dt + \frac{1}{2\pi} \int_{-\pi}^\pi \left(2\sum_{n=1}^\infty f(e^{it})z^ne^{-int}\right)dt\right) = \\
                          & = \Re\left(\frac{1}{2\pi} \int_{-\pi}^\pi f(e^{it})dt + 2\sum_{n=1}^\infty\left(\frac{1}{2\pi} \int_{-\pi}^\pi f(e^{it})e^{-int}dt\right)z^n\right)
              \end{align*}

              Sea
              $$a_n = \frac{1}{2\pi} \int_{-\pi}^\pi f(e^{it})e^{-int}dt, \quad n = 0, 1, 2, \dots$$
              Entonces:
              $$P[f](z) = \Re\left(a_0 + 2\sum_{n=1}^\infty a_nz^n\right)$$
              Esta es una serie de potencias centrada en 0 que converge para todo $z \in \mathbb{D}$.
              Su radio de convergencia es mayor o igual que 1.
              Por tanto, define una función holomorfa en $\mathbb{D}$.
              Entonces la función $z \in \mathbb{D} \mapsto P[f](z)$ es armónica en $\mathbb{D}$.

        \item Veamos que:
              $$\lim_{z \to 1, z \in \mathbb{D}} u(z) = f(1)$$
              Si $z \in \mathbb{D}$, $z = re^{i\theta}$ con $0 \leq r < 1$ y $\theta \in \mathbb{R}$.
              \begin{align*}
                  u(z) - f(1) & = P[f](z) - f(1) = \frac{1}{2\pi} \int_{-\pi}^\pi f(e^{it})P_r(\theta-t)dt - f(1) =                                         \\
                              & = \frac{1}{2\pi} \int_{-\pi}^\pi f(e^{it})P_r(\theta-t)dt - f(1)\frac{1}{2\pi} \int_{-\pi}^\pi P_r(s)ds =                   \\
                              & = \frac{1}{2\pi} \int_{-\pi+\theta}^{\pi+\theta} f(e^{i(\theta-s)})P_r(s)ds - \frac{1}{2\pi} \int_{-\pi}^\pi f(1)P_r(s)ds = \\
                              & = \frac{1}{2\pi} \int_{-\pi}^\pi \left(f(e^{i(\theta-s)}) - f(e^{i0})\right)P_r(s)ds
              \end{align*}
              Así que
              $$|u(z)-f(1)| \leq \frac{1}{2\pi} \int_{-\pi}^\pi \left|f\left(e^{i(\theta-t)}\right) - f\left(e^{i0}\right)\right|P_r(t)dt$$
              Sea $\varepsilon > 0$, como la función $t \in [-\pi, \pi] \mapsto f(e^{it})$ es uniformemente continua, existe $\delta_1 \in (0, \pi)$ tal que si $t, s \in [-\pi, \pi]$ con $|t-s| < \delta_1$, entonces $|f(e^{it})-f(e^{is})| < \frac{\varepsilon}{2}$.
              Como $f$ está acotada en $\partial\mathbb{D}$, existe $M > 0$ tal que $|f(z)| \leq M$ para todo $z \in \partial\mathbb{D}$.
              Tomamos $\delta = \frac{\delta_1}{2} \in \left(0, \frac{\pi}{2}\right)$.
              Como $P_r(t) \xrightarrow[r \to 1^-]{} 0$ uniformemente en $[-\pi, -\delta] \cup [\delta, \pi]$, existe $r_0 \in (0, 1)$ tal que
              $$\begin{cases}
                      r_0 < r < 1 \\
                      t \in [-\pi, -\delta] \cup [\delta, \pi]
                  \end{cases} \Rightarrow P_r(t) < \frac{\varepsilon}{4M}$$
              Entonces, si $r_0 < r < 1$ y $|\theta| < \delta$, vamos a ver que $|u(z)-f(1)| < \varepsilon$.
              \begin{align*}
                  |u(z)-f(1)| \leq & \frac{1}{2\pi} \int_{-\delta}^\delta \left|f\left(e^{i(\theta-t)}\right) - f\left(e^{i0}\right)\right|P_r(t)dt +                        \\
                  +                & \frac{1}{2\pi} \int_{[-\pi, \pi] \setminus [-\delta, \delta]} \left|f\left(e^{i(\theta-t)}\right) - f\left(e^{i0}\right)\right|P_r(t)dt
              \end{align*}
              Si $t \in [-\delta, \delta]$, tenemos que
              $$|\theta-t-0| \leq |\theta| + |t| < \delta + \delta = 2\delta = \delta_1 < \pi$$
              Además, $\theta-t, 0 \in [-\pi, \pi]$.
              Así que:
              $$\left|f\left(e^{i(\theta-t)}\right) - f\left(e^{i0}\right)\right| < \frac{\varepsilon}{2}$$
              Luego:
              \begin{align*}
                  |u(z)-f(1)| & \leq \frac{1}{2\pi} \int_{-\delta}^\delta \frac{\varepsilon}{2}P_r(t)dt + \frac{1}{2\pi} \int_{[-\pi, \pi] \setminus [-\delta, \delta]} 2M\frac{\varepsilon}{4M}dt =                     \\
                              & \leq \frac{\varepsilon}{2}\frac{1}{2\pi} \int_{-\pi}^\pi P_r(t)dt + \frac{1}{2\pi} \int_{-\pi}^\pi \frac{\varepsilon}{2}dt = \frac{\varepsilon}{2} + \frac{\varepsilon}{2} = \varepsilon
              \end{align*}

        \item Veamos que
              $$\lim_{z \to \xi, z \in \mathbb{D}} u(z) = f(\xi), \quad \forall \xi \in \partial\mathbb{D}$$
              Sea $g: \partial\mathbb{D} \to \mathbb{R}$, $g(z) = f(\xi z)$, es continua.
              Sea $v = P[g]: \mathbb{D} \to \mathbb{R}$.
              Sabemos que $\lim\limits_{z \to 1, z \in \mathbb{D}} v(z) = g(1)$.
              Veamos que $v(z) = u(\xi z)$ para todo $z \in \mathbb{D}$.

              Sea $z \in \mathbb{D}$, $z = re^{i\theta}$ con $0 \leq r < 1$ y $\theta \in \mathbb{R}$.
              $$v(z) = P[g](z) = \frac{1}{2\pi} \int_{-\pi}^\pi g(e^{it})P_r(\theta-t)dt = \frac{1}{2\pi} f(\xi e^{it})P_r(\theta-t)dt$$
              Si escribimos $\xi = e^{i\theta_0}$, con $\theta_0 \in \mathbb{R}$, entonces:
              \begin{align*}
                  v(z) & = \frac{1}{2\pi} f(\xi e^{i(\theta_0-t)})P_r(\theta-t)dt = \frac{1}{2\pi} f(\xi e^{is})P_r(\theta+\theta_0-s)ds = P[f](re^{i(\theta-\theta_0)}) = \\
                       & = P[f](\xi z) = u(\xi z)
              \end{align*}
              Entonces:
              $$\lim_{z \to 1, z \in \mathbb{D}} u(\xi z) = f(\xi) \Rightarrow \lim_{z \to \xi, z \in \mathbb{D}} u(z) = f(\xi)$$

        \item Veamos que $u$ es continua en $\bar{\mathbb{D}}$.
              $u$ es continua en $\mathbb{D}$ porque es armónica en $\mathbb{D}$.
              Dado $\xi \in \partial\mathbb{D}$, hay que probar que
              $$\lim_{z \to \xi, z \in \bar{\mathbb{D}}} f(\xi)$$

              Dada $\{z_n\}_{n=1}\infty$ sucesión en $\bar{\mathbb{D}}$ con $z_n \to \xi$, definimos la sucesión $\{w_n\}_{n=1}^\infty$ como sigue.
              \begin{itemize}
                  \item Si $z_n \in \mathbb{D}$, tomamos $w_n = z_n$.
                  \item Si $z_n \in \partial\mathbb{D}$, tomamos $w_n \in \mathbb{D}$ con $|z_n-w_n| < \frac{1}{n}$ y $|u(z_n)-u(w_n)| < \frac{1}{n}$.
                        Esto se puede hacer porque $\lim\limits_{z \to z_n, z \in \mathbb{D}} u(z) = u(z_n)$.

                        Hay que probar que $u(z_n) \to f(\xi)$.
                        $$|u(z_n)-f(\xi)| \leq |u(z_n)-u(w_n)| + |u(w_n)-f(\xi)|$$
                        Observamos que
                        $$|w_n-\xi| \leq |w_n-z_n| + |z_n-\xi| \to 0$$
                        Además, sabemos que $\lim\limits_{z \to \xi, z \in \mathbb{D}} u(z) = f(\xi)$, así que $u(w_n) \to f(\xi)$.
                        Por tanto,
                        $$|u(z_n) - f(\xi)| \to 0$$
              \end{itemize}
    \end{enumerate}
\end{proof}

\begin{theorem}
    Sea $D$ un dominio simplemente conexo en $\mathbb{C}$ con $D \neq \mathbb{C}$.
    Supongamos que existe una aplicación conforme $F$ de $\mathbb{D}$ sobre $D$ que se puede extender a un homeomorfismo de $\bar{\mathbb{D}}$ sobre $\bar{D} = D \cup \partial_\infty D$.
    Entonces $D$ es regular para el problema de Dirichlet.

    En concreto, si $f: \partial_\infty D \to \mathbb{R} \to \mathbb{R}$ continua, la función $u: \bar{D} \to \mathbb{R}$ definida por
    $$u(z) = \begin{cases}
            f(z)                           & z \in \partial_\infty D \\
            u(z) = P[f \circ F](F^{-1}(z)) & z \in D
        \end{cases}$$
    es la solución del problema de Dirichlet en $D$ con valores frontera $f$.
\end{theorem}

\begin{proof}
    $F: \mathbb{D} \to D$ se puede extender a una homeomorfismo $F: \bar{\mathbb{D}} \to \bar{D}$.
    Observamos que $F$ es un homeomorfismo de $\partial\mathbb{D}$ sobre $\partial_\infty D$.
    Sea $g = f \circ F$.
    $$g: \partial\mathbb{D} \xrightarrow{F} \partial_\infty D \xrightarrow{f} \mathbb{R}$$
    $g$ es continua.
    Sea $v$ la solución del problema de Dirichlet en $\mathbb{D}$ con valores frontera $g$.
    $v: \bar{\mathbb{D}} \to \mathbb{R}$ es armónica en $\mathbb{D}$, continua en $\bar{\mathbb{D}}$ y con $v = g$ en $\partial\mathbb{D}$.

    Sea $u: \bar{D} \to \mathbb{R}$, $u = v \circ F^{-1}$.
    $u$ es continua en $\bar{D}$ y armónica en $D$.
    Si $z \in \partial_\infty D$,
    $$u(z) = v(F^{-1}(z)) = g(F^{-1}(z)) = f(z)$$
    Luego $u = f$ en $\partial_\infty D$.
\end{proof}

\begin{example}
    Cualquier disco abierto es un dominio regular para el problema de Dirichlet.

    Sea $a \in \mathbb{C}$ y $R > 0$.
    Sea $D = D(a, R)$.
    Sea $F: \mathbb{D} \to D$, $F(z) = a + Rz$.
    $D$ es regular para el problema de Dirichlet.

    Si $f: \partial D(a, R) \to \mathbb{R}$ es continua, entonces la solución del problema de Dirichlet en $D(a, R)$ con valores frontera $f$ es $u: \bar{D}(a, R) \to \mathbb{R}$, con
    $$u(z) = f(z), \quad z \in \partial D(a, R)$$
    Si $z \in D(a, R)$, $z = a + re^{i\theta}$ con $0 \leq r < R$ y $\theta \in \mathbb{R}$,
    \begin{align*}
        u(z) & = P[f \circ F](F^{-1}(z)) = P[f \circ F]\left(\frac{z-a}{R}\right) = P[f \circ F]\left(\frac{r}{R}e^{i\theta}\right) =                                                                                      \\
             & = \frac{1}{2\pi} \int_{-\pi}^\pi (f \circ F)(e^{it})P_{r/R}(\theta-t)dt = \frac{1}{2\pi} \int_{-\pi}^\pi f(a + Re^{it})\frac{1-\left(\frac{r}{R}\right)^2}{\left|1-\frac{r}{R}e^{i(\theta-t)}\right|^2}dt = \\
             & = \frac{1}{2\pi} \int_{-\pi}^\pi f(a + Re^{it})\frac{R^2-r^2}{|R-re^{i(\theta-t)}|^2}dt
    \end{align*}
\end{example}

Sea $D$ el dominio interior a una curva de Jordan en $\mathbb{C}$.
Entonces $D$ es un dominio simplemente conexo en $D$, con $D \neq \mathbb{C}$.
Por el teorema de Riemann, existe una aplicación conforme $F$ de $\mathbb{D}$ sobre $D$, que se puede extender a un homeomorfismo de $\bar{\mathbb{D}}$ sobre $\bar{D}$ por el teorema de extensión de Carathéodory.
Por el teorema anterior, $D$ es regular para el problema de Dirichlet.
Si $f: \partial D \to \mathbb{R}$ es continua, entonces la función $u: \bar{D} \to \mathbb{R}$ dada por
$$u(z) = \begin{cases}
        f(z)                    & z \in \partial D \\
        P[f \circ F](F^{-1}(z)) & z \in D
    \end{cases}$$
es la solución del problema de Dirichlet en $D$ con valores frontera $f$.

\begin{theorem}[Principio del máximo para funciones armónicas: tercera versión]
    Sea $D$ un dominio acotado en $\mathbb{C}$ y sea $u$ una función armónica y acotada superiormente en $D$.
    Si existen $M \in \mathbb{R}$ y $\{\xi_1, \xi_2, \dots, \xi_n\} \in \partial D$ tales que
    $$\limsup_{z \to \xi, z \in D} u(z) \leq M, \quad \forall \xi \in \partial D \setminus \{\xi_1, \xi_2, \dots \xi_n\}$$
    entonces $u(z) \leq M$ para todo $z \in D$.
\end{theorem}

\begin{proof}
    Sea $\alpha = diam(D) = \sup \{|z-w| : z, w \in D\}$.
    Entonces $0 < \alpha < \infty$.
    Dado $\varepsilon > 0$, sea $u_\varepsilon: D \to \mathbb{R}$ dada por
    $$u_\varepsilon(z) = u(z) + \varepsilon \sum_{i=1}^n \Log\left(\frac{|z-\xi_j|}{\alpha}\right)$$
    Si $z \in D$ y $j \in \{1, \dots, n\}$, como $|z-w| \leq \alpha$ para todo $w \in D$, se tiene que
    $$0 < |z-\xi_j| \leq \alpha \Rightarrow 0 < \frac{|z-\xi_j|}{\alpha} \leq 1 \Rightarrow \Log\left(\frac{|z-\xi_j|}{\alpha}\right) \leq 0$$
    Entonces $u_\varepsilon \leq u$ en $D$.
    Como la función $z \mapsto \frac{z-\xi_j}{\alpha}$ es holomorfa y nunca nula en $D$, se tiene que $\Log\left(\frac{|z-\xi_j|}{\alpha}\right)$ es armónica en $D$ para cada $j$.
    Por tanto, $u_\varepsilon$ es armónica en $D$.

    Sea $\xi \in \partial D \setminus \{\xi_1, \xi_2, \dots, \xi_n\}$.
    Entonces:
    $$\limsup_{z \to \xi, z \in D} u_\varepsilon(z) \leq \limsup_{z \to \xi, z \in D} u(z) \leq M$$
    Sea $j_0 \in \{1, \dots, n\}$.
    $$\limsup_{z \to \xi_{j_0}, z \in D} u_\varepsilon(z) = \limsup_{z \to \xi_{j_0}, z \in D} \left(u(z) + \varepsilon\Log\left(\frac{|z-\xi_{j_0}|}{\alpha}\right) + \varepsilon \sum_{j = 1, j \neq j_0}^n \Log\left(\frac{|z-\xi_j|}{\alpha}\right)\right) = -\infty \leq M$$

    Por la segunda versión del principio del máximo, $u_\varepsilon(z) \leq M$ para todo $z \in D$.
    Entonces hemos probado que, dado $\varepsilon > 0$, se tiene que
    $$u(z) + \varepsilon \sum_{j=1}^n \Log\left(\frac{|z-\xi_j|}{\alpha}\right) \leq M, \quad \forall z \in D$$
    Haciendo $\varepsilon \to 0$, tenemos que $u(z) \leq M$ para todo $z \in D$.
\end{proof}

\begin{theorem}[Principio del máximo para funciones armónicas: cuarta versión]
    Sea $D$ un dominio en $\mathbb{C}$ con exterior no vacío y sea $u$ una función armónica y acotada superiormente en $D$.
    Supongamos que existen $M \in \mathbb{R}$ y $\{\xi_1, \xi_2, \dots, \xi_n\} \in \partial_\infty D$ tales que
    $$\limsup_{z \to \xi, z \in D} u(z) \leq M, \quad \forall \xi \in \partial_\infty D \setminus \{\xi_1, \xi_2, \dots \xi_n\}$$
    Entonces $u(z) \leq M$ para todo $z \in D$.
\end{theorem}

\begin{proof}
    Sea $a \in \mathbb{C}$ un punto exterior a $D$.
    Existe $R > 0$ tal que $D(a, R) \cap D = \emptyset$.
    Entonces $|z-a| \geq R$ para todo $z \in D$.
    Para cada $j \in \{1, \dots, n\}$ vamos a construir una función $h_j$ holomorfa y nunca nula en $D$, con $|h_j| \leq 1$ en $D$ y tal que $\lim\limits_{z \to \xi_j, z \in D} h_j(z) = 0$ y $\lim\limits_{z \to \xi_i, x \in D} h_j(z)$ existe en $\mathbb{C} \setminus \{0\}$ si $i \neq j$.
    \begin{itemize}
        \item Si $\xi_j = \infty$,
              $$h_j(z) = \frac{R}{z-a}, \quad z \in D$$
              $h_j$ es holomorfa y nunca nula en $D$ con $|h_j| \leq 1$ en $D$.

        \item Si $\xi_j \in \mathbb{C}$,
              $$\left|\frac{z-\xi_j}{z-a}\right| = \left|\frac{z-a+a-\xi_j}{z-a}\right| \leq \frac{|z-a|+|a-\xi_j|}{|z-a|} = 1 + \frac{|a-\xi_j|}{|z-a|} \leq 1 + \frac{|a-\xi_j|}{R} = K_j > 0$$
              Sea
              $$h_j(z) = \frac{1}{K_j}\frac{z-\xi_j}{z-a}$$
              $h_j$ es holomorfa y nunca nula en $D$, con
              $$|h_j(z)| = \frac{1}{K_j}\left|\frac{z-\xi_j}{z-a}\right| \leq \frac{1}{K_j}K_j = 1, \quad \forall z \in D$$
              Además,
              $$\begin{cases}
                      \lim_{z \to \xi_j, z \in D} h_j(z) = 0                                                                                  \\
                      \lim_{z \to \xi_i, z \in D} h_j(z) = \frac{1}{K_j}\frac{\xi_i-\xi_j}{\xi_i-a} \in \mathbb{C} \setminus \{0\} & i \neq j
                  \end{cases}$$
              Sea $\varepsilon > 0$, definimos:
              $$u_\varepsilon(z) = u(z) + \varepsilon \sum_{n=1}^\infty \Log|h_j(z)|, \quad z \in D$$
              Se sigue como en la demostración del teorema anterior.
    \end{itemize}
\end{proof}

\begin{theorem}[Teorema de la singularidad evitable para funciones armónicas]
    Sean $a \in \mathbb{C}$ y $R > 0$.
    Si $u$ es armónica y acotada en $D(a, R) \setminus \{a\}$, entonces $u$ se puede extender a una función armónica en $D(a, R)$.
\end{theorem}

\begin{proof}
    Sea $f = u|_{\partial D(a, R/2)}$, $f: \partial D(a, R/2) \to \mathbb{R}$ es continua.
    Como $D(a, R/2)$ es regular para el problema de Dirichlet, existe $v: \bar{D}(a, R/2) \to \mathbb{R}$ armónica en $D(a, R/2)$, continua en $\bar{D}(a, R/2)$ y con $v = f$ en $\partial D(a, R/2)$.

    Sea $D = D(a, R/2) \setminus \{a\}$.
    $u$ y $v$ son armónicas en $D$, $u-v$ y $v-u$ también.
    Aplicamos la tercera versión del principio del máximo a $u-v$ y $v-u$ en $D$.
    $D$ es un dominio acotado en $\mathbb{C}$, $u$ está acotada en $D$ y $v$ es continua en $\bar{D}(a, R/2)$, así que $v$ está acotada en $D$.
    Luego $u-v$ y $v-u$ están acotadas en $D$.

    Si $\xi \in \partial D(a, R/2) = \partial D \setminus \{a\}$,
    \begin{align*}
        \limsup_{z \to \xi, z \in D} (u(z) - v(z)) & \leq u(\xi) - v(\xi) = 0 \\
        \limsup_{z \to \xi, z \in D} (v(z) - u(z)) & \leq v(\xi) - u(\xi) = 0
    \end{align*}

    Por tanto, $u(z) - v(z) \leq 0$ y $v(z) - u(z) \leq 0$ para todo $z \in D$.
    Luego $u(z) = v(z)$ para todo $z \in D = D(a, R/2) \setminus \{a\}$, con $v$ armónica en $D(a, R/2)$.
    Definiendo $u(a) = v(a)$, vemos que $u$ se puede extender a una función armónica en $D(a, R)$.
\end{proof}

\begin{theorem}
    $\mathbb{D} \setminus \{0\}$ es un dominio en $\mathbb{C}$ que no es regular para el problema de Dirichlet.
\end{theorem}

\begin{proof}
    $D = \mathbb{D} \setminus \{0\}$, $\partial D = \partial\mathbb{D} \cup \{0\}$.
    Sea $f: \partial D \to \mathbb{R}$, $f(z) = 0$ si $|z| = 1$ y $f(0) = 1$.
    $f$ es continua en $\partial D$.
    Supongamos que existe $u: \bar{\mathbb{D}} \to \mathbb{R}$ armónica en $\mathbb{D} \setminus \{0\}$, continua en $\bar{\mathbb{D}}$ y con $u = f$ en $\partial D$.
    Como $u$ es armónica y acotada en $\mathbb{D} \setminus \{0\}$, por el teorema anterior $u$ se puede extender a una función armónica en $\mathbb{D}$.
    Entonces $u$ es armónica en $\mathbb{D}$ y continua en $\bar{\mathbb{D}}$.
    Por tanto,
    $$\max_{z \in \bar{\mathbb{D}}} u(z) = \max_{z \in \partial\mathbb{D}} u(z) = 0$$
    Sin embargo, esto contradice que $u(0) = 1 \neq 0$.
\end{proof}

\begin{remark}
    La condición de que $u$ está acotada superiormente no se puede suprimir en la tercera versión del principio del máximo.
\end{remark}

\begin{example}
    Sea $D = \mathbb{D} \setminus \{0\}$ dominio acotado en $\mathbb{C}$ y sea $u(z) = \Log\left(\frac{1}{|z|}\right)$, $z \in D$.
    Como la función $z \mapsto \frac{1}{z}$ es holomorfa y nunca nula en $D$, entonces $u$ es armónica en $D$.
    $u$ no está acotada superiormente en $D$.
    $$\partial D = \partial\mathbb{D} \cup \{0\}$$
    Si $\xi \in \partial\mathbb{D}$,
    $$\limsup_{z \to \xi, z \in D} u(z) = \limsup_{z \to \xi, z \in \mathbb{D}} \Log\left(\frac{1}{|z|}\right) = 0 = M$$
    Pero no es cierto que $u(z) \leq 0$ para todo $z \in D$.
    De hecho, $u(z) > 0$ para todo $z \in D$.
\end{example}

\begin{lemma}
    Sean $a \in \mathbb{C}$ y $R > 0$.
    Sea $u: \bar{D}(a, R) \to \mathbb{R}$ una función continua, con $u \equiv 0$ en $\partial D(a, R)$.
    Supongamos que para cada $z_0 \in D(a, R)$ existe $r_{z_0} > 0$ con $D(z_0, r_{z_0}) \subset D(a, R)$ tal que
    $$u(z_0) = \frac{1}{2\pi} \int_{-\pi}^\pi u(z_0 + re^{it})dt, \quad r \in [0, r_{z_0}]$$
    Entonces $u \equiv 0$ en $D(a, R)$.
\end{lemma}

\begin{proof}
    Sea $M = \max\{u(z) : z \in \bar{D}(a, R)\}$ y sea $K = \{z \in \bar{D}(a, R) : u(z) = M\}$.
    $K$ es un compacto no vacío con $K \subset \bar{D}(a, R)$.
    Como la función $z \in \mathbb{C} \mapsto |z-a|$ es continua, alcanza el máximo en $K$.
    Tomamos $z_0 \in K$ tal que $|z_0-a| = \max\{|z-a| : z \in K\}$.

    Supongamos por reducción al absurdo que $z_0 \in D(a, R)$.
    Tomamos $r_{z_0} > 0$, que existe por hipótesis, y fijamos $r \in (0, r_{z_0})$.
    Entonces
    $$u(z_0) = \frac{1}{2\pi} \int_{-\pi}^\pi u(z_0 + re^{it})dt$$
    Sean $E_1 = \{t \in [-\pi, \pi] : |z_0+re^{it}-a| \leq |z_0-a|\}$ y $E_2 = \{t \in [-\pi, \pi] : |z_0+re^{it}-a| > |z_0-a|\}$.
    Observamos que $E_2 \neq \emptyset$.
    Si $t \in E_2$, se tiene que $z_0 + re^{it} \notin K$.
    Así que $u(z_0 + re^{it}) < M$.
    Entonces
    $$u(z_0) = \frac{1}{2\pi} \int_{E_1} u(z_0 + re^{it})dt + \frac{1}{2\pi} \int_{E_2} u(z_0 + re^{it})dt$$
    Observamos que:
    \begin{align*}
        \frac{1}{2\pi} \int_{E_1} u(z_0 + re^{it})dt \leq \frac{1}{2\pi} \int_{E_1} Mdt \\
        \frac{1}{2\pi} \int_{E_1} u(z_0 + re^{it})dt < \frac{1}{2\pi} \int_{E_2} Mdt
    \end{align*}
    Por tanto,
    $$u(z_0) < \frac{1}{2\pi} \int_{E_1} Mdt + \frac{1}{2\pi} \int_{E_2} Mdt = \frac{1}{2\pi} \int_{-\pi}^\pi Mdt = M$$
    Esto contradice que $z_0 \in K$.
    Entonces $z_0 \in \partial D(a, R)$, así que por hipótesis $u(z_0) = 0$.
    Es decir,
    $$u(z_0) = M = \max\{u(z) : z \in \bar{D}(a, R)\} = 0$$
    Por tanto, $u(z) \leq 0$ para todo $z \in D(a, R)$.

    Aplicando la parte del lema que acabamos de ver a la función $-u$, tenemos que $-u(z) \leq 0$ para todo $z \in D(a, R)$.
    Por tanto, $u \equiv 0$ en $D(a, R)$.
\end{proof}

\begin{theorem}[Caracterización de la armonicidad por la propiedad del valor medio]
    Sea $D$ un abierto en $\mathbb{C}$ y sea $u: D \to \mathbb{R}$ continua.
    Supongamos que para cada $a \in D$ existe $r_a > 0$ con $D(a, r_a) \subset D$, tal que
    $$u(a) = \frac{1}{2\pi} \int_{-\pi}^\pi u(a + re^{it})dt, \quad r \in [0, r_a]$$
    Entonces $u$ es armónica en $D$.
\end{theorem}

\begin{proof}
    Sea $a \in D$.
    Tomamos $r_a > 0$ y $R = \frac{r_a}{2}$.
    Sea $v$ la solución del problema de Dirichlet en $D(a, R)$ con valores frontera $u$.
    Entonces $v: \bar{D}(a, R) \to \mathbb{R}$ es armónica en $D(a, R)$, continua en $\bar{D}(a, R)$ y con $u = v$ en $\partial D(a, R)$.
    Vamos a aplicar el lema a $u-v$.
    $u-v: \bar{D}(a, R) \to \mathbb{R}$ es continua y $u-v \equiv 0$ en $\partial D(a, R)$.
    Sea $z_0 \in D(a, R)$.
    Tomamos $r_{z_0} > 0$, que elegimos suficientemente pequeño para que $D(z_0, r_{z_0}) \subset D(a, R)$.
    Si $r \in [0, r_{z_0})$, tenemos que
    $$u(z_0)-v(z_0) = \frac{1}{2\pi} \int_{-\pi}^\pi u(z_0 + re^{it})dt - frac{1}{2\pi} \int_{-\pi}^\pi v(z_0 + re^{it})dt = frac{1}{2\pi} \int_{-\pi}^\pi (u-v)(z_0 + re^{it})dt$$
    Por el lema, $u-v = 0$ en $D(a, R)$.
    Como $v$ es armónica en $D(a, R)$, $u$ es armónica en $D(a, R)$.
    Para cada $a \in D$ hemos encontrado $R > 0$ con $D(a, R) \subset D$ tal que $u$ es armónica en $D(a, R)$.
    Entonces $u$ es armónica en $D$.
\end{proof}

\begin{remark}
    Esta es la forma débil de la propiedad del valor medio.
    Como $u$ es armónica en $D$, tenemos que $u$ verifica la propiedad del valor medio.
\end{remark}

\begin{theorem}
    Sea $D$ un dominio en $\mathbb{C}$ y sea $\{u_n\}_{n=1}^\infty$ una sucesión de funciones armónicas en $D$.
    Supongamos que $\{u_n\}$ converge uniformemente en cada subconjunto compacto de $D$.
    Sea $u(z) = \lim\limits_{n \to \infty} u_n(z)$, $z \in D$.
    Entonces $u$ es armónica en $D$.
\end{theorem}

\begin{proof}
    Tenemos $u: D \to \mathbb{R}$.
    Si $K \subset D$, $K$ compacto, tenemos que $u_n \to u$ uniformemente en $K$ y $u_n$ es continua en $K$ para cada $n \in \mathbb{N}$.
    Así que $u$ es continua en $K$, luego $u$ es continua en $D$.
    Sea $a \in D$ y sea $R > 0$ con $D(a, R) \subset D$.
    Sea $r \in [0, R)$.
    Tenemos que
    $$u_n(a) = \frac{1}{2\pi} \int_{-\pi}^\pi u_n(a + re^{it})dt, \quad \forall n \in \mathbb{N}$$
    Haciendo $n \to \infty$,
    $$u(a) = \lim_{n \to \infty} \frac{1}{2\pi} \int_{-\pi}^\pi u_n(a + re^{it})dt$$
    Como $u_n \to u$ uniformemente en $\partial D(a, r)$,
    $$u(a) = \lim_{n \to \infty} \frac{1}{2\pi} \int_{-\pi}^\pi u_n(a + re^{it})dt = \frac{1}{2\pi} \int_{-\pi}^\pi \lim_{n \to \infty} u_n(a + re^{it})dt = \frac{1}{2\pi} \int_{-\pi}^\pi u(a + re^{it})dt$$
    Por el teorema anterior, $u$ es armónica en $D$.
\end{proof}

Vamos a ver qué sucede al cambiar la condición de convergencia uniforme en compactos por la condición de que $\{u_n\}$ sea creciente.

\section{Desigualdades de Harnack}
Sean $a \in \mathbb{C}$ y $R > 0$.
Sea $u$ una función armónica y no negativa en $D(a, R)$.
Entonces, si $r \in (0, R)$, se tiene que
$$\frac{R-r}{R+r}u(a) \leq u(z) \leq \frac{R+r}{R-r}, \quad \forall z \in \bar{D}(a, R)$$
Por tanto, tomando $r = \frac{R}{2}$, tenemos
$$\frac{1}{3}u(a) \leq u(z) \leq 3u(a), \quad z \in \bar{D}\left(a, \frac{R}{2}\right)$$

\begin{proof}
    Fijamos $r \in (0, R)$.
    Tomamos $\rho$ con $r < \rho < R$.
    Sea $v: \bar{\mathbb{D}} \to \mathbb{R}$, $v(z) = u(a + \rho z)$.
    \begin{align*}
        v: \mathbb{D} & \to D(a, R) \xrightarrow{u} \mathbb{R} \\
        z             & \mapsto a + \rho z
    \end{align*}
    $v$ es continua en $\bar{\mathbb{D}}$, armónica en $\mathbb{D}$ y no negativa en $\bar{\mathbb{D}}$.

    Sea $z \in \partial D\left(0, \frac{r}{\rho}\right)$.
    $z$ es de la forma $z = \frac{r}{\rho}e^{i\theta}$, con $\theta \in \mathbb{R}$.
    Entonces:
    $$v(z) = \frac{1}{2\pi} \int_{-\pi}^\pi v(e^{it})P_{r/\rho}(\theta-t)dt$$
    Además,
    $$v(0) = \frac{1}{2\pi} \int_{-\pi}^\pi v(e^{it})dt$$
    Recordemos que
    $$\frac{1-r}{1+r} \leq P_r(t) \leq \frac{1+r}{1-r}, \quad 0 \leq r < 1, \; t \in \mathbb{R}$$
    Entonces, si $\theta \in \mathbb{R}$ y $z = \frac{r}{\rho}e^{i\theta}$, tenemos:
    \begin{align*}
         & \frac{1}{2\pi} \int_{-\pi}^\pi v(e^{it})\frac{1-r/\rho}{1+r/\rho}dt \leq \frac{1}{2\pi} \int_{-\pi}^\pi v(e^{it})P_{r/\rho}(\theta-t)dt \leq \frac{1}{2\pi} \int_{-\pi}^\pi v(e^{it})\frac{1+r/\rho}{1-r/\rho}dt \Leftrightarrow                    \\
         & \Leftrightarrow \frac{\rho-r}{\rho+r} \frac{1}{2\pi} \int_{-\pi}^\pi v(e^{it})dt \leq v(z) \leq \frac{\rho+r}{\rho-r} \frac{1}{2\pi} \int_{-\pi}^\pi v(e^{it})dt \Leftrightarrow \frac{\rho-r}{\rho+r}v(0) \leq v(z) \leq \frac{\rho+r}{\rho-r}v(0)
    \end{align*}
    En términos de $u$,
    $$\frac{\rho-r}{\rho+r}u(a) \leq u(a+\rho z) \leq \frac{\rho+r}{\rho-r}u(a), \quad \theta \in \mathbb{R}, \; \rho \in (r, R), \; z = \frac{r}{\rho}e^{i\theta}$$
    Como $u(a+\rho z) = u\left(a + \rho\frac{r}{\rho}e^{i\theta}\right) = u(a + re^{i\theta})$, entonces
    $$\frac{\rho-r}{\rho+r}u(a) \leq u(a+\rho z) \leq \frac{\rho+r}{\rho-r}u(a), \quad \theta \in \mathbb{R}, \; \rho \in (r, R)$$

    Haciendo $\rho \to R^-$, tenemos que:
    $$\frac{R-r}{R+r}u(a) \leq u(a+re^{i\theta}) \leq \frac{R+r}{R-r}u(a), \quad \forall \theta \in \mathbb{R}$$
    Entonces:
    $$\frac{R-r}{R+r}u(a) \leq u(z) \leq \frac{R+r}{R-r}u(a), \quad z \in \partial D(a, r)$$
    Como $u$ es continua en $\bar{D}(a, r)$ y armónica en $D(a, r)$, por el principio del máximo y el principio del mínimo,
    \begin{align*}
        \max_{z \in \bar{D}(a, r)} u(z) & = \max_{z \in \partial D(a, r)} u(z) \\
        \min_{z \in \bar{D}(a, r)} u(z) & = \min_{z \in \partial D(a, r)} u(z)
    \end{align*}
    Entonces tenemos la desigualdad para todo $z \in \bar{D}(a, r)$.
\end{proof}

\begin{proposition}
    Sea $D$ un dominio en $\mathbb{C}$ y sea $\{u_n\}_{n=1}^\infty$ una sucesión de funciones armónicas y no negativas en $D$.
    Si existe $z_0 \in D$ tal que $\lim\limits_{n \to \infty} u_n(z_0) = \infty$, entonces
    $$\lim_{n \to \infty} u_n(z) = \infty, \quad \forall z \in D$$
    siendo la convergencia uniforme en cada subconjunto compacto de $D$.
\end{proposition}

\begin{proof}
    Para cada $n \in \mathbb{N}$ tenemos $u_n: D \to \mathbb{R}$ armónica no negativa.
    Sea
    $$A = \{z \in D : \lim\limits_{n \to \infty} u_n(z) = \infty\}$$
    $A \neq \emptyset$ porque $z_0 \in A$.
    Veamos que $A$ es abierto y cerrado en $D$.

    \begin{enumerate}
        \item Probemos que $A$ es abierto.
              Queremos ver que si $a \in A$ y $R > 0$ tal que $D(a, R) \subset D$, entonces $D(a, R/2) \subset A$ y $u_n \xrightarrow[n \to \infty]{} \infty$ uniformemente en $D(a, R/2)$.

              Basta ver la convergencia uniforme.
              Sea $M > 0$, veamos que existe $n_0 \in \mathbb{N}$ tal que si $n \geq n_0$ y $z \in D(a, R/2)$, entonces $u_n(z) > M$.
              Como $a \in A$, tenemos que $\lim\limits_{n \to \infty} u_n(a) = \infty$.
              Por tanto, existe $n_0 \in \mathbb{N}$ tal que
              $$n \geq n_0 \Rightarrow u_n(a) > 3M$$
              Entonces, si $n \geq n_0$ y $z \in D(a, R/2)$, por las desigualdades de Harnack tenemos
              $$u_n(z) \geq \frac{1}{3}u_n(a) > \frac{1}{3}3M = M$$
              Por tanto, $A$ es abierto.

        \item Veamos que $A$ es cerrado en $D$, es decir, que $D \setminus A$ es abierto.

              Sea $a \in D \setminus A$ y sea $R > 0$ con $D(a, R) \subset D$.
              $D(a, R/2) \subset D \setminus A$, ya que si $z \in D(a, R/2)$, entonces $u_n(z) \leq 3u_n(a)$ para todo $n \in \mathbb{N}$ y $z \in D$.
              Si $z \in A$, $\lim\limits_{n \to \infty} u_n(z) = \infty$ y $\lim\limits_{n \to \infty} u_n(a) = \infty$, pero $a \notin A$.
              Así que $D \setminus A$ es abierto.
    \end{enumerate}

    $A$ es abierto y cerrado en $D$, que es conexo.
    Como $A \neq \emptyset$, tenemos que $A = D$.
    Entonces $\lim\limits_{n \to \infty} u_n(z) = \infty$ para todo $z \in D$.
    Sabemos que, dado $a \in D$ y $R > 0$ con $D(a, R) \subset D$, se tiene que $u_n \xrightarrow[n \to \infty]{} \infty$ uniformemente en $D(a, R/2)$.
    Entonces $u_n \xrightarrow[n \to \infty]{} \infty$ uniformemente en cada subconjunto compacto de $D$.
\end{proof}

\begin{theorem}[Teorema de Harnack]
    Sea $D$ un dominio en $\mathbb{C}$ y sea $\{u_n\}_{n=1}^\infty$ una sucesión creciente de funciones armónicas en $D$.
    Para cada $z \in D$, sea $u(z) = \lim\limits_{n \to \infty} u_n(z) \in \mathbb{R} \cup \{\infty\}$.
    Entonces se da una de las dos siguientes posibilidades:
    \begin{enumerate}
        \item $u \equiv \infty$.
              En este caso, $u_n \xrightarrow[n \to \infty]{} \infty$ uniformemente en cada subconjunto compacto de $D$.
        \item $u(z) \in \mathbb{R}$ para todo $z \in D$.
              En este caso, $u$ es armónica en $D$ y $u_n \xrightarrow[n \to \infty]{} u$ uniformemente en cada subconjunto compacto de $D$.
    \end{enumerate}
\end{theorem}

\begin{proof}
    Supongamos en primer lugar que $u_n$ es no negativa para cada $n \in \mathbb{N}$.
    Hay dos posibilidades:
    \begin{itemize}
        \item Existe $z_0 \in D$ tal que $u(z_0) = \infty$.
              Entonces $u(z) = \lim\limits_{n \to \infty} u_n(z) = \infty$ y $u_n \xrightarrow[n \to \infty]{} u$ uniformemente en cada subconjunto compacto de $D$.
              Se verifica (1).

        \item $u(z) \in \mathbb{R}$ para todo $z \in D$.
              Entonces $u: D \to \mathbb{R}$.
              Sea $a \in D$ y $R > 0$ con $D(a, R) \subset D$.
              Veamos que $\{u_n\}$ es uniformemente de Cauchy en $D(a, R/2)$.

              Sea $\varepsilon > 0$.
              Como $\{u_n(a)\}_{n=1}^\infty$ es una sucesión de Cauchy por ser convergente, tenemos que existe $n_0 \in \mathbb{N}$ tal que
              $$n \geq m > n_0 \Rightarrow 0 \leq u_n(a)-u_m(a) < \frac{\varepsilon}{3}$$
              ya que $u_n-u_m$ es armónica en $D(a, R)$ y no negativa.
              Por tanto, $\{u_n\}_{n=1}^\infty$ converge uniformemente en $D(a, R/2)$.

              Entonces, dado $a \in D$ y $R > 0$ con $D(a, R) \subset D$, hemos visto que $u_n \xrightarrow[n \to \infty]{} u$ uniformemente en $D(a, R/2)$.
              Por tanto, $u_n \xrightarrow[n \to \infty]{} u$ uniformemente en cada subconjunto compacto de $D$.
              Entonces $u$ es armónica en $D$, así que se verifica (2).
    \end{itemize}

    Consideramos ahora el caso general.
    Para cada $n \in \mathbb{N}$, sea $v_n = u_n - u_1$.
    Entonces $\{v_n\}_{n=1}^\infty$ es una sucesión creciente de funciones armónicas no negativas.
    Para cada $z \in D$, sea
    $$v(z) = \lim\limits_{n \to \infty} v_n(z) \in \mathbb{R} \cup \{\infty\}$$
    Si $z \in D$,
    $$v(z) = \lim\limits_{n \to \infty} (u_n(z)-u_1(z)) = \begin{cases}
            \infty        & \text{si } u(z) = \infty       \\
            u(z) - u_1(z) & \text{si } u(z) \in \mathbb{R}
        \end{cases}$$
    Por el caso anterior se da una de las dos siguientes posibilidades:
    \begin{itemize}
        \item $v \equiv \infty$.
              Veamos que $v_n \to \infty$ uniformemente en cada subconjunto compacto de $D$.

              Sea $K \subset D$, $K$ compacto.
              Sea $A = \min_{z \in K} u_1(z)$.
              Como $v_n \to \infty$ uniformemente en $D$, existe $n_0 \in \mathbb{N}$ tal que si $n \geq n_0$ y $z \in K$, entonces $v(z) > M-A$, con $M \in \mathbb{R}$.
              Entonces, si $n \geq n_0$ y $z \in K$, $u_n(z) = u_1(z) + v_n(z) > A + M - A = M$.
              Por tanto, se verifica (1).

        \item $v(z) \in \mathbb{R}$ para todo $z \in D$.
              Veamos que $u$ es armónica y $u_n \xrightarrow[n \to \infty]{} u$ uniformemente en cada subconjunto compacto de $D$.

              Sabemos que
              $$v(z) = u(z) - u_1(z) \Leftrightarrow u(z) = u_1(z) + v(z), \quad z \in D$$
              Entonces $u(z) \in \mathbb{R}$ para todo $z \in D$ y $u$ es armónica en $D$.
              Además, $u_n - u_1 \xrightarrow[n \to \infty]{} u-u_1$ uniformemente en cada subconjunto compacto de $D$.
              Por tanto, $u_n \to u$ uniformemente en cada subconjunto compacto de $D$, así que se verifica (2).
    \end{itemize}
\end{proof}