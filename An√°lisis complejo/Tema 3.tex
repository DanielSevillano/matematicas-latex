\chapter{Familias normales}

\section{Familias normales}
\begin{theorem}[Teorema de convergencia de Weierstrass]
    Sea $D$ abierto en $\mathbb{C}$ y sean $\{f_n\}_{n=1}^\infty$ una sucesión de funciones holomorfas en $D$ y $f: D \to \mathbb{C}$.
    Si $f_n \xrightarrow[n \to \infty]{} f$ uniformemente en cada subconjunto compacto de $D$, entonces $f$ es holomorfa en $D$ y $f_n' \xrightarrow[n \to \infty]{} f'$ uniformemente en cada subconjunto compacto.
    Para todo $k \in \mathbb{N}$, $f^{(k)}_n \xrightarrow[n \to \infty]{} f^{(k)}$ uniformemente en cada compacto.
\end{theorem}

\begin{definition}
    Sea $D$ un abierto en $\mathbb{C}$ y sea $\mathcal{F}$ una familia de funciones holomorfas en $D$.
    Diremos que $\mathcal{F}$ es finitamente normal si para cada sucesión $\{f_n\}_{n=1}^\infty$ en $\mathcal{F}$ existe una subsucesión $\{f_{n_k}\}_{k=1}^\infty$ de $\{f_n\}$ que converge uniformemente en cada subconjunto compacto de $D$.
\end{definition}

\begin{remark}
    El límite $f$ de tal subsucesión es una función holomorfa en $D$, pero no tiene por qué pertenecer a $\mathcal{F}$.
\end{remark}

\begin{definition}
    Sea $D$ un abierto en $\mathbb{C}$ y sea $\mathcal{F}$ una familia de funciones holomorfas en $D$.
    Diremos que $\mathcal{F}$ es compacta si para cada sucesión $\{f_n\}_{n=1}^\infty$ en $\mathcal{F}$ existe una subsucesión $\{f_{n_k}\}_{k=1}^\infty$ de $\{f_n\}$ que converge uniformemente en cada subconjunto compacto de $D$ a una función que pertenece a $\mathcal{F}$.
\end{definition}

En el conjunto $Hol(D)$ de las funciones holomorfas en $D$, con $D$ abierto en $\mathbb{C}$, se puede definir una distancia $d$ tal que $(Hol(D), d)$ es un espacio métrico completo, y en el que:
$$f_n \xrightarrow{d} f \Leftrightarrow f_n \to f \text{ uniformemente en cada subconjunto compacto de } D$$
Si $\mathcal{F} \subset Hol(D)$, $\mathcal{F}$ es finitamente normal si y solo si $\mathcal{F}$ es relativamente compacto.
Los compactos coinciden con la definición de familia compacta dada.

\section{El teorema de Montel}
\begin{lemma}
    Sea $D$ un abierto en $\mathbb{C}$ y $\mathcal{F}$ una familia de funciones holomorfas en $D$.
    Entonces son equivalentes:
    \begin{enumerate}
        \item $\mathcal{F}$ está uniformemente acotada en cada subconjunto compacto de $D$.
        \item Para cada $a \in D$ existe $r_a > 0$ con $D(a, r_a) \subset D$ y $f$ está uniformemente acotada en $D(a, r_a)$.
    \end{enumerate}
\end{lemma}

\begin{lemma}
    Sea $D$ abierto en $\mathbb{C}$ y sean $f_n: D \to \mathbb{C}$ para $n = 1, 2, \dots$ y $f: D \to \mathbb{C}$.
    Entonces son equivalentes:
    \begin{enumerate}
        \item $f_n \to f$ uniformemente en cada subconjunto compacto de $D$.
        \item Para cada $a \in D$ existe $r_a > 0$ con $D(a, r_a) \subset D$ tal que $f_n \to f$ uniformemente en $D(a, r_a)$.
    \end{enumerate}
\end{lemma}

\begin{lemma}
    Sean $C_1, C_2 \in \mathbb{C}$, con $C_1 \cap C_2 = \emptyset$ y $C_1, C_2 \neq \emptyset$.
    Si $C_1$ es compacto y $C_2$ es cerrado, entonces:
    $$dist(C_1, C_2) = \inf\{|z_1-z_2| : z_1 \in C_1, z_2 \in C_2\} > 0$$
\end{lemma}

\begin{remark}
    Si $C_1$ no es compacto no es cierto en general.
\end{remark}

\begin{lemma}
    Sea $A \subset \mathbb{C}$, $A \neq \emptyset$ y sea
    $$F: \mathbb{C} \to \mathbb{R}, \; F(z) = dist(z, A) = \inf \{|z-a| : a \in A\}$$
    Entonces $F$ es continua y $F(z) = 0$ para todo $z \in A$.
    Si además $A$ es cerrado, entonces $F(z) = \min \{|z-a| : a \in A\}$ para todo $z \in \mathbb{C}$.
\end{lemma}

\begin{lemma}
    Sea $A \subset \mathbb{C}$, $A \neq \emptyset$ y sea $\varepsilon > 0$.
    Consideramos los conjuntos:
    \begin{align*}
        B & = \{z \in \mathbb{C} : dist(z, A) < \varepsilon\}    \\
        C & = \{z \in \mathbb{C} : dist(z, A) \leq \varepsilon\}
    \end{align*}
    Entonces $B$ es abierto y $C$ es cerrado, con $A \subset B \subset C$.
    Si además $A$ es acotado, entonces $B$ es acotado y $C$ es compacto.
\end{lemma}

\begin{proposition}
    Sea $D$ un abierto en $\mathbb{C}$ y $\mathcal{F}$ una familia de funciones holomorfas en $D$.
    Supongamos que $\mathcal{F}$ está uniformemente acotada en $D$.
    Sea $K$ un subconjunto compacto de $D$.
    Entonces existe $A > 0$ tal que:
    $$|f(z_2)-f(z_1)| \leq A|z_2-z_1|, \quad \forall z_1, z_2 \in K, \; \forall f \in \mathcal{F}$$
\end{proposition}

\begin{proof}
    Sea $M > 0$ tal que $|f(z)| \leq M$ para todo $z \in D$ y para toda $f \in \mathcal{F}$.
    Sean $K \subset D$, $K$ compacto.
    Sea $d > 0$ con $d < dist(K, \mathbb{C} \setminus D)$.
    Si $D = \mathbb{C}$, tomamos $d > 0$ cualquiera.
    Sea $z_0 \in K$.
    Entonces $D(z_0, d) \subset D$.
    De hecho, podemos tomar $\varepsilon > 0$ tal que $D(z_0, d+\varepsilon) \subset D$.
    Dada $f \in \mathcal{F}$, por la fórmula de Cauchy,
    $$f'(z) = \frac{1}{2\pi i} \int_{|\xi-z_0|=d} \frac{f(\xi)}{(\xi-z)^2}d\xi \quad \text{si } z \in D\left(z_0, \frac{d}{2}\right)$$
    Entonces:
    $$|f'(z)| \leq \frac{1}{2\pi}2\pi \max_{|\xi-z_0|=d} \frac{|f(\xi)|}{|\xi-z|^2}$$
    Podemos acotar:
    $$|\xi-z| = |(\xi-z_0) + (z_0-z)| \geq |\xi-z_0| - |z_0-z| \geq d - \frac{d}{2} = \frac{d}{2}$$
    Así que $|\xi-z|^2 \geq \frac{d^2}{4} > 0$.
    Luego:
    $$|f'(z)| \leq d\frac{M}{d^2/4} = \frac{4M}{d}$$
    Hemos probado que si $z_0 \in K$, $f \in \mathcal{F}$ y $z \in D\left(z_0, \frac{d}{2}\right) \subset D$, entonces $|f'(z)| \leq \frac{4M}{d}$.

    Ahora, sean $z_1, z_2 \in K$ y $f \in \mathcal{F}$.
    Supongamos que $|z_1-z_2| < \frac{d}{2}$.
    Si $\xi \in [z_1, z_2]$, entonces $z_2 \in D\left(z_1, \frac{d}{2}\right) \subset D$ y $|\xi-z_1| \leq |z_1-z_2| < \frac{d}{2}$, $\xi \in D\left(z_1, \frac{d}{2}\right)$.
    Entonces $\xi \in D$ y $|f'(\xi)| \leq \frac{4M}{d}$.
    Por tanto:
    $$|f(z_2)-f(z_1)| = \left|\int_{[z_1, z_2]} f'(\xi)d\xi\right| \leq |z_2-z_1| \max_{\xi \in [z_1, z_2]} |f'(\xi)| \leq |z_2-z_1| \frac{4M}{d}$$
    Entonces, si $z_1, z_2 \in K$, $|z_1-z_2| < \frac{d}{2}$ y $f \in \mathcal{F}$, se tiene que:
    $$|f(z_2)-f(z_1)| \leq A|z_2-z_1|$$

    Ahora, si $z_1, z_2 \in K$, $|z_2-z_1| \geq \frac{d}{2}$ y $f \in \mathcal{F}$, tenemos:
    $$|f(z_2)-f(z_1)| \leq |f(z_2)| + |f(z_1)| \leq 2M = 2M \frac{d}{2}\frac{2}{d} \leq \frac{4M}{d}|z_2-z_1| = A|z_2-z_1|$$
\end{proof}

\begin{theorem}[Teorema de Arzelá-Ascoli]
    Sean $(X_1, d_1)$ y $(X_2, d_2)$ dos espacios métricos, siendo $(X_1, d_1)$ separable y $(X_2, d_2)$ completo.
    Sea $\mathcal{F}$ una familia de aplicaciones continuas de $X_1$ en $X_2$ que verifica:
    \begin{enumerate}
        \item $\mathcal{F}$ es puntualmente equicontinua.
              Es decir, dado $x \in X_1$ se verifica que, para todo $\varepsilon > 0$, existe $\delta > 0$ tal que si $y \in X_1$ con $d_1(x, y) < \delta$, entonces $d_2(f(x), f(y)) < \varepsilon$ para toda $f \in \mathcal{F}$.
        \item Para todo $x \in X_1$, el conjunto $\{f(x) : f \in \mathcal{F}\}$ es relativamente compacto.
    \end{enumerate}
    Entonces, si $\{f_n\}_{n=1}^\infty$ es una sucesión en $\mathcal{F}$, existe una subsucesión $\{f_{n_k}\}_{k=1}^\infty$ de $\{f_n\}$ que converge uniformemente en cada subconjunto compacto de $X_1$.
\end{theorem}

\begin{theorem}[Teorema de Montel]
    Sea $D$ un abierto en $\mathbb{C}$ y sea $\mathcal{F}$ una familia de funciones holomorfas en $D$.
    Entonces son equivalentes:
    \begin{enumerate}
        \item $\mathcal{F}$ es finitamente normal.
        \item $\mathcal{F}$ está uniformemente acotada en cada subconjunto compacto de $D$.
              Es decir, para cada $K \subset D$, $K$ compacto, existe $M_K > 0$ tal que $|f(z)| \leq M_K$ para toda $f \in \mathcal{F}$ y para todo $z \in K$.
    \end{enumerate}
\end{theorem}

\begin{proof}
    \hfill
    \begin{itemize}
        \item[$\Rightarrow$] Sea $D$ abierto en $\mathbb{C}$ y sea $\mathcal{F}$ una familia de funciones holomorfas en $D$, con $\mathcal{F}$ finitamente normal.
            Supongamos por reducción al absurdo que existe $K \subset D$, $K$ compacto, tal que $\mathcal{F}$ no está uniformemente acotada en $K$.
            Entonces existen $\{z_n\}_{n=1}^\infty$ en $K$ y $\{f_n\}_{n=1}^\infty$ en $\mathcal{F}$ tales que $|f_n(z_n)| \to \infty$.

            Como $\mathcal{F}$ es una familia finitamente normal, existe $\{f_{n_k}\}_{k=1}^\infty$ subsucesión de $\{f_n\}$ tal que $\{f_{n_k}\}$ converge uniformemente en cada subconjunto compacto de $D$ a una función $f$ holomorfa en $D$.
            Como $f$ es continua en $K$ y $K$ es compacto, existe $M > 0$ tal que $|f(z)| \leq M$ para todo $z \in K$.
            Por otro lado, como $f_{n_k} \xrightarrow[k \to \infty]{} f$ uniformemente en $K$, existe $k_0 \in \mathbb{N}$ tal que $k \geq k_0$, $z \in K$ $\Rightarrow$ $|f_{n_k}(z)-f(z)| < 1$.
            Entonces $|f_{n_k}(z)| \leq |f_{n_k}(z)-f(z)| + |f(z)| < 1 + M$, $z \in K$, $k \geq k_0$.
            En particular, $|f_{n_k}(z_{n_k})| < 1 + M$ si $k \geq k_0$.
            Esta es una contradicción.

        \item[$\Leftarrow$] Sea $D$ abierto en $\mathbb{C}$ y sea $\mathcal{F}$ una familia de funciones holomorfas en $D$, uniformemente acotada en cada subconjunto compacto de $D$.
            Tomamos $X_1 = D$ y $X_2 = \mathbb{C}$.

            \begin{enumerate}
                \item Sea $z_0 \in \mathbb{C}$.
                      Dado $\varepsilon > 0$, veamos que existe $\delta > 0$ tal que, si $z_1 \in D$, $|z_1-z_0| < \delta$, $f \in \mathcal{F}$, entonces $|f(z_1) - f(z_0)| < \varepsilon$.
                      Sea $R > 0$ con $\overline{D}(0, R) \subset D$.
                      $\mathcal{F}$ está uniformemente acotada en $\overline{D}(z_0, R)$ y por tanto en $D(z_0, R)$.
                      Sea $K = \overline{D}(z_0, \frac{R}{2})$, que es un subconjunto compacto de $D(z_0, R)$.
                      Por la proposición anterior, existe $A > 0$ tal que
                      $$|f(z_2) - f(z_1)| \leq A|z_2-z_1|, \quad \text{si } z_1, z_2 \in K, f \in \mathcal{F}$$
                      Entonces, si $\delta = \min \left(\frac{\varepsilon}{A}, \frac{R}{2}\right)$, $z_1 \in D$, $|z_1-z_0| < \delta$ y $f \in \mathcal{F}$, entonces $z_1 \in \overline{D}(z_0, \frac{R}{2}) = K$, así que:
                      $$|f(z_1) - f(z_0)| \leq A|z_1-z_0| < A\delta \leq A\frac{\varepsilon}{A} = \varepsilon$$

                \item Sea $z \in D$.
                      El conjunto $\{f(z) : f \in \mathcal{F}\}$ está acotado, ya que $\mathcal{F}$ está uniformemente acotada en $\{z\}$.
                      Por tanto, su clausura es compacta.
            \end{enumerate}

            Entonces, por el teorema de Arzelá-Ascoli, existe una subsucesión $\{f_{n_k}\}_{k=1}^\infty$ de $\{f_n\}$ que converge uniformemente en cada subconjunto compacto de $D$.
            Por tanto, $\mathcal{F}$ es finitamente normal.
    \end{itemize}
\end{proof}

\begin{remark}
    \hfill
    \begin{enumerate}
        \item Sea $D$ un abierto en $\mathbb{C}$.
              Si $\mathcal{F}$ es una familia finitamente normal de funciones holomorfas en $D$, entonces la familia $\mathcal{F}' = \{f' : f \in \mathcal{F}\}$ es finitamente normal.
              En general, si $k \in \mathbb{N}$, la familia $\mathcal{F}^{(k)} = \{f^{(k)} : f \in \mathcal{F}\}$ es finitamente normal.

              Sea $\{g_n\}_{n=1}^\infty$ en $\mathcal{F}'$.
              Entonces $g_n = f_n'$, $f_n \in \mathcal{F}$.
              Existe $\{f_{n_k}\}_{k=1}^\infty$ subsucesión de $\{f_n\}$ que converge uniformemente en cada subconjunto compacto de $D$ a una función $f$ holomorfa en $D$.
              Entonces $g_{n_k} = f_{n_k}' \to f'$ uniformemente en cada subconjunto compacto de $D$.

        \item Sea $D$ abierto en $\mathcal{C}$ y sea $\mathcal{G}$ familia finitamente normal de funciones holomorfas en $D$ con $\mathcal{F} \subset \mathcal{G}$.
              Entonces $\mathcal{F}$ es finitamente normal.

        \item Si $a \in \mathbb{C}$, $R > 0$ y $K \subset D(a, R)$, $K$ compacto, entonces existe $r \in (0, R)$ tal que $K \subset \overline{D}(a, r)$.
    \end{enumerate}
\end{remark}

\begin{example}
    \hfill
    \begin{enumerate}
        \item $\mathcal{F} = \{f : f \text{ es entera y } |f(z)| \leq n \text{ si } |z| = n, n = 1, 2, \dots\}$.
              Sea $K \subset \mathbb{C}$, $K$ compacto, y sea $f \in \mathcal{F}$.
              Existe $n_0 \in \mathbb{N}$ tal que $K \subset \overline{D}(0, n_0)$.
              Además, $|f(z_0)| \leq n_0$ si $|z| = n_0$.
              Por el principio del máximo, $|f(z)| \leq n_0$ si $|z| \leq n_0$.
              En particular, $|f(z)| \leq n_0$ si $z \in K$ y $f \in \mathcal{F}$.
              $\mathcal{F}$ está uniformemente acotada en $K$.
              Por el teorema de Montel, $\mathcal{F}$ es finitamente normal.

        \item $\mathcal{P} = \{f : f \text{ es holomorfa en } \mathbb{D}, f(0) = 1, \Re(f(z)) > 0 \; \forall z \in \mathbb{D}\}$.
              Sea $K \subset \mathbb{D}$, $K$ compacto.
              Si $f \in \mathcal{P}$ y $z \in K$,
              $$|f(z)| \leq \frac{1+|z|}{1-|z|}$$
              Existe $R \in (0, 1)$ tal que $K \subset \overline{D}(0, R)$.
              Entonces, si $f \in \mathcal{P}$ y $z \in K$,
              $$|f(z)| \leq \frac{1+|z|}{1-|z|} \leq \frac{1+R}{1-R}$$
              $\mathcal{P}$ está uniformemente acotada en $K$ para todo subconjunto compacto $K$ de $\mathbb{D}$.
              Por el teorema de Montel, $\mathcal{P}$ es finitamente normal.

              \begin{remark}
                  Si quitamos la condición $f(0) = 1$ en $\mathcal{P}$, la familia deja de ser finitamente normal.
                  Por ejemplo, $f_n(z) = n$, $n = 1, 2, \dots$, $\{f_n : n = 1, 2, \dots\} \subset \mathcal{P}$.
                  Si tomamos $K = \{0\}$, $\mathcal{P}$ no está uniformemente acotada en $K$, así que $\mathcal{P}$ no es finitamente normal.
              \end{remark}

              Recordemos que $\mathcal{P} = \{f : f \text{ es holomorfa en } \mathbb{D}, f \prec P\}$, con $P(z) = \frac{1+z}{1-z}$.
              Esto es un caso particular del siguiente ejemplo.

        \item Sea $F$ holomorfa en $\mathbb{D}$ y sea
              $$\mathcal{F}_F = \{f : f \text{ holomorfa en } \mathbb{D}, f \prec F\}$$
              Entonces $\mathcal{F}_F$ es finitamente normal.

        \item Sean $a \in \mathbb{C}$ y $R > 0$.
              Sea $\mathcal{F}$ una familia finitamente normal de funciones holomorfas en $D(a, R)$.
              Para cada $f \in \mathcal{F}$, consideramos el desarrollo de Taylor de $f$ centrado en $a$
              $$f(z) = \sum_{n=0}^\infty a_n(f)(z-a)^n, \quad z \in D(a, R)$$
              Entonces $M_n = \sup_{f \in \mathcal{F}} |a_n(f)| < \infty$ para cada $n$ y la serie de potencias $\sum_{n=0}^\infty M_n(z-a)^n$ tiene radio de convergencia mayor o igual que $R$, y por tanto define una función holomorfa en $D(a, R)$.

              \begin{proof}
                  Fijado $n$, si $f \in \mathcal{F}$ tenemos:
                  $$a_n(f) = \frac{f^{(n)}(a)}{n!} \Rightarrow |a_n(f)| = \frac{|f^{(n)}(a)|}{n!}$$
                  La familia $\mathcal{F}^{(n)}$ es finitamente normal y por tanto está uniformemente acotada en el conjunto $\{a\}$, por lo que $\{f^{(n)}(a) : f \in \mathcal{F}\}$ está acotado.
                  Es decir, $\sup_{f \in \mathcal{F}} |f^{(n)}(a)| < \infty$.
                  Entonces $M_n = \sup_{f \in \mathcal{F}} |a_n(f)| = \sup_{f \in \mathcal{F}} \frac{|f^{(n)}(a)|}{n!} < \infty$.
                  Consideramos la serie de potencias:
                  $$\sum_{n=0}^\infty M_n(z-a)^n$$
                  Si $r \in (0, R)$, tenemos que $\mathcal{F}$ está uniformemente acotada en $\overline{D}(a, r)$, y por tanto existe $M(r) > 0$ tal que $|f(z)| \leq M(r)$ si $z \in \overline{D}(a, r)$ y $f \in \mathcal{F}$.
                  Si $f \in \mathcal{F}$, por la fórmula de Cauchy,
                  $$a_n(f) = \frac{f^{(n)}(a)}{n!} = \frac{1}{2\pi i} \int_{|z-a|=r} \frac{f(z)}{(z-a)^{n+1}}dz, \quad \text{si } r \in (0, R), n = 0, 1, 2, \dots$$
                  Así que:
                  $$|a_n(f)| \leq \frac{1}{2\pi} 2\pi r \max_{|z-a|=r} \frac{|f(z)|}{|z-a|^{n+1}} \leq r\frac{M(r)}{r^{n+1}} = \frac{M(r)}{r^n}, \quad \text{si } r \in (0, R), n = 0, 1, 2, \dots, f \in \mathcal{F}$$
                  Tomando supremo en $f \in \mathcal{F}$ tenemos que:
                  $$|M_n| = M_n \leq \frac{M(r)}{r^n} \Rightarrow \sqrt[n]{M_n} \leq \frac{\sqrt[n]{M(r)}}{r}, \quad \text{si } r \in (0, R), n = 0, 1, 2, \dots$$
                  Por tanto:
                  $$\limsup_{n \to \infty} \sqrt[n]{M_n} \leq \lim_{n \to \infty} \frac{\sqrt[n]{M(r)}}{r} = \frac{1}{r}, \quad \text{si } r \in (0, R)$$
                  Haciendo $r \to R$,
                  $$\limsup_{n \to \infty} \sqrt[n]{M_n} \leq \frac{1}{R}$$
                  Entonces el radio de convergencia es mayor o igual que $R$.
              \end{proof}

        \item $\mathcal{F} = \{f : f \text{ es holomorfa en } \mathbb{D} \text{ y } \iint_\mathbb{D} |f(z)|dxdy \leq M\}$, siendo $M > 0$.
              Veamos que $\mathcal{F}$ es finitamente normal.

              Sea $K \subset \mathbb{D}$, $K$ compacto.
              Tomamos $r \in (0, 1)$ con $K \subset D(0, r)$.
              Sea $f \in \mathcal{F}$ y $z \in K$, por la fórmula de Cauchy
              $$f(z) = \frac{1}{2\pi i} \int_{|z| = \rho} \frac{f(\xi)}{\xi-z}d\xi = \frac{1}{2\pi i} \int_{-\pi}^\pi \frac{f(\rho e^{i\theta})}{\rho e^{i\theta} - z}\rho d\rho, \quad r \leq \rho < 1$$

              $$|f(z)| \leq \frac{1}{2\pi} \int_{-\pi}^\pi \frac{|f(\rho e^{i\theta})|}{|\rho e^{i\theta} - z|}\rho d\rho, \quad r \leq \rho < 1$$

              $$\int_{\frac{1+r}{2}}^1 |f(z)|d\rho \leq \frac{1}{2\pi} \int_{\frac{1+r}{2}}^1 \int_{-\pi}^\pi \frac{|f(\rho e^{i\theta})|}{|\rho e^{i\theta} - z|}\rho d\theta d\rho = \frac{1}{2\pi} \iint_{\frac{1+r}{2} < |w| < 1} \frac{|f(w)|}{|w-z|}dxdy$$
              Como $|w-z| \geq |w| - |z| > \frac{1+r}{2}-r = \frac{1-r}{2} > 0$,
              \begin{align*}
                   & \frac{1}{2\pi} \iint_{\frac{1+r}{2} < |w| < 1} \frac{|f(w)|}{|w-z|}dxdy \leq \frac{1}{2\pi} \frac{2}{1-r} \iint_{\frac{1+r}{2} < |w| < 1} |f(w)| dxdy \leq \\
                   & \leq \frac{1}{\pi(1-r)} \iint_\mathbb{D} |f(w)|dxdy \leq \frac{M}{\pi(1-r)}
              \end{align*}
              Por otro lado,
              $$\int_{\frac{1+r}{2}}^1 |f(z)|d\rho = |f(z)|\left(1 - \frac{1+r}{2}\right) = |f(z)|\frac{1-r}{2}$$
              Entonces:
              $$|f(z)|\frac{1-r}{2} \leq \frac{M}{\pi(1-r)} \Rightarrow |f(z)| \leq \frac{2M}{\pi(1-r)^2}$$
              Por tanto, $\mathcal{F}$ está uniformemente acotada en $K$.
    \end{enumerate}
\end{example}

\begin{theorem}
    Sean $a \in \mathbb{C}$ y $R > 0$.
    Sea $\mathcal{F}$ una familia de funciones holomorfas en $D(a, R)$.
    Las siguientes condiciones son equivalentes:
    \begin{enumerate}
        \item $\mathcal{F}$ es finitamente normal.
        \item Existe una sucesión $\{M_n\}_{n=0}^\infty$ con $M_n \geq 0$ para todo $n$ tal que la serie de potencias $\sum_{n=0}^\infty M_n(z-a)^n$ tiene radio de convergencia mayor o igual que $R$ y tal que, si para cada $f \in \mathcal{F}$,
              $$f(z) = \sum_{n=0}^\infty a_n(f)(z-a)^n, \quad z \in D(a, r)$$
              se tiene que $|a_n(f)| \leq M_n$ para todo $n$ y para toda $f \in \mathcal{F}$.
    \end{enumerate}
\end{theorem}

\begin{proof}
    \hfill
    \begin{itemize}
        \item[$\Rightarrow$] $M_n = \sup_{f \in \mathcal{F}} |a_n(f)|$.
        \item[$\Leftarrow$] Sea $K \subset D(a, R)$, $K$ compacto.
            Existe $r \in (0, R)$ tal que $K \subset \overline{D}(a, r)$.
            Si $z \in K$ y $f \in \mathcal{F}$, se tiene:
            $$|f(z)| = \left|\sum_{n=0}^\infty a_n(f)(z-a)^n\right| \leq \sum_{n=0}^\infty |a_n(f)||z-a|^n \leq \sum_{n=0}^\infty M_n|z-a|^n \leq \sum_{n=0}^\infty M_nr^n < \infty$$
            ya que $\sum_{n=0}^\infty M_n(z-a)^n$ converge para $z = a+r$.
            $\mathcal{F}$ está uniformemente acotada en $K$.
    \end{itemize}
\end{proof}

\section{El teorema de Stieltjes-Vitali}
\begin{theorem}[Teorema de Stieltjes-Vitali]
    Sea $D$ un dominio en $\mathbb{C}$ y $\mathcal{F}$ una familia finitamente normal de funciones holomorfas en $D$.
    Sea $\{f_n\}_{n=1}^\infty$ una sucesión en $\mathcal{F}$.
    Si existe $A \subset D$ tal que $A$ tiene algún punto de acumulación en $D$, para el que existe $\lim\limits_{n \to \infty} f_n(a) \in \mathbb{C}$ para todo $a \in A$, entonces $\{f_n\}$ converge uniformemente en cada subconjunto compacto de $D$.
\end{theorem}

\begin{proof}
    \hfill
    \begin{enumerate}
        \item Veamos que $\{f_n\}$ converge puntualmente en $D$.
              Sea $z^\ast \in D$.
              Supongamos por reducción al absurdo que $\{f_n(z^\ast)\}$ no converge.
              Como $\mathcal{F}$ está uniformemente acotada en el conjunto $\{z^\ast\}$, tenemos que $\{f_n(z^\ast)\}$ está acotado.
              Por tanto, existen $\{f_{n_i}\}_{i=1}^\infty$ y $\{f_{m_i}\}_{i=1}^\infty$ subsucesiones de $\{f_n\}$, y $w_1, w_2 \in \mathbb{C}$ distintos, tales que $f_{n_i}(z^\ast) \xrightarrow[i \to \infty]{} w_1$, $f_{m_i}(z^\ast) \xrightarrow[i \to \infty]{} w_2$.
              Como $\mathcal{F}$ es finitamente normal, existen $\{g_k\}_{k=1}^\infty$ y $\{h_k\}_{k=1}^\infty$ subsucesiones de $\{f_{n_i}\}$ y $\{f_{m_i}\}$, respectivamente, que convergen uniformemente en cada subconjunto compacto de $D$.
              Sean $g$ y $h$ los respectivos límites.
              Entonces $g$ y $h$ son holomorfas en $D$.
              Tenemos que:
              \begin{align*}
                  g_k(z^\ast) & \xrightarrow[k \to \infty]{} w_1, & g(z^\ast) & = w_1 \\
                  h_k(z^\ast) & \xrightarrow[k \to \infty]{} w_2, & h(z^\ast) & = w_2
              \end{align*}
              Si $a \in A$, existe $\lim\limits_{n \to \infty} f_n(a) \in \mathbb{C}$, así que $g(a) = h(a)$.
              $g$ y $h$ son holomorfas en $D$, $g = h$ en $A$ y $A$ tiene algún punto de acumulación en $D$.
              Por el teorema de identidad, $g = h$ en $D$.
              Pero $g(z^\ast) = w_1 \neq w_2 = h(z^\ast)$.
              Esto contradice nuestro supuesto.

        \item Sea $K \subset D$, $K$ compacto.
              Sea $\alpha > 0$ tal que $2\alpha < dist(K, \mathbb{C} \setminus D)$.
              Si $D = \mathbb{C}$, tomamos $\alpha > 0$ cualquiera.
              Sean $G = \{z \in \mathbb{C} : dist(z, K) < \alpha\}$ y $K_1 = \{z \in \mathbb{C} : dist(z, K) \leq \alpha\}$.
              $G$ es abierto, $K_1$ es compacto y $K \subset G \subset K_1 \subset D$.

              Veamos que $K_1 \subset D$.
              Si $z \in K_1$, $dist(z, K) \leq \alpha$.
              Supongamos que $z \in D$.
              Tomamos $w \in K$ con $|z-w| < 2\alpha$.
              Entonces $2\alpha < dist(K, \mathbb{C} \setminus D) \leq |w-z| < 2\alpha$.
              Esto contradice nuestra hipótesis.

              $\mathcal{F}$ está uniformemente acotada en $K_1$ y por tanto en $G$.
              $K \subset G$, $K$ compacto.
              Por una proposición previa, existe $A > 0$ tal que $|f(z_2)-f(z_1)| \leq A|z_2-z_1|$ si $z_1, z_2 \in K$ y $f \in \mathcal{F}$.
              Vamos a ver que $\{f_n\}$ es uniformemente de Cauchy en $K$.
              Sea $\varepsilon > 0$ y sea $\delta = \frac{\varepsilon}{3A} > 0$.
              Tenemos que si $z_1, z_2 \in K$, $|z_1-z_2| < \delta$ y $n \in \mathbb{N}$, entonces $|f_n(z_1)-f_n(z_2)| \leq A|z_1-z_2| < A\delta = \frac{\varepsilon}{3}$.
              Consideramos la familia $\{D(z, \delta) : z \in K\}$.
              Como $K$ es compacto, existen $z_1, z_2, \dots, z_N \in K$ tales que $K \subset \bigcup_{j=1}^N D(z_j, \delta)$.
              Para cada $j \in \{1, \dots, N\}$, la sucesión $\{f_n(z_j)\}_{n=1}^\infty$ es de Cauchy, ya que $\{f_n\}$ converge puntualmente en $D$.
              Por tanto, exsite $n_j \in \mathbb{N}$ tal que $n, m \geq n_j \Rightarrow |f_n(z_j)-f_m(z_j)| < \frac{\varepsilon}{3}$.
              Sea $n_0 = \max \{n_j : j = 1, \dots, N\}$.
              Si $n, n \geq n_j$ y $z \in K$, hay que probar que $|f_n(z)-f_m(z)| < \varepsilon$.
              Tomamos $j \in \{1, \dots, N\}$ con $z \in D(z_j, \delta)$.
              $$|f_n(z)-f_m(z)| \leq |f_n(z)-f_n(z_j)| + |f_n(z_j)-f_m(z_j)| + |f_m(z_j)-f_m(z)| < \frac{\varepsilon}{3} + \frac{\varepsilon}{3} + \frac{\varepsilon}{3} = \varepsilon$$
    \end{enumerate}
\end{proof}

\begin{example}
    Para $x \geq 0$, tenemos que $\left(1 + \frac{x}{n}\right)^n$ es una sucesión creciente y
    $$\lim\limits_{n \to \infty} \left(1 + \frac{x}{n}\right)^n = e^x$$
    Veamos que $\lim\limits_{n \to \infty} \left(1 + \frac{z}{n}\right)^n = e^z$ para todo $z \in \mathbb{C}$, siendo la convergencia uniforme en cada subconjunto compacto de $\mathbb{C}$.

    Sea $D = \mathbb{C}$, $f_n(z) = \left(1 + \frac{z}{n}\right)^n$, $n \in \mathbb{N}$.
    Cada $f_n$ es una función entera.
    Veamos que $\mathcal{F}$ es finitamente normal.

    Sea $K \subset \mathbb{C}$, $K$ compacto.
    Tomamos $R > 0$ con $K \subset \overline{D}(0, R)$.
    Si $k \in K$ y $n \in \mathbb{N}$,
    $$|f_n(z)| = \left|1 + \frac{z}{n}\right|^n \leq \left(1 + \frac{|z|}{n}\right)^n \leq \left(1 + \frac{R}{n}\right)^n \leq e^R$$
    Sea $A = [0, 1]$.
    $A$ tiene puntos de acumulación en $\mathbb{C}$ y $\lim\limits_{n \to \infty} f_n(x) = e^x$ para todo $x \in A$.
    Por el teorema de Stieltjes-Vitali, $\{f_n\}$ converge uniformemente en cada subconjunto compacto de $\mathbb{C}$.
    Sea $f$ el límite, entonces $f$ es entera.
    Si $x \in A$, $\lim\limits_{n \to \infty} f_n(x) = f(x) = e^x$.
    Por el teorema de identidad, $f(z) = e^z$ si $z \in \mathbb{C}$.
\end{example}

\begin{theorem}[Teorema de Lindelöf]
    Sea $f$ holomorfa y acotada en $\mathbb{D}$.
    Sea $\xi \in \partial \mathbb{D}$ y supongamos que existe el límite radial, es decir, $\lim\limits_{r \to 1^-} f(r\xi) = L \in \mathbb{C}$.
    Entonces para todo $\alpha \in (0, \frac{\pi}{2})$ existe el límite tangencial de $f$ en $\xi$, es decir,
    $$\lim\limits_{z \to \xi, z \in S_\alpha(\xi)} f(z) = L$$
    siendo $S_\alpha(\xi)$ el vector de vértice $\xi$ y ángulo $2\alpha$, simétrico con respecto al segmento $[0, \xi]$.
\end{theorem}

\begin{theorem}
    Sea $f$ holomorfa y acotada en $D(1, 1)$.
    Supongamos que existe $\lim\limits_{x \to 0^+} f(x) = L \in \mathbb{C}$.
    Entonces para todo $\alpha \in (0, \frac{\pi}{2})$ existe
    $$\lim\limits_{z \to 0, |\Arg(z)| < \alpha} f(z) = L$$
\end{theorem}

\begin{proof}
    Sea $M > 0$ tal que $|f(z)| \leq M$ si $z \in D(1, 1)$.
    Consideramos la sucesión $\{f_n\}_{n=1}^\infty$, $f_n(z) = f\left(\frac{z}{n}\right)$.
    Cada $f_n$ es holomorfa en $D(1, 1)$.
    La familia $\mathcal{F} = \{f_n : n \in \mathbb{N}\}$ está uniformemente acotada en $D$, porque si $z \in D$ y $n \in \mathbb{N}$ se tiene que $|f_n(z)| = \left|f\left(\frac{z}{n}\right)\right| \leq M$.
    Así que $\mathcal{F}$ es finitamente normal.

    Sea $A = (0, 1)$.
    Si $x \in A$, $\lim\limits_{n \to \infty} f_n(x) = \lim\limits_{n \to \infty} f\left(\frac{x}{n}\right) = L$.
    Por el teorema de Stieltjes-Vitali, $\{f_n\}$ converge uniformemente en cada subconjunto compacto de $D(1, 1)$.
    Sea $g$ el límite, entonces $g$ es holomorfa en $D(1, 1)$ y $g(x) = L$ para todo $x \in A$.
    Por el teorema de identidad, $g(z) = L$ para todo $z \in D(1, 1)$.
    Hemos probado que $f_n \xrightarrow[n \to \infty]{} L$ uniformemente en cada subconjunto compacto de $D$.

    Sea $\alpha \in (0, \frac{\pi}{2})$.
    Sea $K = \left\{z \in \mathbb{C} : \frac{\cos(\alpha)}{2} \leq |z| \leq \cos(\alpha), |\Arg(z)| \leq \alpha\right\}$.
    $K$ es un subconjunto compacto de $D(1, 1)$, así que $f_n \xrightarrow[n \to \infty]{} L$ uniformemente en $K$.
    Es decir, existe $n_0 \in \mathbb{N}$ tal que, si $n \geq n_0$ y $z \in K$, entonces $|f_n(z)-L| < \varepsilon$.

    Sea $\delta = \frac{\cos(\alpha)}{2n_0} > 0$.
    Sea $z$ tal que $0 < |z| < \delta$ y $|\Arg(z)| < \alpha$.
    Observamos que $|z| < \frac{\cos(\alpha)}{2n_0} \leq \frac{\cos(\alpha)}{2}$.
    Tomamos $n_z$ el primer natural para el que $n_z|z| \geq \frac{\cos(\alpha)}{2}$.
    Como $|z| < \frac{\cos(\alpha)}{2n_0} \Leftrightarrow n_0|z| < \frac{\cos(\alpha)}{2}$, entonces $1 \leq n_0 < n_z$.
    Por otro lado,
    $$(n_z-1)|z| < \frac{\cos(\alpha)}{2} \Leftrightarrow n_z|z| - |z| < \frac{\cos(\alpha)}{2} \Leftrightarrow n_z|z| < |z| + \frac{\cos(\alpha)}{2} < \frac{\cos(\alpha)}{2} + \frac{\cos(\alpha)}{2} = \cos(\alpha)$$
    Así que $\frac{\cos(\alpha)}{2} \leq n_z|z| = |n_zz| < \cos(\alpha)$.
    Además, $|\Arg(n_zz)| = |\Arg(z)| < \alpha$.
    Por tanto, $n_zz \in K$.
    Entonces:
    $$|f_{n_k}(n_zz) - L| = |f(z) - L| < \varepsilon$$
\end{proof}

\section{Teoremas de Hurwitz}
\begin{theorem}[Teorema de Rouché]
    Sea $D$ un dominio simplemente conexo en $\mathbb{C}$ y sea $J$ un camino de Jordan en $D$.
    Sean $f$ y $g$ funciones holomorfas en $D$ tales que
    $$|f(z) - g(z)| < |f(z)| \quad \text{si } z \in J$$
    Entonces:
    \begin{enumerate}
        \item $I(J) \subset D$.
        \item Ni $f$ ni $g$ se anulan en $J$.
        \item $f$ y $g$ tienen el mismo número de ceros en $I(J)$.
    \end{enumerate}
\end{theorem}

\begin{theorem}[Primer teorema de Hurwitz]
    Sea $D$ un dominio en $\mathbb{C}$ y sea $\{f_n\}_{n=1}^\infty$ una sucesión de funciones holomorfas y nunca nulas en $D$, que converge uniformemente en cada subconjunto de $D$ a una función $f$.
    Entonces $f$ es nunca nula en $D$ o bien $f$ es idénticamente nula en $D$.
\end{theorem}

\begin{proof}
    Si $f \equiv 0$ en $D$, no hay nada que hacer.
    Supongamos que $f \not\equiv 0$ en $D$.
    Supongamos por reducción al absurdo que existe $a \in D$ con $f(a) = 0$.
    Entonces $a$ es un cero aislado de $f$.
    Podemos tomar $R > 0$ tal que $D(a, 2R) \subset D$ y $f$ no tiene ceros en $D(a, 2R) \setminus \{a\}$.

    Sea $C_R$ la circunferencia $|z-a| = R$.
    Como $f$ no tiene ceros en $C_R$, existe $\alpha > 0$ tal que $|f(z)| > \alpha$ para todo $z \in C_R$.
    Como $f_n \to f$ uniformemente en $C_R$, existe $n_0 \in \mathbb{N}$ tal que $n \geq n_0, z \in C_R \Rightarrow |f_n(z) - f(z)| < \alpha$.
    Entonces, si $n \geq n_0$ y $z \in C_R$, se tiene que
    $$|f_n(z) - f(z)| < \alpha < |f(z)|$$
    Por el teorema de Rouché, $f_n$ y $f$ tienen el mismo número de ceros en $D(a, R)$.
    Pero $f_n$ es nunca nula en $D$, por lo que no tiene ceros en $D(a, R)$, mientras que $f(a) = 0$.
    Esta es una contradicción.
    Entonces $f$ es nunca nula en $D$.
\end{proof}

\begin{theorem}[Segundo teorema de Hurwitz]
    Sea $D$ un dominio en $\mathbb{C}$ y sea $\{f_n\}_{n=1}^\infty$ una sucesión de funciones holomorfas e inyectivas en $D$.
    Si $\{f_n\}$ converge uniformemente a $f$ en cada subconjunto compacto de $D$, entonces $f$ es inyectiva o constante.
\end{theorem}

\begin{proof}
    Sabemos que $f$ es holomorfa en $D$.
    Supongamos que $f$ no es constante.
    Sean $a, b \in D$ con $a \neq b$.
    Veamos que $f(a) \neq f(b)$.
    $D \setminus \{a\}$ es un dominio en $\mathbb{C}$.
    Para cada $n \in \mathbb{N}$, sea $g_n(z) = f_n(z) - f_n(a)$ si $z \in D \setminus \{a\}$.
    Cada $g_n$ es holomorfa y nunca nula en $D \setminus \{a\}$.
    $f_n \to f$ uniformemente en cada subconjunto compacto de $D$.
    Sea $g(z) = f(z) - f(a)$, $z \in D \setminus \{a\}$.
    Entonces $g_n \to g$ uniformemente en cada subconjunto compacto de $D \setminus \{a\}$.
    Por el teorema anterior, $g \equiv 0$ en $D \setminus \{a\}$ o bien $g$ es nunca nula en $D \setminus \{a\}$.

    \begin{enumerate}
        \item Si $g \equiv 0$ en $D \setminus \{a\}$, entonces $f(z) = f(a)$ si $z \in D \setminus \{a\}$ $\Rightarrow$ $f(z) = f(a)$ si $z \in D$.
              $f$ es constante, lo que contradice nuestra hipótesis.
        \item Si $g(z) \neq 0$ si $z \in D \setminus \{a\}$, en particular $g(b) = f(b) - f(a) \neq 0 \Rightarrow f(a) \neq f(b)$.
    \end{enumerate}
\end{proof}