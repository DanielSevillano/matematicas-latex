\chapter{Funciones enteras. Crecimiento y distribución de los ceros}
\begin{theorem}[Teorema de Liouville]
    Si $f$ es una función entera y acotada, entonces $f$ es constante.
\end{theorem}

\begin{theorem}[Generalización del teorema de Liouville]
    Sea $f$ entera tal que existen $\alpha > 0$, $c > 0$ y $R_0 > 0$ con $|f(z)| \leq c|z|^\alpha$ para todo $|z| \geq R_0$.
    Entonces $f$ es un polinomio de grado $E(\alpha)$ y por tanto tiene $E(\alpha)$ ceros.
\end{theorem}

\begin{proof}
    Sea $f(z) = \sum_{n=0}^\infty a_nz^n$.
    Sea $R > 0$,
    $$a_n = \frac{f^{(n)}(0)}{n!} = \frac{1}{2\pi i} \int_{|z|=R} \frac{f(z)}{z^{n+1}}dz \leq \max_{|z|=R} |f(z)|\frac{1}{R^n} \leq c\frac{|z|^\alpha}{R} = cR^{\alpha-n} \xrightarrow[R \to \infty, n > E(\alpha)]{} 0.$$
    Por tanto, $a_n = 0$ para todo $n > E(\alpha)$.
\end{proof}

\begin{remark}
    \hfill
    \begin{enumerate}
        \item Parece que si el crecimiento está controlado, el número de ceros también lo está.
        \item Al revés esto no ocurre.
              Por ejemplo, con la exponencial.
    \end{enumerate}
\end{remark}

\section{La fórmula de Jensen}
La fórmula de Jensen permite controlar el número de ceros de una función holomorfa sabiendo restricciones sobre su crecimiento.

\begin{lemma}
    Sea $s > 0$.
    Entonces
    $$\frac{1}{2\pi} \int_{-\pi}^\pi \log|1+se^{i\theta}|d\theta = \log^+(s),$$
    donde
    $$\log^+(s) = \begin{cases}
            \log(s) & \text{si } s \geq 1,  \\
            0       & \text{si } 0 < s < 1.
        \end{cases}$$
\end{lemma}

\begin{proof}
    Sabemos que $\Log(1-z)$ es holomorfa en $\mathbb{D}$.
    \begin{itemize}
        \item Si $s < 1$, entonces por el teorema del valor medio
              $$\frac{1}{2\pi} \int_{-\pi}^\pi \Log|1-se^{i\theta}|d\theta = 0.$$

        \item Si $s > 1$,
              \begin{align*}
                   & \frac{1}{2\pi} \int_{-\pi}^\pi \log|1-se^{i\theta}|d\theta = \frac{1}{2\pi} \int_{-\pi}^\pi \log\left|se^{i\theta}\left(\frac{1}{s}e^{-i\theta}-1\right)\right|d\theta =                                                                    \\
                   & = \frac{1}{2\pi} \int_{-\pi}^\pi \log(s)d\theta + \frac{1}{2\pi} \int_{-\pi}^\pi \log\left|1-\frac{1}{s}e^{-i\theta}\right|d\theta = \log(s) - \frac{1}{2\pi} \int_{-\pi}^\pi \log\left|1-\frac{1}{s}e^{i\varphi}\right|d\varphi = \log(s).
              \end{align*}

        \item Sea $s = 1$.
              $$|1-e^{i\theta}|^2 = (1-e^{i\theta})(1-e^{-i\theta}) = 2-2\cos(\theta).$$
              Sabemos que
              $$\cos(\theta) = \cos\left(2\frac{\theta}{2}\right) = \cos^2\left(\frac{\theta}{2}\right) - \sin^2\left(\frac{\theta}{2}\right) = 1-2\sin^2\left(\frac{\theta}{2}\right).$$
              Luego $|1-e^{i\theta}|^2 = 4\sin^2\left(\frac{\theta}{2}\right)$.
              \begin{align*}
                   & \frac{1}{2\pi} \int_{-\pi}^\pi \log|1-e^{i\theta}|d\theta = \frac{1}{2\pi} \int_{-\pi}^\pi \log\left(2\left|\sin\left(\frac{\theta}{2}\right)\right|\right)d\theta = \log(2) + \frac{1}{2\pi} \int_{-\pi}^\pi \log\left|\sin\left(\frac{\theta}{2}\right)\right|d\theta = \\
                   & = \log(2) + \frac{1}{\pi} \int_0^{\pi} \log\left(\sin\left(\frac{\theta}{2}\right)\right)d\theta = \log(2) + \frac{2}{\pi} \int_{0}^{\pi/2} \log(\sin(\varphi))d\varphi.
              \end{align*}
              Ahora bien,
              \begin{align*}
                   & \int_0^{\pi/2} \log(\sin(\theta))d\theta = \int_0^{\pi/2} \log\left(2\sin\left(\frac{\theta}{2}\right)\cos\left(\frac{\theta}{2}\right)\right)d\theta =                                        \\
                   & = \frac{\pi}{2}\log(2) + \int_0^{\pi/2} \log\left(\sin\left(\frac{\theta}{2}\right)\right)d\theta + \int_0^{\pi/2} \log\left(\cos\left(\frac{\theta}{2}\right)\right)d\theta =                 \\
                   & = \frac{\pi}{2}\log(2) + \int_0^{\pi/2} \log\left(\sin\left(\frac{\theta}{2}\right)\right)d\theta + \int_0^{\pi/2} \log\left(\sin\left(\frac{\pi}{2} - \frac{\theta}{2}\right)\right)d\theta = \\
                   & = \frac{\pi}{2}\log(2) + 2\int_0^{\pi/2} \log(\sin(\varphi))d\varphi - 2\int_{\pi/2}^{\pi/4} \log(\sin(\varphi))d\varphi = \frac{\pi}{2}\log(2) + 2\int_0^{\pi/2} \log(\sin(\varphi))d\varphi.
              \end{align*}
              Así que
              $$\int_0^{\pi/2} \log(\sin(\theta))d\theta = -\frac{\pi}{2}\log(2).$$
              Por tanto,
              $$\frac{1}{2\pi} \int_{-\pi}^\pi \log|1-e^{i\theta}|d\theta = 0.$$
    \end{itemize}
\end{proof}

\begin{theorem}[Fórmula de Jensen]
    Sea $f$ una función holomorfa en $D(0, R)$ con $f(0) \neq 0$ y tal que tiene ceros $\{a_n\}$ de modo que
    $$|a_1| \leq |a_2| \leq |a_3| \leq \dots$$
    Sea $\rho \in (0, R)$ y $n(\rho, f) = \#\{a_n : |a_n| \leq \rho\}$.
    Entonces
    $$\frac{1}{2\pi} \int_{-\pi}^\pi \log|f(\rho e^{i\theta})|d\theta = \log|f(0)| + \sum_{k=1}^{n(\rho, f)} \log\left(\frac{\rho}{|a_k|}\right).$$
\end{theorem}

\begin{proof}
    Si $\rho = 0$,
    $$\frac{1}{2\pi} \int_{-\pi}^\pi \log|f(0)|d\theta = \log|f(0)|.$$
    Sea ahora $\rho \in (0, R)$.
    Supongamos que $f$ no tiene ceros en $\overline{D(0, \rho)}$.
    Entonces existe $\rho'$ con $0 < \rho < \rho' < R$ tal que $f$ es holomorfa en $D(0, \rho')$ y $f$ no tiene ceros en $D(0, \rho')$.
    Así, como $\Log(f(z))$ es holomorfa en $D(0, \rho')$, tenemos que $\log|f(z)|$ es armónica en $D(0, \rho')$.
    Por el teorema del valor medio,
    $$\frac{1}{2\pi} \int_{-\pi}^\pi \log|f(\rho e^{i\theta})|d\theta = \log|f(0)|.$$
    En general, consideramos
    $$F(z) = \frac{f(z)}{\prod_{k=1}^{n(\rho, f)} (z-a_k)}.$$
    Es holomorfa en $D(0, R)$ y no se anula en $\overline{D(0, \rho)}$.
    Por el caso anterior,
    \begin{align*}
         & \frac{1}{2\pi} \int_{-\pi}^\pi \log\left(\frac{|f(\rho e^{i\theta})|}{\prod_{k=1}^{n(\rho, f)} |\rho e^{i\theta} - a_k|}\right)d\theta = \frac{1}{2\pi} \int_{-\pi}^\pi \log|f(\rho e^{i\theta})|d\theta - \frac{1}{2\pi} \int_{-\pi}^\pi \log\left(\prod_{k=1}^{n(\rho, f)} |\rho e^{i\theta} - a_k|\right)d\theta = \\
         & = \log|f(0)| + \sum_{k=1}^{n(\rho, f)} \log\left(\frac{1}{|a_k|}\right).
    \end{align*}
    Sin embargo, para dar este paso es necesario que las integrales converjan por separado.
    Sea $a_k = \rho_ke^{i\theta_k}$.
    \begin{align*}
         & \frac{1}{2\pi} \int_{-\pi}^\pi \log|\rho e^{i\theta} - \rho_ke^{i\theta_k}|d\theta = \frac{1}{2\pi} \int_{-\pi}^\pi \log\left|\rho e^{i\theta}\left(1-\frac{\rho_k}{\rho}e^{i(\theta_k-\theta)}\right)\right|d\theta = \\
         & = \frac{1}{2\pi} \int_{-\pi}^\pi \log|\rho e^{i\theta}|d\theta + \frac{1}{2\pi} \int_{-\pi}^\pi \log\left|1-\frac{\rho_k}{\rho}e^{i(\theta_k-\theta)}\right|d\theta =                                                  \\
         & = \log(\rho) - \frac{1}{2\pi} \int_{-\pi}^\pi \log\left|1-\frac{\rho_k}{\rho}e^{i\varphi}\right|d\varphi = \log(\rho).
    \end{align*}
\end{proof}

\begin{definition}
    Se llama función contadora de Nevanlinna a la cantidad
    $$N(\rho, f) = \sum_{\{n : |a_n| \leq \rho\}} \log\left(\frac{\rho}{|a_n|}\right).$$
\end{definition}

\begin{remark}
    \hfill
    \begin{enumerate}
        \item $$N(\rho, f) = \sum_{\{n : |a_n| \leq \rho\}} \log\left(\frac{\rho}{|a_n|}\right) = \sum_{k=1}^{n(\rho, f)} \log\left(\frac{\rho}{|a_k|}\right) = \sum_{k=1}^\infty \log^+\left(\frac{\rho}{|a_k|}\right).$$
        \item $$N(\rho, f) = \int_0^\rho \frac{n(t, f)}{t}dt.$$
              \begin{proof}
                  Sean $f$ y $\rho$.
                  Por simplicidad, llamamos $n(\rho) = n(\rho, f)$.
                  Existen $a_1, \dots, a_{n(\rho)}$ ceros de $f$ de modo que
                  $$|a_1| \leq |a_2| \leq \dots \leq |a_{n(\rho)}|.$$
                  Van a existir $p$ radios $\rho_1 < \dots < \rho_p$ tal que cada cero $a_k$ va a tener asociado un único $\rho_j$ de la forma $|a_k| = \rho_j$.
                  Entenderemos que $n(\rho_0) = 0$.
                  \begin{align*}
                       & N(\rho, f) = \sum_{\{n : |a_n| \leq \rho\}} \log\left(\frac{\rho}{|a_n|}\right) = \sum_{j=1}^p \sum_{\{n : |a_n| \leq \rho_j\}} \log\left(\frac{\rho}{|a_n|}\right) = \sum_{j=1}^p \log\left(\frac{\rho}{\rho_j}\right) \sum_{\{n : |a_n| \leq \rho_j\}} 1 =      \\
                       & = \sum_{j=1}^p \log\left(\frac{\rho}{\rho_j}\right)\left(n(\rho_j)-n(\rho_{j-1})\right) =                                                                                                                                                                         \\
                       & = \log\left(\frac{\rho}{\rho_1}\right)(n(\rho_1)-n(\rho_0)) + \log\left(\frac{\rho}{\rho_2}\right)(n(\rho_2)-n(\rho_1)) + \dots + \log\left(\frac{\rho}{\rho_p}\right)(n(\rho_p)-n(\rho_{p-1})) =                                                                 \\
                       & = n(\rho_1)\left(\log\left(\frac{\rho}{\rho_1}\right) - \log\left(\frac{\rho}{\rho_2}\right)\right) + n(\rho_2)\left(\log\left(\frac{\rho}{\rho_2}\right) - \log\left(\frac{\rho}{\rho_3}\right)\right) + \dots + n(\rho_p)\log\left(\frac{\rho}{\rho_p}\right) = \\
                       & = n(\rho_1)\log\left(\frac{\rho_2}{\rho_1}\right) + n(\rho_2)\log\left(\frac{\rho_3}{\rho_2}\right) + \dots + n(\rho_{p-1})\log\left(\frac{\rho_p}{\rho_{p-1}}\right) + n(\rho_p)\log\left(\frac{\rho}{\rho_p}\right) =                                           \\
                       & = \int_{\rho_1}^{\rho_2} \frac{n(t)}{t}dt + \int_{\rho_2}^{\rho^3} \frac{n(t)}{t}dt + \dots + \int_{\rho_{p-1}}^{\rho_p} \frac{n(t)}{dt}dt + \int_{\rho_p}^\rho \frac{n(t)}{t}dt = \int_{\rho_1}^\rho \frac{n(t)}{t}dt = \int_0^\rho \frac{n(t)}{t}dt.
                  \end{align*}
              \end{proof}
    \end{enumerate}
\end{remark}

Existen diversas generalizaciones de la fórmula de Jensen.

\begin{theorem}
    Sea $f(z) = c_Nz^N + c_{N+1}z^{N+1} + \dots$ una función holomorfa en $D(0, R)$ y sea $\rho \in [0, R)$ tal que $\{a_n\}_{n=1}^\infty$ son los ceros de $f$ en $D(0, R)$ y además están ordenados respetando la multiplicidad.
    Entonces:
    $$\frac{1}{2\pi} \int_{-\pi}^\pi \log|f(\rho e^{i\theta})|d\theta = \log|c_N\rho^N| + \sum_{k=1}^{n(\rho, f)} \log\left(\frac{\rho}{|a_k|}\right).$$
\end{theorem}

\begin{proof}
    Consideramos
    $$F(z) = \frac{f(z)}{c_Nz^N}.$$
    $F$ es holomorfa en $D(0, R)$, $F(0) = 1$ y los ceros de $F$ son los mismos que los de $f$.

    Aplicando la fórmula de Jensen,
    \begin{align*}
         & \frac{1}{2\pi} \int_{-\pi}^\pi \log|f(\rho e^{i\theta})d\theta = \log|F(0)| + \sum_{k=1}^{n(\rho, f)} \log\left(\frac{\rho}{|a_k|}\right) \Leftrightarrow                                                             \\
         & \Leftrightarrow \frac{1}{2\pi} \int_{-\pi}^\pi \log|f(\rho e^{i\theta})|d\theta - \frac{1}{2\pi} \int_{-\pi}^\pi \log|c_N\rho^N|d\theta = \sum_{k=1}^{n(\rho, f)} \log\left(\frac{\rho}{|a_k|}\right) \Leftrightarrow \\
         & \Leftrightarrow \frac{1}{2\pi} \int_{-\pi}^\pi \log|f(\rho e^{i\theta})|d\theta = \log|c_N\rho^N| + \sum_{k=1}^{n(\rho, f)} \log\left(\frac{\rho}{|a_k|}\right).
    \end{align*}
\end{proof}

Vamos a ver cómo aplicar la fórmula de Jensen para ver que si el crecimiento de $f$ está controlado entonces sus ceros también lo están.

\begin{definition}
    Llamamos módulo máximo de $f$ a
    $$M_\infty(\rho, f) = \sup_{|z|=\rho} |f(z)|.$$
\end{definition}

\begin{remark}
    Si $f$ es holomorfa,
    $$M_\infty(\rho, f) = \max_{|z|=\rho} |f(z)| = \max_{|z|\leq\rho} |f(z)|.$$
\end{remark}

\begin{theorem}
    Sea $f$ holomorfa en $D(0, R)$ tal que $f(0) \neq 0$ y $0 \leq s \leq \rho < R$.
    Entonces:
    $$n(s, f)\log\left(\frac{\rho}{s}\right) \leq \log(M_\infty(\rho, f)) - \log|f(0)|.$$
\end{theorem}

\begin{proof}
    \begin{align*}
        n(s, f)\log\left(\frac{\rho}{s}\right) & = n(s, f)\int_s^\rho \frac{1}{t}dt \leq \int_s^\rho \frac{n(t, f)}{t}dt \leq \int_0^\rho \frac{n(t, f)}{t}dt =            \\
                                               & = \frac{1}{2\pi} \int_{-\pi}^\pi \log|f(\rho e^{i\theta})|d\theta - \log|f(0)| \leq \log(M_\infty(\rho, f)) - \log|f(0)|.
    \end{align*}
\end{proof}

\begin{example}
    Sea $f$ entera con $|f(0)| = 1$ y sea $\rho = es$.
    Entonces:
    $$n(s, f) \leq \log(M_\infty(es, f)).$$
\end{example}

\begin{remark}
    Si $f(z) = c_Nz^N + \dots$, el teorema anterior se puede escribir de la forma:
    $$n(s, f)\log\left(\frac{\rho}{s}\right) \leq \log(M_\infty(\rho, f)) - \log|c_N\rho^N|.$$
\end{remark}

\section{La fórmula de Poisson-Jensen}
\begin{theorem}
    Sea $R > 0$ tal que $f$ es holomorfa en $D(0, R)$ con la sucesión de ceros ordenados $\{a_n\}_{n=1}^\infty$.
    Sea $s \in (0, R)$ y sea $z \in D(0, s)$ con $f(z) \neq 0$.
    Entonces:
    $$\frac{1}{2\pi} \int_{-\pi}^\pi \log|f(se^{i\theta})|\frac{s^2-|z|^2}{|s-ze^{-i\theta}|^2}d\theta = \log|f(z)| + \sum_{\{n : |a_n| \leq s\}} \log^+\left|\frac{s^2-\overline{a}_nz}{s(z-a_n)}\right|.$$
\end{theorem}

\begin{proof}
    Sea $T_a(\xi) = \frac{\xi + a}{1 + \overline{a}\xi}$.
    Esta transformación manda $\overline{\mathbb{D}} \to \overline{\mathbb{D}}$.
    Entonces, sea $\psi(\xi) = sT_{z/s}(\xi)$,
    \begin{align*}
        \psi(\xi)      & = s\frac{\xi + \frac{Z}{s}}{1 + \frac{\overline{z}}{s}\xi} = \frac{s(\xi s + z)}{s+\overline{z}\xi} \Rightarrow \psi(0) = z,                                 \\
        \psi^{-1}(\xi) & = S_{z/s}\left(\frac{\xi}{s}\right) = \frac{\frac{\xi}{s}-\frac{z}{s}}{1-\frac{\overline{z}}{s}\frac{\xi}{s}} = \frac{\xi-z}{s - \frac{\overline{z}}{s}\xi}.
    \end{align*}
    $f \circ \psi$ es holomorfa en el disco $D(0, \rho)$ con $\rho > 1$ y $\psi(D(0, \rho)) \subset D(0, R)$.

    Aplicamos la fórmula de Jensen a $f \circ \psi$.
    Como $f \circ \psi(\xi) = 0 \Leftrightarrow f(\psi(\xi)) = 0$ y esto ocurre si y solo si existe $n \in \mathbb{N}$ tal que $\psi(\xi) = a_n$, entonces $\xi = \psi^{-1}(a_n)$.
    $$\log|f \circ \psi(0)| = \frac{1}{2\pi} \int_{-\pi}^\pi \log|f \circ \psi(e^{i\theta})|d\theta - \sum_{\{n : |a_n| \leq s\}} \log\left(\frac{1}{|\psi^{-1}(a_n)|}\right).$$
    Si $a_n$ es un cero de $f$ de orden $N$, entonces $\xi = \psi^{-1}(a_n)$ es un cero de $f \circ \psi$ de orden $N$.
    Además,
    $$\frac{1}{|\varphi^{-1}(a_n)|} = \left|\frac{s-\frac{\overline{z}}{s}a_n}{a_n-z}\right| = \left|\frac{s^2-\overline{a_n}z}{s(z-a_n)}\right|.$$

    Por otro lado, haciendo el cambio de variable $se^{i\varphi} = \psi(e^{i\theta})$,
    $$\int_{-\pi}^\pi \log|f \circ \psi(e^{i\theta})|d\theta = \int_{-\pi}^\pi \log|f(se^{i\varphi})|\frac{s^2-|z|^2}{|s-ze^{-i\varphi|^2}}d\varphi.$$
    Por tanto:
    $$\log|f \circ \psi(0)| = \log|f(z)| = \frac{1}{2\pi} \int_{-\pi}^\pi \log|f(se^{i\varphi})|\frac{s^2-|z|^2}{|s-ze^{-i\varphi|^2}}d\varphi - \sum_{\{n : |a_n| \leq s\}} \log\left|\frac{s^2-\overline{a_n}z}{s(z-a_n)}\right|.$$
\end{proof}

\begin{remark}
    \hfill
    \begin{enumerate}
        \item Sea $|z| < s$.
              Entonces:
              $$\frac{|s^2-a_nz|}{|s(z-a_n)|} = \frac{1}{|\psi^{-1}(a_n)} \geq 1 \Rightarrow \log|f(z)| \leq \frac{1}{2\pi} \int_{-\pi}^\pi \log|f(se^{i\theta})|\frac{s^2-|z|^2}{|s-ze^{-i\theta}|^2}d\theta.$$
        \item Sea $z = \rho e^{i\varphi}$ con $0 < \varphi < s$.
              Entonces:
              \begin{align*}
                   & \max_{0 \leq \varphi \leq 2\pi} \log|f(\rho e^{i\varphi})| \leq \max_{0 \leq \varphi \leq 2\pi} \int_{-\pi}^\pi \log|f(se^{i\theta})|\frac{s^2-\rho^2}{|s-\rho e^{i\varphi}e^{-i\theta}|^2}d\theta \leq \\
                   & \leq \frac{1}{2\pi} \int_{-\pi}^\pi \log(M_\infty(s, f))\frac{s+\rho}{s-\rho}d\theta = \log(M_\infty(s, f))\frac{s+\rho}{s-\rho} \leq \log^+(M_\infty(s, f))\frac{s+\rho}{s-\rho}.
              \end{align*}
    \end{enumerate}
\end{remark}

\section{Introducción a los espacios de Hardy}
\begin{theorem}[Condición de Blaschke]
    Sea $f$ una función holomorfa en $\mathbb{D}$ con la sucesión de ceros ordenados $\{a_n\}_{n=1}^\infty$.
    Entonces son equivalentes:
    \begin{enumerate}
        \item $$\sup_{0<\rho<1} \frac{1}{2\pi} \int_{-\pi}^\pi \log|f(\rho e^{i\theta})|d\theta < \infty.$$
        \item $$\sum_{n=1}^\infty (1-|a_n|) < \infty.$$
    \end{enumerate}
\end{theorem}

\begin{proof}
    Si $f(z) = c_Nz^N + \dots$, por la fórmula de Jensen generalizada
    $$\log|c_N\rho^N| + \sum_{\{n : |a_n|\leq\rho\}} \log\left(\frac{\rho}{|a_n|}\right) = \frac{1}{2\pi} \int_{-\pi}^\pi \log|f(\rho e^{i\theta})|d\theta.$$
    \begin{itemize}
        \item[$\boxed{\Rightarrow}$] $$\log|c_N\rho^N| + \sum_{\{n : |a_n|\leq\rho\}} \log\left(\frac{\rho}{|a_n|}\right) \leq \sup_{0<\rho<1} \frac{1}{2\pi} \int_{-\pi}^\pi \log|f(\rho e^{i\theta})|d\theta < \infty.$$
            Tomando límite $\rho \to 1$ a la izquierda,
            $$\log|c_N| + \sum_{n=1}^\infty \log\left(\frac{1}{|a_n|}\right) < \infty \Rightarrow \sum_{n=1}^\infty (1-|a_n|) < \infty.$$
        \item[$\boxed{\Leftarrow}$] $$\frac{1}{2\pi} \int_{-\pi}^\pi \log|f(\rho e^{i\theta})d\theta \leq \log|c_N| + \sum_{n=1}^\infty \log\left(\frac{1}{|a_n|}\right) < \infty.$$
            Por tanto,
            $$\sup_{0<\rho<1} \frac{1}{2\pi} \int_{-\pi}^\pi \log|f(\rho e^{i\theta})|d\theta < \infty.$$
    \end{itemize}
\end{proof}

\begin{definition}
    Se define la función característica de Nevanlinna como
    $$T(\rho, f) = \frac{1}{2\pi} \int_{-\pi}^\pi \log^+|f(\rho e^{i\theta})|d\theta.$$
\end{definition}

\begin{definition}
    Sea $0 < p < \infty$, se define la media integral de orden $p$ como
    $$M_p(\rho, f) = \left(\frac{1}{2\pi} \int_{-\pi}^\pi |f(\rho e^{i\theta})|^pd\theta\right)^{1/p}.$$
\end{definition}

\begin{definition}
    Se define el módulo máximo de $f$ como
    $$M_\infty(\rho, f) = \max_{\theta \in [0, 2\pi]} |f(\rho e^{i\theta})|.$$
\end{definition}

\begin{definition}
    Se define la clase de Nevanlinna como
    $$\mathcal{N} = \{f \in \Hol(\mathbb{D}) : \sup_{0<\rho<1} T(\rho, f) < \infty\}.$$
\end{definition}

\begin{definition}
    Sea $0 < p \leq \infty$, se define el espacio de Hardy como
    $$H^p = \{f \in \Hol(\mathbb{D}) : \sup_{0<\rho<1} M_p(\rho, f) < \infty\}.$$
\end{definition}

\begin{theorem}
    Si $0 < p < q < \infty$, $H^\infty \subset H^q \subset H^p \subset \mathcal{N}$.
\end{theorem}

\begin{proof}
    \hfill
    \begin{enumerate}
        \item Veamos que $H^\infty \subset H^q$, con $0 < q < \infty$.
              $$M_q^q(\rho, f) = \frac{1}{2\pi} \int_{-\pi}^\pi |f(\rho e^{i\theta})|^q d\theta \leq M_\infty^q(\rho, f).$$
              Por tanto, $M_q(\rho, f) \leq M_\infty(\rho, f)$ y tomando supremos se tiene.

        \item Veamos que $H^q \subset H^p$, con $0 < p < q < \infty$.
              $$M_p^p(\rho, f) = \frac{1}{2\pi} \int_{-\pi}^\pi |f(\rho e^{i\theta})^p d\theta.$$
              Usando la desigualdad de Hölder,
              $$\frac{1}{2\pi} \int_{-\pi}^\pi |f(\rho e^{i\theta})^p d\theta \leq \left(\frac{1}{2\pi} \int_{-\pi}^\pi |f(\rho e^{i\theta})|^q d\theta\right)^{p/q} = M_q^{q\frac{p}{q}}(\rho, f) = M_q^p(\rho, f).$$
              Por tanto, $M_p(\rho, f) \leq M_q(\rho, f)$ y tomando supremos se tiene.

        \item Veamos que $H^p \subset \mathcal{N}$, con $0 < p < \infty$.
              Como
              $$\log(x^p) \leq \frac{1}{e}x^p \Leftrightarrow \log(x) \leq \frac{1}{pe}x^p \Rightarrow \log^+(x) \leq \frac{1}{pe}x^p,$$
              entonces
              $$\frac{1}{2\pi} \int_{-\pi}^\pi \log^+|f(\rho e^{i\theta})d\theta \leq \frac{1}{pe}\frac{1}{2\pi} \int_{-\pi}^\pi |f(\rho e^{i\theta})|^pd\theta = \frac{1}{pe}M_p^p(\rho, f).$$
              Tomando supremos se tiene.
    \end{enumerate}
\end{proof}

\begin{theorem}
    Sea $f \in \mathcal{N}$, entonces la sucesión de sus ceros cumple la condición de Blaschke.
\end{theorem}

\begin{proof}
    Sea $f \in \mathcal{N}$ y $\{a_n\}_{n=1}^\infty$ su sucesión de ceros.
    $$\frac{1}{2\pi} \int_{-\pi}^\pi \log|f(\rho e^{i\theta})|d\theta \leq \frac{1}{2\pi} \int_{-\pi}^\pi \log^+|f(\rho e^{i\theta})|d\theta.$$
    Tomando supremos tenemos que
    $$\sup_{0<\rho<1} \frac{1}{2\pi} \int_{-\pi}^\pi \log|f(\rho e^{i\theta})|d\theta < \infty.$$
    Por un teorema anterior, $\{a_n\}_{n=1}^\infty$ cumple la condición de Blaschke.
\end{proof}

\begin{remark}
    Si $f \in H^p$ para $0 < p \leq \infty$, entonces $f \in \mathcal{N}$.
    Por tanto, sus ceros cumplen la condición de Blaschke.
\end{remark}

Sea $\{a_n\}_{n=1}^\infty \subset \mathbb{D}$ una sucesión de puntos que cumplen $\sum (1-|a_n|) < \infty$.
Queremos ver si existe una función $f \in \mathcal{N}$ tal que $f$ tiene como ceros la sucesión $\{a_n\}_{n=1}^\infty$.

\begin{theorem}[Producto infinito de Blaschke]
    Sea $\{a_n\}_{n=1}^\infty \subset \mathbb{D}$ tal que $\{a_n\}_{n=1}^\infty$ cumple la condición de Blaschke.
    Definimos:
    $$b_n = \begin{cases}
            z                                                             & \text{si } a_n = 0,    \\
            \dfrac{\overline{a_n}}{|a_n|}\dfrac{a_n-z}{1-\overline{a_n}z} & \text{si } a_n \neq 0.
        \end{cases}$$
    Consideramos $B(z) = \prod_{n=1}^\infty b_n$.
    Entonces $B(z)$ converge absoluta y uniformemente en compactos de $\mathbb{D}$ y además $|B(z)| < 1$ para todo $z \in \mathbb{D}$.
\end{theorem}

\begin{proof}
    Sea $K$ compacto de $\mathbb{D}$, entonces existe $R < 1$ tal que $K \subset \overline{D(0, R)}$.
    Si $a_n \neq 0$,
    \begin{align*}
         & |1-b_n| = \left|1 - \frac{\overline{a_n}}{|a_n|}\frac{a_n-z}{1-\overline{a_n}z}\right| = \left|1 - \frac{|a_n| - \frac{\overline{a_n}}{|a_n|}z}{1-\overline{a_n}z}\right| = \left|\frac{1 - |a_n| - \overline{a_n}z + \frac{\overline{a_n}}{|a_n|}z}{1 - \overline{a_n}z}\right| = \\
         & = \left|\frac{1-|a_n| + \frac{\overline{a_n}}{|a_n|}z(1-|a_n|)}{1-\overline{a_n}z}\right| = (1-|a_n|) \left|\frac{1 + \frac{\overline{a_n}}{|a_n|}z}{1-\overline{a_n}z}\right| \leq (1-|a_n|)\frac{1+|z|}{1-|z|} \leq (1-|a_n|)\frac{1+R}{1-R}.
    \end{align*}
    Por tanto:
    $$\sum_{\{n : a_n \neq 0\}} |1-b_n| \leq \frac{1+R}{1-R} \sum_{\{n : a_n \neq 0\}} (1-|a_n|) < \infty.$$
\end{proof}

\begin{remark}
    Si $\{a_n\}_{n=1}^\infty$ cumple la condición de Blaschke, entonces existe $f \in H^\infty$ que tiene como ceros $\{a_n\}_{n=1}^\infty$.
\end{remark}

\section{Orden de crecimiento de una función entera}
Sean $h, \varphi: \mathbb{C} \to [0, +\infty)$.
Entendemos por $h(R) = O(\varphi(R))$, con $R \to \infty$, que existen una constante $C$ y un $R_0$ tales que $h(R) \leq C\varphi(R)$ para todo $R \geq R_0$.
Nos va a interesar $M_\infty(s, f) = O(\varphi(s))$, con $s \to \infty$ y $f$ entera.
\begin{itemize}
    \item Si $M_\infty(s, f)$ es acotado, entonces $f$ es constante.
    \item Si $M_\infty(s, f) = O(s^\alpha)$, con $s \to \infty$ y $\alpha > 0$, entonces $f$ es un polinomio.
\end{itemize}

\begin{definition}
    Sea $f$ una función entera, definimos el orden de crecimiento de $f$ como
    $$\rho(f) = \inf\{\alpha \geq 0 : M_\infty(s, f) = O(e^{s^\alpha})\}.$$
\end{definition}

\begin{remark}
    \hfill
    \begin{enumerate}
        \item Si no existe $\alpha \geq 0$ tal que $M_\infty(s, f) = O(e^{s^\alpha})$, diremos que $\rho(f) = \infty$.

        \item Si $M_\infty(s, f) = O(e^{s^{\alpha_0}})$, entonces $M_\infty(s, f) = O(e^{s^\alpha})$ con $\alpha \geq \alpha_0$.

        \item Si $f$ es un polinomio, entonces $\rho(f) = 0$.
              \begin{proof}
                  Sea $\alpha > 0$ y $f(z) = \sum_{n=1}^N a_nz^n$.
                  $$|f(z)| \leq \sum_{n=1}^N |a_n||z|^n \Rightarrow M_\infty(s, f) \leq \sum_{n=1}^N |a_n|s^n.$$
                  Como
                  $$\limsup_{s \to \infty} \sum_{n=1}^N |a_n|\frac{s^n}{e^{s^\alpha}} = 0,$$
                  entonces $\rho(f) < \alpha$ para todo $\alpha > 0$.
                  Por tanto, $\rho(f) = 0$.
              \end{proof}

        \item Si $a \in \mathbb{C} \setminus \{0\}$, entonces $\rho(e^{az^\alpha}) = \alpha$.
              \begin{proof}
                  \hfill
                  \begin{itemize}
                      \item Si $\beta > \alpha$, como $|e^{az^\alpha}| = e^{\Re(az^\alpha)}$, entonces
                            $$M_\infty(s, e^{az^\alpha}) \leq e^{|a|s^\alpha}.$$
                            Así que
                            $$\frac{M_\infty(s, e^{az^\alpha})}{e^{s^\beta}} \leq e^{|a|s^\alpha - s^\beta} \to 0.$$
                            Luego:
                            $$M_\infty(s, e^{az^\alpha}) = O(e^{s^\beta}), \quad \forall \beta > \alpha.$$
                            Por tanto, $\rho(e^{az^\alpha}) \leq \alpha$.

                      \item Si $\beta < \alpha$, como $|e^{az^\alpha}| = e^{\Re(az^\alpha)}$, entonces
                            $$M_\infty(s, e^{az^\alpha}) \geq e^{|a|s^\alpha}.$$
                            Así que
                            $$\frac{M_\infty(s, e^{az^\alpha})}{e^{s\beta}} \geq \frac{e^{|a|s^\alpha}}{e^{s\beta}} \xrightarrow[s \to \infty]{} +\infty.$$
                            Luego:
                            $$\limsup_{s \to \infty} \frac{M_\infty(s, e^{az^\alpha})}{e^{s\beta}} = \infty.$$
                            Por tanto, $\rho(f) \geq \beta$ para todo $\beta < \alpha$, así que $\rho(f) \geq \alpha$.
                  \end{itemize}
              \end{proof}

        \item $f(z) = e^{e^z}$ tiene orden infinito.
              $M_\infty(s, f) = e^{e^s}$.
    \end{enumerate}
\end{remark}

\begin{proposition}
    Sea $f$ entera y no constante.
    Entonces
    $$\rho(f) = \limsup_{s \to \infty} \frac{\log(\log(M_\infty(s, f)))}{\log(s)} = \limsup_{s \to \infty} \frac{\log(T(s, f))}{\log(s)},$$
    donde
    $$T(s, f) = \frac{1}{2\pi} \int_0^{2\pi} \log^+|f(se^{i\theta})|d\theta.$$
\end{proposition}

% Observaciones y demostración

\section{Propiedades aritméticas}
\begin{lemma}
    \hfill
    \begin{enumerate}
        \item $x = x^+ - (-x)^+$, $x \in \mathbb{R}$.
        \item $(x+y)^+ \leq x^+ + y^+$, $x, y \in \mathbb{R}$.
        \item $\log(a) = \log^+(a) - \log^+\left(\frac{1}{a}\right)$, $a > 0$.
        \item $\log(a+b) \leq \log(2) + \log^+(a) + \log^+(b)$, $a, b > 0$.
        \item $\log^+(a+b) \leq \log(2) + \log^+(a) + \log^+(b)$, $a, b > 0$.
        \item $\log^+(a-b) \leq \log^+(a) + \log^+(b)$, $a, b > 0$.
        \item $\log^+(ab) \leq \log^+(a) + \log^+(b)$, $a, b > 0$.
    \end{enumerate}
\end{lemma}

\begin{remark}
    Veamos un par de observaciones sobre
    $$\rho(f) = \limsup_{s \to \infty} \frac{\log(\log(M_\infty(s, f)))}{\log(s)} = \limsup_{s \to \infty} \frac{\log(T(s, f))}{\log(s)}.$$
    \begin{enumerate}
        \item Si $\rho(f) < \alpha$, entonces existe $s_\alpha$ tal que si $s \geq s_\alpha$ entonces
              $$\log(M_\infty(s, f)) \leq s^\alpha \Leftrightarrow M_\infty(s, f) \leq e^{s^\alpha}$$
              y $T(s, f) \leq s^\alpha$.

        \item Si existen constantes $c, c_1, c_2, \beta > 0$ y $s > 0$ tales que
              $$M_\infty(s, f) \leq c + c_1s^\beta + c_2s^\alpha, \quad \forall s \geq s_0,$$
              entonces $\rho(f) \leq \alpha$.
              \begin{proof}
                  Como
                  $$\lim_{s \to \infty} \frac{c + c_1s^\beta}{e^{s^\alpha}} = 0,$$
                  entonces existe $s_1 \geq s_0$ tal que
                  $$c + c_1s^\beta \leq e^{s^\alpha}, \quad \forall s \geq s_1.$$
                  Por tanto, si $s \geq s_1$, $M_\infty(s, f) \leq (c_2+1)e^{s^\alpha}$.
                  Luego
                  $$\log(M_\infty(s, f)) \leq \log(c_2+1) + s^\alpha \Leftrightarrow \log(\log(M_\infty(s, f))) \leq \log(\log(c_1+1) +  s^\alpha).$$
                  Por el lema previo,
                  $$\log(\log(c_1+1) +  s^\alpha) \leq \log(2) + \log^+(\log(c_2+1)) + \log^+(s^\alpha).$$
                  Como existe $s_2 > 0$ tal que $s^\alpha > 1$ si $s \geq s_2$, entonces
                  $$\log(2) + \log^+(\log(c_2+1)) + \log^+(s^\alpha) = \log(2) + \log^+(\log(c_2+1)) + \alpha\log(s), \quad s \geq s_2.$$
                  Por tanto, $\rho(f) \leq \alpha$.
              \end{proof}

        \item Si existen constantes $c_1, c_2, c_3, \alpha > 0$ tales que
              $$T(s, f) \leq c_1 + c_2\log(s) + c_3s^\alpha,$$
              si $s \geq s_0 > 1$ entonces $\rho(f) \leq \alpha$.
    \end{enumerate}
\end{remark}

\begin{theorem}[Teorema del álgebra del orden de crecimiento]
    Sean $f$ y $g$ enteras.
    Entonces
    \begin{enumerate}
        \item $\rho(cf) = \rho(f)$, si $c \in \mathbb{C} \setminus \{0\}$.
        \item $\rho(f+g) \leq \max\{\rho(f), \rho(g)\}$, con igualdad si $\rho(f) \neq \rho(g)$.
        \item $\rho(fg) \leq \max\{\rho(f), \rho(g)\}$, con igualdad si $\rho(f) \neq \rho(g)$.
        \item $\rho(f) = \rho(f')$.
    \end{enumerate}
\end{theorem}

% Demostración

\section{Orden de los factores primos de Weierstrass}
Recordemos que los factores primos de Weierstrass son
\begin{align*}
    E_0(z) & = 1-z,                                                                      \\
    E_p(z) & = (1-z)\exp\left(\sum_{k=1}^p \frac{z^k}{k}\right), \quad p \in \mathbb{N}.
\end{align*}

Como $E_0$ es un polinomio, $\rho(E_0) = 0$.
Veamos que $\rho(E_p) = p$.
Como $\rho(e^c) = 0$ y $\rho(e^{c_1z}) = 1$,
$$\rho(e^{c+c_1z}) = \rho(e^ce^{c_1z}) = \max\{\rho(e^c), \rho(e^{c_1z})\} = \max\{0, 1\} = 1.$$
Por inducción se prueba el resultado.

\section{Orden de una función entera expresada en sus coeficientes de Taylor}
Queremos ver si existe una función entera con orden cualquier número $\rho > 0$.

Sea $f(z) = \sum_{n=0}^\infty c_nz^n$ entera.
Si $R$ es su radio de convergencia, como $R = \infty$,
$$\frac{1}{R} = \limsup_{n \to \infty} |c_n|^{1/n} \Rightarrow \limsup_{n \to \infty} |c_n|^{1/n} = 0.$$

\begin{theorem}
    Sea $f$ una función entera no constante tal que $f(z) = \sum_{n=0}^\infty c_nz^n$.
    Entonces:
    $$\rho(f) = \limsup_{n \to \infty} \frac{\log(n)}{\log|c_n|^{-1/n}} = \limsup_{n \to \infty} \frac{n\log(n)}{\log\left(\frac{1}{|c_n|}\right)}.$$
\end{theorem}

% Demostración

Sea $\rho > 0$,
$$\frac{\log(n)}{\log|c_n|^{-1/n}} = \rho \Rightarrow \log(n) = \log|c_n|^{-\rho/n} \Rightarrow |c_n|^{\rho/n} = n^{-1} \Rightarrow |c_n| = n^{-n/\rho}.$$

\begin{corollary}
    Sea
    $$F(z) = \sum_{n=1}^\infty n^{-n/\rho}z^n,$$
    con $\rho > 0$.
    Entonces $F$ es entera y su orden de crecimiento es $\rho$.
\end{corollary}

\begin{proof}
    En primer lugar calculamos $\frac{1}{R}$, con $R$ su radio de convergencia.
    $$\limsup_{n \to \infty} |n^{-n/\rho}|^{1/n} = \limsup_{n \to \infty} \frac{1}{n^{1/\rho}} = 0.$$
    Por tanto, $F$ es entera.
    Por otro lado,
    $$\rho(f) = \limsup_{n \to \infty} \frac{\log(n)}{\log|c_n|^{-1/n}} = \limsup_{n \to \infty} \frac{\log(n)}{\frac{1}{\rho}\log(n)} = \rho.$$
\end{proof}

\section{Relación entre el orden de crecimiento y el exponente de convergencia de los ceros}
Sea $\{a_n\}_{n=1}^\infty \subset \mathbb{C} \setminus \{0\}$, se definía el exponente de convergencia como
$$\sigma = \inf\left\{s > 0 : \sum_{n=1}^\infty \frac{1}{|a_n|^s} < \infty\right\}.$$
El género de $\{a_n\}_{n=1}^\infty$ es el menor $p \in \mathbb{N} \cup \{0\}$ de modo que
$$\sum_{n=1}^\infty \frac{1}{|a_n|^{p+1}} < \infty.$$
Teníamos que $p \leq \sigma \leq p+1$.

\begin{example}
    \hfill
    \begin{enumerate}
        \item Si en cada circunferencia de radio $n^2$ ponemos un cero para todo $n \in \mathbb{N}$,
              $$\sum_{n=1}^\infty \frac{1}{|a_n|^s} = \sum_{n=1}^\infty \frac{1}{n^{2s}} < \infty, \quad \forall s > \frac{1}{2}.$$
              Por tanto, $\sigma = \frac{1}{2}$.

        \item Si en cada circunferencia de radio $n^2$ ponemos dos ceros para todo $n \in \mathbb{N}$,
              $$\sum_{n=1}^\infty \frac{1}{|a_n|^s} = \sum_{n=1}^\infty \frac{1}{n^{s}} < \infty, \quad \forall s > 1.$$
              Por tanto, $\sigma = 1$.
    \end{enumerate}
\end{example}

\begin{lemma}
    Sea $\{a_n\} \subset \mathbb{C} \setminus \{0\}$, con $|a_n| \to \infty$.
    Entonces
    $$\sigma = \limsup_{R \to \infty} \frac{\log(n(R))}{\log(R)},$$
    con $n(R) = \#\{a_n : |a_n| \leq R\}$.
\end{lemma}

% Demostración

\begin{theorem}
    Sea $f$ una función entera no constante cuyo orden de crecimiento es $\rho(f)$ y tiene como ceros no nulos la sucesión $\{a_n\}_{n=1}^\infty$.
    Entonces:
    $$\sigma(\{a_n\}_{n=1}^\infty) \leq \rho(f).$$
\end{theorem}

% Demostración

\section{Orden de los productos canónicas de Weierstrass}
Queremos ver si existen funciones tales que $\sigma(f) = \rho(f)$.

\begin{theorem}
    Sea $\{a_n\}_{n=1}^\infty \subset \mathbb{C} \setminus \{0\}$ tal que el exponente de convergencia de $\{a_n\}_{n=1}^\infty$ es finito y tiene género $p$.
    Entonces la función
    $$F(z) = \prod_{n=1}^\infty E_p\left(\frac{z}{a_n}\right)$$
    tiene orden de convergencia $\rho(F) = \rho(\{a_n\})$.
\end{theorem}

\begin{example}
    Sea $\alpha > 0$ y sea $\{a_n\}_{n=1}^\infty$ con $a_n = n^\alpha$ para $n \in \mathbb{N}$.
    Sabemos que $\sigma(\{a_n\}_{n=1}^\infty) = \frac{1}{\alpha}$.
    Entonces su género es $p = E(1/\alpha)$ y la función
    $$F(z) = \prod_{n=1}^\infty E_p\left(\frac{z}{n^\alpha}\right)$$
    tiene orden de crecimiento $\rho(F) = \frac{1}{\alpha}$.
\end{example}

\section{Factorización de Hadamard}
Sea $f$ una función entera no constante con $\{a_n\}_{n=1}^\infty$ su sucesión de ceros no nulos, tal que $\sigma(\{a_n\}) < \infty$ y con género $p$.
Entonces la factorización canónica de Weierstrass permite expresar
$$f(z) = e^{g(z)}z^N \prod_{n=1}^\infty E_p\left(\frac{z}{a_n}\right),$$
donde $N$ es la multiplicidad del origen como 0.
Además, $g$ es única salvo constantes $2\pi i$.

\begin{proposition}
    $\rho(f) = \max\{\rho(e^g), \sigma\}$.
\end{proposition}

\begin{proof}
    Por el teorema del álgebra del orden de crecimiento,
    $$\rho(f) \leq \max\left\{\rho(e^g), \rho\left(\prod_{n=1}^\infty E_p\left(\frac{z}{a_n}\right)\right)\right\} = \max\{\rho(e^g), \sigma\}.$$
    Además, se tiene la igualdad si $\sigma \neq \rho(e^g)$.
    Si $\sigma = \rho(e^g)$, entonces por un teorema previo
    $$\sigma \leq \rho(f) \leq \sigma \Rightarrow \rho(f) = \sigma = \max\{\rho(e^g), \sigma\}.$$
\end{proof}

\begin{theorem}[Factorización de Hadamard]
    Sea $f$ una función entera no constante con ceros no nulos $\{a_n\}_{n=1}^\infty$, $\sigma = \sigma(\{a_n\}) < \infty$ y género $p$.
    Además, sea $N$ la multiplicidad del origen como cero de $f$.
    Entonces existe un polinomio $g$ con $\deg(p) \leq \rho(f)$, satisfaciendo que
    $$f(z) = e^{g(z)}z^N \prod_{n=1}^\infty E_p\left(\frac{z}{a_n}\right).$$
    $g$ es única salvo constantes $2\pi i$ y $\rho(f) = \max\{\deg(p), \sigma\}$.
\end{theorem}

% Demostración