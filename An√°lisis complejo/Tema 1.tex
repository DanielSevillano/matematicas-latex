\chapter{Introducción}

\section{Series complejas con índice en los enteros}
\begin{definition}
    Dada $\{z_n\}_{-\infty}^\infty \subset \mathbb{C}$, la serie $\sum_{-\infty}^\infty z_n$ converge si el límite
    $$\lim_{N \to \infty, M \to \infty} \sum_{n=-M}^N z_n$$
    existe en $\mathbb{C}$.

    Aquí, $N \to \infty$ y $M \to \infty$ de forma independiente.
    Es decir, existe $z^\ast \in \mathbb{C}$ tal que para todo $\varepsilon > 0$ existen dos naturales $N_0, M_0$ tales que si $N \geq N_0$, $M \geq M_0$, entonces
    $$\left|\sum_{n=-M}^N z_n-z^\ast\right| < \varepsilon.$$
\end{definition}

\begin{definition}
    Dada $\{z_n\}_{-\infty}^\infty \subset \mathbb{C}$, diremos que $\sum_{-\infty}^\infty z_n$ converge absolutamente si $\sum_{-\infty}^\infty |z_n|$ converge.
\end{definition}

\begin{remark}
    Dada $\{z_n\}_{-\infty}^\infty \subset \mathbb{C}$, si $\sum_{-\infty}^\infty z_n$ converge a $z^\ast \in \mathbb{C}$ entonces
    $$\lim_{N \to \infty} \sum_{n=-N}^N z_n = z^\ast.$$
    El recíproco no es cierto.
    Por ejemplo, si $z_n = n$ para todo $n$, se tiene que $\sum_{n=-N}^N z_n = 0$ para todo $N$, pero se puede comprobar que $\sum_{-\infty}^\infty z_n$ no converge.
\end{remark}

Veamos ahora que una serie de este tipo se puede poner como suma de dos series de las que ya conocíamos.

\begin{proposition}
    Dada $\{z_n\}_{-\infty}^\infty \subset \mathbb{C}$, se tiene que $\sum_{-\infty}^\infty z_n$ converge si y solo si $\sum_{-\infty}^{-1} z_n$ y $\sum_{n=0}^\infty z_n$ convergen.
    En ese caso,
    $$\sum_{-\infty}^\infty z_n = \sum_{-\infty}^{-1} z_n + \sum_{n=0}^\infty z_n.$$
\end{proposition}

\begin{remark}
    La serie $\sum_{-\infty}^{-1} z_n$ es de las que ya conocíamos, pues $\sum_{-\infty}^{-1} z_n = \sum_{n=1}^\infty z_{-n}$.
\end{remark}

% Demostración

De forma más general se tiene lo siguiente.

\begin{proposition}
    Dada $\{z_n\}_{-\infty}^\infty \subset \mathbb{C}$ y $n_0 \in \mathbb{Z}$, se tiene que $\sum_{-\infty}^\infty z_n$ converge si y solo si $\sum_{-\infty}^{n_0} z_n$ y $\sum_{n=n_0+1}^\infty z_n$ convergen.
    En ese caso,
    $$\sum_{-\infty}^\infty z_n = \sum_{-\infty}^{n_0} z_n + \sum_{n=n_0+1}^\infty z_n.$$
\end{proposition}

\begin{proposition}
    Dada $\{z_n\}_{-\infty}^\infty \subset \mathbb{C}$, si $\sum_{-\infty}^\infty z_n$ converge absolutamente entonces $\sum_{-\infty}^\infty z_n$ converge.
\end{proposition}

% Demostración

También existe un resultado más general para la convergencia absoluta.
\begin{proposition}
    Dada $\{z_n\}_{-\infty}^\infty \subset \mathbb{C}$ y $n_0 \in \mathbb{Z}$, se tiene que $\sum_{-\infty}^\infty z_n$ converge absolutamente si y solo si $\sum_{-\infty}^{n_0} z_n$ y $\sum_{n=n_0+1}^\infty z_n$ convergen absolutamente.
    En ese caso,
    $$\sum_{-\infty}^\infty z_n = \sum_{-\infty}^{n_0} z_n + \sum_{n=n_0+1}^\infty z_n.$$
\end{proposition}

\section{Singularidades aisladas}
\begin{definition}
    Si $a \in \mathbb{C}$ y $0 \leq R_1 < R_2 \leq \infty$, se define la corona de centro $a$ y radios $R_1$ y $R_2$ como:
    $$A(a, R_1, R_2) = \{z \in \mathbb{C} : R_1 < |z-a| < R_2\}.$$
\end{definition}

\begin{theorem}
    Si $a \in \mathbb{C}$, $0 \leq R_1 < R_2 \leq \infty$ y $f$ es holomorfa en $A(a, R_1, R_2)$, entonces existe una única sucesión $\{a_n\}_{-\infty}^\infty$ en $\mathbb{C}$ tal que:
    \begin{itemize}
        \item $\sum_{-\infty}^\infty a_n(z-a)^n$ converge para todo $z \in A(a, R_1, R_2)$.
        \item $f(z) = \sum_{-\infty}^\infty a_n(z-a)^n$ para todo $z \in A(a, R_1, R_2)$.
    \end{itemize}

    Para cada $n \in \mathbb{Z}$,
    $$a_n = \frac{1}{2\pi i} \int_\gamma \frac{f(z)}{(z-a)^{n+1}}dz,$$
    siendo $\gamma$ cualquiera camino que esté en $A(a, R_1, R_2)$ con $n(\gamma, a) = 1$

    Además, la serie $\sum_{-\infty}^\infty a_n(z-a)^n$ converge absoluta y uniformemente a cada subconjunto compacto de $A(a, R_1, R_2)$.

    A esta serie se le llama desarrollo de Laurent de $f$ en $A(a, R_1, R_2)$.
\end{theorem}

\begin{definition}
    $f$ tiene una singularidad aislada en $a \in \mathbb{C}$ si existe $R > 0$ tal que $f$ está definida y es holomorfa en $D(a, R) \setminus \{a\} = A(a, 0, R)$.
\end{definition}

Podemos considerar el desarrollo de Laurent de $f$ en $D(a, R) \setminus \{a\}$.
Existe una única sucesión en $\mathbb{C}$, $\{a_n\}_{-\infty}^\infty$, tal que:
$$f(z) = \sum_{-\infty}^\infty a_n(z-a)^n, \quad z \in D(a, R) \setminus \{a\}.$$

Como la sucesión $\{a_n\}_{-\infty}^\infty$ no depende de $R$, a este desarrollo se le puede llamar desarrollo de Laurent de $f$ en $a$ o en un entorno perforado de $a$.

\begin{proposition}
    Sea $f$ una función con una singularidad aislada en $a \in \mathbb{C}$ y sea $\sum_{-\infty}^\infty a_n(z-a)^n$ el desarrollo de Laurent de $f$ en $a$.
    Entonces:
    \begin{enumerate}
        \item $a$ es una singularidad evitable de $f$ $\Leftrightarrow$ $a_n = 0$ si $n < 0$ $\Leftrightarrow$ $\{n < 0 : a_n \neq 0\} = \emptyset$.
        \item $a$ es un polo de orden $N$ de $f$ $\Leftrightarrow$ $a_{-N} \neq 0$ y $a_n = 0$ si $n < -N$.
              Luego $a$ es un polo de $f$ $\Leftrightarrow$ $\{n < 0 : a_n \neq 0\}$ es finito y no vacío.
        \item $a$ es una singularidad esencial de $f$ $\Leftrightarrow$ $\{n < 0 : a_n \neq 0\}$ es infinito.
    \end{enumerate}
\end{proposition}

\begin{definition}
    $f$ tiene una singularidad aislada en $\infty$ si existe $R > 0$ tal que $f$ es holomorfa en $\{z \in \mathbb{C} : |z| > R\}$.
    \begin{enumerate}
        \item Es una singularidad evitable de $f$ si $\lim\limits_{z \to \infty} f(z)$ existe en $\mathbb{C}$.
        \item Es un polo de $f$ si $\lim\limits_{z \to \infty} f(z) = \infty$.
        \item Es una singularidad esencial en otro caso.
    \end{enumerate}
\end{definition}

Si $f$ tiene una singularidad aislada en $\infty$, entonces $f$ es holomorfa en $\{z \in \mathbb{C} : |z| > R\}$ para un cierto $R > 0$.
Entonces la función $g(z) = f\left(\frac{1}{z}\right)$ es holomorfa en $D\left(0, \frac{1}{R}\right) \setminus \{0\}$, por lo que tiene una singularidad aislada en 0.

Entonces:
\begin{enumerate}
    \item $f$ tiene una singularidad evitable en $\infty$ $\Leftrightarrow$ $g$ tiene una singularidad evitable en 0.
    \item $f$ tiene un polo en $\infty$ $\Leftrightarrow$ $g$ tiene un polo en 0.
    \item $f$ tiene una singularidad esencial en $\infty$ $\Leftrightarrow$ $g$ tiene una singularidad esencial en 0.
\end{enumerate}

\begin{proposition}
    Sea $f$ una función con una singularidad aislada en $\infty$.
    Entonces:
    \begin{enumerate}
        \item $\infty$ es una singularidad evitable de $f$ $\Leftrightarrow$ $f$ está acotada en un entorno perforado de $\infty$.
              Es decir, si existe $R > 0$ tal que $f$ es holomorfa y está acotada en $\{z \in \mathbb{C} : |z| > R\}$.
        \item $\infty$ es un polo de $f$ $\Leftrightarrow$ existe $N \in \mathbb{N}$ tal que $\lim\limits_{z \to \infty} \frac{f(z)}{z^N}$ existe en $\mathbb{C}$ y es distinto de 0.
              En este caso, $N$ es único y se denomina el orden de $\infty$ como polo de $f$.
        \item $\infty$ es una singularidad esencial de $f$ $\Leftrightarrow$ $f(\{z \in \mathbb{C} : |z| > R\})$ es denso en $\mathbb{C}$ para todo $R > 0$ suficientemente grande.
    \end{enumerate}
\end{proposition}

\begin{remark}
    En (2), el orden de $\infty$ como polo de $f$ coincide con el orden de 0 como polo de $f\left(\frac{1}{z}\right)$.
\end{remark}

Si $f$ tiene una singularidad aislada en $\infty$, entonces existe $R > 0$ tal que $f$ es holomorfa en $\{z \in \mathbb{C} : |z| > R\} = A(0, R, \infty)$.
Podemos considerar el desarrollo de Laurent de $f$ en $A(0, R, \infty)$: existe una única sucesión $\{a_n\}_{-\infty}^\infty$ en $\mathbb{C}$ tal que:
$$f(z) = \sum_{-\infty}^\infty a_nz^n, \quad \text{para todo } z \in \mathbb{C} \text{ con } |z| > R.$$

Como no depende de $R$, se le puede llamar desarrollo de Laurent de $f$ en $\infty$.

\begin{proposition}
    Sea $f$ una función con una singularidad aislada en $\infty$ y sea $\sum_{-\infty}^\infty a_nz^n$ el desarrollo de Laurent de $f$ en $\infty$.
    Entonces:
    \begin{enumerate}
        \item $\infty$ es una singularidad evitable de $f$ $\Leftrightarrow$ $a_n = 0$ si $n > 0$.
        \item $\infty$ es un polo de $f$ de orden $N$ $\Leftrightarrow$ $a_N \neq 0$ y $a_n = 0$ si $n > N$.
        \item $\infty$ es una singularidad esencial de $f$ $\Leftrightarrow$ $\{n > 0 : a_n \neq 0\}$ es infinito.
    \end{enumerate}
\end{proposition}

\begin{definition}
    Si $f$ tiene una singularidad aislada en $a \in \mathbb{C}$ y $\sum_{-\infty}^\infty a_n(z-a)^n$ es el desarrollo de Laurent de $f$ en $a$, se define $\Res(f, a) = a_{-1}$.
\end{definition}

\begin{proposition}
    Sea $a \in \mathbb{C}$ y $f$ una función con una singularidad aislada en $a$.
    Sea $R > 0$ tal que $f$ es holomorfa en $D(a, R) \setminus \{a\}$.
    Entonces, para todo $r \in (0, R)$, se tiene que:
    $$\Res(f, a) = \frac{1}{2\pi i} \int_{|z-a| = r} f(z)dz.$$
\end{proposition}

\begin{proposition}
    Sea $f$ una función con una singularidad aislada en $\infty$.
    Sea $R > 0$ tal que $f$ es holomorfa en $\{z \in \mathbb{C} : |z| > R\}$.
    Se define:
    $$\Res(f, \infty) = \frac{-1}{2\pi i} \int_{|z| = r} f(z)dz, \quad \text{siendo } r > R$$.
\end{proposition}

\begin{proposition}
    Si $f$ tiene una singularidad aislada en $\infty$ y $\sum_{-\infty}^\infty a_nz^n$ es el desarrollo de Laurent de $f$ en $\infty$, entonces $\Res(f, \infty) = -a_{-1}$.
\end{proposition}

\begin{theorem}[Teorema de los residuos]
    Sea $D$ un dominio en $\mathbb{C}$ y sea $f$ holomorfa en $D$ salvo por singularidades aisladas, es decir, existe $A \subset D$, $A$ sin puntos de acumulación en $D$, tal que $f$ es holomorfa en $D \setminus A$.
    Sea $\gamma$ un camino cerrado en $D \setminus A$, con $n(\gamma, z) = 0$ para todo $z \in \mathbb{C} \setminus D$.
    Entonces:
    $$\frac{1}{2\pi i} \int_\gamma f(z)dz = \sum_{a \in A} \Res(f, a)n(\gamma, a).$$
\end{theorem}

\begin{theorem}[Teorema de la función inversa]
    Sea $D$ un dominio en $\mathbb{C}$ y sea $f$ holomorfa en $D$, con $a \in D$ tal que $f'(a) \neq 0$.
    Entonces existen $U$ y $V$ abiertos en $\mathbb{C}$ con $a \in U \subset D$, $f(a) \in V$, tales que:
    \begin{enumerate}
        \item $f$ es inyectiva en $U$.
        \item $f(U) = V$.
        \item $f'(z) \neq 0$ para todo $z \in U$.
        \item $f^{-1}: V \to U$ es holomorfa y además:
              $$(f^{-1})'(f(z)) = \frac{1}{f'(z)}, \quad \forall z \in U.$$
    \end{enumerate}
\end{theorem}

\begin{theorem}
    Sea $D$ un dominio en $\mathbb{C}$ y sean $f$ holomorfa en $D$ no constante y $a \in D$.
    Sea $n$ el orden de $a$ como cero de $f-f(a)$, es decir, el primer natural para el que $f^{(n)}(a) \neq 0$.
    Entonces $f$ es localmente una aplicación $n \to 1$ cerca de $a$.
    Es decir, existe $\alpha > 0$ con $D(a, \alpha) \subset D$ tal que para todo $0 < \varepsilon < \alpha$ existe $\delta > 0$ tal que cada punto $w \in D(f(a), \delta) \setminus \{f(a)\}$ es la imagen de exactamente $n$ puntos distintos $z_1, z_2, \dots z_n \in D(a, \varepsilon) \setminus \{a\}$.
    En particular, $f(D(a, \varepsilon)) \supset D(f(a), \delta)$.
\end{theorem}

\begin{definition}
    Sea $D$ abierto en $\mathbb{C}$ y sea $f$ holomorfa en $D$ salvo por polos.
    Si $a \in D$ es un polo de $f$, se tiene que $\lim\limits_{z \to a} f(z) = \infty$.
    Definimos $f(a) = \infty$.
    Entonces $f: D \to \mathbb{C}^\ast$ y es continua.
    Se dice que $f$ es meromorfa en $D$.
\end{definition}

\begin{theorem}
    Sea $D$ un dominio en $\mathbb{C}$ y sea $f$ meromorfa en $D$, con $a \in D$ un polo de orden $n$ de $f$.
    Entonces $f$ es localmente una aplicación $n \to 1$ cerca de $a$.
    Es decir, existe $\alpha > 0$ tal que $D(a, \alpha) \subset D$, $f$ es holomorfa en $D(a, \alpha) \setminus \{a\}$ y se verifica que para todo $0 < \varepsilon < \alpha$ existe $R > 0$ tal que cada punto $w \in \mathbb{C}$ con $|w| > R$ es la imagen de exactamente $n$ puntos distintos $z_1, z_2, \dots z_n \in D(a, \varepsilon) \setminus \{a\}$.
    En particular, $f(D(a, \varepsilon) \setminus \{a\}) \supset \{w \in \mathbb{C} : |w| > R\}$.
\end{theorem}


\begin{theorem}
    Sea $f$ una función con un polo de orden $n$ en $\infty$.
    Entonces $f$ es localmente una aplicación $n \to 1$ cerca de $\infty$.
    Es decir, existe $R_0 > 0$ tal que $f$ es holomorfa en $\{z \in \mathbb{C} : |z| > R_0\}$ y se verifica que para todo $R > R_0$ existe $R' > 0$ tal que cada punto $w \in \mathbb{C}$ con $|w| > R'$ es la imagen de exactamente $n$ puntos distintos $z_1, \dots, z_n$ de $\{z \in \mathbb{C} : |z| > R\}$.
    En particular, $f(\{z \in \mathbb{C} : |z| > R\}) \supset \{w \in \mathbb{C} : |w| > R\}$.
\end{theorem}

\begin{theorem}[Teorema de la aplicación abierta]
    Sea $D$ un dominio en $\mathbb{C}$ y sea $f: D \to \mathbb{C}$ holomorfa y no constante.
    Entonces $f$ es una aplicación abierta.
    En particular, $f(D)$ es un dominio.
\end{theorem}

\begin{lemma}
    Sea $D$ un dominio en $\mathbb{C}$ y sea $f$ holomorfa en $D$.
    \begin{itemize}
        \item Sea $a \in D$.
              Entonces $f'(a) \neq 0$ si y solo si $f$ es inyectiva en un entorno de $a$.
        \item Si $f$ es inyectiva en $D$, entonces $f'(z) \neq 0$ para todo $z \in D$.
    \end{itemize}
\end{lemma}

\section{Aplicaciones conformes}
\begin{definition}
    Sea $D$ un dominio en $\mathbb{C}$ y sea $f: D \to \mathbb{C}$ holomorfa e inyectiva.
    Sea $D' = f(D)$.
    Entonces:
    \begin{itemize}
        \item $D'$ es un dominio en $D$.
        \item $f: D \to D'$ es biyectiva.
        \item $f^{-1}: D' \to D$ es holomorfa.
    \end{itemize}

    En ese caso decimos que $f$ es una aplicación conforme de $D$ sobre $D'$.
\end{definition}

\begin{remark}
    \hfill
    \begin{enumerate}
        \item Si $f$ es una aplicación conforme de $D$ sobre $D'$, entonces $f^{-1}$ es una aplicación conforme de $D'$ sobre $D$.
        \item Si $D_1$, $D_2$ y $D_3$ son dominios en $\mathbb{C}$ con $f$ aplicación conforme de $D_1$ sobre $D_2$ y $g$ aplicación conforme de $D_2$ sobre $D_3$, entonces $g \circ f$ es una aplicación cnforme de $D_1$ sobre $D_3$.
    \end{enumerate}
\end{remark}

\begin{definition}
    Si $D_1$ y $D_2$ son dominios en $\mathbb{C}$, se dice que $D_1$ y $D_2$ son conformemente equivalentes si existe una aplicación conforme $f$ de $D_1$ sobre $D_2$.

    En el conjunto de los dominios en $\mathbb{C}$, se tiene la relación de equivalencia "ser conformemente equivalentes".
\end{definition}

\begin{definition}
    Sea $D$ un dominio en $\mathbb{C}$.
    $D$ es simplemente conexo si $\mathbb{C}^\ast \setminus D$ es conexo.
    Equivalentemente, $D$ es simplemente conexo si todo camino cerrado $\gamma$ en $D$ es homólogo a cero módulo $D$, es decir, $n(\gamma, z) = 0$ para todo $z \in \mathbb{C} \setminus D$.
\end{definition}

\begin{theorem}
    Sean $D_1$ y $D_2$ dos dominios en $\mathbb{C}$ que son conformemente equivalentes.
    Entonces $D_1$ es simplemente conexo si y solo si $D_2$ es simplemente conexo.
\end{theorem}

\begin{definition}
    Si $z_1, z_2 \in \mathbb{C} \setminus \{0\}$, el ángulo formado por $z_1$ y $z_2$ se define como:
    $$\theta(z_1, z_2) = \arg \frac{z_2}{z_1} \in (-\pi, \pi].$$
\end{definition}

\begin{remark}
    Si $z_1, z_2 \in \mathbb{C} \setminus \{0\}$ y $\lambda_1, \lambda_2 > 0$, entonces $\theta(\lambda_1z_1, \lambda_2z_2) = \theta(z_1, z_2)$.
\end{remark}

\begin{definition}
    Sea $\gamma$ un camino con origen en un punto $a \in \mathbb{C}$.
    Se dice que $\gamma$ es regular en $a$ si existe una parametrización $\mathcal{C}^1$ a trozos de $\gamma$, $\gamma: [0, 1] \to \mathbb{C}$, tal que $\gamma'(0) \neq 0$.
\end{definition}

\begin{definition}
    Sean $\gamma_1$ y $\gamma_2$ dos caminos con origen $a \in \mathbb{C}$ que son regulares en $a$.
    El ángulo que forman $\gamma_1$ y $\gamma_2$ en $a$, $\theta_a(\gamma_1, \gamma_2)$, se define como sigue.

    Sean $\gamma_1, \gamma_2: [0, 1] \to \mathbb{C}$ parametrizaciones $\mathcal{C}^1$ a trozos de $\gamma_1, \gamma_2$ respectivamente tales que $\gamma_1'(0), \gamma_2'(0) \neq 0$.
    Entonces $\theta_a(\gamma_1, \gamma_2) = \theta(\gamma_1'(0), \gamma_2'(0))$.
\end{definition}

\begin{definition}
    Si $\gamma$ es una curva en $\mathbb{C}$ y $f: \sop(\gamma) \to \mathbb{C}$ es continua, se define la curva imagen de $\gamma$ por $f$ como la curva $\Gamma$ que tiene por parametrización $f \circ \gamma$, siendo $\gamma$ una parametrización de $\gamma$.
\end{definition}

\begin{definition}
    Sea $D$ un dominio en $\mathbb{C}$ y sean $f$ holomorfa en $D$ y $a \in D$.
    Diremos que $f$ preserva ángulos en $a$ o que $f$ es conforme en $a$ si se verifica lo siguiente.

    Si $\gamma_1$ y $\gamma_2$ son caminos con origen $a$, regulares en $a$, entonces las curvas imagen de $\Gamma_1$ y $\Gamma_2$ por $f$ de $\gamma_1$ y $\gamma_2$ respectivamente son caminos con oriden $f(a)$, que son regulares en $f(a)$ y se tiene que:
    $$\theta_{f(a)}(\Gamma_1, \Gamma_2) = \theta_a(\gamma_1, \gamma_2).$$
\end{definition}

\begin{theorem}
    Sea $D$ un dominio en $\mathbb{C}$ y sean $f$ holomorfa en $D$ y $a \in D$.
    Si $f'(a) \neq 0$, entonces $f$ es conforme en $a$.
\end{theorem}

\begin{proof}
    Sean $\gamma_1$ y $\gamma_2$ caminos en $D$, con origen en $a$ y regulares en $a$.
    Sean $\gamma_1, \gamma_2: [0, 1] \to \mathbb{C}$ parametrizaciones de $\gamma_1$ y $\gamma_2$ respectivamente, ambas $\mathcal{C}^1$ a trozos con $\gamma_1'(0), \gamma_2'(0) \neq 0$.
    Consideramos las curvas imagen de $\gamma_1$ y $\gamma_2$ por $f$:
    \begin{align*}
        \Gamma_1 & = f \circ \gamma_1: [0, 1] \to \mathbb{C}, \\
        \Gamma_2 & = f \circ \gamma_2: [0, 1] \to \mathbb{C}.
    \end{align*}
    $\Gamma_1$ y $\Gamma_2$ son $\mathcal{C}^1$ a trozos.
    Además, $\Gamma_1$ y $\Gamma_2$ son caminos con origen $f(a)$, porque:
    $$\Gamma_1(0) = f(\gamma_1(0)) = f(a) = f(\gamma_2(0)) = \Gamma_2(0)$$
    Observamos que $\Gamma_1$ y $\Gamma_2$ son regulares en $a$:
    \begin{align*}
        \Gamma_1'(0) & = f'(\gamma_1(0))\gamma_1'(0) = f'(a)\gamma_1(0) \neq 0, \\
        \Gamma_2'(0) & = f'(\gamma_2(0))\gamma_2'(0) = f'(a)\gamma_2(0) \neq 0.
    \end{align*}
    Por tanto:
    $$\theta_{f(a)}(\Gamma_1, \Gamma_2) = \theta(\Gamma_1'(0), \Gamma_2'(0)) = \arg \frac{\Gamma_2'(0)}{\Gamma_1'(0)} = \theta(\gamma_1'(0), \gamma_2'(0)) = \theta_a(\gamma_1, \gamma_2).$$
\end{proof}

\begin{example}[Contraejemplo]
    Sean $D = \mathbb{C}$, $f(z) = z^2$ y $a = 0$.
    Observamos que $f'(a) = 0$.
    Sea $\gamma_1$ el segmento $[0, 1]$ y $\gamma_2$ el segmento $[0, i]$.
    Es claro que $\theta_0(\gamma_1, \gamma_2) = \frac{\pi}{2}$.
    Si consideramos las curvas imagen de $\gamma_1$ y $\gamma_2$ por $f$, $\Gamma_1$ y $\Gamma_2$, podemos ver que $\Gamma_1$ es el segmento $[0, 1]$ y $\Gamma_2$ el segmento $[0, -1]$, que tienen $\theta_0(\Gamma_1, \Gamma_2) = \pi \neq \frac{\pi}{2}$.
\end{example}

De hecho, se tiene la equivalencia.
Sea $D$ un dominio en $\mathbb{C}$ y sean $f$ holomorfa en $D$ y $a \in D$.
Entonces $f'(a) \neq 0$ $\Leftrightarrow$ $f$ es conforme en $a$.

\section{Transformaciones de Möbius}
\begin{definition}
    Una transformación de Möbius es una aplicación $T: \mathbb{C}^\ast \to \mathbb{C}^\ast$ de la forma
    $$T(z) = \frac{\alpha z + \beta}{\gamma z + \delta}, \quad \text{con } \alpha, \beta, \gamma, \delta \in \mathbb{C}, \; \alpha\delta - \beta\gamma \neq 0.$$
\end{definition}

\begin{remark}
    Las aplicaciones conformes de $\mathbb{C}^\ast$ sobre $\mathbb{C}^\ast$ son las transformaciones de Möbius.
\end{remark}

Sea $\mathcal{M}$ el conjunto de las transformaciones de Möbius.
A cada $T \in \mathcal{M}$ se le hace corresponder la matriz de los coeficientes:
$$T(z) = \frac{\alpha z + \beta}{\gamma z + \delta} \quad \to \quad A = \begin{pmatrix}
        \alpha & \beta  \\
        \gamma & \delta
    \end{pmatrix}.$$
Esta matriz no es única.
Por tanto, no tenemos una aplicación, sino una correspondencia.
Se tiene que
\begin{itemize}
    \item Si $T_1, T_2 \in \mathcal{M}$ con matrices $A_1$ y $A_2$ respectivamente, entonces $T_2 \circ T_1 \in \mathcal{M}$ y le corresponde la matriz producto $A_2A_1$.
    \item A la transformación de Möbius identidad, $T(z) = z$, le corresponde la matriz identidad.
    \item Si $T \in \mathcal{M}$ con matriz $A$, entonces $T^{-1} \in \mathcal{M}$ y le corresponde la matriz inversa $A^{-1}$.
          También le corresponde la matriz indicada a continuación:
          $$A = \begin{pmatrix}
                  \alpha & \beta  \\
                  \gamma & \delta
              \end{pmatrix}, \quad A^{-1} = \frac{1}{\alpha\delta - \beta\gamma}\begin{pmatrix}
                  \delta  & -\beta \\
                  -\gamma & \alpha
              \end{pmatrix} \leftrightarrow \begin{pmatrix}
                  \delta  & -\beta \\
                  -\gamma & \alpha
              \end{pmatrix}.$$
          Es decir, no es necesario dividir entre el determinante.
\end{itemize}

Por tanto, $\mathcal{M}$ es un grupo con la composición.
El grupo $(\mathcal{M}, \circ)$ está generado por las transformaciones de Möbius siguientes:
\begin{itemize}
    \item Traslaciones:
          $$T(z) = z+a, \quad a \in \mathbb{C}.$$
    \item Rotaciones:
          $$T(z) = e^{i\theta}z, \quad \theta \in \mathbb{R}.$$
    \item Dilataciones ($r > 1$), contracciones ($r < 1$) e identidad ($r = 1$):
          $$T(z) = rz, \quad r > 0.$$
    \item Inversión con respecto a la circunferencia unidad:
          $$T(z) = \frac{1}{z}.$$
          Observemos que $T(0) = \infty$ y $T(\infty) = 0$.
          Además, si $r > 0$ y $\theta \in \mathbb{R}$ se tiene que
          $$T(re^{i\theta}) = \frac{1}{re^{i\theta}} = \frac{1}{r}e^{-i\theta},$$
          con lo que podemos ver en qué se transforman las circunferencias de centro 0 y las semirrectas con origen 0.
\end{itemize}

El hecho de que el grupo $\mathcal{M}$ esté generado por estas transformaciones de Möbius quiere decir que cualquier transformación de Möbius se puede expresar como composición de un número finito de transformaciones de Möbius de los tipos anteriores.

\begin{lemma}
    Si $T$ es una transformación de Möbius que fija tres puntos, entonces $T$ es la identidad.
\end{lemma}

\begin{proposition}
    Si $z_1$, $z_2$ y $z_3$ son tres puntos distintos de $\mathbb{C}^\ast$ y $w_1$, $w_2$ y $w_3$ son otros tres puntos distintos de $\mathbb{C}^\ast$, entonces existe una única transformación de Möbius $T$ con $T(z_j) = w_j$, para $j = 1, 2, 3$.
\end{proposition}

\begin{lemma}
    La ecuación general de una circunferencia o recta en $\mathbb{C}$ viene dada por
    $$\alpha|z|^2 + \beta z + \overline{\beta}\overline{z} + \gamma = 0, \quad \beta \in \mathbb{C}, \; \alpha, \gamma \in \mathbb{R}, \; |\beta|^2 > \alpha\gamma.$$
    El caso $\alpha = 0$ corresponde a una recta y el caso $\alpha \neq 0$ a una circunferencia.
\end{lemma}

\begin{remark}
    Cuando se condiera una recta $R$ en $\mathbb{C}^\ast$ se supone que $\infty \in R$.
    Es decir, $R = R' \cup \{\infty\}$, siendo $R'$ una recta en $\mathbb{C}$.
    Por tanto, $R$ será de la forma
    $$R = \{z \in \mathbb{C} : \beta z + \overline{\beta}\overline{z} + \gamma = 0\} \cup \{\infty\}, \quad \beta \in \mathbb{C}, \; \beta \neq 0, \; \gamma \in \mathbb{R}.$$
\end{remark}

En el contexto de las transformaciones de Möbius se consideran las rectas en $\mathbb{C}^\ast$.

\begin{proposition}
    Sea $T$ una transformación de Möbius.
    Si $R$ es una recta, entonces $T(R)$ es una circunferencia o una recta.
    Si $C$ es una circunferencia, entonces $T(C)$ es una circunferencia o una recta.
\end{proposition}

\begin{remark}
    En $\mathbb{C}^\ast$, tres puntos distintos determinan una circunferencia o una recta.
    Es decir, dados $z_1$, $z_2$ y $z_3$ tres puntos distintos de $\mathbb{C}^\ast$, existe una única circunferencia o recta $C$ con $z_1, z_2, z_3 \in C$.
\end{remark}

\begin{remark}
    Si $C_1$ y $C_2$ son circunferencias o rectas, entonces existe $T \in \mathcal{M}$ con $T(C_1) = C_2$.
\end{remark}

\begin{remark}
    Si $T \in \mathcal{M}$, entonces $T$ preserva ángulos en $a$ para todo $a \in \mathbb{C}$ tal que $a$ no es un polo de $T$.
\end{remark}

Sea $T$ una transformación de Möbius, sea $C_1$ una circunferencia o recta y sea $C_2 = T(C_1)$.
Entonces $C_2$ es una circunferencia o recta.
Sean $A_1$ y $B_1$ las dos componentes conexas de $\mathbb{C}^\ast \setminus C_1$ y sean $A_2$ y $B_2$ las dos componentes conexas de $\mathbb{C}^\ast \setminus C_2$.
Entonces $T(A_1) = A_2$ y $T(B_1) = B_2$, o bien $T(A_1) = B_2$ y $T(B_1) = A_2$.
Ahora bien, si fijamos una orientación en $C_1$, esta determina mediante $T$ una orientación en $C_2$.
Enrtonces, respecto a esas orientaciones, la componente de $\mathbb{C}^\ast \setminus C_1$ que queda a la derecha de $C_1$ se transforma en la componente de $\mathbb{C}^\ast \setminus C_2$ que queda a la derecha de $C_2$, y la que queda a la izquierda se transforma en la que queda a la izquierda.

\begin{remark}
    Dados una pareja de disco abierto, exterior de disco y semiplano, existe $T \in \mathcal{M}$ que aplica el uno en el otro.
\end{remark}

\begin{example}
    La transformación de Möbius
    $$P(z) = \frac{1+z}{1-z}$$
    aplica $\partial\mathbb{D}$ en el eje imaginario y $\mathbb{D}$ en el semiplano de la derecha.
\end{example}

\begin{proposition}
    Sea $T \in \mathcal{M}$.
    Entonces $T(\mathbb{R} \cup \{\infty\}) = \mathbb{R} \cup \{\infty\}$ si y solo si existen $\alpha, \beta, \gamma, \delta \in \mathbb{R}$ con $\alpha\delta - \beta\gamma \neq 0$ tales que
    $$T(z) = \frac{\alpha z + \beta}{\gamma z + \delta}.$$
\end{proposition}

\begin{lemma}
    Sea $T \in \mathcal{M}$ con $T(\mathbb{R} \cup \{\infty\}) = \mathbb{R} \cup \{\infty\}$.
    Entonces $T(\overline{z}) = \overline{T(z)}$, para todo $z \in \mathbb{C}^\ast$.
\end{lemma}

\begin{remark}
    Se considera $\overline{\infty} = \infty$.
\end{remark}

\begin{remark}
    Sea $T \in \mathcal{M}$ con $T(\mathbb{R} \cup \{\infty\}) = \mathbb{R} \cup \{\infty\}$.
    Si $A \subset \mathbb{C}^\ast$ y $A$ es simétrico respecto al eje real, entonces $T(A)$ también es simétrico con respecto al eje real.
\end{remark}

\begin{definition}
    Sea $C$ una circunferencia o una recta y sea $a \in \mathbb{C}^\ast$.
    Se define $a^\ast$ como el simétrico de $a$ con respecto a $C$ de la siguiente forma.
    Sea $T \in \mathcal{M}$ con $T(C) = \mathbb{R} \cup \{\infty\}$.
    Definimos
    $$a^\ast = T^{-1}(\overline{T(a)}).$$
\end{definition}

\begin{remark}
    \hfill
    \begin{enumerate}
        \item $a^\ast = a \Leftrightarrow a \in \mathbb{C}$.
        \item $(a^\ast)^\ast = a$.
        \item Si $C$ es una recta, el simétrico es el usual, siendo $\infty^\ast = \infty$.
        \item Sea $C$ una circunferencia o recta y sea $a \in \mathbb{C}^\ast$.
              Entonces, si $T \in \mathcal{M}$ con $T(C) = \mathbb{R} \cup \{\infty\}$, se tiene que $T(a^\ast) = \overline{T(a)}$.
    \end{enumerate}
\end{remark}

\begin{lemma}
    Sean $C_1$ y $C_2$ dos circunferencias o rectas y sea $S \in \mathcal{M}$ con $S(C_1) = C_2$.
    Entonces $S$ transforma puntos simétricos con respecto a $C_1$ en puntos simétricos con respecto a $C_2$.
\end{lemma}

\begin{lemma}
    Si $C = \partial\mathbb{D}$, se tiene que $0^\ast = \infty$, $\infty^\ast = 0$ y
    $$z^\ast = \frac{1}{\overline{z}}, \quad \text{si } z \in \mathbb{C}^\ast \setminus \{0, \infty\}.$$
\end{lemma}

\begin{remark}
    Si $z = re^{i\theta}$, entonces $z^\ast = \frac{1}{r}e^{i\theta}$.
\end{remark}

\begin{lemma}
    Sea $a \in \mathbb{C}$ y $R > 0$.
    Si $C = \{z \in \mathbb{C} : |z-a| = R\}$ se tiene que $a^\ast = \infty$, $\infty^\ast = a$ y
    $$z^\ast = a + \frac{R^2}{\overline{z-a}}, \quad \text{si } z \in \mathbb{C}^\ast \setminus \{a, \infty\}.$$
    De forma geométrica, si $z \in \mathbb{C}$, $z \neq a$, entonces $z^\ast$ está en la semirrecta con origen $a$ que pasa por $z$, siendo
    $$\frac{|z^\ast-a|}{R} = \frac{R}{|z-a|}.$$
    Por tanto, si $0 < |z-a| < R$ entonces $|z^\ast-a| > R$, y si $|z-a| > R$ entonces $|z^\ast-a| < R$.
\end{lemma}

\begin{remark}
    Si $z = a + re^{i\theta}$, entonces $z^\ast = a + \frac{R^2}{r}e^{i\theta}$.
\end{remark}

\begin{notation}
    Si $a \in \mathbb{C}$ con $|a| \neq 1$, consideramos las transformaciones de Möbius siguientes:
    $$S_a(z) = \frac{z-a}{1-\overline{a}z}, \quad T_a(z) = \frac{z+a}{1+\overline{a}z}, \quad \varphi_a(z) = \frac{a-z}{1-\overline{a}z}.$$
\end{notation}

Se tiene que:
\begin{itemize}
    \item $T_a = S_{-a}$.
    \item $\varphi_a = -S_a$.
    \item $S_a$ y $T_a$ son inversas la una de la otra.
    \item $\varphi_a \circ \varphi_a$ es la identidad.
\end{itemize}

\begin{proposition}
    El conjunto de las transformaciones de Möbius que aplican $\partial\mathbb{D}$ en $\partial\mathbb{D}$ es
    \begin{align*}
         & \{\lambda S_a : \lambda \in \mathbb{C}, |\lambda| = 1, a \in \mathbb{C}, |a| \neq 1\} \cup \left\{\frac{\lambda}{z} : \lambda \in \mathbb{C}, |\lambda| = 1\right\} =       \\
         & = \{\lambda T_a : \lambda \in \mathbb{C}, |\lambda| = 1, a \in \mathbb{C}, |a| \neq 1\} \cup \left\{\frac{\lambda}{z} : \lambda \in \mathbb{C}, |\lambda| = 1\right\} =     \\
         & = \{\lambda\varphi_a : \lambda \in \mathbb{C}, |\lambda| = 1, a \in \mathbb{C}, |a| \neq 1\} \cup \left\{\frac{\lambda}{z} : \lambda \in \mathbb{C}, |\lambda| = 1\right\}.
    \end{align*}
\end{proposition}

\begin{proposition}
    El conjunto de las transformaciones de Möbius que aplican $\mathbb{D}$ en $\mathbb{D}$ es
    $$\{\lambda T_a : \lambda \in \mathbb{C}, |\lambda| = 1, a \in \mathbb{D}\} = \{\lambda S_a : \lambda \in \mathbb{C}, |\lambda| = 1, a \in \mathbb{D}\} = \{\lambda \varphi_a : \lambda \in \mathbb{C}, |\lambda| = 1, a \in \mathbb{D}\}.$$
\end{proposition}