\chapter{Conformalidad y funciones abiertas en el disco unidad}
\section{Funciones meromorfas}

\begin{definition}
    Sea $D$ un abierto en $\mathbb{C}^\ast$.
    La función $f: D \to \mathbb{C}^\ast$ es meromorfa en $D$ si dado $a \in D$ se verifica una de las siguientes posibilidades:
    \begin{itemize}
        \item $a \in \mathbb{C}$ y $f$ es holomorfa en $a$.
        \item $a \in \mathbb{C}$ y $f$ tiene un polo en $a$, es decir, $f(a) = \infty$..
        \item $a = \infty$ y $f$ tiene una singularidad evitable en $\infty$, es decir, $\lim\limits_{z \to \infty} f(z) = f(\infty) \in \mathbb{C}$.
        \item $a = \infty$ y $f$ tiene un polo en $a$, es decir, $f(\infty) = \infty$.
    \end{itemize}

    Entonces $f: D \to \mathbb{C}^\ast$ es continua.
\end{definition}

\begin{remark}
    En el caso $D \subset \mathbb{C}$, la definición es la que ya conocíamos de función meromorfa.
    Si además $f(D) \subset \mathbb{C}$, se tiene una función holomorfa en $D$.
\end{remark}

\begin{remark}
    Sea $D$ abierto en $\mathbb{C}^\ast$ y sea $f: D \to \mathbb{C}^\ast$ continua.
    Supongamos que $f$ es holomorfa en $\{z \in D \cap \mathbb{C} : f(z) \in \mathbb{C}\}$ y que el conjunto $\{z \in D : f(z) = z\}$ no tiene puntos de acumulación en $D$.
    Entonces $f$ es meromorfa en $D$.
\end{remark}

\begin{remark}
    Sea $D$ abierto en $\mathbb{C}^\ast$ y sea $f: D \to \mathbb{C}^\ast$, $f$ meromorfa e inyectiva en $ A = \{z \in D \cap \mathbb{C} : f(z) \in \mathbb{C}\}$.
    Entonces $f$ tiene a lo sumo un polo y tal polo es simple.
    Además, $f'(a) \neq 0$ para todo $a \in A$, por lo que $f$ es conforme en $a$ para todo $a \in A$.
\end{remark}

\begin{theorem}[Teorema de la aplicación abierta]
    Sea $D$ un dominio en $\mathbb{C}^\ast$ y sea $f: D \to \mathbb{C}^\ast$ una función meromorfa y no constante en $D$.
    Entonces $f$ es una aplicación abierta.
    En particular, $f(D)$ es un dominio en $\mathbb{C}^\ast$.
\end{theorem}

Sea $D$ un dominio en $\mathbb{C}^\ast$ y sea $f: D \to \mathbb{C}^\ast$ meromorfa e inyectiva, con $D' = f(D)$.
Entonces:
\begin{enumerate}
    \item $D'$ es un dominio en $\mathbb{C}^\ast$.
    \item $f^{-1}: D' \to D$ es meromorfa e inyectiva.
\end{enumerate}

Veamos que (2) es cierto.
Como $f$ es una aplicación abierta, se tiene que $f^{-1}$ es continua.
Sea $w \in D' \cap \mathbb{C}$ tal que $z = f^{-1}(w) \in \mathbb{C}$, veamos que $f^{-1}$ es holomorfa en $w$.
Como $z \in \mathbb{C} \cap D$ y $f(z) \in \mathbb{C}$, $f$ es holomorfa en $z$ con $f'(z) \neq 0$.
Por el teorema de la función inversa, $f^{-1}$ es holomorfa en $w$.

\section{Aplicaciones conformes}
\begin{definition}
    Sea $D$ un dominio en $\mathbb{C}^\ast$ y sea $f: D \to \mathbb{C}^\ast$ meromorfa e inyectiva en $D$.
    Sea $D' = f(D)$.
    Entonces diremos que $f$ es una aplicación conforme de $D$ sobre $D'$.
\end{definition}

En este caso, se tiene que $D'$ es un dominio en $\mathbb{C}^\ast$ y que $f^{-1}: D' \to D$ es meromorfa e inyectiva en $D'$.
Por tanto, $f: D \to D'$ es un homeomorfismo, con $f$ y $f^{-1}$ meromorfas.

\begin{remark}
    \hfill
    \begin{enumerate}
        \item Si $f$ es una aplicación conforme de $D$ sobre $D'$, entonces $f^{-1}$ es una aplicación conforme de $D'$ sobre $D$.
        \item Si $D_1$, $D_2$ y $D_3$ son dominios en $\mathbb{C}^\ast$, con $f$ aplicación conforme de $D_1$ sobre $D_2$ y $g$ aplicación conforme de $D_2$ sobre $D_3$, entonces $g \circ f$ es una aplicación conforme de $D_1$ sobre $D_3$.
    \end{enumerate}
\end{remark}

Se puede comprobar que, sean $G_1, G_2$ abiertos en $\mathbb{C}^\ast$ y $f: G_1 \to \mathbb{C}, g: G_2 \to \mathbb{C}$ meroformas tal que $f(G_1) \subset G_2$, entonces $g \circ f: G_1 \to \mathbb{C}^\ast$ es meromorfa.

\begin{definition}
    Sean $D_1$ y $D_2$ dominios en $\mathbb{C}^\ast$.
    Diremos que $D_1$ y $D_2$ son conformemente equivalentes si existe una aplicación conforme $f$ de $D_1$ sobre $D_2$.

    En el conjunto de los dominios en $\mathbb{C}^\ast$, el ser conformemente equivalentes es una relación de equivalencia.
\end{definition}

\begin{definition}
    Sea $D$ un dominio en $\mathbb{C}^\ast$.
    Diremos que $D$ es simplemente conexo si $\mathbb{C}^\ast \setminus D$ es conexo.
\end{definition}

\begin{example}
    \hfill
    \begin{itemize}
        \item $D = \mathbb{C}$.
        \item $D = \mathbb{C}^\ast \setminus \{a\}$, $a \in \mathbb{C}$.
        \item $D = \mathbb{C}^\ast \setminus \overline{D}(a, R)$, $a \in \mathbb{C}$, $R > 0$.
        \item $D = D(a, R)$, $a \in \mathbb{C}$, $R > 0$.
        \item Un semiplano sin $\infty$.
        \item Un sector sin $\infty$.
        \item El plano menos dos semirrectas.
        \item $D = \mathbb{C} \setminus \{a\}$, $a \in \mathbb{C}$, no es simplemente conexo, porque $\mathbb{C}^\ast \setminus D = \{a, \infty\}$ no es conexo.
    \end{itemize}
\end{example}

\begin{lemma}
    Dado $a \in \mathbb{C}$, la transformación $T: \mathbb{C}^\ast \to \mathbb{C}^\ast$, $T(z) = \frac{1}{z-a}$ si $z \in \mathbb{C} \setminus \{a\}$, $T(a) = \infty$ y $T(\infty) = 0$, es una aplicación conforme de $\mathbb{C}^\ast$ sobre $\mathbb{C}^\ast$.
\end{lemma}

\begin{lemma}
    Sea $H$ un homeomorfismo de $\mathbb{C}^\ast$ sobre $\mathbb{C}^\ast$.
    Si $D$ es un dominio simplemente conexo en $\mathbb{C}^\ast$, entonces $H(D)$ es un dominio simplemente conexo en $\mathbb{C}^\ast$.
\end{lemma}

\begin{proof}
    Como $H: \mathbb{C}^\ast \to \mathbb{C}^\ast$ y $D$ es abierto y conexo en $\mathbb{C}^\ast$, entonces $H(D)$ es abierto y conexo en $\mathbb{C}^\ast$.
    Luego $H(D)$ es un dominio en $\mathbb{C}^\ast$.
    Como además $\mathbb{C}^\ast \setminus D$ es conexo, entonces $\mathbb{C}^\ast \setminus H(D) = H(\mathbb{C}^\ast \setminus D)$ es conexo.
    Por tanto, $H(D)$ es un dominio simplemente conexo.
\end{proof}

\begin{theorem}
    Sean $D_1$ y $D_2$ dos dominios en $\mathbb{C}^\ast$ que son conformemente equivalentes.
    Entonces $D_1$ es simplemente conexo si y solo si $D_2$ es simplemente conexo.
\end{theorem}

\begin{proof}
    Sea $F: D_1 \to D_2$ aplicación conforme.
    Consideramos todos los posibles casos teniendo en cuenta que los papeles de $D_1$ y $D_2$ son intercambiables.
    \begin{itemize}
        \item Si $D_1, D_2 \subset \mathbb{C}$, se cumple.
        \item Si $D_1 = \mathbb{C}^\ast$, como $\mathbb{C}^\ast$ es cerrado y $F$ es un homeomorfismo, se tiene que $D_2$ es compacto y por tanto cerrado.
              Entonces $D_2$ es abierto y cerrado en $\mathbb{C}^\ast$, que es conexo.
              Por tanto, $D_2 = \mathbb{C}^\ast = D_1$, ambos simplemente conexos.
        \item Si $D_1, D_2 \neq \mathbb{C}^\ast$, consideramos dos casos.
              \begin{itemize}
                  \item Supongamos que $\infty \notin D_1$ y $\infty \in D_2$.
                        $D$ es un dominio en $\mathbb{C}$ y $D_2$ es un dominio en $\mathbb{C}^\ast$.
                        Sea $a \in \mathbb{C}^\ast \setminus D_2$, de hecho $a \in \mathbb{C} \setminus D_2$.
                        Tomamos la aplicación conforme $T: \mathbb{C}^\ast \to \mathbb{C}^\ast$, $T(z) = \frac{1}{z-a}$ si $z \in \mathbb{C}^\ast \setminus \{a\}$, $T(a) = \infty$.
                        Tenemos el diagrama:
                        $$D_1 \xrightarrow{F} D_2 \xrightarrow{T} T(D_2)$$
                        $T(D_2)$ es un dominio en $\mathbb{C}^\ast$.
                        Como $a \notin D_2$, entonces $T(a) = \infty \notin T(D_2)$.
                        Así que $T(D_2)$ es un dominio en $\mathbb{C}$ conformemente equivalente a $D_1$.
                        Luego $D_1$ es simplemente conexo si y solo si $T(D_2)$ es simplemente conexo.
                        Por el lema anterior, esto es equivalente a que $D_2$ sea simplemente conexo.

                  \item Supongamos que $\infty \in D_1, D_2$.
                        Se sigue de un razonamiento similar usando el apartado anterior.
              \end{itemize}
    \end{itemize}
\end{proof}

\section{Dominios conformemente equivalentes al plano complejo y al plano complejo extendido}
Veremos que hay tres clases de equivalencia de dominios simplemente conexos en $\mathbb{C}^\ast$: $\mathbb{C}^\ast$, $\mathbb{C}$ y el disco unidad $\mathbb{D} = D(0, 1) = \{z \in \mathbb{C} : |z| < 1\}$.

$\mathbb{C}^\ast$ es compacto.
Si $D$ es un dominio en $\mathbb{C}^\ast$ que es conformemente equivalente a $\mathbb{C}^\ast$, entonces $D$ es compacto y por tanto cerrado.
Como $D$ es abierto, entonces $D = \mathbb{C}^\ast$.

$\mathbb{C}$ y $\mathbb{D}$ son homeomorfos.
Por ejemplo, $T: \mathbb{D} \to \mathbb{C}$, $T(z) = \frac{z}{1-|z|}$ es un homeomorfismo.

\begin{proposition}
    $\mathbb{C}$ y $\mathbb{D}$ no son conformemente equivalentes.
\end{proposition}

\begin{proof}
    Supongamos que existe una aplicación conforme $F$ de $\mathbb{C}$ sobre $\mathbb{D}$.
    Entonces $F: \mathbb{C} \to \mathbb{D}$ es entera y acotada.
    Por el teorema de Liouville, $F$ es constante.
    Esto contradice que $F$ sea una aplicación conforme.
\end{proof}

\begin{proposition}
    Sea $f$ entera e inyectiva, entonces $f$ es de la forma
    $$f(z) = \alpha z + \beta, \quad \alpha, \beta \in \mathbb{C}, \; \alpha \neq 0$$
\end{proposition}

\begin{proof}
    Sea $f(z) = \sum_{n=0}^\infty a_nz^n$, $z \in \mathbb{C}$, el desarrollo de Taylor de $f$ en 0.
    Entonces $\infty$ es una singularidad aislada de $f$ y el desarrollo anterior coincide con el desarrollo de Laurent de $f$ en $\infty$.
    \begin{itemize}
        \item Si $\infty$ es una singularidad evitable de $f$, entonces $a_n = 0$ si $n \geq 1$, así que $f$ es constante.
              Esto no es posible.
        \item Si $\infty$ es un polo de orden $N$ de $f$, entonces $a_N \neq 0$ y $a_n = 0$ si $n > N$.
              Luego $f$ es un polinomio de grado $N$.
              $f'$ es un polinomio de grado $N-1$, con $f'(z) \neq 0$ para todo $z \in \mathbb{C}$.
              Así que $f'$ es constante, por tanto $N - 1 = 0 \Rightarrow N = 1$.
        \item Si $\infty$ es una singularidad esencial de $f$, entonces $f(\{z \in \mathbb{C} : |z| > 1\})$ es denso en $\mathbb{C}$.
              Por el teorema de la aplicación abierta, $f(\{z \in \mathbb{C} : |z| < 1\})$ es abierto en $\mathbb{C}$.
              Estos conjuntos son disjuntos por ser $f$ inyectiva, y esto no es posible.
    \end{itemize}
\end{proof}

Si $D$ es un dominio en $\mathbb{C}$ que es conformemente equivalente a $\mathbb{C}$, entonces $D = \mathbb{C}$.
Veamos que esto es verdad.
Sea $f: \mathbb{C} \to D$ aplicación conforme.
$f$ es entera e inyectiva, así que $f(z) = \alpha z + \beta$, con $\alpha, \beta \in \mathbb{C}$, $\alpha \neq 0$.
Luego $D = f(\mathbb{C}) = \mathbb{C}$.

Las aplicaciones conformes de $\mathbb{C}$ sobre $\mathbb{C}$ son de la forma:
$$f(z) = \alpha z + \beta, \quad \alpha, \beta \in \mathbb{C}, \; \alpha \neq 0$$

Sea $D$ un dominio en $\mathbb{C}^\ast$ que es conformemente equivalente a $\mathbb{C}$.
\begin{itemize}
    \item Si $\infty \notin D$, entonces $D$ es un dominio en $\mathbb{C}$ conformemente equivalente a $\mathbb{C}$ y, por tanto, $D = \mathbb{C}$.
    \item Si $\infty \in D$, consideramos $F: \mathbb{C} \to D$ aplicación conforme.
          Como sabemos que $D \neq \mathbb{C}^\ast$, existe $\alpha \in \mathbb{C}^\ast \setminus D$, de hecho $\alpha \in \mathbb{C} \setminus D$.
          Sea
          $$T: D \to T(D), \quad T(z) = \frac{1}{z-\alpha}$$
          Tenemos el diagrama:
          $$\mathbb{C} \xrightarrow[]{F} D \xrightarrow[]{T} T(D) = D'$$
          $D'$ es un dominio en $\mathbb{C}$ conformemente equivalente a $\mathbb{C}$, así que $D' = \mathbb{C} = \mathbb{C}^\ast \setminus \{\infty\}$.
          Por tanto, $D = \mathbb{C}^\ast \setminus \{\alpha\}$.
\end{itemize}

Hemos probado que si $D$ es un dominio en $\mathbb{C}^\ast$ conformemente equivalente a $\mathbb{C}$, entonces $D = \mathbb{C}$ o $D = \mathbb{C}^\ast \setminus \{\alpha\}$, con $\alpha \in \mathbb{C}$.
Es decir, $D = \mathbb{C}^\ast \setminus \{\alpha\}$, $\alpha \in \mathbb{C}^\ast$.

Los dominios en $\mathbb{C}^\ast$ que son conformemente equivalentes a $\mathbb{C}$ son $\mathbb{C}^\ast \setminus \{\alpha\}$, $\alpha \in \mathbb{C}^\ast$.

\subsection*{Aplicaciones conformes de $\mathbb{C}^\ast$ sobre $\mathbb{C}^\ast$}
Sea $T: \mathbb{C}^\ast \to \mathbb{C}^\ast$ aplicación conforme.
Sea $a \in \mathbb{C}^\ast$ tal que $T(a) = \infty$.
Consideramos dos casos:
\begin{enumerate}
    \item Si $a = \infty$, $T(\infty) = \infty$.
          $T: \mathbb{C} \to \mathbb{C}$ es una aplicación conforme, así que $T(z) = \alpha z + \beta$, con $\alpha, \beta \in \mathbb{C}$, $\alpha \neq 0$.
    \item Si $a \in \mathbb{C}$, $T(a) = \infty$.
          $T$ es holomorfa en $\mathbb{C} \setminus \{a\}$, así que $a$ es un polo simple de $T$.
          Consideramos el desarrollo de Laurent de $T$ en $a$.
          $$T(z) = \frac{A_{-1}}{z-a} + \sum_{n=0}^\infty A_n(z-a)^n, \quad z \in \mathbb{C} \setminus \{a\}, \; A_{-1} \neq 0$$
          $\infty$ es una singularidad aislada de $T$.
          De hecho, es una singularidad evitable.

          Sea $F(z) = T(z) - \frac{A_{-1}}{z-a}$, $z \in \mathbb{C} \setminus \{a\}$.
          $F$ es holomorfa en $\mathbb{C} \setminus \{a\}$.
          $a$ es singularidad evitable de $F$ y $\lim\limits_{z \to \infty} F(z) = T(\infty) \in \mathbb{C}$, así que $\infty$ es una singularidad evitable también.
          Evitando la singularidad de $F$ en $a$, tenemos que $F$ es entera y acotada.
          Por tanto $F$ es constante.
          Así que $F(z) = a_0$, para todo $z \in \mathbb{C}$.
          Entonces:
          $$T(z) = F(z) + \frac{A_{-1}}{z-a} = a_0 + \frac{A_{-1}}{z-a} = \frac{a_0z + (A_{-1}-a_0a)}{z-a}$$
\end{enumerate}

En cualquiera de los dos casos, $T$ es de la forma:
$$T(z) = \frac{\alpha z + \beta}{\gamma z + \delta}, \quad \alpha, \beta, \gamma, \delta \in \mathbb{C}$$

No todas las aplicaciones de esta forma son aplicaciones conformes de $\mathbb{C}^\ast$ sobre $\mathbb{C}^\ast$.

\begin{example}[Contraejemplo]
    No es una aplicación conforme si $\alpha = \beta = 0$ o $(\alpha, \beta)$ y $(\gamma, \delta)$ son proporcionales.
    Por ejemplo:
    $$T(z) = \frac{3z + 2}{6z + 4} = \frac{1}{2}$$
\end{example}

Para que las aplicaciones de esa forma sean aplicaciones conformes de $\mathbb{C}^\ast$ sobre $\mathbb{C}^\ast$, se tiene que verificar que:
$$\begin{vmatrix}
        \alpha & \beta  \\
        \gamma & \delta
    \end{vmatrix} \neq 0$$

En el caso (1), $T(z) = Az + B = \frac{Az + B}{0z + 1}$, con $A, B \in \mathbb{C}$, $A \neq 0$, luego:
$$\begin{vmatrix}
        A & B \\
        0 & 1
    \end{vmatrix} = A \neq 0$$

En el caso (2),
$$\begin{vmatrix}
        a_0 & A_{-1} - a_0a \\
        1   & -a
    \end{vmatrix} = -a_0a - A_{-1} + a_0a = -A_{-1} \neq 0$$

\begin{theorem}
    Las aplicaciones conformes de $\mathbb{C}^\ast$ en $\mathbb{C}^\ast$ son de la forma:
    $$T(z) = \frac{\alpha z + \beta}{\gamma z + \delta}, \quad \alpha, \beta, \gamma, \delta \in \mathbb{C}, \; \alpha\delta - \beta\gamma \neq 0$$
\end{theorem}

\begin{proof}
    Sea $T: \mathbb{C}^\ast \to \mathbb{C}^\ast$ una aplicación de esa forma.
    \begin{itemize}
        \item Si $\gamma = 0$, entonces:
              $$T(z) = \frac{\alpha z + \beta}{\delta} = \frac{\alpha}{\delta}z + \frac{\beta}{\delta}, \quad \alpha, \delta \neq 0$$

              $T$ es una aplicación conforme de $\mathbb{C}$ sobre $\mathbb{C}$, con $\lim\limits_{z \to \infty} T(z) = \infty$.
              Definiendo $T(\infty) = \infty$, tenemos que $T: \mathbb{C}^\ast \to \mathbb{C}^\ast$ es una aplicación conforme.

        \item Si $\gamma \neq 0$, entonces $T: \mathbb{C}^\ast \to \mathbb{C}^\ast$,
              \begin{align*}
                   & T(z) = \frac{\alpha z + \beta}{\gamma z + \delta}, \quad z \in \mathbb{C} \setminus \left\{-\frac{\delta}{\gamma}\right\} \\
                   & T\left(-\frac{\delta}{\gamma}\right) = \infty                                                                             \\
                   & T(\infty) = \frac{\alpha}{\gamma}
              \end{align*}

              $T$ es meromorfa en $\mathbb{C}^\ast$ y $T$ es inyectiva.

              Veamos que $T$ es sobreyectiva.
              Sea $w \in \mathbb{C} \setminus \{\frac{\alpha}{\gamma}\}$ y sea $z \in \mathbb{C} \setminus \left\{-\frac{\delta}{\gamma}\right\}$.
              Entonces:
              $$T(z) = w \Leftrightarrow \frac{\alpha z + \beta}{\gamma z + \delta} = w \Leftarrow \alpha z + \beta = \gamma zw + \delta w \Leftrightarrow (\alpha - \gamma w)z = \delta w - \beta \Leftrightarrow z = \frac{\delta w - \beta}{-\gamma w + \alpha}$$

              Por tanto, $T$ es una aplicación conforme de $\mathbb{C}^\ast$ sobre $\mathbb{C}^\ast$.
    \end{itemize}

    Además, hemos probado que $T^{-1}$ es de la forma:
    $$T^{-1}(z) = \frac{\delta z - \beta}{-\gamma z + \alpha}, \quad \begin{vmatrix}
            \delta  & -\beta \\
            -\gamma & \alpha
        \end{vmatrix} = \alpha\delta - \beta\gamma \neq 0$$
    Si $\gamma = 0$, también es válida esta expresión.
\end{proof}

\section{Funciones holomorfas en el disco unidad}
\begin{theorem}[Lema de Schwarz]
    Sea $\varphi$ una función holomorfa en $\mathbb{D}$ tal que $\varphi(0) = 0$ y $\varphi(\mathbb{D}) \subset \mathbb{D}$.
    Entonces:
    \begin{enumerate}
        \item $|\varphi(z)| \leq |z|$ para todo $z \in \mathbb{D}$.
        \item $|\varphi'(0)| \leq 1$.
    \end{enumerate}

    Además, se da la igualdad en (1) para algún $z \in \mathbb{D}$ con $z \neq 0$ o bien se da la igualdad en (2) si y solo si $\varphi$ es una rotación de $\mathbb{D}$, es decir, si existe $\lambda \in \mathbb{C}$ con $|\lambda| = 1$ tal que $\varphi(z) = \lambda z$ para todo $z \in \mathbb{D}$.
\end{theorem}

\begin{remark}
    Si $\varphi$ es una rotación, entonces se da la igualdad en (1) para todo $z \in \mathbb{D}$ y se da la igualdad en (2).
\end{remark}

\begin{remark}
    El teorema se puede enunciar de forma equivalente con la condición $\varphi(\mathbb{D}) \subset \overline{\mathbb{D}}$ en lugar de $\varphi(\mathbb{D}) \subset \mathbb{D}$.
    Es decir, si $\varphi$ es holomorfa en $\mathbb{D}$ con $\varphi(0) = 0$ y $\varphi(\mathbb{D}) \subset \overline{\mathbb{D}}$, entonces $\varphi(\mathbb{D}) \subset \mathbb{D}$. \\
    Veamos que esto es cierto.
    Supongamos que existe $z_0 \in \mathbb{D}$ con $|\varphi(z_0)| = 1$.
    Como $|\varphi(z)| \leq 1$ para todo $z \in \mathbb{D}$, por el principio del máximo $\varphi$ es constante, luego $\varphi \equiv \varphi(0) = 0$.
    Esto contradice que $|\varphi(z_0)| = 1$.
\end{remark}

Sea $f$ holomorfa en $\mathbb{D}$ con $(\mathbb{D}) \subset \mathbb{D}$.
Sea $a \in \mathbb{D}$ y $b = f(a) \in \mathbb{D}$.
Definimos:
\begin{align*}
    T_a(z) & = \frac{z+a}{1 + \bar{a}z}, & T_a \in \mathcal{M}, \; T_a(\mathbb{D}) = \mathbb{D}, \; T_a(0) = a \\
    S_b(z) & = \frac{z-b}{1 - \bar{b}z}, & S_b \in \mathcal{M}, \; S_b(\mathbb{D}) = \mathbb{D}, \; S_b(b) = 0
\end{align*}

Sea $\varphi = S_b \circ f \circ T_a$.
$\varphi$ es holomorfa en $\mathbb{D}$, con $\varphi(\mathbb{D}) \subset \mathbb{D}$ y $\varphi(0) = 0$.
Por el lema de Schwarz,
\begin{enumerate}
    \item $|\varphi(z)| \leq |z|$ para todo $z \in \mathbb{D}$.
    \item $|\varphi(0)| \leq 1$.
\end{enumerate}

Además, se da la igualdad en (1) para algún $z \in \mathbb{D}$, $z \neq 0$, o bien se da la igualdad en (2) si y solo si $\varphi$ es una rotación.

Desarrollamos las dos expresiones:
\begin{enumerate}
    \item Sea $z \in \mathbb{D}$.
          Consideramos $T_a^{-1}(z) \in \mathbb{D}$.
          \begin{align*}
               & |\varphi(T_a^{-1}(z))| \leq |T_a^{-1}(z)| \Leftrightarrow |S_b(f(z))| \leq |S_a(z)| \Leftrightarrow \left|\frac{f(z)-b}{1-\bar{z}f(z)}\right| \leq \left|\frac{z-a}{1-\bar{a}z}\right| \Leftrightarrow \\
               & \Leftrightarrow \left|\frac{f(z)-f(a)}{1-\overline{f(a)}f(z)}\right| \leq \left|\frac{z-a}{1-\bar{a}z}\right|, \quad \forall z \in \mathbb{D}
          \end{align*}

          Además, si se da la igualdad para algún $z \in \mathbb{D}$, $z \neq a$, entonces $\varphi$ es una rotación.
          Entonces, $f = S_b^{-1} \circ \varphi \circ T_a^{-1} \in \mathcal{M}$, con $f(\mathbb{D}) = \mathbb{D}$.

    \item Por la regla de la cadena, $\varphi'(0) = T_a'(0) f'(a) S_b'(b)$.
          \begin{align*}
              T_a'(z) & = \frac{1 + \bar{a}z - (z+a)\bar{a}}{(1 + \bar{a}z)^2}, & T_a'(0) & = 1 - |a|^2                                     \\
              S_b'(z) & = \frac{1 - \bar{b}z + (z-b)\bar{b}}{(1 - \bar{b}z)^2}, & S_b'(b) & = \frac{1-|b|^2}{(1-|b|^2)} = \frac{1}{1-|b|^2}
          \end{align*}

          Así que:
          $$\varphi'(0) = (1-|a|^2)f'(a)\frac{1}{1-|b|^2}$$

          Por tanto:
          $$|\varphi'(0)| \leq 1 \Leftrightarrow (1-|a|^2)f'(a)\frac{1}{1-|b|^2} \leq 1 \Leftrightarrow \frac{|f'(a)|}{1-|f(a)|^2} \leq \frac{1}{1-|a|^2}$$

          Además, si se da la igualdad, entonces $\varphi$ es una rotación y por tanto $f \in \mathcal{M}$, con $f(\mathbb{D}) \subset \mathbb{D}$.
\end{enumerate}

Por tanto, hemos probado lo siguiente:
\begin{enumerate}
    \item Para todo $z \in \mathbb{D}$,
          $$\left|\frac{f(z)-f(a)}{1-\overline{f(a)}f(z)}\right| \leq \left|\frac{z-a}{1-\bar{a}z}\right|$$
          Si se da la igualdad para algún $z \in \mathbb{D}$ con $z \neq a$ entonces $f \in \mathcal{M}$ y $f(\mathbb{D}) = \mathbb{D}$.
    \item $$\frac{|f'(a)|}{1-|f(a)|^2} \leq \frac{1}{1-|a|^2}$$
          Si se da la igualdad entonces $f \in \mathcal{M}$ y $f(\mathbb{D}) = \mathbb{D}$.
\end{enumerate}

\section{El teorema de Schwarz-Pick}
\begin{theorem}[Teorema de Schwarz-Pick]
    Sea $f$ holomorfa en $\mathbb{D}$ con $f(\mathbb{D}) \subset \mathbb{D}$.
    Entonces:
    \begin{enumerate}
        \item Para todo $z_1, z_2 \in \mathbb{D}$,
              $$\left|\frac{f(z_2)-f(z_1)}{1-\overline{f(z_1)}f(z_2)}\right| \leq \left|\frac{z_2-z_1}{1-\overline{z_1}z_2}\right|$$
        \item Para todo $z \in \mathbb{D}$,
              $$\frac{|f'(z)|}{1-|f(z)|^2} \leq \frac{1}{1-|z|^2}$$
    \end{enumerate}

    Además, se da la igualdad en (1) para algún par de puntos $z_1, z_2 \in \mathbb{D}$ con $z_1 \neq z_2$ o bien se da la igualdad en (2) para algún $z \in \mathbb{D}$ si y solo si $f \in \mathcal{M}$ y $f(\mathbb{D}) = \mathbb{D}$, en cuyo caso se da la igualdad en (1) para todo $z_1, z_2 \in \mathbb{D}$ y se da la igualdad en (2) para todo $z \in \mathbb{D}$.
\end{theorem}

\begin{proposition}
    Sea $T \in \mathcal{M}$ con $T(\mathbb{D}) = \mathbb{D}$.
    Entonces:
    \begin{enumerate}
        \item Para todo $z_1, z_2 \in \mathbb{D}$,
              $$\left|\frac{T(z_2)-T(z_1)}{1-\overline{T(z_1)}T(z_2)}\right| = \left|\frac{z_2-z_1}{1-\overline{z_1}z_2}\right|$$
        \item Para todo $z \in \mathbb{D}$,
              $$\frac{|T'(z)|}{1-|T(z)|^2} \leq \frac{1}{1-|z|^2}$$
    \end{enumerate}
\end{proposition}

\begin{definition}
    Dados $z_1, z_2 \in \mathbb{D}$, definimos:
    $$\rho(z_1, z_2) = \left|\frac{z_2-z_1}{1-\overline{z_1}z_2}\right|$$

    Observamos que si $1-\overline{z_1}z_2 = 0$ entonces $\overline{z_1}z_2 = 1 \Rightarrow |z_1||z_2| = 1$.
    Como esto no ocurre, $\rho$ está bien definida.
\end{definition}

La primera parte del teorema de Schwarz-Pick se puede reescribir usando $\rho$.

Sea $f$ holomorfa en $\mathbb{D}$ con $f(\mathbb{D}) \subset \mathbb{D}$.
Entonces:
$$\rho(f(z_1), f(z_2)) \leq \rho(z_1, z_2), \quad \text{si } z_1, z_2 \in \mathbb{D}$$
Además, se da la igualdad para algún par de puntos distintos $z_1, z_2 \in \mathbb{D}$ si y solo si $f \in \mathcal{M}$ y $f(\mathbb{D}) = \mathbb{D}$, en cuyo caso se da la igualdad para todo $z_1, z_2 \in \mathbb{D}$.

Vamos a ver que $\rho$ es una distancia en $\mathbb{D}$.
\begin{align*}
    \rho: & D \times D \to \mathbb{R}                                                            \\
          & (z_1, z_2) \mapsto \rho(z_1, z_2) = \left|\frac{z_2-z_1}{1-\overline{z_1}z_2}\right|
\end{align*}

\begin{itemize}
    \item $\rho(z_1, z_2) \leq 0$.
    \item $\rho(z_1, z_2) = \rho(z_2, z_1)$.
    \item $\rho(z_1, z_2) = 0 \Leftrightarrow z_1 = z_2$.
    \item $\rho(z_1, z_3) \leq \rho(z_1, z_2) + \rho(z_2, z_3)$.
\end{itemize}

\begin{lemma}
    Para todo $a, z \in \mathbb{D}$,
    $$\frac{||z|-|a||}{1-|a||z|} \leq \left|\frac{z-a}{1-\bar{a}z}\right| \leq \frac{|z|+|a|}{1+|a||z|}$$
\end{lemma}

Observamos que si $a, z \in \mathbb{D}$, tenemos:
$$\rho(a, z) = \left|\frac{z-a}{1-\bar{a}z}\right| = |S_a(z)| < 1, \quad S_a \in \mathcal{M}, S_a(\mathbb{D}) = \mathbb{D}$$

Dados $a \in \mathbb{D}$ y $0 < r < 1$, denotamos:
$$\Delta(a, r) = \{z \in \mathbb{D} : \rho(z, a) < r\}$$
Entonces, dado $z \in \mathbb{D}$, se tiene que:
$$z \in \Delta(a, r) \Leftrightarrow \rho(z, a) < r \Leftrightarrow |S_a(z)| < r \Leftrightarrow S_a(z) \in D(0, r) \Leftrightarrow z \in S_a^{-1}(D(0, r)) \Leftrightarrow z \in T_a(D(0, r))$$
Entonces $\Delta(a, r) = T_a(D(0, r))$.

$T_a(\partial D(0, r))$ es una circunferencia $C$ contenida en $\mathbb{D}$.
Sean $c$ y $R$ el centro y el radio de $C$, con $c \in \mathbb{C}$, $R > 0$.
Entonces $T_a(D(0, r)) = D(c, R)$.
Por tanto:
$$\Delta(a, r) = T_a(D(0, r)) = D(c, R)$$
Así que $\Delta(a, r)$ es un disco euclídeo.
Como $T_a(0) = a$ tenemos que $a \in \Delta(a, r)$, pero $a$ no tiene por qué ser el centro del disco.

Vamos a calcular $c$ y $R$.
Si $a = 0$, $T_a(z) = z$ luego $T_a(D(0, r)) = D(0, r)$.
Supongamos que $a \neq 0$.
Sea $L$ la recta que pasa por 0 y $a$.
Calculamos $S_a(L)$ hallando la imagen de tres puntos.
\begin{align*}
    S_a(0)                            & = -a     \\
    S_a(a)                            & = 0      \\
    S_a\left(\frac{1}{\bar{a}}\right) & = \infty
\end{align*}
$L' = S_a(L)$ es la recta que pasa por 0 y por $-a$, luego $L'$ coincide con $L$.
Como $L'$ es perpendicular a $\partial D(0, r)$ en los dos puntos de corte y $T_a$ preserva ángulos en esos dos puntos, entonces $L$ es perpendicular a $C$.
Por tanto $c$ está en $L$.

El diámetro $\left[-r\frac{a}{|a|}, r\frac{a}{|a|}\right]$ se aplica mediante $T_a$ en un diámetro de $C$, que es:
$$\left[T_a\left(-r\frac{a}{|a|}\right), T_a\left(r\frac{a}{|a|}\right)\right]$$
Entonces:
\begin{align*}
    c & = \frac{1}{2} \left(T_a\left(-r\frac{a}{|a|}\right) + T_a\left(r\frac{a}{|a|}\right)\right) \\
    R & = \frac{1}{2} \left|T_a\left(r\frac{a}{|a|}\right) - T_a\left(-r\frac{a}{|a|}\right)\right|
\end{align*}

Calculamos:
\begin{align*}
    T_a\left(-r\frac{a}{|a|}\right) = \frac{-r\frac{a}{|a|}+a}{1-\bar{a}r\frac{a}{|a|}} = \frac{-ra+a|a|}{|a|-r|a|^2} = \frac{a(|a|-r)}{|a|(1-r|a|)} \\
    T_a\left(r\frac{a}{|a|}\right) = \frac{r\frac{a}{|a|}+a}{1+\bar{a}r\frac{a}{|a|}} = \frac{ra+a|a|}{|a|+r|a|^2} = \frac{a(|a|+r)}{|a|(1+r|a|)}
\end{align*}
Se llega a que:
\begin{align*}
    c & = \frac{1-r^2}{1-r^2|a|^2}a     \\
    R & = \frac{r(1-|a|^2)}{1-r^2|a|^2}
\end{align*}

Observamos que los puntos de mayor y menor módulo de $C$ son $T_a\left(-r\frac{a}{|a|}\right)$ y $T_a\left(r\frac{a}{|a|}\right)$
Veamos que, de hecho,
$$\left|T_a\left(-r\frac{a}{|a|}\right)\right| = \frac{||a|-r|}{1-r|a|} \leq \frac{r+|a|}{1+r|a|} = \left|T_a\left(r\frac{a}{|a|}\right)\right|$$
\begin{itemize}
    \item Si $|a| \geq r$,
          $$\frac{|a|-r}{1-r|a|} \leq \frac{r+|a|}{1+r|a|} \Leftrightarrow |a| + r|a|^2 - r - r^2|a| \leq r + |a| - r^2|a| - r|a|^2 \Leftrightarrow 2r|a|^2 \leq 2r \Leftrightarrow |a| \leq 1$$
    \item Si $|a| < r$ se razona de forma análoga.
\end{itemize}

Entonces, para todo $z \in \partial D(0, r)$ se tiene que:
\begin{align*}
     & \frac{||a|-r|}{1-r|a|} \leq T_a(z) \leq \frac{r+|a|}{1+r|a|} \Leftrightarrow \frac{||a|-r|}{1-r|a|} \leq \left|\frac{z+a}{1+\bar{a}z}\right| \leq \frac{r+|a|}{1+r|a|} \Leftrightarrow \\
     & \Leftrightarrow \frac{||a|-|z||}{1-|z||a|} \leq \left|\frac{z+a}{1+\bar{a}z}\right| \leq \frac{|z|+|a|}{1+|z||a|} \leq |z|+|a|
\end{align*}
Hemos probado esto para $a, z \in \mathbb{D}$, $a, z \neq 0$.
Pero si $a = 0$ o $z = 0$ la desigualdad es trivial.
Por tanto, esta cadena de desigualdades es cierta para todo $a, z \in \mathbb{D}$.

Cambiando $a$ por $-a$, tenemos:
$$\frac{||a|-|z||}{1-|z||a|} \leq \left|\frac{z-a}{1-\bar{a}z}\right| \leq \frac{|z|+|a|}{1+|z||a|} \leq |z|+|a|, \quad z, a \in \mathbb{D}$$
Las desigualdades primera y segunda corresponden al último lema.

Por otro lado,
$$\rho(a, z) \leq |z| + |a|, \quad z, a \in \mathbb{D}$$
Como $\rho(z_1, 0) = |z_1|$ y $\rho(0, z_2) = |z_2|$, entonces:
$$\rho(a, z) \leq \rho(a, 0) + \rho(0, z), \quad a, z \in \mathbb{D}$$
Esto es un caso particular de la desigualdad triangular.

Sean $z_1, z_2, z_3 \in \mathbb{D}$.
Tenemos, usando el teorema de Schwarz-Pick,
\begin{align*}
    \rho(z_1, z_3) & = \rho(S_{z_2}(z_1), S_{z_2}(z_3)) \leq \rho(S_{z_2}(z_1, 0)) + \rho(0, S_{z_2}(z_3)) =                 \\
                   & = \rho(S_{z_2}(z_1), S_{z_2}(z_2)) + \rho(S_{z_2}(z_2), S_{z_2}(z_3)) = \rho(z_1, z_2) + \rho(z_2, z_3)
\end{align*}
Así que $\rho$ verifica la desigualdad triangular.
Por tanto, $\rho$ es una distancia en $\mathbb{D}$ que se denomina distancia pseudohiperbólica en $\mathbb{D}$.
$$\rho(z_1, z_2) = \left|\frac{z_2-z_1}{1-\overline{z_1}z_2}\right| = |S_{z_1}(z_2)| < 1$$

Si $a \in \mathbb{D}$ y $0 < r < 1$, el disco pseudohiperbólico de centro $a$ y radio $r$ es:
$$\Delta(a, r) = \{z \in \mathbb{D} : \rho(z, a) < r\}$$
No consideramos $r \geq 1$ porque $\Delta(a, r) = \mathbb{D}$.
Sabemos que $\Delta(a, r)$ es un disco euclídeo, en concreto un disco abierto de centro $\frac{1-r^2}{1-r^2|a|^2}a$ y radio $\frac{r(1-|a|^2)}{1-r^2|a|^2}$.
Si $a = 0$, $\Delta(a, r) = D(0, r)$.

Esta distancia es equivalente a la distancia euclídea en $\mathbb{D}$.

Si $T \in \mathcal{M}$ con $T(\mathbb{D}) = \mathbb{D}$, se tiene que:
$$\rho(T(z_1), T(z_2)) = \rho(z_1, z_2), \quad z_1, z_2 \in \mathbb{D}$$
Además, si $f$ es holomorfa en $\mathbb{D}$ y $f(\mathbb{D}) \subset \mathbb{D}$, se tiene que:
$$\rho(f(z_1), f(z_2)) \leq \rho(z_1, z_2), \quad z_1, z_2 \in \mathbb{D}$$

\section{Subordinación}
\begin{definition}
    Sean $f, F$ holomorfas en $\mathbb{D}$.
    Diremos que $f$ está subordinada a $F$, $f \prec F$, si existe $w$ holomorfa en $\mathbb{D}$, con $w(0) = 0$ y $w(\mathbb{D}) \subset \mathbb{D}$ tal que $f = F \circ w$.
\end{definition}

\begin{remark}
    $w$ está en las condiciones del lema de Schwarz.
\end{remark}

Veamos algunas propiedades:
\begin{itemize}
    \item $f(0) = F(w(0)) = F(0)$.
    \item $f(\mathbb{D}) = F(w(\mathbb{D})) \subset F(\mathbb{D})$.
    \item Si $0 < r < 1$, veamos que
          $$f(D(0, r)) \subset F(D(0, r))$$
          Si $z \in D(0, r)$, $f(z) = F(w(z))$.
          Por el lema de Schwarz,
          $$|w(z)| \leq |z| < r$$
    \item Si $0 < r < 1$, veamos que
          $$\max_{|z|=r} |f(z)| \leq \max_{|z|=r} |F(z)|$$
          Si $|z| = r$, como por el lema de Schwarz $|w(z)| \leq |z| = r$, entonces:
          $$|f(z)| = |F(w(z))| \leq \max_{|z| \leq r} |F(z)| \Rightarrow \max_{|z|=r} |f(z)| \leq \max_{|z| \leq r} |F(z)|$$
    \item Si $|z| = r$, como por el lema de Schwarz $|w'(0)| \leq 1$ y además $f'(0) = F'(w(0))w'(0) = F'(0)w'(0)$, entonces:
          $$|f(z)| = |F(w(z))| \leq \max_{|z| \leq r} |F(z)| \Rightarrow \max_{|z|=r} |f(z)| \leq \max_{|z| \leq r} |F(z)|$$
    \item No se verifica para todo $r \in (0, 1)$ que
          $$\max_{|z|=r} |f'(z)| \leq \max_{|z|=r} |F'(z)|$$
\end{itemize}

\begin{example}[Contraejemplo]
    Sean $f(z) = z^2$ y $F(z) = z$.
    Podemos tomar $w(z) = z^2$, que verifica $w(0) = 0$ y $w(\mathbb{D}) \subset \mathbb{D}$, luego $f \prec F$.
    Si $0 < r < 1$,
    \begin{align*}
        \max_{|z|=r} |f'(z)| & = \max_{|z|=r} 2|z| = 2r \\
        \max_{|z|=r} |F'(z)| & = 1
    \end{align*}
    Observamos que no se cumple que $2r \leq 1$ para todo $r \in (0, 1)$.
\end{example}

Por la segunda parte del teorema de Schwarz-Pick,
$$\frac{|w'(z)|}{1-|w(z)|^2} \leq \frac{1}{1-|z|^2}, \quad z \in \mathbb{D}$$
Entonces, si $z \in \mathbb{D}$,
$$|f'(z)| = |F'(w(z))||w'(z)| \leq |F'(w(z))|\frac{1-|w(z)|^2}{1-|z|^2} \Leftrightarrow (1-|z|^2)|f'(z)| \leq (1-|w(z)|^2)|F'(w(z))|$$
Entonces, si $0 < r \leq 1$, tenemos que:
$$\sup_{|z|<r} (1-|z|^2)|f'(z)| \leq \sup_{|z|<r} (1-|z|^2)|F'(z)|$$

Veamos que esto es cierto.
Si $|z| < r$, como $|w(z)| \leq |z| < r$,
$$(1-|z|^2)|f'(z)| \leq (1-|w(z)|^2)|F'(w(z))| \leq \sup_{|z|<r} (1-|z|^2)|F'(z)|$$

\begin{proposition}
    Sean $f, F$ holomorfas en $\mathbb{D}$, con $f \prec F$.
    Entonces:
    \begin{enumerate}
        \item $f(0) = F(0)$.
        \item $f(\mathbb{D}) \subset F(\mathbb{D})$.
        \item Para todo $r \in (0, 1)$,
              $$f(D(0, r)) \subset F(D(0, r))$$
        \item Para todo $r \in (0, 1)$,
              $$\max_{|z|=r} |f(z)| \leq \max_{|z|=r} |F(z)|$$
        \item $|f'(0)| \leq |F'(0)|$.
        \item Para todo $r \in (0, 1]$,
              $$\sup_{|z|<r} (1-|z|^2)|f'(z)| \leq \sup_{|z|<r} (1-|z|^2)|F'(z)|$$
    \end{enumerate}
\end{proposition}

La última propiedad tiene mucha relación con el espacio de Bloch $\mathcal{B}$ de las funciones holomorfas en $\mathbb{D}$ que satisfacen:
$$\sup_{z \in \mathbb{D}} (1-|z|^2)|f'(z)| < \infty$$

\begin{remark}
    Veamos qué se puede decir sobre los coeficientes de Taylor.
    Sean $f$ y $F$ holomorfas en $\mathbb{D}$ con $f \prec F$.
    Consideramos los desarrollos de Taylor de $f$ y $F$ para $z \in \mathbb{D}$:
    \begin{align*}
        f(z) & = \sum_{n=0}^\infty a_nz^n \\
        F(z) & = \sum_{n=0}^\infty A_nz^n
    \end{align*}

    Usando (1), observamos que:
    $$\begin{cases}
            a_0 = f(0) \\
            A_0 = F(0)
        \end{cases} \Rightarrow a_0 = A_0$$

    Con (2), vemos que:
    $$\begin{cases}
            a_1 = f'(0) \\
            A_1 = F'(0)
        \end{cases} \Rightarrow |a_1| \leq |A_1|$$

    No podemos decir nada más.
    Por ejemplo, dado $N \geq 2$, podemos considerar $f(z) = z^N$ y $F(z) = z$.
    Observamos que $f \prec F$ con $w(z) = z^N$.
    Observamos que $a_N = 1$ y $A_N = 0$, luego no es cierto que $|a_n| \leq |A_n|$.
\end{remark}

Veamos ahora un ejemplo importante de subordinación.
Sea $F$ una aplicación conforme de $\mathbb{D}$ sobre $D$, siendo $D$ un dominio en $\mathbb{C}$.
Si $f$ es holomorfa en $\mathbb{D}$ tal que $f(\mathbb{D}) \subset D$ y $f(0) = F(0)$, entonces $f \prec F$.

Sea $w = F^{-1} \circ f$.
$w$ es holomorfa en $\mathbb{D}$, $w(0) = F^{-1}(f(0)) = F^{-1}(F(0)) = 0$ y $w(\mathbb{D}) \subset \mathbb{D}$.
Además, $f = F \circ w$.

Por ejemplo:
$$P(z) = \frac{1+z}{1-z}$$
Esta es una transformación de Möbius que aplica $\partial \mathbb{D}$ en el eje imaginario.
$P(\mathbb{D})$ es el semiplano de la derecha $\{z \in \mathbb{C} : Re(z) > 0\}$ y $P(0) = 1$.
Entonces, si $f$ es holomorfa en $\mathbb{D}$, $f(\mathbb{D}) \subset \{z \in \mathbb{C} : Re(z) > 0\}$ y $f(0) = P(0)$, entonces $f \prec P$.
Es decir, si $f$ es holomorfa en $\mathbb{D}$, $Re(f(z)) > 0$ para todo $z \in \mathbb{D}$ y $f(0) = 1$, entonces $f \prec F$.

Sea $\mathcal{P} = \{f \text{ holomorfa en } \mathbb{D} : Re(f(z)) > 0 \; \forall z \in \mathbb{D}, f(0) = 1\}$.
Entonces:
\begin{itemize}
    \item $P \in \mathcal{P}$.
    \item $f \in \mathcal{P} \Rightarrow f \prec P$.
          De hecho, $\mathcal{P} = \{f \text{ holomorfa en } \mathbb{D} : f \prec P\}$.
    \item $f \in \mathcal{P} \Rightarrow \frac{1}{f} \in \mathcal{P}$.
\end{itemize}

\begin{theorem}
    Si $f \in \mathcal{P}$, entonces:
    \begin{enumerate}
        \item Para todo $z \in \mathbb{D}$,
              $$\frac{1-|z|}{1+|z|} \leq |f(z)| \leq \frac{1+|z|}{1-|z|}$$
        \item $|f'(0)| \leq 2$
    \end{enumerate}
\end{theorem}

Veamos cuáles son las aplicaciones conformes de $\mathbb{D}$ sobre $\mathbb{D}$.

Sea $f$ una aplicación conforme de $\mathbb{D}$ sobre $\mathbb{D}$.
Sea $a = f(0) \in \mathbb{D}$.
Aplicando la segunda parte del teorema de Schwarz-Pick a $f$ en 0, tenemos:
$$\frac{|f'(0)|}{1-|a|^2} \leq 1$$
y si se diera igualdad, $f$ sería una transformación de Möbius.

Sea $g = f^{-1}$, que es holomorfa en $\mathbb{D}$ y $g(\mathbb{D}) \subset \mathbb{D}$.
Aplicando lo mismo en el punto $a$ tenemos:
$$|g'(a)| \leq \frac{1}{1-|a|^2}$$
y si se diera igualdad, $g$ sería una transformación de Möbius.

Tenemos que:
$$|f'(0)| \leq 1-|a|^2 \leq \frac{1}{|g'(a)|} = |f'(0)|$$
Por tanto se da igualdad, así que $f$ es una transformación de Möbius con $f(\mathbb{D}) = \mathbb{D}$.
En conclusión, las aplicaciones conformes de $\mathbb{D}$ sobre $\mathbb{D}$ son:
$$\{\lambda T_a : \lambda \in \mathbb{C}, |\lambda| = 1, a \in \mathbb{D}\} = \{\lambda S_a : \lambda \in \mathbb{C}, |\lambda| = 1, a \in \mathbb{D}\} = \{\lambda \varphi_a : \lambda \in \mathbb{C}, |\lambda| = 1, a \in \mathbb{D}\}$$

\section{La métrica de Poincaré}
Si $\gamma$ es un camino en $\mathbb{C}$ y $f: sop(\gamma) \to \mathbb{C}$ es continua, entonces:
\begin{align*}
    \int_\gamma f(z)dz   & = \int_a^b f(\gamma(t))\gamma'(t)dt   \\
    \int_\gamma f(z)|dz| & = \int_a^b f(\gamma(t))|\gamma'(t)|dt
\end{align*}
siendo $\gamma: [a, b] \to \mathbb{C}$ una parametrización $\mathcal{C}^1$ a trozos de $\gamma$.

Veamos algunas propiedades:
\begin{enumerate}
    \item Si $f$ es real, entonces $\int_\gamma f(z)|dz| \in \mathbb{R}$.
          Si además $f$ es no negativa, entonces $\int_\gamma f(z)|dz| \geq 0$.
    \item Si $f(z) = 1$,
          $$\int_\gamma f(z)|dz| = \int_a^b |\gamma'(t)|dt = long(\gamma)$$
    \item $$\left|\int_\gamma f(z)dz\right| \leq \int_\gamma |f(z)||dz| \leq \max_{z \in sop(\gamma)} |f(z)|long(\gamma)$$
    \item Si $f, g: sop(\gamma) \to \mathbb{R}$ continuas y $f \leq g$, entonces:
          $$\int_\gamma f(z)|dz| \leq \int_\gamma g(z)|dz|$$
    \item $$\int_{\gamma_1 + \gamma_2} f(z)|dz| = \int_{\gamma_1} f(z)|dz| + \int_{\gamma_2} f(z)|dz|$$
    \item $$\int_{-\gamma} f(z)|dz| = \int_\gamma f(z)|dz|$$
    \item $$\int_\gamma (af(z) + bg(z))|dz| = a\int_\gamma f(z)|dz| + b\int_\gamma g(z)|dz|, \quad a, b \in \mathbb{C}$$
\end{enumerate}

Sean $z_1, z_2 \in \mathbb{D}$.
Sea $\gamma$ un camino en $\mathbb{D}$ con origen $z_1$ y extremo $z_2$.
Podemos considerar la integral
$$\int_\gamma \frac{1}{1-|z|^2}|dz| = \int_\gamma \frac{|dz|}{1-|z|^2}$$
Como la función $z \in \mathbb{D} \mapsto \frac{1}{1-|z|^2}$ es real y positiva, entonces la integral es no negativa.
Definimos:
$$\delta(z_1, z_2) = \inf \left\{\int_\gamma \frac{|dz|}{1-|z|^2} : \gamma \text{ camino en } \mathbb{D} \text{ con origen } z_1 \text{ y extremo } z_2\right\}$$
Entonces:
\begin{itemize}
    \item $\delta(z_1, z_2) \geq 0$.
    \item $\delta(z_1, z_2) = \delta(z_2, z_1)$.
    \item $\delta(z_1, z_2) = 0 \Leftrightarrow z_1 = z_2$.
\end{itemize}

Sean $z_1, z_2, z_3 \in \mathbb{D}$.
Consideramos:
$$A_{12} = \left\{\int_\gamma \frac{|dz|}{1-|z|^2} : \gamma \text{ camino en } \mathbb{D} \text{ con origen } z_1 \text{ y extremo } z_2\right\}$$
Se definen de manera análoga $A_{13}$ y $A_{23}$.
Observamos que $A_{12} + A_{23} \subset A_{13}$.
Por tanto:
$$\inf(A_{12} + A_{13}) = \inf A_{12} + \inf A_{23} \geq \inf A_{13} \Leftrightarrow \delta(z_1, z_3) \leq \delta(z_1, z_2) + \delta(z_2, z_3)$$

$\delta$ es una distancia en $\mathbb{D}$, denominada distancia hiperbólica en $\mathbb{D}$.

\begin{proposition}
    \hfill
    \begin{enumerate}
        \item Si $f$ es holomorfa en $\mathbb{D}$ con $f(\mathbb{D}) \subset \mathbb{D}$, entonces:
              $$\delta(f(z_1), f(z_2)) \leq \delta(z_1, z_2), \quad \forall z_1, z_2 \in \mathbb{D}$$
        \item Si $T \in \mathcal{M}$ con $T(\mathbb{D}) = \mathbb{D}$, entonces:
              $$\delta(T(z_1), T(z_2)) = \delta(z_1, z_2), \quad \forall z_1, z_2 \in \mathbb{D}$$
    \end{enumerate}
\end{proposition}

\begin{proof}
    \hfill
    \begin{enumerate}
        \item Sea $f$ holomorfa en $\mathbb{D}$ con $f(\mathbb{D}) \subset \mathbb{D}$ y sean $z_1, z_2 \in \mathbb{D}$.
              \begin{align*}
                  \delta(z_1, z_2)       & = \inf \left\{\int_\gamma \frac{|dz|}{1-|z|^2} : \gamma \text{ camino de } z_1 \text{ a } z_2\right\}       \\
                  \delta(f(z_1), f(z_2)) & = \inf \left\{\int_\Gamma \frac{|dw|}{1-|w|^2} : \Gamma \text{ camino de } f(z_1) \text{ a } f(z_2)\right\}
              \end{align*}

              Sea $\gamma$ un camino en $\mathbb{D}$ con origen $z_1$ y extremo $z_2$, con parametrización $\mathcal{C}^1$ a trozos $\gamma: [a, b] \to \mathbb{C}$.
              Entonces $\Gamma = f \circ \gamma: [a, b] \to \mathbb{C}$ es una parametrización $\mathcal{C}^1$ a trozos de un camino $\Gamma$ en $\mathbb{D}$ con origen $f(z_1)$ y extremo $f(z_2)$.
              Tenemos:
              $$\int_\Gamma \frac{|dw|}{1-|w|^2} = \int_a^b \frac{|\Gamma'(t)|}{1-|\Gamma(t)|^2}dt = \int_a^b \frac{|f'(\gamma(t))||\gamma'(t)|}{1-|f(\gamma(t))|^2}dt$$
              Usando la segunda parte del teorema de Schwarz-Pick:
              $$\int_a^b \frac{|f'(\gamma(t))||\gamma'(t)|}{1-|f(\gamma(t))|^2}dt \leq \int_a^b \frac{|\gamma'(t)|}{1-|\gamma(t)|^2}dt = \int_\gamma \frac{|dz|}{1-|z|^2}$$

              Luego tenemos que:
              $$\delta(f(z_1), f(z_2)) \leq \int_\gamma \frac{|dz|}{1-|z|^2}, \quad \forall \gamma$$
              Por tanto, $\delta(f(z_1), f(z_2)) \leq \gamma(z_1, z_2)$.

        \item Se tiene aplicando (1) a $T$ y $T^{-1}$.
    \end{enumerate}
\end{proof}

\begin{proposition}
    Si $z_1, z_2 \in \mathbb{D}$, entonces:
    $$\delta(z_1, z_2) = \frac{1}{2} \Log \frac{1+\rho(z_1, z_2)}{1-\rho(z_1, z_2)}$$
\end{proposition}

\begin{proof}
    Si $z_1 = z_2$ es trivial.
    Supongamos $z_1 \neq z_2$.
    Consideramos:
    $$S_{z_1}(z) = \frac{z-z_1}{1-\overline{z_1}z}$$
    Se tiene que $S_{z_1} \in \mathcal{M}$, $S_{z_1}(\mathbb{D}) = \mathbb{D}$ y $S_{z_1}(z_1) = 0$.
    Sabemos que $S_{z_1}(z_2) \neq 0$.
    Además,
    $$\delta(z_1, z_2) = \delta(0, S_{z_1}(z_2))$$

    Tomamos $\lambda \in \mathbb{C}$, $|\lambda| = 1$, tal que $\lambda S_{z_1}(z_2) \in (0, 1)$.
    Sea $r = \lambda S_{z_1}(z_2)$.
    Entonces:
    $$\delta(z_1, z_2) = \delta(0, S_{z_1}(z_2)) = \delta(0, r)$$
    Además, $r = |\lambda S_{z_1}(z_2)| = |S_{z_1}(z_2)| = \rho(z_1, z_2)$.
    Calculamos $\delta(0, r)$.
    $$\delta(0, r) = \inf \left\{\int_\gamma \frac{|dz|}{1-|z|^2} : \gamma \text{ camino en } \mathbb{D} \text{ con origen } 0 \text{ y extremo } r\right\}$$
    Si $\gamma = [0, r]$,
    \begin{align*}
         & \int_\gamma \frac{|dz|}{1-|z|^2} = \int_0^r \frac{dt}{1-t^2} = \frac{1}{2} \int_0^r \left(\frac{1}{1-t} + \frac{1}{1+t}\right)dt = = \frac{1}{2} \left[-\Log(1-t) + \Log(1+t)\right]_0^r = \\
         & = \frac{1}{2} \left[\Log \frac{1+t}{1-t}\right]_0^r = \frac{1}{2} \Log\frac{1+r}{1-r}
    \end{align*}
    Luego $\delta(0, r) \leq \frac{1}{2} \Log\frac{1+r}{1-r}$.

    Sea $\gamma$ un camino en $\mathbb{D}$ con origen 0 y extremo $r$.
    Veamos que
    $$\int_\gamma \frac{|dz|}{1-|z|^2} \geq \frac{1}{2} \Log\frac{1+r}{1-r}$$
    Sea $\gamma: [a, b] \to \mathbb{C}$ una parametrización $\mathcal{C}^1$ a trozos de $\gamma$.
    Sean $u = Re(f)$ y $v= Im(f)$, de forma que $\gamma = u + iv$.
    $u, v: [a, b] \to \mathbb{R}$, $\mathcal{C}^1$ a trozos.
    $$\int_\gamma \frac{|dz|}{1+|z|^2} = \int_a^b \frac{|\gamma'(t)|}{1-|\gamma(t)|^2}dt$$
    Tenemos que:
    $$\begin{cases}
            |\gamma(t)|^2 \geq u(t)^2 \Rightarrow 0 < 1-|\gamma(t)|^2 \leq 1-u(t)^2 \Rightarrow \frac{1}{1-|\gamma(t)|^2} \geq \frac{1}{1-u(t)^2} \\
            |\gamma'(t)| \geq |u'(t)| \geq 0
        \end{cases}$$
    Así que:
    $$\frac{|\gamma'(t)|}{1-|\gamma(t)|^2} \geq \frac{|u'(t)|}{1-u(t)^2} \geq \frac{u'(t)}{1-u(t)^2}$$
    Luego:
    \begin{align*}
         & \int_a^b \frac{|\gamma'(t)|}{1-|\gamma(t)|^2}dt \geq \int_a^b \frac{u'(t)}{1-u(t)^2}dt = \frac{1}{2} \int_a^b \left(\frac{u'(t)}{1-u(t)} + \frac{u'(t)}{1+u(t)}\right)dt = \\
         & = \frac{1}{2} \left[-\Log(1-u(t)) + \Log(1+u(t))\right]_a^b = \frac{1}{2} \left[\Log \frac{1+u(t)}{1-u(t)}\right]_a^b = \frac{1}{2} \Log\frac{1+r}{1-r}
    \end{align*}
    porque $u(a) = 0$ y $u(b) = r$.
    Por tanto,
    $$\delta(z_1, z_2) = \delta(0, r) = \frac{1}{2} \Log\frac{1+r}{1-r} = \frac{1}{2} \log\frac{1+\rho(z_1, z_2)}{1-\rho(z_1, z_2)}$$
\end{proof}

\begin{remark}
    Sea $h(x) = \frac{1}{2} \Log \frac{1+x}{1-x}$, $x \in [0, 1)$.
    Observamos que si $x < 1$, entonces $1+x \geq 1-x > 0 \Rightarrow \frac{1+x}{1-x} \geq 1$, así que $h: [0, 1) \to [0, \infty)$.
    $h$ es creciente, con $h(0) = 0$ y $\lim\limits_{x \to 1^-} h(x) = \infty$.
    Podemos escribir $\delta = h \circ \rho : \mathbb{D} \times \mathbb{D} \xrightarrow{\rho} [0, 1) \xrightarrow{h} [0, \infty)$.
    Fijado $a \in \mathbb{D}$, si $\{z_n\}_{n=1}^\infty$ está en $\mathbb{D}$ con $|z_n| \to 1$, entonces:
    $$\rho(a, z_n) = |S_a(z_n)| \xrightarrow[n \to \infty]{} 1$$
    Por tanto, $\delta(a, z_n) \xrightarrow[n \to \infty]{} \infty$.
\end{remark}

$\mathbb{D}$ con esta distancia $\delta$ es un modelo de la geometría hiperbólica.
Si $\gamma$ es un camino en $\mathbb{D}$, la longitud de $\gamma$ es
$$\int_\gamma \frac{|dz|}{1-|z|^2}$$
La geodésica que une $z_1, z_2 \in \mathbb{D}$ es el camino $\gamma$ para el que:
$$\delta(z_1, z_2) = \int_\gamma \frac{|dz|}{1-|z|^2}$$
Las geodésicas con respecto a $\delta$ son los diámetros de $\partial \mathbb{D}$ y los arcos de circunferencia ortogonales a $\partial \mathbb{D}$.