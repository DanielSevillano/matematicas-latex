\chapter{Introducción}
\section{Superficies regulares}

\begin{definition}
    $S \subset \mathbb{R}^3$ es superficie regular si para todo $p \in S$ existe una aplicación $X : U \to V$, con $U \subset \mathbb{R}^2$ abierto y $V \subset S$ entorno abierto de $p$ en $S$, que verifica:
    \begin{itemize}
        \item $X$ es diferenciable.
        \item $X$ es homeomorfismo.
        \item $dX_q : \mathbb{R}^2 \to \mathbb{R}^3$ es inyectiva para todo $q \in U$.
    \end{itemize}
\end{definition}

\begin{definition}
    \hfill
    \begin{itemize}
        \item $V$ se llama un entorno coordenado de $p$ en $S$.
        \item $X$ es una parametrización de $S$ en $p$ o un sistema local de coordenadas.
        \item $(U, X)$ es una carta en $p$.
        \item $\{ (U_i, X_i), i \in I : \bigcup_i X_i(U_i) = S \}$ es un atlas.
    \end{itemize}
\end{definition}

\begin{remark}
    Las parametrizaciones de una superficie regular no son únicas.
\end{remark}

\begin{proposition}
    Si $f : U \to \mathbb{R}$ es una función diferenciable definida sobre un abierto $U$, entonces el grafo de $f$
    $$G(f) = \{ (x, y, f(x, y)) : (x, y) \in U \}$$
    es una superficie regular.
\end{proposition}

\begin{definition}
    Sea $F : U \subset \mathbb{R}^n \to \mathbb{R}^m$, con $U$ abierto.
    \begin{itemize}
        \item $p \in U$ es un punto crítico de $F$ si $dF_p$ no es sobreyectiva. Entonces $F(p) \in \mathbb{R}^m$ es un valor crítico.
        \item $p$ es un punto regular si no es crítico. Análogamente, $F(p)$ es valor regular si no es crítico.
    \end{itemize}
\end{definition}

\begin{remark}
    Si $f : U \subset \mathbb{R}^3 \to \mathbb{R}$, entonces $df_p$ es sobreyectiva o $df_p = (0,0,0)$.
    Luego $a \in f(U)$ es valor regular de $f$ si y solo si $f_x, f_y, f_z$ no se anulan simultáneamente en ningún punto de $f^{-1}(a)$.
\end{remark}

\begin{proposition}
    Si $f: U \subset \mathbb{R}^2 \to \mathbb{R}$ diferenciable y $a \in f(U)$ es valor regular de $f$, entonces $f^{-1}(a)$ es superficie regular de $\mathbb{R}^3$.
\end{proposition}

\begin{proposition}
    Sea $S \subset \mathbb{R}^3$ una superficie regular.
    Entonces dado $p \in S$, existe $V$ entorno abierto de $p$ en $S$ tal que $V$ es el grafo de una función diferenciable de una de las tres formas siguientes:
    $$z = f(x, y), \quad y = g(x, z), \quad \text{o} \quad x = h(y, z)$$
\end{proposition}

\begin{proposition}
    Sea $S$ una superficie regular y sea $X : U \subset \mathbb{R}^2 \to S$ una aplicación diferenciable, inyectiva y tal que $dX_q$ es inyectiva para todo $q \in U$.
    Entonces $X^{-1} : X(U) \to U$ es continua y, en consecuencia, $X$ es una parametrización de $S$.
\end{proposition}

\begin{remark}
    Es necesario que $S$ sea superficie regular.
\end{remark}

\begin{definition}
    Una superficie parametrizada es una aplicación $X : U \subset \mathbb{R}^2 \to \mathbb{R}^3$ diferenciable.
    Se dice que $X(U) \subset \mathbb{R}^3$ es la traza de $X$.\\
    $X$ es regular si $dX_q : \mathbb{R}^2 \to \mathbb{R}^3$ es inyectiva para todo $q \in U$.
\end{definition}

\begin{proposition}
    Sea $X : U \subset \mathbb{R}^2 \to \mathbb{R}^3$ superficie parametrizada regular y sea $q \in U$.
    Entonces existe $V$ entorno abierto de $q$ en $U$ tal que $X(U)$ es superficie regular.
\end{proposition}

\begin{definition}
    $C \subset \mathbb{R}^3$ es una curva regular si para todo $p \in C$ existe $V$ entorno abierto de $p$ y $\alpha : I \to U \int C$, con $I$ intervalo abierto, tal que:
    \begin{itemize}
        \item $\alpha$ es diferenciable.
        \item $\alpha$ es homeomorfismo.
        \item $d\alpha_t$ = $\alpha'(t)$ es inyectiva para todo $t$.
    \end{itemize}
\end{definition}

\begin{definition}
    Una superficie de revolución es un subconjunto $S \subset \mathbb{R}^3$ obtenida al rotar una curva plana regular $C$ alrededor de un eje contenido en el mismo plano que la curva y que no corte a la curva.
\end{definition}

\section{Cálculo diferencial en superficies regulares}

\begin{definition}
    Sea $f: S \subset \mathbb{R}^3 \to \mathbb{R}$, $X$ parametrización de $S$ y $p \in \mathbb{R}^3$.
    $f$ es diferenciable en $p$ si y solo si $f \circ X$ es diferenciable en $X^{-1}(p)$.
\end{definition}

\begin{remark}
    Esta definición no depende de la parametrización de $S$.
\end{remark}

\begin{definition}
    $f: O \subset S \to \mathbb{R}$ es una función diferenciable en $p \in O$ si para alguna parametrización $X : U \to S$, $p \in X(U)$, se tiene que $f \circ X : U \to \mathbb{R}$ es diferenciable en $X^{-1}(p)$.
\end{definition}

\begin{definition}
    $\varphi : S_1 \to S_2$ es difeomorfismo si es diferenciable, biyectiva y $\varphi^{-1}$ es diferenciable.
\end{definition}

\begin{definition}
    Un vector tangente a $S$ en $p$ es un vector tangente a una curva diferenciable parametrizada que pase por $p$.\\
    Es decir, una curva $\alpha : (-\varepsilon, \varepsilon) \to S$, con $\alpha(0) = p$, $\alpha'(0) \in T_pS$.
\end{definition}

\begin{proposition}
    Sea $X$ una parametrización, $dX_q(\mathbb{R}^2) = T_pS$, con $q = X^{-1}(p)$.
\end{proposition}

\begin{definition}
    Sea $\varphi : O \subset S_1 \to S_2$ una aplicación diferenciable definida en un abierto $O$ de $S_1$ y $p \in O$.
    Consideramos la diferencial de $\varphi$ en $p$
    $$d\varphi_p : T_pS_1 \to T_{\varphi(p)}S_2$$
    Sean $w \in T_pS_1$ y $\alpha : (-\varepsilon, \varepsilon) \to S_1$ curva diferenciable parametrizada con $\alpha(0) = p, \alpha'(0) = w$.
    Entonces $d\varphi_p(w) = (\varphi \circ \alpha)'(0)$.\\
    Se tiene:
    \begin{itemize}
        \item $\varphi \circ \alpha$ es una curva diferenciable parametrizada sobre $S_2$. $(\varphi \circ \alpha)'(0) \in T_{\varphi(p)}S_2$.
        \item $d\varphi_p$ no depende de $\alpha$ y es lineal.
    \end{itemize}
\end{definition}

\begin{definition}
    Sea $S$ superficie regular, $O$ abierto de $S$, $f : O \subset S \to \mathbb{R}$ diferenciable.
    Consideramos la diferencial de $f$ en $p$
    $$df_p : T_pS \to \mathbb{R}$$
    Sean $w \in T_pS$ y $\alpha : (-\varepsilon, \varepsilon) \to S$ curva diferenciable parametrizada con $\alpha(0) = p, \alpha'(0) = w$.
    Entonces $df(w) = (f \circ \alpha)'(0)$.\\
    Verifica:
    \begin{itemize}
        \item Está bien definida y es lineal.
        \item Si $f = F|_S$ con $F : O \subset \mathbb{R}^3 \to \mathbb{R}$, entonces $df_p = dF_p|_{T_pS}$.
    \end{itemize}
\end{definition}

\section{Primera forma fundamental}

\begin{definition}
    Sea $S$ superficie regular.
    Para cada $p \in S$ el producto escalar en $\mathbb{R}^3$ induce una métrica en $T_pS$.
    $$\left\langle w_1, w_2 \right\rangle _p = \left\langle w_1, w_2 \right\rangle, \quad \forall w_1, w_2 \in T_pS$$
    La forma cuadrática asociada se llama primera forma fundamental de $S$ en $p$.
    $$I_p : T_pS \to \mathbb{R}$$
    $$I_p(w) = \left\langle w, w \right\rangle _p = |w|^2 \geq 0$$
\end{definition}

\begin{note}
    Dada una parametrización $X$ de $S$, $\{ X_u, X_v \}_q$ base de $T_pS$, $q = X^{-1}(p)$.
    Llamamos $E$, $F$ y $G$ a los coeficientes de la primera forma fundamental.
    $$E_q = \left\langle X_u(q), X_u(q) \right\rangle$$
    $$F_q = \left\langle X_u(q), X_v(q) \right\rangle$$
    $$G_q = \left\langle X_v(q), X_v(q) \right\rangle$$
    Estas son funciones diferenciables en $X(U)$.
\end{note}

\section{Propiedades de las curvas}

\begin{definition}
    Sea $\alpha : I \to S$ una curva diferenciable parametrizada.
    Se define la longitud de arco como una aplicación $s : I \to \mathbb{R}$, con $t_0 \in I$ fijo, dada por:
    $$s(t) = \int^t_{t_0} |\alpha'(r)|dr = \int^t_{t_0} \sqrt{\left\langle \alpha'(r), \alpha'(r) \right\rangle} dr$$
\end{definition}

\begin{remark}
    Sea $(u_0, v_0) \in U$ fijo.
    $$u \mapsto X(u, v_0), \quad v \mapsto X(u_0, v)$$
    $$s_{v_0}(u) = \int^u_{u_0} |X_u| dr = \int^u_{u_0} \sqrt{E} dr$$
    $v = v_0$ está parametrizada por el arco si y solo si $E(u, v_0) = 1$ para todo $u$.
    $$s_{u_0}(v) = \int^v_{v_0} |X_v| dr = \int^v_{v_0} \sqrt{G} dr$$
    $u = u_0$ está parametrizada por el arco si y solo si $G(u_0, v) = 1$ para todo $v$.\\
    En general, todas las curvas coordenadas de $X$ están parametrizadas por el arco si y solo si $E(u, v) = 1$ y $G(u, v) = 1$, para todo $(u, v) \in U$.
\end{remark}

\begin{definition}
    El ángulo de dos curvas es el menor ángulo que forman las rectas tangentes.
    Sean $\alpha, \beta : I \to S$, con $\alpha(t_0) = \beta(t_0)$.
    $$\cos \theta(t_0) = \frac{\left\langle \alpha'(t_0), \beta'(t_0) \right\rangle}{|\alpha'(t_0)||\beta'(t_0)|}$$
\end{definition}

\begin{remark}
    El ángulo de las curvas coordenadas $u = u_0$ y $v = v_0$ en $X(u_0, v_0)$ es
    $$\cos \theta(u_0, v_0) = \frac{\left\langle X_u, X_v \right\rangle}{|X_u||X_v|} (u_0, v_0) = \frac{F}{\sqrt{EG}} (u_0, v_0)$$
    Las curvas coordenadas de una parametrización $u = cte$, $v = cte$ son ortogonales en todos los puntos de $U$ si y solo si $F(u, v) = 0$ para todo $(u, v) \in U$.\\
    En ese caso se dice que $X$ es una parametrización ortogonal.
\end{remark}

\section{Orientación en superficies}

\begin{properties}
    \hfill
    \begin{itemize}
        \item Dos bases ordenadas de un mismo espacio vectorial representan la misma orientación si el determinante de la matriz de cambio de base es positivo.
              Cada clase de equivalencia es una orientación en $U$.
        \item $\{X_u, X_v\}$ determina una orientación en $S$.
              $\{X_u, X_v, X_u \land X_v\}$ es una base positiva, con $N = \frac{X_u \land X_v}{|X_u \land X_v|}$.
    \end{itemize}
\end{properties}

\begin{definition}
    $S$ es orientable si se puede recubrir con una familia de entornos coordenados tal que en la intersección de dos tales entornos el determinante jacobiano del cambio de coordenadas tiene determinante positivo.
    La elección de tal familia se llama una orientación de $S$ y se dice que $S$ está orientada.
\end{definition}

\begin{definition}
    Sea $V$ abierto de $S$, se llama campo diferenciable de vectores normales unitarios a $N : V \to \mathbb{R}^3$ diferenciable tal que para todo $p \in S$ se tiene que $N(p) \perp T_pS$ y $|N(p)| = 1$.
\end{definition}

\begin{theorem}
    $S$ es orientable si y solo si existe un campo diferenciable $N$ de vectores normales unitarios sobre $S$.
\end{theorem}

\begin{note}
    La elección de dicho campo normal $N$ determina una orientación en $S$.
\end{note}

\begin{proposition}
    Si $S$ es imagen inversa de un valor regular, entonces $S$ es orientable.
\end{proposition}

\section{Segunda forma fundamental}

\begin{definition}
    Sea $S \subset \mathbb{R}^3$ orientable y orientada con orientación $N : S \to \mathbb{R}^3$, con $|N(p)| = 1$ para todo $p \in S$.
    Luego $N(p) \in S^2(1) \equiv S^2 = \{ (x, y, z) \in \mathbb{R}^3 : x^2+y^2+z^2 = 1 \}$.\\
    La aplicación diferenciable $N : S \to S^2$ es la aplicación de Gauss.
\end{definition}

\begin{note}
    Se mira $N$ como aplicación diferenciable entre superficies, no como un campo de vectores.
    El vector normal se toma con origen en el origen de $\mathbb{R}^3$ y el extremo da un punto de $S^2$.
\end{note}

\begin{remark}
    Si se cambia la orientación, también se cambia la aplicación de Gauss.
\end{remark}

\begin{definition}
    La diferencial de la aplicación de Gauss es, para $p \in S$, $dN_p : T_pS \to T_{N(p)}S^2$ lineal.
    Como $N(p) \perp T_pS$ y $N(p) \perp T_{N(p)}S^2$, estos son planos paralelos, así que se pueden identificar y considerar $dN_p : T_pS \to T_pS$ endomorfismo de $T_pS$.
    Este mide la variación en dirección de $N$ sobre las curvas que pasan por $p$ en un entorno de $p$.
    $$dN_p(w) = \frac{d}{dt}|_{t=0} (N \circ \alpha)(t) = (N \circ \alpha)'(0)$$
\end{definition}

\begin{proposition}
    $dN_p$ es autoadjunta respecto a $\left\langle , \right\rangle$, es decir, $\left\langle dN_p(v), w \right\rangle = \left\langle v, dN_p(w) \right\rangle, \forall v, w \in T_pS$.
    Luego se puede asociar la forma bilineal simétrica con la forma cuadrática.
\end{proposition}

\begin{definition}
    Dicha forma cuadrática $\amalg_p : T_pS \to \mathbb{R}$
    $$\amalg_p(w) = -\left\langle dN_p(w), w \right\rangle$$
    es la segunda forma fundamental de $S$ en $p$.
\end{definition}

\begin{remark}
    $S_p = -dN_p$ es el operador de Weingarten en $p$.
\end{remark}

\begin{definition}
    Sea $C$ una curva regular en $S$ que pasa por $p$.
    Se define la curvatura normal de $C$ en $p$ como
    $$k_n(p) = k(p) \left\langle n(p), N(p) \right\rangle$$
\end{definition}

\begin{remark}
    $k_n$ cambia de signo si cambia la orientación en $S$ y no depende de la orientación en $C$.
\end{remark}

\begin{theorem}[Teorema de Meusnier]
    Todas las curvas sobre una superficie regular orientable $S$ que tienen la misma recta tangente en $p \in S$ tienen la misma curvatura normal en $p$.
\end{theorem}

\begin{remark}
    \hfill
    \begin{itemize}
        \item $\amalg_p(w) = k_n(p)$, siendo $w$ unitario.
        \item La curvatura de la sección normal a lo largo de $w$ de $S$ en $p$ es el valor absoluto de la curvatura normal en $p$ de cualquier curva sobre $S$ que pase por $p$ con vector tangente $w$.
              $$k(p) = |\amalg_p(w)|$$
    \end{itemize}
\end{remark}

\begin{theorem}
    Sea $S$ superficie orientable con aplicación de Gauss $N : S \to S^2$.
    Para todo $p \in S$ existe $\{e_1, e_2\}$ base ortonormal de $T_pS$ con $dN_p(e_1) = -k_1e_1, dN_p(e_2) = -k_2e_2$.
    Además, $k_1$ y $k_2$ son el máximo y el mínimo de $\amalg_p$ sobre la circunferencia unidad de $T_pS$, es decir, los valores extremos de las curvaturas normales en $p$.
\end{theorem}

\begin{theorem}[Fórmula de Euler]
    Sea $w \in T_pS$, $|w| = 1$.
    $$\amalg_p(w) = \cos^2(t) k_1 + \sin^2(t) k_2$$
    con $\cos(t) = \left\langle w, e_1 \right\rangle, \sin(t) = \left\langle w, e_2 \right\rangle$, es decir, $t$ es el ángulo que forma $w$ con $e_1$ en la orientación de $T_pS$.
\end{theorem}

\begin{definition}
    $k_1$ y $k_2$ son las curvaturas principales de $S$ en $p$.
    Las direcciones asociadas se llaman direcciones principales en $p$.
\end{definition}

\begin{definition}
    Una curva regular conexa $C$ en $S$ es línea de curvatura de $S$ si para todo $p \in C$ la recta tangente a $C$ en $p$ es una dirección principal en $p$.
\end{definition}

\begin{definition}
    \hfill
    \begin{itemize}
        \item La curvatura de Gauss de $S$ en $p$ se define como $K(p) = det(dN_p)$.
        \item La curvatura media de $S$ en $p$ se define como $H(p) = -\frac{1}{2} tr(dN_p)$.
    \end{itemize}
\end{definition}

\begin{remark}
    En una base ortonormal de direcciones principales
    $$dN_p \equiv
        \begin{pmatrix}
            -k_1 & 0    \\
            0    & -k_2
        \end{pmatrix} \Rightarrow
        K = k_1 k_2, \quad H = \frac{k_1+k_2}{2}$$
\end{remark}

\begin{remark}
    Ante un cambio de orientación, $k_1$, $k_2$ y $H$ cambian de signo, mientras que $K$ no cambia.
\end{remark}

\begin{definition}
    Sea $p \in S$, se puede clasificar según su curvatura de Gauss.
    \begin{itemize}
        \item Si $K(p) > 0$, decimos que $p$ es elíptico.
        \item Si $K(p) < 0$, decimos que $p$ es hiperbólico.
        \item Si $K(p) = 0$ y $dN_p \neq 0$, decimos que $p$ es parabólico.
        \item Si $dN_p \equiv 0$, decimos que $p$ es plano.
    \end{itemize}
\end{definition}

\begin{remark}
    \hfill
    \begin{itemize}
        \item Si $k_1 = k_2$, todas las direcciones son principales.
        \item Si $k_1 \neq k_2$, hay dos direcciones principales y son perpendiculares.
    \end{itemize}
\end{remark}

\begin{definition}
    Un punto $p$ es umbilical si $k_1(p) = k_2(p)$.
    En ese caso, $k_n = k_1 = k_2$ y todas las direcciones de $T_pS$ son principales.
\end{definition}

\begin{remark}
    Los puntos umbilicales solo pueden ser elípticos o planos.
\end{remark}

\begin{theorem}
    Si todos los puntos de una superficie conexa $S$ son umbilicales, entonces $S$ es un abierto de un plano o de una esfera.
\end{theorem}

\begin{definition}
    Una dirección asintótica de $S$ en $p$ es una dirección de $T_pS$ para la cual la curvatura normal es cero.
\end{definition}

\begin{definition}
    Una curva o línea asintótica es una curva regular conexa $C$ en $S$ tal que, para todo $p \in C$, la recta tangente a $C$ en $p$ es una dirección asintótica.
\end{definition}

\begin{note}
    $C$ es una curva asintótica si y solo si $k_n(p) = k(p) \left\langle n, N \right\rangle = 0$.
\end{note}

\begin{remark}
    \hfill
    \begin{itemize}
        \item En un punto elíptico no hay direcciones asintóticas.\\
              $k_1$ y $k_2$ tienen el mismo signo, así que $k_2 < k_n < k_1$.
        \item En un punto plano todas las direcciones son asintóticas.
        \item En un punto hiperbólico hay dos direcciones asintóticas.\\
              $k_1 \cos^2(t) + k_2 \sin^2(t) = 0$ tiene dos soluciones.
        \item En un punto parabólico hay una dirección asintótica.
              $k_1 \cos^2(t) = 0$ tiene una solución.
    \end{itemize}
\end{remark}

\begin{proposition}
    Sea $S$ orientada, con $N = \frac{X_u \land X_v}{|X_u \land X_v|}$. En la base $\{ X_u, X_v \}$
    $$\amalg_p =
        \begin{pmatrix}
            e & f \\
            f & g
        \end{pmatrix},$$
    $$e = \left\langle N, X_{uu} \right\rangle = \frac{det(X_u, X_v, X_{uu})}{\sqrt{EG-F^2}}$$
    $$g = \left\langle N, X_{vv} \right\rangle = \frac{det(X_u, X_v, X_{uv})}{\sqrt{EG-F^2}}$$
    $$f = \left\langle N, X_{uv} \right\rangle = \frac{det(X_u, X_v, X_{vv})}{\sqrt{EG-F^2}}$$
\end{proposition}

\begin{proposition}
    $$dN_p = -I_p^{-1} \amalg_p$$
\end{proposition}

\begin{proposition}
    $$K = \frac{eg-f^2}{EG-F^2}$$
    $$H = \frac{1}{2} \frac{eG - 2fF + gE}{EG-F^2}$$
    $K$ y $H$ son diferenciables.
\end{proposition}

\begin{remark}
    Si una parametrización de una superficie regular es tal que $F = f = 0$, entonces las matrices de $I_p$ y $\amalg_p$ son diagonales y $dN_p$ es diagonal en $\{X_u, X_v\}$.
    $$dN_p = -
        \begin{pmatrix}
            E & 0 \\
            0 & G
        \end{pmatrix}^{-1}
        \begin{pmatrix}
            e & 0 \\
            0 & g
        \end{pmatrix} = -
        \begin{pmatrix}
            \frac{e}{E} & 0           \\
            0           & \frac{g}{G}
        \end{pmatrix}$$
    Así que las curvaturas principales son $\frac{e}{E}$ y $\frac{g}{G}$.
\end{remark}

\section{Geometría intrínseca}

\begin{definition}
    $\varphi : S \to \bar{S}$ es isometría si:
    \begin{itemize}
        \item $\varphi$ es difeomorfismo.
        \item $d\varphi_p : T_pS \to T_{\varphi(p)}\bar{S}$ es isometría lineal para todo $p \in S$.
    \end{itemize}
    Se dice que $S$ y $\bar{S}$ son isométricas.
\end{definition}

\begin{remark}
    Como $\varphi$ es difeomorfismo, entonces $d\varphi_p$ es isomorfismo.
    Luego $d\varphi_p$ es isometría lineal si y solo si conserva el producto escalar.\\
    Esto es, si para cada $w_1, w_2 \in T_pS$
    $$\left\langle w_1, w_2 \right\rangle _p = \left\langle d\varphi_p(w_1), d\varphi_p(w_2) \right\rangle _{\varphi(p)}$$
\end{remark}

\begin{definition}
    Sea $V$ entorno abierto de $p$ en $S$.
    $\varphi : V \to \bar{S}$ es isometría local si existe $\bar{V}$ entorno abierto de $\varphi(p)$ en $\bar{S}$ tal que $\varphi : V \to \bar{V}$ es isometría.\\
    $S$ es localmente isométrica a $\bar{S}$ si para todo $p \in S$ existe isometría local a $\bar{S}$.
    Si además $\bar{S}$ es localmente isométrica a $S$, se dice que $S$ y $\bar{S}$ son localmente isométricas.
\end{definition}

\begin{remark}
    Ser localmente isométricas no implica que exista $\varphi : S \to \bar{S}$ isometría local, porque puede que no sea la misma en todos los puntos.
\end{remark}

\begin{proposition}
    Sean $X: U \subset \mathbb{R}^2 \to S, \bar{X}: U \subset \mathbb{R}^2 \to \bar{S}$ parametrizaciones tales que $E = \bar{E}, F = \bar{F}, G = \bar{G}$ en $U$.
    Entonces $\varphi = \bar{X} \circ X^{-1}: X(U) \to \bar{X}(U)$ es isometría, es decir, $\varphi: X(U) \to \bar{S}$ es isometría local.
\end{proposition}

\begin{proposition}
    Sea $\varphi: S \to \bar{S}$ isometría, $X: U \to S$ parametrización.
    Entonces $\bar{X} = \varphi \circ X: U \to \bar{S}$ es una parametrización de $\bar{S}$ en $\varphi(p)$ con $E = \bar{E}, F = \bar{F}, G = \bar{G}$ en $U$.
\end{proposition}

\begin{note}
    La inversa y la composición de isometrías son isometrías.
\end{note}

\begin{proposition}
    Sea $\varphi: S \to \bar{S}$ difeomorfismo.
    \begin{enumerate}
        \item $\varphi$ es isometría si y solo si $\varphi$ conserva la longitud de arco de las curvas parametrizadas en $S$.
        \item Si $\varphi$ es isometría entonces $\varphi$ conserva ángulos y áreas.
    \end{enumerate}
\end{proposition}

\begin{definition}
    $\varphi: S \to \bar{S}$ es aplicación conforme si
    \begin{itemize}
        \item $\varphi$ es difeomorfismo.
        \item Para todo $p \in S$, $w_1, w_2 \in T_pS$
              $$\left\langle d\varphi_p(w_1), d\varphi_p(w_2) \right\rangle _{\varphi(p)} = \lambda^2(p) \left\langle w_1, w_2 \right\rangle _p$$
              donde $\lambda^2$ es una función diferenciable no nula sobre $S$.
    \end{itemize}
    Se dice que $S$ y $\bar{S}$ son conformes.
\end{definition}

\begin{definition}
    $\varphi: V \to \bar{S}$, con $V$ entorno abierto de $p$ en $S$, es aplicación conforme local en $p$ si existe un entorno abierto $\bar{V}$ de $\varphi(p)$ en $\bar{S}$ tal que $\varphi: V \to \bar{V}$ es aplicación conforme.
    Si para todo $p \in S$ existe una aplicación conforme local en $p$, entonces $S$ es localmente conforme a $\bar{S}$.
\end{definition}

\begin{remark}
    Una isometría es una aplicación conforme con $\lambda(p) = 1, \forall p$.
\end{remark}

\begin{note}
    Una aplicación conforme no conserva longitudes de curvas, mientras que conserva los ángulos.
\end{note}

\begin{proposition}
    Sean $X: U \subset \mathbb{R}^2 \to S, \bar{X}: U \subset \mathbb{R}^2 \to \bar{S}$ parametrizaciones tales que $\bar{E} = \lambda^2 E, \bar{F} = \lambda^2 F, \bar{G} = \lambda^2 G$ en $U$, con $\lambda^2$ diferenciable y $\lambda^2 \neq 0$ en $U$.
    Entonces $\varphi = \bar{X} \circ X^{-1}: X(U) \to \bar{S}$ es una aplicación conforme local.
\end{proposition}

\begin{corollary}
    Dos superficies cualesquiera son localmente conformes.
\end{corollary}

\begin{proposition}
    Sea $\varphi: S \to \bar{S}$ aplicación conforme, $X: U \to S$ parametrización.
    Entonces $\bar{X} = \varphi \circ X: U \to \bar{S}$ es una parametrización de $\bar{S}$ en $\varphi(p)$ con $\bar{E} = \lambda^2 E, \bar{F} = \lambda^2 F, \bar{G} = \lambda^2 G$ en $U$.
\end{proposition}

\section{El teorema de Gauss}

\begin{proposition}
    Sea $S$ una superficie regular orientable y orientada, con orientación $N: S \to S^2$ y sea $X: U \subset \mathbb{R}^2 \to S$ una parametrización de $S$ compatible con la orientación $N$.
    Es decir, $N = \frac{X_u \land X_v}{|X_u \land X_v|}$ en $U$.
    A cada punto de $X(U)$ se le puede asignar un triedro $\{X_u, X_v, N\}$, que es una base de $\mathbb{R}^3$.
    Derivando estos vectores con respecto de $u$ y $v$ obtenemos las siguientes ecuaciones:
    $$\left\{
        \begin{array}{lcl}
            X_{uu} & = & \Gamma^1_{11}X_u + \Gamma^2_{11}X_v + eN         \\
            X_{uv} & = & \Gamma^1_{12}X_u + \Gamma^2_{12}X_v + fN = X{vu} \\
            X_{vv} & = & \Gamma^1_{22}X_u + \Gamma^2_{22}X_v + gN         \\
            N_u    & = & a_{11}X_u + a_{21}X_v                            \\
            N_v    & = & a_{12}X_u + a_{22}X_v
        \end{array}
        \right.$$
    Estas se conocen como ecuaciones de Gauss-Weingarten.
    Los $\Gamma^k_{ij}$ se denominan los símbolos de Christoffel de $S$ en la parametrización $X$.
\end{proposition}

\begin{proposition}
    Consideramos las relaciones
    $$\left\{
        \begin{array}{lcl}
            (X_{uu})_v - (X_{uv})_u & = & 0 \\
            (X_{vv})_u - (X_{vu})_v & = & 0 \\
            N_{uv} - N_{vu}         & = & 0
        \end{array}
        \right.$$
    Estas son las condiciones que hacen que el sistema de ecuaciones en derivadas parciales que definen las ecuaciones de Gauss-Weingarten sea integrable dada una condición inicial.\\
    Sustituyendo las ecuaciones de Gauss-Weingarten en estas relaciones y resolviendo llegamos a la fórmula de Gauss si $E \neq 0$:
    $$\Gamma^1_{11} \Gamma^2_{12} + \Gamma^2_{11} \Gamma^2_{22} + (\Gamma^2_{11})_v - \Gamma^1_{12} \Gamma^2_{11} - \Gamma^2_{12} \Gamma^2_{12} - (\Gamma^2_{12})_u = EK$$
\end{proposition}

\begin{remark}
    Como $E \neq 0$, la fórmula de Gauss permite obtener la curvatura de Gauss a partir de los símbolos de Christoffel, es decir, que $K$ es intrínseco.
\end{remark}

\begin{note}
    Existen otras dos versiones de la fórmula de Gauss cuando $F \neq 0$ y $G \neq 0$, respectivamente.
\end{note}

\begin{theorem}[Teorema Egregium de Gauss]
    La curvatura de Gauss de una superficie regular es invariante frente a isometrías locales.
\end{theorem}

\begin{remark}
    El recíproco del teorema Egregium no es cierto en general.
    Sin embargo, si dos superficies tienen la misma curvatura de Gauss constante, entonces dos entornos cualesquiera suficientemente pequeños de esas superficies son isométricos.
\end{remark}

\begin{note}
    Siempre es posible calcular los símbolos de Christoffel en términos de los coeficientes de la primera forma fundamental y de sus derivadas.
\end{note}

\begin{remark}
    Todos los conceptos geométricos y propiedades que se expresan en términos de los símbolos de Christoffel son invariantes por isometrías, puesto que solo dependen de la primera forma fundamental.
\end{remark}

\begin{proposition}
    Las ecuaciones de Mainardi-Codazzi son
    $$\left\{
        \begin{array}{lcl}
            e_v - f_u & = & e\Gamma^1_{12} + f(\Gamma^2_{12} - \Gamma^1_{11}) - g\Gamma^2_{11} \\
            f_v - g_u & = & e\Gamma^1_{22} + f(\Gamma^2_{22} - \Gamma^1_{12}) - g\Gamma^2_{12}
        \end{array}
        \right.$$
\end{proposition}

\begin{note}
    Las fórmulas de Gauss y las ecuaciones de Mainardi-Codazzi se conocen como ecuaciones de compatibilidad de la teoría de superficies.
\end{note}

\begin{theorem}[Teorema de Bonnet]
    Sean $E, F, G, e, f, g$ funciones diferenciables definidas en un conjunto abierto $V \subset \mathbb{R}^2$ con $E > 0$, $G > 0$, y que verifican la fórmula de Gauss y las ecuaciones de Mainardi-Codazzi y $EG-F^2 > 0$.
    Entonces para cada $q \in V$ existe un entorno abierto $U$ de $q$ en $V$ y un difeomorfismo
    $$X: U \to X(U) \subset \mathbb{R}^3$$
    tal que la superficie regular $X(U)$ tiene a $E, F, G$ y a $e, f, g$ como coeficientes de la primera y la segunda formas fundamentales respectivamente.\\
    Además, si $U$ es conexo y si
    $$\bar{X}: U \to \bar{X}(U) \subset \mathbb{R}^3$$
    es otro difeomorfismo que satisface las mismas condiciones, entonces existe un movimiento rígido directo de $\mathbb{R}^3$, $\psi: \mathbb{R}^3 \to \mathbb{R}^3$, tal que $\bar{X} = \psi \circ X$.
\end{theorem}

\section{Parametrizaciones especiales}

\begin{definition}[Parametrización ortogonal]
    Se dice que una parametrización de una superficie regular $S$, $X: U \subset \mathbb{R}^2 \to S$, $(u, v) \in U$, es ortogonal si la función $\left\langle X_u, X_v \right\rangle$ es idénticamente cero en $U$, es decir, si las curvas coordenadas de $X$ se cortan ortogonalmente en cualquier punto de $X(U)$.
\end{definition}

\begin{remark}
    En una parametrización ortogonal $F = 0$ en $X(U)$, es decir, la matriz de $I_p$ en la base $\{X_u, X_v\}_q$ de $T_pS$ es diagonal.
    $$I_p \equiv
        \begin{pmatrix}
            E & 0 \\
            0 & G
        \end{pmatrix}$$
    Esto permite obtener la siguiente expresión de la curvatura de Gauss.
    $$K = -\frac{1}{2\sqrt{EG}} \left( \left( \frac{E_v}{\sqrt{EG}}\right)_v + \left( \frac{G_u}{\sqrt{EG}}\right)_u \right) $$
\end{remark}

\begin{theorem}
    Dado un punto $p$ cualquiera de una superficie regular $S$ existe una parametrización ortogonal de $S$ en $p$.
\end{theorem}

\begin{definition}[Parametrización por líneas de curvatura]
    Una parametrización de una superficie regular $S$, $X: U \to S$, es una parametrización por líneas de curvatura si $F = f = 0$, es decir, si las curvas coordenadas son las líneas de curvatura.
\end{definition}

\begin{remark}
    En este caso, respecto a $\{X_u, X_v\}$, las matrices de $I_p$ y $\amalg_p$ son diagonales.
    $$I_p \equiv
        \begin{pmatrix}
            E & 0 \\
            0 & F
        \end{pmatrix}, \quad
        \amalg_p \equiv
        \begin{pmatrix}
            e & 0 \\
            0 & g
        \end{pmatrix}$$
    Luego, calculando $dN_p$, obtenemos que $\frac{e}{E}$ y $\frac{g}{G}$ son las curvaturas principales.
\end{remark}

\begin{proposition}
    Las curvas coordenadas de una parametrización en un entorno sin puntos umbilicales son las líneas de curvatura si y solo si $f = F = 0$ en todos los puntos del entorno.
\end{proposition}

\begin{corollary}
    Las curvas coordenadas de una parametrización en un entorno de un punto no umbilical son las líneas de curvatura si y solo si $F = f = 0$ en el entorno.
\end{corollary}

\begin{theorem}
    Sea $p$ un punto no umbilical de una superficie regular $S$.
    Entonces existe una parametrización por líneas de curvatura en un entorno de $p$.
\end{theorem}