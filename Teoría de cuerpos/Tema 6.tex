\chapter{El teorema fundamental de la teoría de Galois}
\section{Grupo de Galois}

\begin{example}
    \begin{align*}
         & Gal(\mathbb{Q}(\sqrt[3]{2})/\mathbb{Q}) = \{Id\}                                                                \\
         & Gal(\mathbb{Q}(\sqrt{2})/\mathbb{Q}) = \{ Id, \sigma \}, \text{ donde } \sigma(\sqrt{2}) = -\sqrt{2}            \\
         & Gal(\mathbb{R}/\mathbb{Q}) = \{Id\}                                                                             \\
         & Gal(\mathbb{Q}(x)/\mathbb{Q}) = \{ x \mapsto \frac{ax + b}{cx + d} : a, b, c, d \in \mathbb{Q}, ad-bc \neq 0 \}
    \end{align*}
    Observamos que $\mathbb{R}/\mathbb{Q}$ y $\mathbb{Q}(x)/\mathbb{Q}$ son extensiones infinitas cuyo grupos de Galois son respectivamente finito e infinito.
\end{example}

\begin{lemma}
    Sea $K/F$ algebraica. Entonces todo $F$-homomorfismo $\sigma : K \to K$ es $F$-automorfismo.
\end{lemma}

\begin{theorem}
    Si $K/F$ es una extensión finita, entonces $Gal(K/F)$ es un grupo finito.
\end{theorem}

\section{Subgrupos y cuerpos intermedios}

\begin{theorem}
    Sea $K/F$ una extensión de cuerpos, $E$ un cuerpo intermedio y $H$ un subgrupo de $G = Gal(K/F)$. Entonces:
    $$K^H = \{ \alpha \in K : \sigma(\alpha) = \alpha, \text{ para todo } \sigma \in H \}$$
    es un cuerpo intermedio de $K/F$ y
    $$Gal(K/E) = \{ \sigma \in G : \sigma(\alpha) = \alpha, \text{ para todo } \alpha \in E \}$$
    es un subgrupo de $G$.
\end{theorem}

\begin{example}
    Sea $G$ el grupo de Galois de $\mathbb{Q}(\sqrt{2}, \sqrt{3})$ sobre $\mathbb{Q}$.
    Podemos comprobar que $G$ es un grupo de orden 4.
    Como sus elementos están completamente determinados por su acción en $\sqrt{2}$ y $\sqrt{3}$, podemos describirlos así:
    \begin{align*}
        \sigma_1           & = Id                                           \\
        \sigma_2(\sqrt{2}) & = \sqrt{2}  & \sigma_2(\sqrt{3}) & = -\sqrt{3} \\
        \sigma_3(\sqrt{2}) & = -\sqrt{2} & \sigma_3(\sqrt{3}) & = \sqrt{3}  \\
        \sigma_4(\sqrt{2}) & = -\sqrt{2} & \sigma_4(\sqrt{3}) & = -\sqrt{3}
    \end{align*}
    Observamos que $\sigma_2, \sigma_3, \sigma_4$ son elementos de orden 2 en $G$.
    Luego $G$ es isomorfo al grupo $C_2 \times C_2$.\\
    El subgrupo de $G$ asociado al cuerpo intermedio $\mathbb{Q}(\sqrt{3})$ es:
    $$Gal(\mathbb{Q}(\sqrt{2}, \sqrt{3}) / \mathbb{Q}(\sqrt{3}))$$
    el grupo de $\mathbb{Q}(\sqrt{3})$-automorfismos de $\mathbb{Q}(\sqrt{2}, \sqrt{3})$. Este es $< \sigma_3 >$.\\
    Por otro lado, sea $H$ el subgrupo generado por $\sigma_2 : H = < \sigma_2 >$.
    El correspondiente cuerpo intermedio es:
    $$\mathbb{Q}(\sqrt{2}, \sqrt{3})^H = \{ \alpha \in \mathbb{Q}(\sqrt{2}, \sqrt{3}) : \sigma(\alpha) = \alpha, \text{ para todo } \sigma \in H \} = \{ \alpha \in \mathbb{Q}(\sqrt{2}, \sqrt{3}) : \sigma_2(\alpha) = \alpha \}$$
    Obervamos que $\mathbb{Q}(\sqrt{2}) \subseteq \mathbb{Q}(\sqrt{2}, \sqrt{3})^H$.
    Se puede demostrar que se da la igualdad considerando la acción de $\sigma_2$ en cada elemento de $\mathbb{Q}(\sqrt{2}, \sqrt{3})$, recordando que son de la forma $a + b\sqrt{2} + c\sqrt{3} + d\sqrt{2}\sqrt{3}$, con $a, b, c, d \in \mathbb{Q}$.
\end{example}

\begin{definition}
    Sea $K/F$ una extensión de cuerpos y $H$ un subgrupo de $Gal(K/F)$.
    El cuerpo intermedio $K^H$ se llama subcuerpo de $K$ fijo por $H$.
\end{definition}

\section{Extensiones de Galois}

\begin{definition}
    Una extensión de Galois es una extensión de cuerpos $K/F$ tal que $K^G = F$, donde $G = Gal(K/F)$.
\end{definition}

\begin{example}
    El grupo de Galois de $\mathbb{Q}(\sqrt[3]{2})$ sobre $\mathbb{Q}$ es trivial.
    Luego todos los elementos de $\mathbb{Q}(\sqrt[3]{2})$ son fijos por todos los elementos del grupo de Galois, es decir:
    $$\mathbb{Q}(\sqrt[3]{2})^G = \mathbb{Q}(\sqrt[3]{2})$$
    y por tanto $\mathbb{Q}(\sqrt[3]{2})$ no es una extensión de Galois de $\mathbb{Q}$.
\end{example}

\begin{example}
    Veamos si $\mathbb{Q}(\sqrt[4]{2})/\mathbb{Q}$ es una extensión de Galois.\\
    Su grupo de Galois $G$ tiene dos elementos: la identidad y el elemento $\sigma$ determinado por:
    $$\sigma(\sqrt[4]{2}) = -\sqrt[4]{2}$$
    Observamos que $\sqrt{2} \in \mathbb{Q}(\sqrt[4]{2})$ es fijo por $Id$ y $\sigma$, así que $\mathbb{Q}(\sqrt{2}) \subseteq \mathbb{Q}(\sqrt[4]{2})^G$ y por tanto la extensión no es de Galois.
\end{example}

\begin{theorem}
    Sea $K/F$ una extensión algebraica. Entonces son equivalentes:
    \begin{enumerate}
        \item $K/F$ es una extensión de Galois.
        \item $K/F$ es normal y separable.
    \end{enumerate}
\end{theorem}

\section{El teorema fundamental}

\begin{proposition}
    Sea $K/F$ algebraico. Si $K/F$ es de Galois y $E$ es un cuerpo intermedio, entonces $K/E$ es de Galois.
\end{proposition}

\begin{example}
    La extensión $\mathbb{Q}(i, \sqrt{3}, \sqrt[3]{2})/\mathbb{Q}$ es de Galois pero $\mathbb{Q}(\sqrt[3]{2})/\mathbb{Q}$ no es de Galois.
\end{example}

\begin{theorem}
    Sea $K/F$ una extensión finita. Entonces:
    \begin{enumerate}
        \item $|Gal(K/F)|$ divide a $[K : F]$.
        \item $K/F$ es de Galois si y solo si $|Gal(K/F)| = [K : F]$.
    \end{enumerate}
\end{theorem}

\begin{theorem}
    Sea $K/F$ una extensión finita y $E$ un cuerpo intermedio. Si $K/F$ es de Galois, son equivalentes:
    \begin{enumerate}
        \item $E = \sigma E$ para todo $\sigma \in Gal(K/F)$.
        \item $Gal(K/E)$ es un subgrupo normal de $Gal(K/F)$.
        \item $E/F$ es de Galois.
        \item $E/F$ es una extensión normal.
    \end{enumerate}
\end{theorem}

\begin{theorem}
    Sean $F \subset E \subset K$ extensiones finitas con $E/F$ y $K/F$ de Galois.
    Entonces $Gal(K/E)$ es un subgrupo normal de $Gal(K/F)$ y existe un isomorfismo de grupos:
    $$Gal(K/F)/Gal(K/E) \equiv Gal(E/F)$$
\end{theorem}

\begin{theorem}[Teorema fundamental]
    Sea $K/F$ una extensión finita de Galois.
    Existe una correspondencia uno a uno entre el conjunto de cuerpos intermedios de $K/F$ y el conjunto de subgrupos del grupo de Galois $G = Gal(K/F)$.
    Esta correspondencia viene dada por $E \mapsto Gal(K/E)$ y satisface las siguientes condiciones:
    \begin{enumerate}
        \item Si $F \subseteq E \subseteq K$, entonces $K/E$ es una extensión de Galois y su grupo de Galois $Gal(K/E)$ es un subgrupo de $G = Gal(K/F)$. Además, $$[K : E] = |Gal(K/E)| \text{ y } |Gal(K/F) : Gal(K/E)| = [E : F]$$
        \item $E/F$ es de Galois si y solo si $Gal(K/E)$ es un subgrupo normal de $G = Gal(K/F)$. En este caso, $Gal(E/F)$ es isomorfo a $G/Gal(K/E)$.
    \end{enumerate}
\end{theorem}