\chapter{Cuerpos finitos}

\begin{proposition}
    Sea $F$ un cuerpo finito. Entonces:
    \begin{enumerate}
        \item La característica de $F$ es un primo $p$.
        \item $F$ es una extensión finita de su cuerpo primo $\pi(F)$ (isomorfo a $\mathbb{Z}_p$) y $$|F| = p^n, \text{ donde } n = [F : \pi(F)]$$
    \end{enumerate}
\end{proposition}

\begin{proposition}
    Sea $F$ un cuerpo finito con $q = p^n$ elementos. Entonces:
    \begin{enumerate}
        \item $\alpha^q = \alpha$ para todo $\alpha \in F$.
        \item $x^q - x = \prod_{\alpha \in F} (x-\alpha)$.
        \item $F$ es un cuerpo de descomposición sobre $\pi(F)$ de $x^q - x$.
    \end{enumerate}
\end{proposition}

\begin{example}
    $$x^3-x = x(x-1)(x-2) \in \mathbb{Z}_3[x]$$
    La factorización de $x^9-x$ en factores irreducibles en $\mathbb{Z}_3[x]$ es:
    $$x^9-x = x(x-1)(x+1)(x^2+1)(x^2+x+2)(x^2+2x+2) \in \mathbb{Z}_3[x]$$
    La proposición previa afirma que si $F$ es un cuerpo finito con 9 elementos, entonces sus elementos serán las raíces de $x^9-x$.
    Para expresar  cada una de las raíces de $x^9-x$ contruimos un cuerpo finito de 9 elementos como un cuerpo stem de un polinomio irreducible de grado 2 sobre $\mathbb{Z}_3[x]$.
    Por ejemplo, consideramos el polinomio irreducible $x^2+1 \in \mathbb{Z}_3[x]$. Entonces:
    \begin{center}
        $\mathbb{F} = \mathbb{Z}_3[x]/(x^2+1)$ es una extensión de $\mathbb{Z_3}$ de grado 2.
    \end{center}
    Recordamos que este cociente es isomorfo a $\mathbb{Z}_3(\alpha)$ para cada raíz $\alpha$ de $x^2+1$.
    Como $\{ 1, \alpha \}$ es una base de $\mathbb{Z}_3(\alpha)$ sobre $\mathbb{Z}$, tenemos que los 9 elementos de $F = \mathbb{Z}_3(\alpha)$ son:
    $$0, \; 1, \; 2, \; \alpha, \; 1+\alpha, \; 2+\alpha, \; 2\alpha, \; 1+2\alpha, \; 2+2\alpha$$
\end{example}

\begin{corollary}
    Sea $p$ primo y $n$ un entero positivo. Entonces:
    \begin{enumerate}
        \item Existe un cuerpo finito con $p^n$ elementos.
        \item Dos cuerpos finitos con $p^n$ elementos son isomorfos.
    \end{enumerate}
\end{corollary}

\begin{corollary}
    El polinomio $x^{p^n}-x \in \mathbb{Z}_p[x]$ es el producto de todos los polinomios irreducibles mónicos de $\mathbb{Z}_p[x]$ de grado $d$, con $d$ divisor de $n$.
\end{corollary}

\begin{example}
    Descomponemos el polinomio $x^{5^2} - x$ sobre $\mathbb{Z}_5$:
    \begin{align*}
         & x(x+1)(x+2)(x+3)(x+4)(x^2+2)(x^2+3)      \\
         & (x^2+x+1)(x^2+x+2)(x^2+2x+3)(x^2+2x+4)   \\
         & (x^2+3x+3)(x^2+3x+4)(x^2+4x+1)(x^2+4x+2)
    \end{align*}
\end{example}

\section{Extensiones finitas de cuerpos finitos}

\begin{proposition}
    Sea $F$ un cuerpo finito con $q = p^m$ elementos y $n$ un entero positivo.
    Entonces existe una extensión finita $K$ de grado $n$ sobre $F$ y es única salvo $F$-isomorfismos.
\end{proposition}

\begin{corollary}
    Si $F$ es un cuerpo finito y $n$ es un entero positivo, existe un polinomio irreducible $f \in F[x]$ de grado $n$.
\end{corollary}

\begin{theorem}
    Sea $F$ un cuerpo finito con $p^m$ elementos y $K/F$ una extensión finita de grado $[K : F] = n$.
    Entonces $K/F$ es de Galois y su grupo de Galois es:
    $$G \equiv \mathbb{Z}/n\mathbb{Z}$$
    Además, $G$ es generado por:
    \begin{align*}
        \sigma : K & \to K            \\
        \alpha     & \mapsto \alpha^q
    \end{align*}
\end{theorem}