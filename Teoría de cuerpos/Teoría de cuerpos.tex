\documentclass{report}
\usepackage[spanish]{babel}
\usepackage{amssymb, amsmath, amsthm, hyperref}

\title{Teoría de cuerpos}
\author{}

\newtheorem{theorem}{Teorema}[chapter]
\newtheorem{corollary}[theorem]{Corolario}
\newtheorem{lemma}[theorem]{Lema}
\newtheorem{proposition}[theorem]{Proposición}

\theoremstyle{remark}
\newtheorem*{remark}{Observación}

\theoremstyle{definition}
\newtheorem{definition}{Definición}[chapter]
\theoremstyle{definition}
\theoremstyle{definition}
\newtheorem*{example}{Ejemplo}

\begin{document}
\maketitle
\tableofcontents

\chapter{Extensiones de cuerpos}
\section{Introducción}

\begin{definition}
    Un anillo es un conjunto no vacío $R$ con dos operaciones internas tal que:
    \begin{enumerate}
        \item $(R, +)$ es un grupo abeliano.
        \item La multiplicación es asociativa.
        \item $a(b + c) = ab + ac$ y $(a + b)c = ac + bc$ para todo $a, b, c \in R$.
    \end{enumerate}
    Si además la multiplicación es conmutativa entonces $R$ es un anillo conmutativo.
    Si $R$ contiene un elemento neutro para la multiplicación, entonces $R$ es un anillo unitario.
\end{definition}

\begin{example}
    $\mathbb{Z}$, $\mathbb{Z}_n$ y $\mathbb{Q}[x]$ son anillos conmutativos unitarios.
\end{example}

\begin{definition}
    Sea $R$ un anillo conmutativo unitario, con neutro no nulo.
    Un elemento $a \in R$ no nulo es un divisor de cero si existe un elemento $b \in R$ no nulo tal que $ab = 0$.\\
    $R$ se llama dominio de integridad si no tiene divisores de cero.
    Si además todo elemento no nulo es invertible, entonces $R$ es un cuerpo.
\end{definition}

\begin{example}
    $\mathbb{Z}_n$ es un dominio de integridad si y solo si $n$ es primo.
\end{example}

\begin{example}
    $\mathbb{Z}$ es un dominio de integridad, pero no todo elemento de $\mathbb{Z}$ es invertible. Luego $\mathbb{Z}$ no es un cuerpo.\\
    En $\mathbb{Q}[x]$, no todo polinomio tiene una inversa. Sin embargo, como $\mathbb{Q}[x]$ es un dominio de integridad, podemos construir su cuerpo de fracciones.
    $$\mathbb{Q}(x) = \{ \frac{p(x)}{q(x)} : p, q \in \mathbb{Q}[x], q(x) \neq 0 \}$$
    Este es el cuerpo más pequeño que contiene a $\mathbb{Q}[x]$.
\end{example}

\begin{definition}
    Sea $R$ un anillo. Un subconjunto $S$ de $R$ no vacío es un subanillo si es cerrado bajo las operaciones de $R$ y $S$ es un anillo.\\
    Si $K$ es un cuerpo, entonces un subconjunto $F$ de $K$ no vacío es un subcuerpo si, bajo las operaciones de $K$, $F$ es un cuerpo. El elemento neutro de $F$ será $1_K$.\\
    El subcuerpo más pequeño de $K$ se llama cuerpo primal de $K$.
\end{definition}

\begin{example}
    El cuerpo primal de $\mathbb{R}$ o $\mathbb{C}$ es $\mathbb{Q}$.
\end{example}

\section{Extensiones de cuerpos}

\begin{definition}
    Sean $F$ y $K$ cuerpos. Decimos que $K$ es una extensión de $F$ cuando $F$ es un subcuerpo de $K$. Lo denotamos $K/F$.
    Se tiene que $(K, +, *_F)$ es un espacio vectorial sobre $F$.
\end{definition}

\begin{definition}
    Un espacio vectorial es un conjunto $V$ con un cuerpo $F$ y dos operaciones tales que:
    \begin{enumerate}
        \item $(V, +)$ es un grupo abeliano.
        \item La multiplicación escalar $*_F : F \times V \to V$ satisface las siguientes propiedades:
              \begin{enumerate}
                  \item $a(v + w) = av + aw$, para todo $a \in F, v, w \in V$.
                  \item $(a + b)v = av + bv$, para todo $a, b \in F, v \in V$.
                  \item $(ab)v = a(bv)$, para todo $a, b \in F, v \in V$.
                  \item $1_Fv = v$ para todo $v \in V$.
              \end{enumerate}
    \end{enumerate}
\end{definition}

\begin{example}
    $\mathbb{C}$ es una extensión de cuerpos de $\mathbb{R}$. Como $\mathbb{C}$ es un espacio vectorial sobre $\mathbb{R}$, ha de tener una base.
    En efecto, los elementos 1 e $i$ son linealmente independientes sobre $\mathbb{R}$ y constituyen una base de $\mathbb{C}$.\\
    Por tanto, $\mathbb{C}/\mathbb{R}$ es un espacio vectorial de dimensión 2.
\end{example}

\begin{definition}
    Sea $K/F$ una extensión de cuerpos. El grado de $K/F$, denotado por $[K : F]$, es la dimensión de $K$ como espacio vectorial sobre $F$.\\
    Si el grado de $K/F$ es finito, decimos que la extensión es finita. En caso contrario, decimos que la extensión es infinita.
\end{definition}

\begin{example}
    Consideramos de nuevo el cuerpo $\mathbb{Q}(x)$, que es el cuerpo de fracciones de $\mathbb{Q}[x]$. Este es una extensión de cuerpos de $\mathbb{Q}$.\\
    En el espacio vectorial $\mathbb{Q}(x)$ sobre $\mathbb{Q}$, el conjunto $\{1, x, x^2, \dots\}$ es linealmente independiente, con infinitos elementos.
    Por tanto, no existe una base finita, luego la extensión $\mathbb{Q}(x)/\mathbb{Q}$ es infinita.
\end{example}

\begin{remark}
    Una extensión de cuerpos $K/F$ tiene grado 1 si y solo si $K = F$.\\
    Si el grado es 1, todo elemento no nulo de $K$ es una base sobre $K$, en particular $1_K = 1_F$. Por tanto, $K = \{ a1_F : a \in F \}  = F$.
\end{remark}

\begin{theorem}[Teorema de la torre]
    Sea $F \subseteq K \subseteq L$ una sucesión de extensiones de cuerpos.
    \begin{enumerate}
        \item Si $[K : F] = \infty$ o $[L : K] = \infty$, entonces $[L : F] = \infty$.
        \item Si $[K : F] < \infty$ y $[L : K] < \infty$. Entonces $[L : F] = [L : K] [K : F]$.
    \end{enumerate}
\end{theorem}

\section{Elementos algebraicos y trascendentes}

\begin{definition}
    Sea $K/F$ una extensión de cuerpos y $\alpha \in K$. Decimos que $\alpha$ es algebraico sobre $F$ si existe un polinomio $f \in F[x]$ no constante tal que $f(\alpha) = 0$.\\
    Si $\alpha$ no es algebraico sobre $F$, entonces decimos que es trascendente sobre $F$.
\end{definition}

\begin{example}
    $\sqrt{2} \in \mathbb{R}$ es algebraico sobre $\mathbb{Q}$, puesto que es una raíz de $x^2 - 2 \in \mathbb{Q}[x]$.
\end{example}

\begin{example}
    Veamos que $\sqrt{2} + \sqrt{3}$ es también algebraico sobre $\mathbb{Q}$. Sea $\alpha = \sqrt{2} + \sqrt{3}$.
    \begin{align*}
        (\alpha - \sqrt{2})^2              & = 3               \\
        \alpha^2 - 2\sqrt{2}\alpha + 2 - 3 & = 0               \\
        \alpha^2 - 1                       & = 2\sqrt{2}\alpha \\
        \alpha^4 + 1 - 2\alpha^2           & = 8\alpha^2       \\
        \alpha^4 - 10\alpha^2 + 1          & = 0
    \end{align*}
    Luego $\alpha = \sqrt{2} + \sqrt{3}$ es raíz de $x^4 - 10x^2 + 1$.
\end{example}

\begin{example}
    $\pi$ es trascendente sobre $\mathbb{Q}$. Sin embargo, $\pi$ es algebraico sobre $\mathbb{R}$, pues es raíz de $x - \pi \in \mathbb{R}[x]$.
\end{example}

\begin{definition}
    Se dice que una extensión de cuerpos es algebraica cuando cada elemento en $K$ es algebraico sobre $F$.
    De lo contrario, se dice que la extensión es trascendente.
\end{definition}

\begin{example}
    $\mathbb{R}/\mathbb{Q}$ es trascendente puesto que $\mathbb{R}$ tiene elementos transcendentes sobre $\mathbb{Q}$.\\
    Sin embargo, $\mathbb{C}/\mathbb{R}$ es algebraico, porque todo elemento complejo $a + bi$ es raíz del polinomio:
    $$(x - (a+bi))(x - (a-bi)) = x^2 - 2ax + a^2 + b^2 \in \mathbb{R}[x]$$
\end{example}

\begin{remark}
    En una extensión de cuerpos $K/F$ cada elemento de $F$ es algebraico sobre $F$.
    Además, si $\alpha \in K$ es algebraico sobre $F$, entonces también es algebraico sobre todo cuerpo $F'$ entre $F$ y $K$.
\end{remark}

\section{Nociones de anillos de polinomios}

\begin{theorem}[Algoritmo de la división]
    Sea $R$ un anillo unitario y $f, g \in \mathbb{R}[x]$ polinomios no nulos tales que el coeficiente líder de $g$ sea un elemento neutro de $R$.
    Entonces existen dos únicos polinomios $q, r \in \mathbb{R}[x]$ tales que:
    $$f(x) = q(x)g(x) + r(x), \quad\text{con } r(x)=0 \text{ o } grad(r(x)) < grad(g(x))$$
\end{theorem}

\begin{corollary}[Algoritmo de Euclides e identidad de Bezout]
    Sea $F$ un campo, $f, g \in F[x]$ polinomios no constantes y $d = MCD(f, g)$. Entonces existen polinomios $a, b \in F[x]$ tales que:
    $$d(x) = a(x)f(x) + b(x)g(x)$$
\end{corollary}

\begin{corollary}
    Todo ideal en $F[x]$ es principal, es decir, $F[x]$ es un dominio de ideales maximales.
\end{corollary}

\begin{corollary}
    Si $f$ es irreducible, entonces el ideal $(f)$ es maximal en $F[x]$.
\end{corollary}

\section{Polinomio mínimo}

\begin{proposition}
    Si $\alpha \in K$ es algebraico sobre $F$, entonces existe un único polinomio mónico $f \in F[x]$ tal que:
    \begin{enumerate}
        \item $f(\alpha) = 0$.
        \item Si $g \in F[x]$ y $g(\alpha) = 0$, entonces $f$ divide a $g$ en $F[x]$.
    \end{enumerate}
\end{proposition}

\begin{definition}
    Dicho polinomio se llama polinomio mínimo de $\alpha$ sobre $F$.
\end{definition}

\begin{proposition}
    Sea $\alpha \in K$ algebraico sobre $F$ y $f \in F[x]$ un polinomio mónico no constante.
    Entonces las siguientes afirmaciones son equivalentes.
    \begin{enumerate}
        \item $f$ es el polinomio mínimo de $\alpha$ sobre $F$.
        \item $f$ tiene grado mínimo entre los polinomios con raíz $\alpha$.
        \item $f$ es irreducible y $f(\alpha) = 0$.
    \end{enumerate}
\end{proposition}

\begin{example}
    Trabajaremos frecuentemente con el elemento $\xi_n = e^{\frac{2\pi i}{n}}$. Es claro que $\xi_n^n = 1$, así que podemos decir que $\xi_n$ es algebraico sobre $\mathbb{Q}$ y que su polinomio mínimo es un factor irreducible de $x^n - 1$.
    El polinomio mínimo de $\xi_n$ sobre $\mathbb{Q}$ se llama polinomio ciclotómico de orden $n$. Si $n$ es primo, este polinomio ciclotómico es:
    $$x^{n-1} + x^{n-2} + \dots + x + 1$$
\end{example}

\begin{example}
    Hemos visto que $f(x) = x^4 - 10x^3 + x^2 + 1$ se anula en $\sqrt{2} + \sqrt{3}$. Si comprobamos que $f$ es irreducible, podremos afirmar que este es el polinomio mínimo de $\sqrt{2} + \sqrt{3}$ sobre $\mathbb{Q}$.
\end{example}

\section{Construcción de extensiones}

\begin{definition}
    Sea $K/F$ una extensión de cuerpos y $X$ un subconjunto de $K$.
    El menor subcuerpo de $K$ que contiene a $F \bigcup X$ se denota por $F(X)$ y se llama subcuerpo de $K$ generado por $X$ sobre $F$.
\end{definition}

\begin{definition}
    El subanillo más pequeño de $K$ que contiene a $F \bigcup X$ se denota por $F[X]$.
    Siempre se tiene que $F[X] \subseteq F(X) \subseteq K$ y $F[X]$ es un dominio de integridad.
\end{definition}

\begin{definition}
    Si $X$ es finito, $X = \{u_1, \dots, u_n\}$, escribimos $F(X) =\\F(u_1, \dots, u_n)$ y decimos que la extensión es finitamente generada sobre $F$.\\
    Una extensión de cuerpos de la forma $F(u)$ se llama extensión simple.
\end{definition}

\begin{proposition}
    Sea $K/F$ una extensión de cuerpos, $u, u_i \in K, X \subseteq K$.
    \begin{enumerate}
        \item $F[u] = \{ f(u) : f(x) \in F[x] \}$
        \item $F[u_1, \dots, u_n] = \{ f(u_1, \dots, u_n) : f(x_1, \dots, x_n) \in F[x_1, \dots, x_n] \}$
        \item $F[X] = \{ f(u_1, \dots, u_n) : n \in N, u_i \in X, f(x_1, \dots, x_n) \in F[x_1, \dots, x_n] \}$
        \item $F(u) = \{ \frac{f(u)}{g(u)} : f(x), g(x) \in F[x], g(x) \neq 0 \}$
        \item $F(u_1, \dots, u_n) = \{ \frac{a}{b} : a, b \in F[u_1, \dots, u_n], b \neq 0 \}$
        \item $F(X) = \{ \frac{a}{b} : a, b \in F[x], b \neq 0 \}$
    \end{enumerate}
\end{proposition}

\begin{example}
    Consideramos el anillo $\mathbb{Q}[\sqrt{3}]$.
    $$\mathbb{Q}[\sqrt{3}] = \{ f(\sqrt{3}) : f(x) \in \mathbb{Q}[x] \}$$
    Sea $f(x) \in \mathbb{Q}[x]$. Entonces podemos dividir $f(x)$ por el polinomio mínimo de $\sqrt{3}$ sobre $\mathbb{Q}$, $x^2 - 3$. De esta forma, obtenemos polinomios $q(x), r(x) \in \mathbb{Q}[x]$ únicos, con $grad(r(x)) < 2$, tales que:
    $$f(x) = (x^2 - 3)q(x) + r(x)$$
    Entonces, existen $a, b \in \mathbb{Q}$ tales que $r(x) = a + bx$. Evaluando en $\sqrt{3}$:
    $$f(\sqrt{3}) = 0 + r(\sqrt{3})$$
    Por tanto:
    $$\mathbb{Q}[\sqrt{3}] = \{ a + b\sqrt{3} : a, b \in \mathbb{Q} \}$$
\end{example}

\begin{proposition}
    Sea $K/F$ una extensión de cuerpos. Si $\alpha \in K$ es algebraico sobre $F$, entonces:
    \begin{enumerate}
        \item $F[\alpha] = F(\alpha)$.
        \item $\{1, \alpha, \dots, \alpha^{n-1} \}$ es una base de $F(\alpha)$ sobre $F$, donde $n$ es el grado del polinomio mínimo de $\alpha$ sobre $F$.
        \item $[F(\alpha) : F] = n$.
    \end{enumerate}
\end{proposition}

\begin{example}
    $\mathbb{Q}(\sqrt{3})$ es una extensión de cuerpos de $\mathbb{Q}$. Tiene grado 2 y una base es $\{ 1, \sqrt{3} \}$.
\end{example}

\begin{example}
    $\mathbb{Q}(\sqrt{2}, \sqrt{3}) = \mathbb{Q}(\sqrt{2})(\sqrt{3}) = \mathbb{Q}(\sqrt{3})(\sqrt{2}).$
    $$\mathbb{Q} \subseteq \mathbb{Q}(\sqrt{2}) \subseteq \mathbb{Q}(\sqrt{2})(\sqrt{3})$$
    Por el teorema de la torre:
    $$[\mathbb{Q}(\sqrt{2}, \sqrt{3}) : \mathbb{Q}] = [\mathbb{Q}(\sqrt{2}, \sqrt{3}) : \mathbb{Q}(\sqrt{2})] [\mathbb{Q}(\sqrt{2}) : \mathbb{Q}]$$
    Además, una base para $\mathbb{Q}(\sqrt{2}, \sqrt{3})$ sobre $\mathbb{Q}$ es $\{ 1, \sqrt{2}, \sqrt{3}, \sqrt{6} \}$.
\end{example}

\begin{example}
    $\mathbb{Q}(\pi)$ es una extensión finitamente generada sobre $\mathbb{Q}$.
    Sin embargo, no es una extensión finita, puesto que $\{ 1, \pi, \pi^2, \dots \}$ es linealmente independiente.
    En caso contrario, $\pi$ sería algebraico sobre $\mathbb{Q}$.
\end{example}

\section{Extensiones algebraicas}

\begin{proposition}
    Sea $K/F$ una extensión finita. Entonces:
    \begin{enumerate}
        \item $K/F$ es algebraica.
        \item El grado del polinomio mínimo de $\alpha \in K$ sobre $F$ divide a $[K : F]$.
    \end{enumerate}
\end{proposition}

\begin{remark}
    No toda extensión algebraica es finita. Consideramos por ejemplo:
    $$\bar{\mathbb{Q}} = \{ \alpha : \alpha \text{ es algebraico sobre } \mathbb{Q} \}$$
    Esta es una extensión algebraica infinita de $\mathbb{Q}$.
\end{remark}

\begin{proposition}
    Sea $K/F$ una extensión de cuerpos. Entonces $[K : F] < \infty$ si y solo si existen $\alpha_1, \dots, \alpha_m \in K$ algebraicos sobre $F$ tales que $K = F(\alpha_1, \dots, \alpha_m)$.
\end{proposition}

\begin{proposition}
    Dada una extensión de cuerpos $K/F$, el siguiente subconjunto es un cuerpo intermedio de $K/F$:
    $$M = \{ \alpha \in K : \alpha \text{ es algebraico sobre } F \}$$
\end{proposition}

\begin{proposition}
    Sea $F \subseteq K \subseteq L$ una sucesión de cuerpos. Si $\alpha \in L$ es algebraico sobre $K$ y $K$ es algebraico sobre $F$, entonces $\alpha$ es algebraico sobre $F$.
\end{proposition}

\chapter{Homomorfismos de cuerpos}
\section{Homomorfismos de anillos y cuerpos}

\begin{definition}
    Sean $R$ y $S$ anillos. Una aplicación $f : R \to S$ es un homomorfismo de anillos si para todo $a, b \in R$ se verifica:
    \begin{enumerate}
        \item $f(a + b) = f(a) + f(b)$
        \item $f(ab) = f(a) f(b)$
    \end{enumerate}
\end{definition}

\begin{example}
    Consideramos el homomorfismo de anillos:
    $$\varphi : \mathbb{Q}[x] \to \mathbb{Q}[\sqrt{3}]$$
    Esta aplicación deja fijo a todo número racional y envía $x$ a $\sqrt{3}$.
    Entonces, si $p(x) = a_0 + a_1x + \dots + a_mx^m \in \mathbb{Q}[x]$:
    \begin{align*}
        \varphi(p(x)) & = \varphi(a_0 + a_1x + \dots + a_mx^m)                                     \\
                      & = \varphi(a_0) + \varphi(a_1x) + \dots + \varphi(a_mx^m)                   \\
                      & = \varphi(a_0) + \varphi(a_1)\varphi(x) + \dots + \varphi(a_m)\varphi(x^m) \\
                      & = a_0 + a_1\sqrt{3} + \dots + a_m\sqrt{3^m} = p(\sqrt{3})
    \end{align*}
    Su núcleo es el conjunto:
    $$Ker(\varphi) = \{ f(x) \in \mathbb{Q}[x] : f(\sqrt{3}) = 0 \} = (x^2 - 3)$$
    Como $\varphi$ is sobreyectiva, por el primer teorema de isomorfía tenemos que:
    $$\mathbb{Q}[\sqrt{3}] \equiv \mathbb{Q}[x] / (x^2 - 3)$$
    Como $x^2 - 3$ es irreducible y $\mathbb{Q}[x]$ es un dominio de ideales principales, entonces $(x^2 - 3)$ es un ideal maximal y por tanto $\mathbb{Q}[\sqrt{3}]$ es un cuerpo.
\end{example}

\begin{remark}
    Si $f$ es un homomorfismo de anillos, entonces $f(0) = 0$. Como siempre trabajaremos con anillos conmutativos unitarios, consideraremos homomorfismos entre anillos unitarios y añadiremos la condición $f(1_R) = 1_S$.
\end{remark}

\begin{definition}
    Sean $K_1, K_2$ cuerpos. Un homomorfismo de cuerpos de $K_1$ a $K_2$ es una aplicación $f : K_1 \to K_2$ tal que para todo $a, b \in R$ se verifica:
    \begin{enumerate}
        \item $f(a+b) = f(a) + f(b)$
        \item $f(ab) = f(a) f(b)$
        \item $f(1_{K_1}) = 1_{K_2}$
    \end{enumerate}
\end{definition}

\begin{remark}
    Si $f : K_1 \to K_2$ es un homomorfismo de cuerpos, entonces $f(-a) = -f(a)$ para todo $a \in K_1$.
    Si además $a$ es un elemento no nulo, entonces $f(a^{-1}) = f(a)^{-1}$.
\end{remark}

\begin{example}
    Consideramos el isomorfismo de anillos del ejemplo anterior.
    $$\mathbb{Q}[x] / (x^2 - 3) \equiv \mathbb{Q}(\sqrt{3})$$
    Observamos que es de hecho un isomorfismo de cuerpos que identifica cada clase del cociente $\mathbb{Q}[x] / (x^2 - 3)$ con un elemento de $\mathbb{Q}(\sqrt{3})$.
    En efecto, para cada polinomio $p(x) \in \mathbb{Q}[x]$, su clase $p(x) + (x^2 -3)$ puede ser representada por el resto de la división de $p(x)$ por $x^2 - 3$, que será un polinomio de la forma $a + bx$, con $a, b \in \mathbb{Q}$.
    Entonces, el elemento correspondiente en $\mathbb{Q}(\sqrt{3})$ es $\varphi(a + bx) = a + b\sqrt{3}$.
\end{example}

\begin{remark}
    Todo homomorfismo de cuerpos es inyectivo, es decir, todo homomorfismo de cuerpos es un monomorfismo.
\end{remark}

\section{F-homomorfismos}

\begin{definition}
    Sean $K/F$, $K'/F$ extensiones de cuerpos.
    Un $F$-homomorfismo de $K$ a $K'$ es un homomorfismo de cuerpos $\varphi : K \to K'$ tal que $\varphi(a) = a$ para todo $a \in F$.
    Podemos definir análogamente los conceptos de $F$-endomorfismo, $F$-isomorfismo, $F$-monomorfismo y $F$-automorfismo.
\end{definition}

\begin{theorem}
    Sea $K/F$ una extensión de cuerpos, $u \in K$ trascendente sobre $F$. Entonces existe un $F$-isomorfismo entre $F(u)$ y $F(x)$.
\end{theorem}

\begin{example}
    Existe un $\mathbb{Q}$-isomorfismo entre $\mathbb{Q}(\pi)$ y $\mathbb{Q}(x)$ que identifica $\pi$ con $x$.
\end{example}

\begin{theorem}
    Sea $K/F$ una extensión de cuerpos, $\alpha \in K$ algebraico sobre $F$ y $f(x) \in F[x]$ el polinimio mínimo de $\alpha$ sobre $F$. Entonces existe un $F$-isomorfismo entre $F(\alpha)$ y $F[x]/(f(x))$.
\end{theorem}

\begin{example}
    Sea $p$ un número primo, $\zeta_p = e^{2\pi i / p}$.
    Entonces $\mathbb{Q}(\zeta_p)$ es isomorfo a $\mathbb{Q}[x]/(x^{p-1} + x^{p-2} + \dots + x + 1)$.
    El isomorfismo fija a los números racionales e identifica $\zeta_p$ con la clase del polinomio $x$ módulo el ideal $(x^{p-1} + x^{p-2} + \dots + x + 1)$.
\end{example}

\section{Extensiones de monomorfismos de anillos}

\begin{definition}
    Sea $\sigma : F_1 \to F_2$ un homomorfismo de cuerpos y
    $$f(x) = a_nx^n + a_{n-1}x^{n-1} + \dots + a_1x + a_0 \in F_1[x]$$
    Denotaremos por $f^\sigma(x)$ al polinomio
    $$f^\sigma(x) = \sigma(a_n)x^n + \sigma(a_{n-1})x^{n-1} + \dots + \sigma(a_1)x + \sigma(a_0) \in F_2[x]$$
\end{definition}

\begin{remark}
    Recordamos que todo homomorfismo de cuerpos es un monomorfismo.
    Los monomorfismos de cuerpos también se llaman inmersiones.
\end{remark}

\begin{theorem}
    Sea $\sigma : F_1 \to F_2$ un isomorfismo de cuerpos.
    Consideramos $u, v$ elementos trascendentes sobre $F_1$ y $F_2$, respectivamente.
    Entonces existe un isomorfismo $\bar{\sigma} : F_1(u) \to F_2(v)$ que extiende a $\sigma$ y tal que $\bar{\sigma}(u) = v$.
\end{theorem}

\begin{example}
    Consideramos el isomorfismo identidad $1_\mathbb{Q} : \mathbb{Q} \to \mathbb{Q}$ y el elemento trascendente $\pi$.
    Entonces, podemos definir un isomorfismo $\bar{\sigma} : \mathbb{Q}(\pi) \to \mathbb{Q}(v)$ que deje fijos a los números racionales y envíe $\pi$ a cualquier elemento $v$ trascendente sobre $\mathbb{Q}$.
    Observamos que en lugar de un isomorfismo podríamos considerar una inmersión, como la inclusión $\mathbb{Q} \to \mathbb{R}$, y podríamos extenderla a una inmersión $\mathbb{Q}(\pi) \to \mathbb{R}$ que deje fijo todo elemento de $\mathbb{Q}(\pi)$.
    Ahora la imagen de $\pi$ es algebraica sobre $\mathbb{R}$.
\end{example}

\begin{proposition}
    Sean $K_1/F_1$ y $K_2/F_2$ extensiones de cuerpos, $\sigma : F_1 \to F_2$ un homomorfismo de cuerpos, $\alpha \in K_1$ algebraico sobre $F_1$ y $f(x)$ su polinomio mínimo sobre $F_1$.
    Si $\bar{\sigma} : K_1 \to K_2$ es un homomorfismo de cuerpos que extiende a $\sigma$, entonces $\bar{\sigma}(\alpha)$ es una raíz de $f^\sigma(x) \in F_2[x]$.
\end{proposition}

\begin{example}
    Consideramos el isomorfismo $\tau : \mathbb{Q}(\sqrt{2}) \to \mathbb{Q}(\sqrt{2})$ dado por $\tau(a + b\sqrt{2}) = a - b\sqrt{2}, a, b \in \mathbb{Q}$.
    Si queremos extenderlo a un automorfismo $\bar{\tau}$ de $\mathbb{Q}(\sqrt{2}, \sqrt[4]{2})$, la imagen de $\sqrt[4]{2}$ tiene que ser una raíz de $f^\tau(x)$, donde $f(x) = x^2 - \sqrt{2} \in \mathbb{Q}(\sqrt{2})[x]$ es el polinimio mínimo de $\sqrt[4]{2}$ sobre $\mathbb{Q}(\sqrt{2})$.
    Como $f^\tau(x) = x^2 - \tau(\sqrt{2}) = x^2 - (-\sqrt{2}) = x^2 + \sqrt{2}$, tenemos que $\bar{\tau}(\sqrt[4]{2})$ tiene que ser una raíz de $x^2 + \sqrt{2}$.
\end{example}

\begin{theorem}
    Sea $\sigma : F_1 \to F_2$ un isomorfismo de cuerpos.
    Si $\alpha$ es algebraico sobre $F_1$ con polinimio mínimo $f(x) \in F_1[x]$ y $\beta$ es una raíz de $f^\sigma(x) \in F_2[x]$, entonces existe un isomorfismo $\bar{\sigma} : F_1(\alpha) \to F_2(\beta)$ que extiende a $\sigma$ y tal que $\bar{\sigma}(\alpha) = \beta$.
\end{theorem}

\begin{corollary}
    Sean $K/F, L/F$ extensiones de cuerpos, $\alpha \in K, \beta \in L$ algebraicos sobre $F$.
    Entonces $\alpha$ y $\beta$ son raíces del mismo polinimio irreducible $f(x) \in F[x]$ si y solo si existe un $F$-isomorfismo $\sigma : F(\alpha) \to F(\beta)$ tal que $\sigma(\alpha) = \beta$.
\end{corollary}

\begin{definition}
    Si $\alpha$ y $\beta$ son raíces del mismo polinomio irreducible $f(x) \in F[x]$, decimos que son raíces conjugadas sobre $F$.
\end{definition}

\begin{remark}
    Dos números complejos conjugados son raíces conjugadas sobre $\mathbb{R}$. Si $a, b \in \mathbb{R}$, $a + bi$ y $a - bi$ son raíces del polinomio
    $$(x - (a+bi))(x - (a-bi)) = (x-a)^2 + b^2 = x^2 - 2ax + a^2 + b^2 \in \mathbb{R}[x]$$
    Observamos que este polinomio es irreducible cuando $b \neq 0$.
    Además, si $a, b \in \mathbb{Q}, b \neq 0$, entonces este polinomio es irreducible en $\mathbb{Q}[x]$.
    Sin embargo, $a + bi$ y $a - bi$ no son raíces conjugadas sobre $\mathbb{Q}(i)$.
\end{remark}

\begin{corollary}
    Sean $F_1, F_2, K$ cuerpos, $\sigma : F_1 \to F_2$ una inmersión, $F_2 \subset K$. Sea $\alpha$ un elemento en alguna extensión de cuerpos de $F_1$, algebraico sobre $F_1$, con polinomio mínimo $f(x) \in F_1[x]$. Entonces existe una inmersión $\bar{\sigma} : F_1(\alpha) \to K$ que entiende a $\sigma$ si y solo si $\bar{\sigma}(\alpha)$ es una raíz de $f^\sigma(x)$.
    En particular, si denotamos por $Emb_\sigma(F_1(\alpha), K)$ el conjunto de inmersiones de $F_1(\alpha)$ en $K$ extendiendo a $\sigma$, tenemos que
    $$|Emb_\sigma(F_1(\alpha), K)| = \text{número de raíces distintas de } f^\sigma(x) \text{ en } K$$
\end{corollary}

\begin{definition}
    Sea $K/F$ extensión de cuerpos.
    El grupo de $F$-automorfismos de $K$ se llama el grupo de Galois de $K$ sobre $F$.
    Se denota por $Gal(K/F)$ o por $Gal_F(K)$.
\end{definition}

\section{Cuerpo primo y característica}

\begin{definition}
    Sea $F$ un cuerpo y $X \subset F$, consideramos todos los subcuerpos de $F$ que contienen a $X$.
    Esta es una familia no vacía cuya intersección es el subcuerpo de $F$ más pequeño que contiene a $X$.
    Este se llama el subcuerpo de $F$ generado por $X$.
    Si $X = \emptyset, {0_F}, {1_F}$ o ${0_F, 1_F}$, el subcuerpo de $F$ más pequeño generado por $X$ es el subcuerpo más pequeño de $F$.
    Este se llama cuerpo primo de $F$ y se denota por $\pi(F)$.
\end{definition}

\begin{proposition}
    Si $F$ es un cuerpo, $\pi(F)$ es isomofo a $\mathbb{Q}$ o a $\mathbb{Z}_p$, para algún número primo $p$.
\end{proposition}

\begin{definition}
    Cuando $\pi(F)$ es isomorfo a $\mathbb{Q}$ decimos que la característica de $F$ es 0.
    Si $\pi(F)$ es isomorfo a $\mathbb{Z}_p$, decimos que la característica de $F$ es $p$.
\end{definition}

\chapter{Extensiones normales}
\section{Cuerpo stem}

\begin{definition}
    Sea $K/F$ una extensión de cuerpos y $f(x) \in F[x]$ no constante.
    Decimos que $K$ es un cuerpo stem de $f$ sobre $F$ si existe $\alpha \in K$ raíz de $f$ tal que $K = F(\alpha)$.
\end{definition}

\begin{example}
    $\mathbb{Q}(\sqrt{2})$ es un cuerpo stem de $x^2 - 2$ sobre $\mathbb{Q}$.
    Contiene todas las raíces de $x^2 - 2$.
    Por otro lado, $\mathbb{Q}(\sqrt[3]{2})$ es un cuerpo stem de $x^3 - 2$ sobre $\mathbb{Q}$ que no contiene todas las raíces de $x^3 - 2$.
\end{example}

\begin{theorem}[Existencia del cuerpo stem]
    Sea $F$ un cuerpo y $f(x) \in F[x]$ de grado $n > 0$. Entonces existe una extensión simple $F(\alpha)$ de $F$ tal que:
    \begin{enumerate}
        \item $\alpha$ es una raíz de $f(x)$.
        \item Si $f(x)$ es irreducible en $F[x]$, entonces el cuerpo $F(\alpha)$ es único salvo $F$-isomorfismos.
    \end{enumerate}
\end{theorem}

\begin{example}
    Construyamos un cuerpo stem de $f(x) = x^4 + x^2 - x + 1$ sobre $\mathbb{Z}_5$.
    En primer lugar observamos que $f(x) \in \mathbb{Z}_5[x]$ es irreducible.
    Entonces $F = \mathbb{Z}_5[x]/(x^4 + x^2 - x + 1)$ es un cuerpo stem.
    Además, cualquier otro cuerpo stem de $f$ sobre $\mathbb{Z}_5$ es $\mathbb{Z}_5$-isomorfo a $F$.
    Observamos que $F$ tiene $5^4$ elementos y que $[F : \mathbb{Z}_5] = 4$.
\end{example}

\begin{example}
    Construyamos un cuerpo stem de $f(x) = x^5 + 3x^3 + x^2 + 2x +2$ sobre $\mathbb{Z}_5$.
    Como se tiene que $f(x) = (x^2 + 2)(x^3 + x + 1)$, podemos construir dos cuerpos stem no isomorfos, cada uno con un factor irreducible: $\mathbb{Z}_5[x]/(x^2+2)$ y $\mathbb{Z}_5[x]/(x^3+x+1)$.
    Podemos asegurar que no son isomorfos porque son cuerpos finitos con distinto número de elementos.
    Tienen respectivamente $5^2$ y $5^3$ elementos.
\end{example}

\begin{example}
    Consideramos el polinomio $f(x) = x^4 + 1$ sobre $\mathbb{Z}_5$.
    Obsservamos que es reducible: $f(x) = (x^2+2)(x^2+3)$.
    Podemos construir un cuerpo stem con cada factor irreducible: $\mathbb{Z}_5[x]/(x^2+2)$ y $\mathbb{Z}_5[x]/(x^2+3)$.
    Sin embargo, en este caso son isomorfos, aunque ningún isomorfismo envía $\alpha = x + (x^2+2)$ a $\beta = x + (x^2+3)$, puesto que son raíces de polinomios irreducibles diferentes.
\end{example}

\begin{example}
    El polinomio $f(x) = (x^2-5)(x^2-11) \in \mathbb{Q}[x]$ tiene dos cuerpos stem no isomorfos: $\mathbb{Q}(\sqrt{5})$ y $\mathbb{Q}(\sqrt{11})$.
    Consideramos el polinomio $f(x) = (x^2-5)(x^2-2x-4) \in \mathbb{Q}[x]$. Podemos construir un cuerpo stem para cada factor irreducible: $\mathbb{Q}(\sqrt{5})$ y $\mathbb{Q}(\alpha)$, donde $\alpha$ es una raíz de $x^2-2x-4$.
    Sin embargo, como las raíces de este último factor son $1 \pm \sqrt{5}$, luego ambos cuerpos stem son el mismo.
\end{example}

\section{Cuerpo de descomposición}

\begin{definition}
    Sea $K/F$ una extensión de cuerpos y $f(x) \in F[x]$ un polinomio no constante.
    Decimos que $f(x)$ se descompone completamente sobre $K$ si $f(x) = a(x-\alpha_1) \dots (x-\alpha_r)$, con $a \in F, \alpha_i \in K$.
\end{definition}

\begin{example}
    $x^2 - 2$ se descompone completamente sobre $\mathbb{Q}(\sqrt{2})$, $\mathbb{Q}(\sqrt{2}$, $\sqrt{3})$, $\mathbb{R}$ y $\mathbb{C}$.
    No se descompone sobre $\mathbb{Q}$, $\mathbb{Q}(\sqrt{3})$ y $\mathbb{Q}(i)$.
\end{example}

\begin{definition}
    Sea $F$ un cuerpo y $f(x) \in F[x]$ no constante. Una extensión $K$ de $F$ es un cuerpo de descomposición de $f$ sobre $F$ si:
    \begin{enumerate}
        \item $f(x)$ se descompone completamente sobre $K$.
        \item $K = F(\alpha_1, \dots, \alpha_r)$, con $\alpha_1, \dots, \alpha_r$ las raíces de $f(x)$ en $K$.
    \end{enumerate}
\end{definition}

\begin{example}
    $\mathbb{Q}(\sqrt{2}, \sqrt{3})$ es cuerpo de descomposición de $(x^2-2)(x^2-3)$ sobre $\mathbb{Q}$.
\end{example}

\begin{example}
    $\mathbb{Q}(\sqrt[4]{2}, i)$ es cuerpo de descomposición de $x^4-2$ sobre $\mathbb{Q}$.
    $$x^4-2 = (x - \sqrt[4]{2})(x + \sqrt[4]{2})(x - i\sqrt[4]{2})(x + i\sqrt[4]{2}) \text{ y } \mathbb{Q}(\pm \sqrt[4]{2}, \pm i\sqrt[4]{2}) = \mathbb{Q}(\sqrt[4]{2}, i)$$
\end{example}

\begin{example}
    \begin{enumerate}
        \item $\mathbb{Q}(i)$ es cuerpo de descomposición de $x^2+1$ sobre $\mathbb{Q}$.
        \item $\mathbb{C}$ es cuerpo de descomposición de $x^2+1$ sobre $\mathbb{R}$.
        \item $\mathbb{C}$ es cuerpo de descomposición de $x^2+1$ sobre $\mathbb{C}$.
    \end{enumerate}
\end{example}

\begin{theorem}[Existencia de cuerpo de descomposición]
    Sea $f \in F[x]$ un polinomio de grado $n > 0$.
    Entonces existe un cuerpo de descomposición $K$ de $f$ sobre $F$.
\end{theorem}

\begin{theorem}
    Sea $\sigma : F_1 \to F_2$ un isomorfismo de cuerpos y $f(x) \in F_1[x]$ de grado $n > 0$.
    Si $K_1$ es cuerpo de descomposición de $f(x)$ sobre $F_1$ y $K_2$ es cuerpo de descomposición de $f^\sigma(x)$ sobre $F_2$, entonces $\sigma$ puede ser extendido a un isomorfismo $\bar{\sigma} : K_1 \to K_2$.
\end{theorem}

\begin{corollary}[Unicidad del cuerpo de descomposición]
    Dos cuerpos de descomposición de $f(x) \in F[x]$ sobre $F$ son $F$-isomorfos.
\end{corollary}

\begin{definition}
    Si $S$ es un subconjunto de $F[x]$, una extensión de cuerpos $K$ de $F$ es cuerpo de descomposición de $S$ sobre $F$ cuando:
    \begin{enumerate}
        \item $f(x)$ se descompone completamente sobre $K$ para cada $f(x) \in S$.
        \item $K = F(X)$ con $X = \{ \alpha \in K : \alpha \text{ es raíz de algún } f(x) \in S \}$.
    \end{enumerate}
\end{definition}

\begin{remark}
    Observamos que el cuerpo de descomposición de una familia finita de polinomios $f_1(x), \dots, f_m(x) \in F[x]$ es el mismo que el cuerpo de descomposición del producto de polinomios $f(x) = \prod_{i = 1}^m f_i(x) \in F[x]$.
    Por tanto, la definición previa solo es interesante cuando la familia $S$ de polinomios es infinita.
\end{remark}

\section{Clausura algebraica y cuerpos algebraicamente cerrados}

\begin{definition}
    Un cuerpo $F$ es algebraicamente cerrado si no tiene extensiones algebraicas.
    Es decir, si $K$ es una extensión algebraica de $F$, entonces $K = F$.
\end{definition}

\begin{proposition}
    $F$ es algebraicamente cerrado si y solo si todo polinomio irreducible en $F[x]$ tiene grado 1.
\end{proposition}

\begin{definition}
    Una extensión $K/F$ es clausura algebraica de $F$ si $K$ es algebraico sobre $F$ y $K$ es algebraicamente cerrado.
\end{definition}

\begin{proposition}
    $K$ es clausura algebraica de $F$ si y solo si $K$ es cuerpo de descomposición de $F[x]$ sobre $F$.
\end{proposition}

\begin{theorem}[Existencia de clausura algebraica]
    Sea $F$ un cuerpo. Entonces existe una clausura algebraica de $F$.
\end{theorem}

\begin{corollary}
    Sea $S \subseteq F[x]$. Entonces existe un cuerpo de descomposición de $S$ sobre $F$.
\end{corollary}

\begin{theorem}
    Sea $K/F$ una extensión algebraica y $L$ un cuerpo algebraicamente cerrado.
    Todo homomorfismo de cuerpos $F \to L$ puede ser extendido a un homomorfismo de cuerpos $K \to L$.
\end{theorem}

\begin{corollary}[Unicidad de la clausura algebraica]
    Dos clausuras algebraicas de un cuerpo $F$ son $F$-isomorfas.
\end{corollary}

\begin{corollary}[Unicidad del cuerpo de descomposición]
    Sea $S \subseteq F[x]$. Dos cuerpos de descomposición de $S$ sobre $F$ son $F$-isomorfos.
\end{corollary}

\section{Extensiones normales}

\begin{definition}
    Una extensión algebraica $K/F$ es normal si todo polinomio irreducible $f(x) \in F[x]$ que tiene una raíz en $K$ se descompone completamente sobre $K$.
\end{definition}

\begin{example}
    Observamos que $\mathbb{Q}(\sqrt[3]{2})$ no es una extensión normal de $\mathbb{Q}$.
\end{example}

\begin{theorem}
    Sea $K$ cuerpo de descomposición de un polinomio $f(x) \in F[x]$ sobre $F$. Entonces $K/F$ es normal.
\end{theorem}

\begin{remark}
    Si $K$ es el cuerpo de descomposición de un conjunto $S \in F[x]$, entonces $K/F$ es normal.
\end{remark}

\begin{theorem}
    Sea $K/F$ una extensión algebraica. Entonces $K/F$ es finita y normal si y solo si $K$ es cuerpo de descomposición sobre $F$ de algún $f(x) \in F[x]$.
\end{theorem}

\begin{remark}
    Existe una equivalencia entre extensiones normales y cuerpos de descomposición cuando la extensión es infinita.\\
    Sabemos que un cuerpo de descomposición de una familia de polinomios $S \subseteq F[x]$ es una extensión normal de $F$.\\
    Supongamos ahhora que $K/F$ es normal e infinita y tomamos una base $\{ u_i : i \in I \}$ de $K$ sobre $F$. Consideramos el conjunto:
    $$S = \{ f_i(x) \in F[x] : i \in I \}$$
    donde cada $f_i(x) \in F[x]$ es el polinomio mínimo de $\alpha_i$ sobre $F$. Entonces $K$ es el cuerpo de descomposición de $S$ sobre $F$.
\end{remark}

\begin{example}
    Sea $S = \{ \sqrt{p} : p \text{ primo en } \mathbb{Z} \}$. $\mathbb{Q}(S)$ es una extensión infinita de $\mathbb{Q}$.\\
    Es claro que $\mathbb{Q}(S)$ es el cuerpo de descomposición de la familia de polinomios $\{ x^2-p : p \text{ primo en } \mathbb{Z} \}$.
    Entonces es una extensión normal de $\mathbb{Q}$, que es infinito.
\end{example}

\begin{definition}
    Sea $K/F$ algebraica. Una extensión $L$ de $K$ se llama clausura normal de $K/F$ si:
    \begin{enumerate}
        \item $L/F$ es normal.
        \item Si $K \subseteq M \subseteq L$ y $M/F$ es normal, entonces $M = L$.
    \end{enumerate}
\end{definition}

\begin{example}
    Sabemos que $\mathbb{Q}(\sqrt[3]{2})$ no es normal sobre $\mathbb{Q}$. Si adjuntamos las raíces conjugadas de $\sqrt[3]{2}$ sobre $\mathbb{Q}$, obtenemos:
    $$L = \mathbb{Q}(\sqrt[3]{2}, i\sqrt{3})$$
    que es una extensión normal de $\mathbb{Q}$ porque es el cuerpo de descomposición de $x^3-2$.\\
    Cualquier otra extensión de $\mathbb{Q}(\sqrt[3]{2})$ normal sobre $\mathbb{Q}$ debe contener a las raíces conjugadas de $\sqrt[3]{2}$ sobre $\mathbb{Q}$.
    Por tanto, $L$ es clausura normal de $\mathbb{Q}(\sqrt[3]{2})$ sobre $\mathbb{Q}$.
\end{example}

\begin{theorem}[Existencia y unicidad de la clausura normal]
    Sea $K/F$ algebraica. Entonces existe una clausura normal de $K/F$. De hecho, dos clausuras normales de $K/F$ son $K$-isomorfas.
\end{theorem}

\begin{proposition}
    Sea $K/F$ algebraica y $L$ su clausura normal. Entonces $K/F$ es finita si y solo si $L/F$ es finita.
\end{proposition}

\chapter{Extensiones separables}
\section{Polinomios separables}

\begin{definition}
    Un polinomio no constante $f(x) \in F[x]$ se llama separable si sus raíces en un cuerpo de descomposición son todas simples.
\end{definition}

\begin{example}
    Para dar un ejemplo de un polinomio no separable basta con elegir un polinomio reducible con factores repetidos, como:
    \begin{align*}
        x^2-2x+1 = (x-1)^2     & \in \mathbb{Q}[x] \\
        x^4-4x^2+4 = (x^2-2)^2 & \in \mathbb{Q}[x]
    \end{align*}
\end{example}

\begin{proposition}
    Sea $f \in F[x]$ un polinomio mónico y no constante. Entonces las afirmaciones siguientes son equivalentes:
    \begin{enumerate}
        \item $f$ es separable.
        \item $f$ y su derivada $f'$ son primos relativos en $F[x]$.
    \end{enumerate}
    En particular, si $f \in F[x]$ es irreducible, entonces $f$ es separable si y solo si $f' \neq 0$.
\end{proposition}

\begin{corollary}
    Sea $f(x) \in F[x]$ un polinimio irreducible de grado $n > 0$. Si se verifica una de las siguientes condiciones, entonces $f(x)$ es separable:
    \begin{itemize}
        \item $char(F) = 0$.
        \item $char(F) = p$ y $p$ no divide a $n$.
    \end{itemize}
\end{corollary}

\section{Cuerpos perfectos y extensiones separables}

\begin{definition}
    Un cuerpo $F$ es perfecto si todo polinomio irreducible $f(x) \in F[x]$ es separable.
\end{definition}

\begin{theorem}
    Un cuerpo $F$ es perfecto si y solo si se verifica una de las siguientes condiciones:
    \begin{itemize}
        \item $char(F) = 0$.
        \item $char(F) = p$ y para cada $a \in F$ existe $b \in F$ tal que $a = b^2$.
    \end{itemize}
\end{theorem}

\begin{remark}
    Para cada cuerpo $F$ de característica $p$, la aplicación
    $$\varphi : F \to F$$
    definida por $\varphi(a) = a^p$ para cada $a \in F$ es un endomorfismo inyectivo.
    Se conoce como el endomorfismo de Frobenius.\\
    Observamos que, por el teorema previo, $F$ es perfecto si y solo si $\varphi$ es un automorfismo.\\
    Cuando $F$ es finito, $\varphi$ es una aplicación inyectiva entre conjuntos del mismo cardinal finito, así que tiene que ser sobreyectiva.
    Luego todo cuerpo finito es perfecto.
\end{remark}

\begin{definition}
    Si $K/F$ es una extensión de cuerpos y $\alpha \in K$ es algebraico sobre $F$, decimos que $\alpha$ es separable sobre $F$ cuando su polinomio mínimo sobre $F$ es separable.\\
    Una extensión algebraica $K/F$ es separable si cada elemento de $K$ es separable sobre $F$.
    Equivalentemente, $K/F$ es separable si cada polinomio irreducible $f(x) \in F[x]$ con una raíz en $K$ es separable.
\end{definition}

\begin{proposition}
    Toda extensión algebraica de un cuerpo perfecto es separable.
\end{proposition}

\begin{proposition}
    Si toda extensión finita de un cuerpo $F$ es separable, entonces $F$ es perfecto.
\end{proposition}

\begin{example}
    Para dar un ejemplo de un polinomio irreducible y no separable, necesitamos un cuerpo no perfecto.
    Hemos visto que no puede tener característica 0 y no puede ser finito ni una extensión algebraica de un cuerpo finito.
    Luego $F$ tiene que ser una extensión trascendente de su cuerpo primo.
    Tomaremos $F = \mathbb{Z}_p(t)$ para cierto primo $p$ y una variable $t$.
    Por otro lado, sabemos que un polinomio irreducible de grado $n$, con $p$ no dividiendo a $n$, es separable.
    Recordando además que la derivada de un polinomio irreducible no separable es el polinomio nulo, elegiremos el polinomio:
    $$f(x) = x^p - t \in F[x]$$
    Veamos que $f(x)$ es irreducible y no separable.\\
    Procedemos por reducción al absurdo. Sea $\alpha$ una raíz de $f(x)$:
    $$f(x) = (x-a)^p$$
    Si $\alpha \in F$, entonces:
    $$\alpha = \frac{g(t)}{h(t)}$$
    con $g(t), h(t) \in \mathbb{Z}_p[t], h(t) \neq 0$.
    Esto implica que $h(t)^pt = g(t)^p$. Sin embargo, esto es imposible.\\
    Por tanto, $x^p - t$ es irreducible en $F[x]$.
\end{example}

\section{Inmersiones y separabilidad}

\begin{theorem}
    Sea $K/F$ una extensión finita con $[K : F] = n$ y $\sigma : F \to L$ una inmersión.
    \begin{enumerate}
        \item El número de inmersiones $\bar{\sigma} : K \to L$ que extienden a $\sigma$ es a lo sumo $n$.
        \item Si $K/F$ no es separable, entonces el número de inmersiones $\bar{\sigma} : K \to L$ que extienden a $\sigma$ es menor que $n$.
        \item Si $K = F(\alpha_1, \dots, \alpha_r)$, con $\alpha_1, \dots, \alpha_r$ separables sobre $F$, entonces existe una extensión $L'$ de $L$ tal que el número de inmersiones $\bar{\sigma} : K \to L$ que extienden a $\sigma$ es $n$.
        \item Si $K/F$ es separable, entonces existe una extensión $L'$ de $L$ tal que el número de inmersiones $\bar{\sigma} : K \to L$ que extienden a $\sigma$ es $n$.
    \end{enumerate}
\end{theorem}

\begin{corollary}
    Si $K = F(\alpha_1, \dots, \alpha_r)$, con $\alpha_1, \dots, \alpha_r$ separables sobre $F$, entonces $K/F$ es separable.
\end{corollary}

\begin{corollary}
    Sea $K/F$ una extensión de cuerpos y
    $$K_s = \{ \alpha \in K : \alpha \text{ separable sobre } F \}$$
    Entonces $K_s$ es un cuerpo intermedio de $K/F$. Se llama clausura separable de $K/F$.
\end{corollary}

\section{Grado de separabilidad}

\begin{definition}
    Sea $K/F$ una extensión finita y
    $$K_s = \{ \alpha \in K : \alpha \text{ separable sobre } F \}$$
    El grado de separabilidad de $K$ sobre $F$ es el grado de $[K_s : F]$. Se denota por $[K : F]_s$.\\
    El grado de inseparabilidad es $[K : K_s]$ y se denota por $[K : F]_i$.
\end{definition}

\begin{lemma}
    Sea $K/F$ una extensión de cuerpos con $char(F) = p \neq 0$.
    Si $\alpha \in K$ es algebraico sobre $F$, entonces existe un entero $m > 0$ tal que $\alpha^{p^m}$ es separable sobre $F$.
\end{lemma}

\begin{proposition}
    Sea $K/F$ una extensión finita, $L$ clausura algebraica de $F$ y $\sigma : F \to L$ una inmersión.
    Entonces el número de inmersiones de $K$ a $L$ que extienden a $\sigma$ es precisamente $[K : F]_s$.\\
    En particular, $[K : F]_s$ es el número de $F$-inmersiones de $K$ a $N$, donde $N$ es clausura normal de $K/F$.
\end{proposition}

\chapter{Teorema del elemento primitivo}

\begin{definition}
    Sea $K/F$ una extensión de cuerpos. Si existe un elemento $\alpha \in K$ tal que $K = F(\alpha)$, decimos que $K/F$ es simple y que $\alpha$ es un elemento primitivo de la extensión.
\end{definition}

\begin{example}
    $\mathbb{Q}(\sqrt{2}, \sqrt{3}) = \mathbb{Q}(\sqrt{2} + \sqrt{3})$ es una extensión simple de $\mathbb{Q}$ y $\sqrt{2} + \sqrt{3}$ es un elemento primitivo.
\end{example}

\begin{lemma}
    Sea $G$ un grupo abeliano finito $G$ y $a \in G$ un elemento de orden máximo $m$. Entonces el orden de todo elemento de $G$ es un divisor de $m$.
\end{lemma}

\begin{lemma}
    Si $F$ es un cuerpo finito, entonces el grupo multiplicativo $F^\times = \{ a \in F : a \neq 0 \}$ es un grupo cíclico.
\end{lemma}

\begin{theorem}[Teorema del elemento primitivo]
    Toda extensión separable finita es simple.
\end{theorem}

\begin{example}
    Por el teorema, si una extensión finita no es simple, entonces ha de ser no separable.\\
    Sea $F = \mathbb{Z}_p(t, s)$, donde $t, s$ son variables, $\alpha$ una raíz de $x^p-t$ y $\beta$ una raíz de $x^p-s$.
    Veamos que la extensión $F(\alpha, \beta)/F$ no es simple.
    Para ello, observamos que $[F(\alpha) : F] = p$, pues $x^p-t$ es irreducible en $F[x]$.
    Con un argumento similar podemos obtener que $x^p-s$ es irreducible en $F(\alpha)[x]$, así que $[F(\alpha, \beta) : F(\alpha)] = p$.
    Podemos concluir que:
    $$[F(\alpha, \beta) : F] = p^2$$
    Si la extensión $F(\alpha, \beta)/F$ es simple, debe existir un elemento primitivo $\gamma \in F(\alpha, \beta)$ de grado $p^2$ sobre $F$.
    Pero podemos comprobar que todo elemento de $F(\alpha, \beta)/F$ tiene grado $p$.\\
    Sea $\gamma \in F(\alpha, \beta)$. Entonces:
    \begin{align*}
        \gamma   & = \frac{g(\alpha, \beta)}{h(\alpha, \beta)}, \quad g, h \in F[x, y], h \neq 0 \\
        \gamma^p & = \frac{\bar{g}(\alpha^p, \beta)^p}{\bar{h}(\alpha^p, \beta^p)}
    \end{align*}
    donde los coeficientes de $\bar(g), \bar{h}$ son la potencia $p$-ésima de los coeficientes de $g$ y $h$, respectivamente. Luego:
    $$\gamma^p = \frac{\bar{g}(t, s)}{\bar{h}(t, s)} \in F$$
    Por tanto $\gamma$ es una raíz de un polinomio de la forma $x^p - a \in F[x]$, luego el grado de $\gamma$ sobre $F$ no es $p^2$.\\
    Concluimos que $F(\alpha, \beta)/F$ es finita pero no es simple.
\end{example}

\begin{remark}
    El recíproco del teorema del elemento primitivo no es cierto.
\end{remark}

\begin{theorem}[Steiniz]
    Una extensión finita $K/F$ es simple si y solo si tiene un número finito de cuerpos intermedios.
\end{theorem}

\begin{proposition}
    Sean $K/F$ y $L/K$ extensiones finitas. Entonces $K/F$ y $L/K$ son separables si y solo si $L/F$ es separable.
\end{proposition}

\chapter{El teorema fundamental de la teoría de Galois}
\section{Grupo de Galois}

\begin{example}
    \begin{align*}
         & Gal(\mathbb{Q}(\sqrt[3]{2})/\mathbb{Q}) = \{Id\}                                                                \\
         & Gal(\mathbb{Q}(\sqrt{2})/\mathbb{Q}) = \{ Id, \sigma \}, \text{ donde } \sigma(\sqrt{2}) = -\sqrt{2}            \\
         & Gal(\mathbb{R}/\mathbb{Q}) = \{Id\}                                                                             \\
         & Gal(\mathbb{Q}(x)/\mathbb{Q}) = \{ x \mapsto \frac{ax + b}{cx + d} : a, b, c, d \in \mathbb{Q}, ad-bc \neq 0 \}
    \end{align*}
    Observamos que $\mathbb{R}/\mathbb{Q}$ y $\mathbb{Q}(x)/\mathbb{Q}$ son extensiones infinitas cuyo grupos de Galois son respectivamente finito e infinito.
\end{example}

\begin{lemma}
    Sea $K/F$ algebraica. Entonces todo $F$-homomorfismo $\sigma : K \to K$ es $F$-automorfismo.
\end{lemma}

\begin{theorem}
    Si $K/F$ es una extensión finita, entonces $Gal(K/F)$ es un grupo finito.
\end{theorem}

\section{Subgrupos y cuerpos intermedios}

\begin{theorem}
    Sea $K/F$ una extensión de cuerpos, $E$ un cuerpo intermedio y $H$ un subgrupo de $G = Gal(K/F)$. Entonces:
    $$K^H = \{ \alpha \in K : \sigma(\alpha) = \alpha, \text{ para todo } \sigma \in H \}$$
    es un cuerpo intermedio de $K/F$ y
    $$Gal(K/E) = \{ \sigma \in G : \sigma(\alpha) = \alpha, \text{ para todo } \alpha \in E \}$$
    es un subgrupo de $G$.
\end{theorem}

\begin{example}
    Sea $G$ el grupo de Galois de $\mathbb{Q}(\sqrt{2}, \sqrt{3})$ sobre $\mathbb{Q}$.
    Podemos comprobar que $G$ es un grupo de orden 4.
    Como sus elementos están completamente determinados por su acción en $\sqrt{2}$ y $\sqrt{3}$, podemos describirlos así:
    \begin{align*}
        \sigma_1           & = Id                                           \\
        \sigma_2(\sqrt{2}) & = \sqrt{2}  & \sigma_2(\sqrt{3}) & = -\sqrt{3} \\
        \sigma_3(\sqrt{2}) & = -\sqrt{2} & \sigma_3(\sqrt{3}) & = \sqrt{3}  \\
        \sigma_4(\sqrt{2}) & = -\sqrt{2} & \sigma_4(\sqrt{3}) & = -\sqrt{3}
    \end{align*}
    Observamos que $\sigma_2, \sigma_3, \sigma_4$ son elementos de orden 2 en $G$.
    Luego $G$ es isomorfo al grupo $C_2 \times C_2$.\\
    El subgrupo de $G$ asociado al cuerpo intermedio $\mathbb{Q}(\sqrt{3})$ es:
    $$Gal(\mathbb{Q}(\sqrt{2}, \sqrt{3}) / \mathbb{Q}(\sqrt{3}))$$
    el grupo de $\mathbb{Q}(\sqrt{3})$-automorfismos de $\mathbb{Q}(\sqrt{2}, \sqrt{3})$. Este es $< \sigma_3 >$.\\
    Por otro lado, sea $H$ el subgrupo generado por $\sigma_2 : H = < \sigma_2 >$.
    El correspondiente cuerpo intermedio es:
    \begin{align*}
        \mathbb{Q}(\sqrt{2}, \sqrt{3})^H & = \{ \alpha \in \mathbb{Q}(\sqrt{2}, \sqrt{3}) : \sigma(\alpha) = \alpha, \text{ para todo } \sigma \in H \} \\
                                         & = \{ \alpha \in \mathbb{Q}(\sqrt{2}, \sqrt{3}) : \sigma_2(\alpha) = \alpha \}
    \end{align*}
    Obervamos que $\mathbb{Q}(\sqrt{2}) \subseteq \mathbb{Q}(\sqrt{2}, \sqrt{3})^H$.
    Se puede demostrar que se da la igualdad considerando la acción de $\sigma_2$ en cada elemento de $\mathbb{Q}(\sqrt{2}, \sqrt{3})$, recordando que son de la forma $a + b\sqrt{2} + c\sqrt{3} + d\sqrt{2}\sqrt{3}$, con $a, b, c, d \in \mathbb{Q}$.
\end{example}

\begin{definition}
    Sea $K/F$ una extensión de cuerpos y $H$ un subgrupo de $Gal(K/F)$.
    El cuerpo intermedio $K^H$ se llama subcuerpo de $K$ fijo por $H$.
\end{definition}

\section{Extensiones de Galois}

\begin{definition}
    Una extensión de Galois es una extensión de cuerpos $K/F$ tal que $K^G = F$, donde $G = Gal(K/F)$.
\end{definition}

\begin{example}
    El grupo de Galois de $\mathbb{Q}(\sqrt[3]{2})$ sobre $\mathbb{Q}$ es trivial.
    Luego todos los elementos de $\mathbb{Q}(\sqrt[3]{2})$ son fijos por todos los elementos del grupo de Galois, es decir:
    $$\mathbb{Q}(\sqrt[3]{2})^G = \mathbb{Q}(\sqrt[3]{2})$$
    y por tanto $\mathbb{Q}(\sqrt[3]{2})$ no es una extensión de Galois de $\mathbb{Q}$.
\end{example}

\begin{example}
    Veamos si $\mathbb{Q}(\sqrt[4]{2})/\mathbb{Q}$ es una extensión de Galois.\\
    Su grupo de Galois $G$ tiene dos elementos: la identidad y el elemento $\sigma$ determinado por:
    $$\sigma(\sqrt[4]{2}) = -\sqrt[4]{2}$$
    Observamos que $\sqrt{2} \in \mathbb{Q}(\sqrt[4]{2})$ es fijo por $Id$ y $\sigma$, así que $\mathbb{Q}(\sqrt{2}) \subseteq \mathbb{Q}(\sqrt[4]{2})^G$ y por tanto la extensión no es de Galois.
\end{example}

\begin{theorem}
    Sea $K/F$ una extensión algebraica. Entonces son equivalentes:
    \begin{enumerate}
        \item $K/F$ es una extensión de Galois.
        \item $K/F$ es normal y separable.
    \end{enumerate}
\end{theorem}

\section{El teorema fundamental}

\begin{proposition}
    Sea $K/F$ algebraico. Si $K/F$ es de Galois y $E$ es un cuerpo intermedio, entonces $K/E$ es de Galois.
\end{proposition}

\begin{example}
    La extensión $\mathbb{Q}(i, \sqrt{3}, \sqrt[3]{2})/\mathbb{Q}$ es de Galois pero $\mathbb{Q}(\sqrt[3]{2})/\mathbb{Q}$ no es de Galois.
\end{example}

\begin{theorem}
    Sea $K/F$ una extensión finita. Entonces:
    \begin{enumerate}
        \item $|Gal(K/F)|$ divide a $[K : F]$.
        \item $K/F$ es de Galois si y solo si $|Gal(K/F)| = [K : F]$.
    \end{enumerate}
\end{theorem}

\begin{theorem}
    Sea $K/F$ una extensión finita y $E$ un cuerpo intermedio. Si $K/F$ es de Galois, son equivalentes:
    \begin{enumerate}
        \item $E = \sigma E$ para todo $\sigma \in Gal(K/F)$.
        \item $Gal(K/E)$ es un subgrupo normal de $Gal(K/F)$.
        \item $E/F$ es de Galois.
        \item $E/F$ es una extensión normal.
    \end{enumerate}
\end{theorem}

\begin{theorem}
    Sean $F \subset E \subset K$ extensiones finitas con $E/F$ y $K/F$ de Galois.
    Entonces $Gal(K/E)$ es un subgrupo normal de $Gal(K/F)$ y existe un isomorfismo de grupos:
    $$Gal(K/F)/Gal(K/E) \equiv Gal(E/F)$$
\end{theorem}

\begin{theorem}[Teorema fundamental]
    Sea $K/F$ una extensión finita de Galois.
    Existe una correspondencia uno a uno entre el conjunto de cuerpos intermedios de $K/F$ y el conjunto de subgrupos del grupo de Galois $G = Gal(K/F)$.
    Esta correspondencia viene dada por $E \mapsto Gal(K/E)$ y satisface las siguientes condiciones:
    \begin{enumerate}
        \item Si $F \subseteq E \subseteq K$, entonces $K/E$ es una extensión de Galois y su grupo de Galois $Gal(K/E)$ es un subgrupo de $G = Gal(K/F)$. Además, $$[K : E] = |Gal(K/E)| \text{ y } |Gal(K/F) : Gal(K/E)| = [E : F]$$
        \item $E/F$ es de Galois si y solo si $Gal(K/E)$ es un subgrupo normal de $G = Gal(K/F)$. En este caso, $Gal(E/F)$ es isomorfo a $G/Gal(K/E)$.
    \end{enumerate}
\end{theorem}

\chapter{Cuerpos finitos}

\begin{proposition}
    Sea $F$ un cuerpo finito. Entonces:
    \begin{enumerate}
        \item La característica de $F$ es un primo $p$.
        \item $F$ es una extensión finita de su cuerpo primo $\pi(F)$ (isomorfo a $\mathbb{Z}_p$) y $$|F| = p^n, \text{ donde } n = [F : \pi(F)]$$
    \end{enumerate}
\end{proposition}

\begin{proposition}
    Sea $F$ un cuerpo finito con $q = p^n$ elementos. Entonces:
    \begin{enumerate}
        \item $\alpha^q = \alpha$ para todo $\alpha \in F$.
        \item $x^q - x = \prod_{\alpha \in F} (x-\alpha)$.
        \item $F$ es un cuerpo de descomposición sobre $\pi(F)$ de $x^q - x$.
    \end{enumerate}
\end{proposition}

\begin{example}
    $$x^3-x = x(x-1)(x-2) \in \mathbb{Z}_3[x]$$
    La factorización de $x^9-x$ en factores irreducibles en $\mathbb{Z}_3[x]$ es:
    $$x^9-x = x(x-1)(x+1)(x^2+1)(x^2+x+2)(x^2+2x+2) \in \mathbb{Z}_3[x]$$
    La proposición previa afirma que si $F$ es un cuerpo finito con 9 elementos, entonces sus elementos serán las raíces de $x^9-x$.
    Para expresar  cada una de las raíces de $x^9-x$ contruimos un cuerpo finito de 9 elementos como un cuerpo stem de un polinomio irreducible de grado 2 sobre $\mathbb{Z}_3[x]$.
    Por ejemplo, consideramos el polinomio irreducible $x^2+1 \in \mathbb{Z}_3[x]$. Entonces:
    \begin{center}
        $\mathbb{F} = \mathbb{Z}_3[x]/(x^2+1)$ es una extensión de $\mathbb{Z_3}$ de grado 2.
    \end{center}
    Recordamos que este cociente es isomorfo a $\mathbb{Z}_3(\alpha)$ para cada raíz $\alpha$ de $x^2+1$.
    Como $\{ 1, \alpha \}$ es una base de $\mathbb{Z}_3(\alpha)$ sobre $\mathbb{Z}$, tenemos que los 9 elementos de $F = \mathbb{Z}_3(\alpha)$ son:
    $$0, 1, 2, \alpha, 1+\alpha, 2+\alpha, 2\alpha, 1+2\alpha, 2+2\alpha$$
\end{example}

\begin{corollary}
    Sea $p$ primo y $n$ un entero positivo. Entonces:
    \begin{enumerate}
        \item Existe un cuerpo finito con $p^n$ elementos.
        \item Dos cuerpos finitos con $p^n$ elementos son isomorfos.
    \end{enumerate}
\end{corollary}

\begin{corollary}
    El polinomio $x^{p^n}-x \in \mathbb{Z}_p[x]$ es el producto de todos los polinomios irreducibles mónicos de $\mathbb{Z}_p[x]$ de grado $d$, con $d$ divisor de $n$.
\end{corollary}

\begin{example}
    Descomponemos el polinomio $x^{5^2} - x$ sobre $\mathbb{Z}_5$:
    \begin{align*}
         & x(x+1)(x+2)(x+3)(x+4)(x^2+2)(x^2+3)      \\
         & (x^2+x+1)(x^2+x+2)(x^2+2x+3)(x^2+2x+4)   \\
         & (x^2+3x+3)(x^2+3x+4)(x^2+4x+1)(x^2+4x+2)
    \end{align*}
\end{example}

\section{Extensiones finitas de cuerpos finitos}

\begin{proposition}
    Sea $F$ un cuerpo finito con $q = p^m$ elementos y $n$ un entero positivo.
    Entonces existe una extensión finita $K$ de grado $n$ sobre $F$ y es única salvo $F$-isomorfismos.
\end{proposition}

\begin{corollary}
    Si $F$ es un cuerpo finito y $n$ es un entero positivo, existe un polinomio irreducible $f \in F[x]$ de grado $n$.
\end{corollary}

\begin{theorem}
    Sea $F$ un cuerpo finito con $p^m$ elementos y $K/F$ una extensión finita de grado $[K : F] = n$.
    Entonces $K/F$ es de Galois y su grupo de Galois es:
    $$G \equiv \mathbb{Z}/n\mathbb{Z}$$
    Además, $G$ es generado por:
    \begin{align*}
        \sigma : K & \to K            \\
        \alpha     & \mapsto \alpha^q
    \end{align*}
\end{theorem}

\chapter{Extensiones ciclotómicas}

\begin{definition}
    Una extensión de cuerpos de la forma $\mathbb{Q}(\zeta_n)/\mathbb{Q}$, con $\zeta_n = e^{2\pi i/n}$ se llama la $n$-ésima extensión ciclotómica.
\end{definition}

\begin{proposition}
    $\mathbb{Q}(\zeta_n)/\mathbb{Q}$ es cuerpo de descomposición de $x^n-1$ sobre $\mathbb{Q}$. De hecho, es una extensión de Galois finita.
\end{proposition}

\begin{definition}
    Si $\theta$ es una raíz de $x^n-1$ tal que $\theta^k \neq 1$ para todo $1 \leq k < n$, decimos que $\theta$ es una raíz primitiva $n$-ésima de la unidad.
\end{definition}

\begin{remark}
    $\zeta_n = e^{2\pi i/n}$ es una raíz primitiva $n$-ésima de la unidad.
\end{remark}

\begin{proposition}
    El conjunto de raíces primitivas $n$-ésimas de la unidad es:
    $$\{ \zeta^k_n : MCD(k, n) = 1 \}$$
\end{proposition}

\begin{theorem}
    Las raíces conjugadas de $\zeta_n$ sobre $\mathbb{Q}$ son:
    $$\{ \zeta^k_n : MCD(k, n) = 1 \}$$
    las raíces primitivas $n$-ésimas de la unidad.\\
    En consecuencia, $[\mathbb{Q}(\zeta_n) : \mathbb{Q}] = \phi(n)$, donde $\phi$ es la función de Euler.
\end{theorem}

\begin{example}
    Por el teorema previo, el polinomio mímimo de $\zeta_{15}$ sobre $\mathbb{Q}$ tiene grado $\phi(15) = 8$.
    \begin{align*}
        x^{15}-1 & = ((x^5)^3-1) = (x^5-1)((x^5)^2+(x^5)+1) = \\
                 & = (x-1)(x^4+x^3+x^2+x+1)(x^{10}+x^5+1)
    \end{align*}
    Sabemos que $x^4+x^3+x^2+x+1$ es irreducible.
    Como el polinimio mímimo de $\zeta_{15}$ sobre $\mathbb{Q}$ divide a $x^{15}-1$ y tiene grado 8, debe ser un factor irreducible de $x^{10}+x^5+1$.
    También sabemos que $x^3-1$ divide a $x^15-1$, porque sus raíces son las raíces 3-ésimas de la unidad, luego también son raíces 15-ésimas de la unidad, aunque no sean primitivas.
    Entonces $x^{10}+x^5+1$ es múltiplo de $x^2+x+1$. Dividiendo por $x^2+x+1$, obtenemos:
    $$x^{10}+x^5+1 = (x^2+x+1)(x^8-x^7+x^5-x^4+x^3-x+1)$$
    Podemos concluir que $x^8-x^7+x^5-x^4+x^3-x+1$ es el polinimio mínimo de $\zeta_{15}$ sobre $\mathbb{Q}$.
\end{example}

\begin{definition}
    El polinimio mínimo de $\zeta_n$ sobre $\mathbb{Q}$ se llama $n$-ésimo polinomio ciclotómico.
\end{definition}

\begin{example}
    $x^8-x^7+x^5-x^4+x^3-x+1$ es el 15-ésimo polinomio ciclotómico.\\
    Sus raíces son $\zeta_{15}, \zeta^2_{15}, \zeta^4_{15}, \zeta^7_{15}, \zeta^8_{15}, \zeta^{11}_{15}, \zeta^{13}_{15}, \zeta^{14}_{15}$.
\end{example}

\begin{remark}
    Si $n = p$ es primo, el $p$-ésimo polinomio ciclotómico es:
    $$x^{p-1} + x^{p-2} + \dots + x + 1$$
\end{remark}

\begin{proposition}
    Sea $\Phi_n(x)$ el $n$-ésimo polinomio ciclotómico. Entonces:
    $$x^n-1 = \prod_{d | n} \Phi_d(x)$$
\end{proposition}

\begin{theorem}
    El grupo de Galois de $\mathbb{Q}(\zeta_n)$ sobre $\mathbb{Q}$ es isomorfo al grupo multiplicativo de los enteros módulo $n$:
    $$\mathbb{Z}^*_n = \{ k : 1 \leq k < n, MCD(k, n) = 1 \}$$
    Para cada $k \in \mathbb{Z}^*_n$, el correspondiente automorfismo en el grupo de Galois envía $\zeta_n$ a $\zeta^k_n$.
\end{theorem}

\section{Ciclos gaussianos}

\begin{theorem}[Gauss]
    Sea $p$ un primo, $\mathbb{Q}(\zeta_p)/\mathbb{Q}$ la $p$-ésima extensión ciclotómica, $H$ un subgrupo de $\mathbb{Z}^\times_p$. Entonces:
    $$\gamma_H = \sum_{a \in H} \zeta^a_p$$
    es un elemento primitivo de $\mathbb{Q}(\zeta_p)^H/\mathbb{Q}$.
\end{theorem}

\begin{example}
    Consideramos el subgrupo de $\mathbb{Z}^*_{19}$ generado por 8:
    $$H = < 8 > = \{ 1, 7, 8, 11, 12, 18 \}$$
    Este es un ciclo. La suma:
    $$\gamma_H = \zeta_{19} + \zeta^7_{19} + \zeta^8_{19} + \zeta^{11}_{19} + \zeta^{12}_{19} + \zeta^{18}_{19}$$
    es un elemento primitivo de $\mathbb{Q}(\zeta_{19})^H$.\\
    Los otros dos conjuntos complementarios de $H$ en $\mathbb{Z}^*_{19}$ son también ciclos:
    \begin{align*}
         & \{ 2, 3, 5, 14, 16, 17 \}, \\
         & \{ 4, 6, 9, 10, 13, 15 \}
    \end{align*}
    Y las correspondientes sumas:
    \begin{align*}
        \zeta^2_{19} + \zeta^3_{19} + \zeta^5_{19} + \zeta^{14}_{19} + \zeta^{16}_{19} + \zeta^{17}_{19}, \\
        \zeta^4_{19} + \zeta^6_{19} + \zeta^9_{19} + \zeta^{10}_{19} + \zeta^{13}_{19} + \zeta^{15}_{19}
    \end{align*}
    son las raíces conjugadas de $\gamma_H$ sobre $\mathbb{Q}$, que también generan el cuerpo intermedio $\mathbb{Q}(\zeta_{19})^H$.
\end{example}

\chapter{Construcciones geométricas}
\section{Números construibles}

\begin{definition}
    Un número complejo $\alpha$ es construible si existe una  secuencia finita de construcciones con regla y compás que empieza con 0 y 1 y acaba con $\alpha$.
\end{definition}

\begin{proposition}
    El conjunto $\mathcal{C} = \{ \alpha \in \mathbb{C} : \alpha \text{ es construible} \}$ es un subcuerpo de $\mathbb{C}$. Además:
    \begin{enumerate}
        \item Sea $\alpha = a + bi \in \mathbb{C}$. Entonces $\alpha \in \mathcal{C}$ si y solo si $a, b \in \mathcal{C}$.
        \item Si $\alpha \in \mathcal{C}$ entonces $\sqrt{\alpha} \in \mathcal{C}$.
    \end{enumerate}
\end{proposition}

\begin{theorem}
    Sea $\alpha$ un número complejo. Entonces $\alpha$ es construible si y solo si existe una sucesión de cuerpos
    $$\mathbb{Q} = F_0 \subset F_1 \subset \dots \subset F_{n-1} \subset F_n \subset \mathbb{C}$$
    tal que $\alpha \in F_n$ y $[F_i : F_{i-1}] = 2$ para todo $i = 1, \dots, n$.
\end{theorem}

\begin{corollary}
    El cuerpo de números construibles es el subcuerpo más pequeño de $\mathbb{C}$ que es cerrado para la raíz cuadrada.
\end{corollary}

\begin{corollary}
    Si $\alpha \in \mathbb{C}$ es un número construible, entonces $[\mathbb{Q}(\alpha) : \mathbb{Q}] = 2^m$ para algún $m \geq 0$.
\end{corollary}

\begin{theorem}
    Sea $\alpha \in \mathbb{C}$ algebraico sobre $\mathbb{Q}$ y sea $K$ cuerpo de descomposición del polinimio mínimo de $\alpha$ sobre $\mathbb{Q}$.
    Entonces $\alpha$ es construible si y solo si $[K : \mathbb{Q}]$ es una potencia de 2.
\end{theorem}

\section{Algunas construcciones imposibles}
\begin{itemize}
    \item Trisección del ángulo
    \item Duplicación del cubo
    \item Cuadratura del círculo
\end{itemize}

\section{Polígonos regulares}

\begin{definition}
    Un primo impar $p$ es un primo de Fermat si se puede escribir como $p = 2^{2^m}+1$ para algún entero $m \geq 0$.
\end{definition}

\begin{example}
    Los números primos de Fermat conocidos son 3, 5, 17, 257 y 65537.
\end{example}

\begin{lemma}
    Sea $k$ un entero positivo. Si $p = 2^k+1$ es un primo impar, entonces $p$ es un primo de Fermat.
\end{lemma}

\begin{theorem}
    Sea $n > 2$ un entero. Entonces un $n$-ágono regular puede ser construido por regla y compás si y solo si
    $$n = 2^s p_1 \dots p_r$$
    donde $s \geq 0$ es un entero y $p_1, \dots, p_r$ son distintos primos de Fermat.
\end{theorem}

\chapter{Solubilidad por radicales}
\section{Extensiones radicales y solubles}

\begin{definition}
    Una extensión $F \subset K$ es radical si existen cuerpos intermedios
    $$F = F_0 \subset F_1 \subset \dots \subset F_{m-1} \subset F_m = K$$
    tales que para cada $i = 1, \dots m$ existe $\gamma_i \in F_i$, con $F_i = F_{i-1}(\gamma_i)$ y $\gamma^{m_i}_i \in F_{i-1}$ para algún $m_i > 0$.
\end{definition}

\begin{example}
    $\mathbb{Q} \subset \mathbb{Q}(\sqrt[3]{2+\sqrt{2}})$ es una extensión radical.
\end{example}

\begin{example}
    El polinomio $f(x) = x^3+x^2-2x-1 \in \mathbb{Q}[x]$ es irreducible.
    Podemos comprobar que $f(x)$ tiene tres soluciones reales. Además, el discriminante de $f$ es 49, que es un cuadrado.
    Se puede comprobar que el grupo de Galois de un polinimio irreducible de orden $n$ cuyo discriminante es un cuadrado es un subgrupo del grupo alternado $A_n$.
    En este caso, podemos concluir que $Gal_\mathbb{Q}(f)$ es cíclico de grado 3.
    Por tanto, el cuerpo de descomposición es de la forma $\mathbb{Q}(\alpha)$, donde $\alpha$ es una raíz de $f$.\\
    Si $\mathbb{Q}(\alpha)$ es radical sobre $\mathbb{Q}$, debe existir un elemento primitivo $\gamma \in \mathbb{Q}(\alpha)$ tal que $\gamma^m \in \mathbb{Q}$ para algún $m \geq 3$.
    Pero entonces, sus raíces conjugadas pertenecen al conjunto $\{ \gamma, \gamma\zeta_m, \dots, \gamma\zeta^{m-1}_m \}$.
    Como $\zeta_m$ no pertenece a $\mathbb{Q}(\alpha) \subset \mathbb{R}$, esto no es posible.
\end{example}

\begin{definition}
    Una extensión de cuerpos $F \subset K$ es solucble si existe una extensión $L$ de $K$ tal que $L/F$ es radical.
\end{definition}

\begin{definition}
    Sea $F$ un cuerpo. Si $f \in F[x]$, la ecuación $f(x) = 0$ se dice que es soluble por radicales cuando el cuerpo de descomposición de $f$ sobre $F$ es una extensión soluble de $F$.
\end{definition}

\begin{proposition}
    Sea $K/F$ una extensión de cuerpos y $E, E'$ cuerpos intermedios. Denotamos por $EE'$ al subcuerpo más pequeño de $K$ que contiene a $E$ y a $E'$. Entonces:
    \begin{enumerate}
        \item Si $K/E$ y $E/F$ son radicales, entonces $K/F$ es radical.
        \item Si $E'/F$ es radical, entonces $EE'/E$ es radical.
        \item Si $E/F$ y $E'/F$ son radicales, entonces $EE'/F$ es radical.
    \end{enumerate}
\end{proposition}

\begin{proposition}
    Si $K/F$ es una extensión radical y $\bar{K}/F$ es clausuranormal de $K/F$, entonces $\bar{K}/F$ es radical.
\end{proposition}

\section{Teorema de Galois}

\begin{lemma}
    Sea $K/F$ una extensión de Galois finita de grado $m$ y $\zeta$ una raíz primitiva $m$-ésima de la unidad.
    Entonces $K(\zeta)/F(\zeta)$ es una extensión de Galois cuyo grado divide a $m$.
\end{lemma}

\begin{lemma}
    Sea $K/F$ una extensión de Galois finita con grupo de Galois cíclico de orden primo $p$.
    Si $F$ contiene una raíz primitiva $p$-ésima de la unidad $\zeta_p$, entonces existe $\alpha \in K$ tal que $K = F(\alpha)$ y $\alpha^p \in F$.
\end{lemma}

\begin{theorem}[Galois]
    Sea $K/F$ una extensión de Galois finita con $F$ de característica 0. Entonces son equivalentes:
    \begin{enumerate}
        \item $K/F$ es soluble.
        \item $Gal(K/F)$ es un grupo soluble.
    \end{enumerate}
\end{theorem}

\begin{corollary}[Galois]
    Sea $F$ un cuerpo de característica 0 y $f \in F[x]$.
    Entonces $f(x) = 0$ es soluble por radicales si y solo si $Gal_F(f)$ es un grupo soluble.
\end{corollary}

\begin{corollary}[Teorema de Abel-Ruffini]
    Sea $F$ un cuerpo de característica 0.
    La ecuacuón general de grado $n \geq 5$ no es soluble por radicales sobre $F$.
\end{corollary}

\end{document}