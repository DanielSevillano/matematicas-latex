\chapter{Extensiones separables}
\section{Polinomios separables}

\begin{definition}
    Un polinomio no constante $f(x) \in F[x]$ se llama separable si sus raíces en un cuerpo de descomposición son todas simples.
\end{definition}

\begin{example}
    Para dar un ejemplo de un polinomio no separable basta con elegir un polinomio reducible con factores repetidos, como:
    \begin{align*}
        x^2-2x+1 = (x-1)^2     & \in \mathbb{Q}[x] \\
        x^4-4x^2+4 = (x^2-2)^2 & \in \mathbb{Q}[x]
    \end{align*}
\end{example}

\begin{proposition}
    Sea $f \in F[x]$ un polinomio mónico y no constante. Entonces las afirmaciones siguientes son equivalentes:
    \begin{enumerate}
        \item $f$ es separable.
        \item $f$ y su derivada $f'$ son primos relativos en $F[x]$.
    \end{enumerate}
    En particular, si $f \in F[x]$ es irreducible, entonces $f$ es separable si y solo si $f' \neq 0$.
\end{proposition}

\begin{corollary}
    Sea $f(x) \in F[x]$ un polinimio irreducible de grado $n > 0$. Si se verifica una de las siguientes condiciones, entonces $f(x)$ es separable:
    \begin{itemize}
        \item $char(F) = 0$.
        \item $char(F) = p$ y $p$ no divide a $n$.
    \end{itemize}
\end{corollary}

\section{Cuerpos perfectos y extensiones separables}

\begin{definition}
    Un cuerpo $F$ es perfecto si todo polinomio irreducible $f(x) \in F[x]$ es separable.
\end{definition}

\begin{theorem}
    Un cuerpo $F$ es perfecto si y solo si se verifica una de las siguientes condiciones:
    \begin{itemize}
        \item $char(F) = 0$.
        \item $char(F) = p$ y para cada $a \in F$ existe $b \in F$ tal que $a = b^2$.
    \end{itemize}
\end{theorem}

\begin{remark}
    Para cada cuerpo $F$ de característica $p$, la aplicación
    $$\varphi : F \to F$$
    definida por $\varphi(a) = a^p$ para cada $a \in F$ es un endomorfismo inyectivo.
    Se conoce como el endomorfismo de Frobenius.\\
    Observamos que, por el teorema previo, $F$ es perfecto si y solo si $\varphi$ es un automorfismo.\\
    Cuando $F$ es finito, $\varphi$ es una aplicación inyectiva entre conjuntos del mismo cardinal finito, así que tiene que ser sobreyectiva.
    Luego todo cuerpo finito es perfecto.
\end{remark}

\begin{definition}
    Si $K/F$ es una extensión de cuerpos y $\alpha \in K$ es algebraico sobre $F$, decimos que $\alpha$ es separable sobre $F$ cuando su polinomio mínimo sobre $F$ es separable.\\
    Una extensión algebraica $K/F$ es separable si cada elemento de $K$ es separable sobre $F$.
    Equivalentemente, $K/F$ es separable si cada polinomio irreducible $f(x) \in F[x]$ con una raíz en $K$ es separable.
\end{definition}

\begin{proposition}
    Toda extensión algebraica de un cuerpo perfecto es separable.
\end{proposition}

\begin{proposition}
    Si toda extensión finita de un cuerpo $F$ es separable, entonces $F$ es perfecto.
\end{proposition}

\begin{example}
    Para dar un ejemplo de un polinomio irreducible y no separable, necesitamos un cuerpo no perfecto.
    Hemos visto que no puede tener característica 0 y no puede ser finito ni una extensión algebraica de un cuerpo finito.
    Luego $F$ tiene que ser una extensión trascendente de su cuerpo primo.
    Tomaremos $F = \mathbb{Z}_p(t)$ para cierto primo $p$ y una variable $t$.
    Por otro lado, sabemos que un polinomio irreducible de grado $n$, con $p$ no dividiendo a $n$, es separable.
    Recordando además que la derivada de un polinomio irreducible no separable es el polinomio nulo, elegiremos el polinomio:
    $$f(x) = x^p - t \in F[x]$$
    Veamos que $f(x)$ es irreducible y no separable.\\
    Procedemos por reducción al absurdo. Sea $\alpha$ una raíz de $f(x)$:
    $$f(x) = (x-a)^p$$
    Si $\alpha \in F$, entonces:
    $$\alpha = \frac{g(t)}{h(t)}$$
    con $g(t), h(t) \in \mathbb{Z}_p[t], h(t) \neq 0$.
    Esto implica que $h(t)^pt = g(t)^p$. Sin embargo, esto es imposible.\\
    Por tanto, $x^p - t$ es irreducible en $F[x]$.
\end{example}

\section{Inmersiones y separabilidad}

\begin{theorem}
    Sea $K/F$ una extensión finita con $[K : F] = n$ y $\sigma : F \to L$ una inmersión.
    \begin{enumerate}
        \item El número de inmersiones $\bar{\sigma} : K \to L$ que extienden a $\sigma$ es a lo sumo $n$.
        \item Si $K/F$ no es separable, entonces el número de inmersiones $\bar{\sigma} : K \to L$ que extienden a $\sigma$ es menor que $n$.
        \item Si $K = F(\alpha_1, \dots, \alpha_r)$, con $\alpha_1, \dots, \alpha_r$ separables sobre $F$, entonces existe una extensión $L'$ de $L$ tal que el número de inmersiones $\bar{\sigma} : K \to L$ que extienden a $\sigma$ es $n$.
        \item Si $K/F$ es separable, entonces existe una extensión $L'$ de $L$ tal que el número de inmersiones $\bar{\sigma} : K \to L$ que extienden a $\sigma$ es $n$.
    \end{enumerate}
\end{theorem}

\begin{corollary}
    Si $K = F(\alpha_1, \dots, \alpha_r)$, con $\alpha_1, \dots, \alpha_r$ separables sobre $F$, entonces $K/F$ es separable.
\end{corollary}

\begin{corollary}
    Sea $K/F$ una extensión de cuerpos y
    $$K_s = \{ \alpha \in K : \alpha \text{ separable sobre } F \}$$
    Entonces $K_s$ es un cuerpo intermedio de $K/F$. Se llama clausura separable de $K/F$.
\end{corollary}

\section{Grado de separabilidad}

\begin{definition}
    Sea $K/F$ una extensión finita y
    $$K_s = \{ \alpha \in K : \alpha \text{ separable sobre } F \}$$
    El grado de separabilidad de $K$ sobre $F$ es el grado de $[K_s : F]$. Se denota por $[K : F]_s$.\\
    El grado de inseparabilidad es $[K : K_s]$ y se denota por $[K : F]_i$.
\end{definition}

\begin{lemma}
    Sea $K/F$ una extensión de cuerpos con $char(F) = p \neq 0$.
    Si $\alpha \in K$ es algebraico sobre $F$, entonces existe un entero $m > 0$ tal que $\alpha^{p^m}$ es separable sobre $F$.
\end{lemma}

\begin{proposition}
    Sea $K/F$ una extensión finita, $L$ clausura algebraica de $F$ y $\sigma : F \to L$ una inmersión.
    Entonces el número de inmersiones de $K$ a $L$ que extienden a $\sigma$ es precisamente $[K : F]_s$.\\
    En particular, $[K : F]_s$ es el número de $F$-inmersiones de $K$ a $N$, donde $N$ es clausura normal de $K/F$.
\end{proposition}