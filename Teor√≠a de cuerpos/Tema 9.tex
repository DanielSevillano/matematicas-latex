\chapter{Construcciones geométricas}
\section{Números construibles}

\begin{definition}
    Un número complejo $\alpha$ es construible si existe una  secuencia finita de construcciones con regla y compás que empieza con 0 y 1 y acaba con $\alpha$.
\end{definition}

\begin{proposition}
    El conjunto $\mathcal{C} = \{ \alpha \in \mathbb{C} : \alpha \text{ es construible} \}$ es un subcuerpo de $\mathbb{C}$. Además:
    \begin{enumerate}
        \item Sea $\alpha = a + bi \in \mathbb{C}$. Entonces $\alpha \in \mathcal{C}$ si y solo si $a, b \in \mathcal{C}$.
        \item Si $\alpha \in \mathcal{C}$ entonces $\sqrt{\alpha} \in \mathcal{C}$.
    \end{enumerate}
\end{proposition}

\begin{theorem}
    Sea $\alpha$ un número complejo. Entonces $\alpha$ es construible si y solo si existe una sucesión de cuerpos
    $$\mathbb{Q} = F_0 \subset F_1 \subset \dots \subset F_{n-1} \subset F_n \subset \mathbb{C}$$
    tal que $\alpha \in F_n$ y $[F_i : F_{i-1}] = 2$ para todo $i = 1, \dots, n$.
\end{theorem}

\begin{corollary}
    El cuerpo de números construibles es el subcuerpo más pequeño de $\mathbb{C}$ que es cerrado para la raíz cuadrada.
\end{corollary}

\begin{corollary}
    Si $\alpha \in \mathbb{C}$ es un número construible, entonces $[\mathbb{Q}(\alpha) : \mathbb{Q}] = 2^m$ para algún $m \geq 0$.
\end{corollary}

\begin{theorem}
    Sea $\alpha \in \mathbb{C}$ algebraico sobre $\mathbb{Q}$ y sea $K$ cuerpo de descomposición del polinimio mínimo de $\alpha$ sobre $\mathbb{Q}$.
    Entonces $\alpha$ es construible si y solo si $[K : \mathbb{Q}]$ es una potencia de 2.
\end{theorem}

\section{Algunas construcciones imposibles}
\begin{itemize}
    \item Trisección del ángulo
    \item Duplicación del cubo
    \item Cuadratura del círculo
\end{itemize}

\section{Polígonos regulares}

\begin{definition}
    Un primo impar $p$ es un primo de Fermat si se puede escribir como $p = 2^{2^m}+1$ para algún entero $m \geq 0$.
\end{definition}

\begin{example}
    Los números primos de Fermat conocidos son 3, 5, 17, 257 y 65537.
\end{example}

\begin{lemma}
    Sea $k$ un entero positivo. Si $p = 2^k+1$ es un primo impar, entonces $p$ es un primo de Fermat.
\end{lemma}

\begin{theorem}
    Sea $n > 2$ un entero. Entonces un $n$-ágono regular puede ser construido por regla y compás si y solo si
    $$n = 2^s p_1 \dots p_r$$
    donde $s \geq 0$ es un entero y $p_1, \dots, p_r$ son distintos primos de Fermat.
\end{theorem}